%%% ====================================================================
%%% @LaTeX-file{
%%%   filename  = "technote.tex",
%%%   version   = "2.0",
%%%   date      = "1999/11/15",
%%%   time      = "15:07:20 EST",
%%%   checksum  = "58506 233 1647 12005",
%%%   author    = "American Mathematical Society",
%%%   copyright = "Copyright 1995, 1999 American Mathematical Society,
%%%                all rights reserved.  Copying of this file is
%%%                authorized only if either:
%%%                (1) you make absolutely no changes to your copy,
%%%                including name; OR
%%%                (2) if you do make changes, you first rename it
%%%                to some other name.",
%%%   address   = "American Mathematical Society,
%%%                Technical Support,
%%%                Electronic Products and Services,
%%%                P. O. Box 6248,
%%%                Providence, RI 02940,
%%%                USA",
%%%   telephone = "401-455-4080 or (in the USA and Canada)
%%%                800-321-4AMS (321-4267)",
%%%   FAX       = "401-331-3842",
%%%   email     = "tech-support@ams.org (Internet)",
%%%   supported = "yes",
%%%   keywords  = "amsmath, latex",
%%%   abstract  = "This is part of the AMS-\LaTeX{} distribution. It
%%%                is a document discussing some technical issues in the
%%%                contents and interface of AMS-\LaTeX{} packages."
%%%   docstring = "The checksum field above contains a CRC-16 checksum
%%%                as the first value, followed by the equivalent of
%%%                the standard UNIX wc (word count) utility output of
%%%                lines, words, and characters.  This is produced by
%%%                Robert Solovay's checksum utility.",
%%% }
%%% ====================================================================
\NeedsTeXFormat{LaTeX2e}% LaTeX 2.09 can't be used (nor non-LaTeX)
[1994/12/01]% LaTeX date must December 1994 or later
\documentclass{amsdtx}

\title{Technical notes on the \pkg{amsmath} package}
\author{American Mathematical Society\\Michael Downes}
\date{1999/10/29}

\markboth
  {Technical notes on the \pkg{amsmath} package}
  {Technical notes on the \pkg{amsmath} package}
\renewcommand{\sectionmark}[1]{}

\providecommand{\etc}[1]{etc.}

\providecommand{\begend}[1]{%
  {\ntt \symbol{92}begin\symbol{123}#1\symbol{125}}
    \ldots\ {\ntt\symbol{92}end\symbol{123}#1\symbol{125}}%
}

\MakeShortVerb{\|}

\begin{document}
\maketitle

\section{Introduction}

These notes are miscellaneous remarks on some technical questions
applicable to version 2.0 of \amslatex/.

\section{Deprecated and disallowed commands}

Certain commands that were in the \pkg{amstex} package (the predecessor,
in \amslatex/ 1.0 and 1.1, of the \pkg{amsmath} package) were moved into
\pkg{amsxtra} because they seemed to be little-used relics:
\cn{accentedsymbol}, \qq{sup accents} (\cn{sptilde}, \cn{sphat}, etc.).

Certain other commands\mdash e.g., \cn{over}, \cn{pmatrix}, and
\cn{cases}\mdash are changed by the \pkg{amsmath} package to produce
warning messages or error messages, depending on their history; these
are discussed in more detail in the following sections.

\section{Why do \cn{matrix}, \cn{pmatrix}, and \cn{cases} stop working
when I add the \pkg{amsmath} package?}

If you used the \fn{plain.tex} versions of \cn{matrix}, \cn{pmatrix}, or
\cn{cases} in a document and then later converted the document to use
the \pkg{amsmath} package (or one of the AMS documentclasses, which
automatically call the \pkg{amsmath} package internally), the instances
of those commands will produce error messages. The problem is that when
\latex/ was originally created, it adopted most of its mathematics
features straight from \fn{plain.tex}. But in the case of \cn{matrix},
\cn{pmatrix}, \cn{cases} this was a mistake\mdash the \fn{plain.tex}
syntax for them is decidedly non-\latex/ in style, for example the fact
that they use \cs{cr} instead of \cn{\\} to mark line breaks, and they
don't use \cn{begin} and \cn{end}. In basic \latex/ this mistake will be
perpetuated at least until \latex/3 appears, in order to avoid breaking
existing documents. But no existing documents that were written with the
\pkg{amsmath} package have that syntactic problem, as \pkg{amsmath}
provides proper \latex/-syntax versions of \cn{matrix} and the others.
The possibility of optionally allowing the \fn{plain.tex} variants to
make document conversion easier seems ill-advised since those variants
are so blatantly wrong in a \latex/ context. The \env{array} environment
ought to have been used instead.

\section{Why do \cn{over}, \cn{atop}, \cn{above} [\dots{\ntt withdelims}]
 give a warning when I add the \pkg{amsmath} package?}

Using the six generalized fraction commands \cn{over},
\cn{overwithdelims}, \cn{atop}, \cn{atopwithdelims}, \cn{above},
\cn{abovewithdelims} is not really a good idea in \latex/, for reasons
explained below. I construe \latex/'s provision of \cs{frac}, and the
lack of any mention in the \latex/ book of the primitive fraction
commands, as an implicit injunction against their use, although I don't
think Lamport actually spent a lot of time pondering the issue, and the
basic \latex/ version of \cn{frac} provides access only to \cn{over},
not to \cn{atop}, \cn{above}, or the \verb'withdelims' variants. The
\pkg{amsmath} package, however, provides a \cn{genfrac} command that
gives user-level access to all six of the generalized fraction
primitives in a way that conforms to the syntactic conventions followed
by all other \latex/ commands.

Not only is the unusual syntax of the \tex/ primitives rather out of
place in \latex/, but furthermore that syntax seems to be responsible
for one of the most significant flaws in \tex/'s mathematical
typesetting capabilities: the fact that the current mathstyle at any
given point in a math formula cannot be determined until the end of the
formula, because of the possibility that a following generalized
fraction command will change the mathstyle of the \emph{preceding}
material. To cite two of the worst side effects: \cn{mathchoice} must
actually typeset all four of its arguments, instead of being able to
immediately select only one; and, were it possible to always know the
current math style at a given point, math font selection would be
greatly simplified and the upper limit of 16 different math font
\cn{fam}s would never be a problem as \cn{text,script[script]font}
assignments for any \cn{fam} could take immediate effect and therefore
could be changed arbitrarily often within a single formula. More
concretely, math font selection difficulties are responsible for many of
the more convoluted passages in the source code of \latex/'s NFSS (that
does font loading on demand) and of the \pkg{amsmath} package, and by
extension it has historically been responsible for significant delays in
making new features available to end users and for making those features
more prone to bugs.

There are additional bad consequences following from the syntax of those
generalized fraction commands that only become evident when you do some
writing of nontrivial macros for math use. For example, as things
currently stand you cannot measure the size of any object in math
without going through \cn{mathchoice} and \emph{leaving and reentering
math mode} via \verb'\hbox{$' (which then introduces complications
regarding \cn{everymath} and \cn{mathsurround}). And it seems that
uncertainty about the current mathstyle is the only barrier to allowing
the use of mu units with \cn{vrule}, to make vertical struts in
constructing compound symbols or notation. And so on and so forth.

\section{The \opt{fleqn} option and \cn{mathindent}}

Strictly speaking, the \pkg{amsmath} package doesn't use \cn{mathindent}
to control the left indent of displayed equations when the \opt{fleqn}
option is in effect: it uses an internal parameter \cs{@mathmargin}
instead. However, for compatibility with existing \latex/ documentation,
\pkg{amsmath} turns \cn{mathindent} into an alias for \cs{@mathmargin}.
There is a small risk here: In the plain \latex/ implementation,
\cn{mathindent} is a dimen register, but with \pkg{amsmath}
\cs{@mathmargin} is a skip register and, by association, so is
\cn{mathindent}. If any package or documentclass uses \cn{mathindent} in
a way that depends on it being a dimen register, when used in
conjunction with the \pkg{amsmath} package it may be vulnerable to a
well-known pitfall having to do with the primitive \tex/ lookahead for a
\texttt{plus} or \texttt{minus} key word. However if the standard
\latex/ commands \cn{setlength} and \cn{addtolength} are used to modify
\cn{mathindent} then this problem will not arise.

\section{Why can't I use abbreviations for \protect\begend{align}?}

Authors often like to use abbreviations such as |\beq| |\eeq| for
|\begin{equation}| |\end{equation}|. For some environments defined by
the \pkg{amsmath} package, such as \env{align}, \env{gather},
\env{multline}, and others of the same general type, this does not work:
An attempt to define |\bal| |\eal| as shorthand for |\begin{align}|
|\end{align}| will fail with a puzzling error message. This has to do
with unfortunately nontrivial technical complications: the given
environments must read their contents as a delimited macro argument
because they do multipass processing of the contents using algorithms
inherited from Spivak's \fn{amstex.tex}. The obvious
solution\mdash substitution of different algorithms that do box shuffling
instead of token shuffling for the multipass calculations\mdash would
require rewriting these display environments from the ground up; while
that is a worthy goal, it was beyond the original scope of the
\amslatex/ project. Work is under way on an auxiliary package called
\pkg{breqn} that addresses not only this problem but a number of others;
at the time of this writing, however [September 1999] it has only
progressed as far as a beta release.

Some workarounds:
\begin{itemize}
\item |\def\bal#1\eal{\begin{align}#1\end{align}}|
\item Define |\newcommand{\env}[2]{\begin{#1}#2\end{#1}}| and then use
  |\env{align}{...}|
\end{itemize}

\section{The \pkg{upref} package}

The reason for splitting out the \pkg{upref} package instead of
automatically incorporating it in the \cls{amsart} and \cls{amsbook}
classes is this: It involves low-level surgery on an important \latex/
command. This means that if ever this command changes in the future (as
it did between versions 2.09 and 2e of \latex/) we have a maintenance
problem. And the benefit that \pkg{upref} provides is something that
most end users don't care much about.

\section{The \pkg{amsintx} package}

After a very preliminary trial release the \pkg{amsintx} package was
withdrawn to await further development. It is still considered a worthy
project but has simply not yet made it to the head of the priority queue
in the AMS \latex/ development program, as it has had to contend with a
number of other equally worthy projects for development time.

\section{Hyphenation in the documentation}

Hyphenation was allowed for certain long command names in
\fn{amsldoc.tex}; this presented technical difficulties because \latex/
normally deactivates hyphenation for tt fonts. The method chosen to
reinstate hyphenation was to turn off the encoding-specific function
\cs{OT1+cmtt} that disables the \cs{hyphenchar} for tt fonts; see the
definition of \cn{allowtthyphens} in \fn{amsdtx.dtx}. Then a list of all
tt words in the document was gathered (from the \fn{.idx} file, produced
by the \cn{cn}, \cn{fn}, \cn{pkg}, etc. commands) and \cn{showhyphens}
was applied to this list. The result was another list in the resulting
\tex/ log, containing those words in a form suitable for the argument of
\cn{hyphenation}. That list was then edited by hand to overrule
undesirable hyphenations; words with acceptable hyphenations were
dropped from the list, as they don't need to be repeated there.

\end{document}
