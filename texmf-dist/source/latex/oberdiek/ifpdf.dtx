% \iffalse meta-comment
%
% File: ifpdf.dtx
% Version: 2011/01/30 v2.3
% Info: Provides the ifpdf switch
%
% Copyright (C) 2001, 2005-2011 by
%    Heiko Oberdiek <heiko.oberdiek at googlemail.com>
%
% This work may be distributed and/or modified under the
% conditions of the LaTeX Project Public License, either
% version 1.3c of this license or (at your option) any later
% version. This version of this license is in
%    http://www.latex-project.org/lppl/lppl-1-3c.txt
% and the latest version of this license is in
%    http://www.latex-project.org/lppl.txt
% and version 1.3 or later is part of all distributions of
% LaTeX version 2005/12/01 or later.
%
% This work has the LPPL maintenance status "maintained".
%
% This Current Maintainer of this work is Heiko Oberdiek.
%
% The Base Interpreter refers to any `TeX-Format',
% because some files are installed in TDS:tex/generic//.
%
% This work consists of the main source file ifpdf.dtx
% and the derived files
%    ifpdf.sty, ifpdf.pdf, ifpdf.ins, ifpdf.drv, ifpdf-test1.tex.
%
% Distribution:
%    CTAN:macros/latex/contrib/oberdiek/ifpdf.dtx
%    CTAN:macros/latex/contrib/oberdiek/ifpdf.pdf
%
% Unpacking:
%    (a) If ifpdf.ins is present:
%           tex ifpdf.ins
%    (b) Without ifpdf.ins:
%           tex ifpdf.dtx
%    (c) If you insist on using LaTeX
%           latex \let\install=y% \iffalse meta-comment
%
% File: ifpdf.dtx
% Version: 2011/01/30 v2.3
% Info: Provides the ifpdf switch
%
% Copyright (C) 2001, 2005-2011 by
%    Heiko Oberdiek <heiko.oberdiek at googlemail.com>
%
% This work may be distributed and/or modified under the
% conditions of the LaTeX Project Public License, either
% version 1.3c of this license or (at your option) any later
% version. This version of this license is in
%    http://www.latex-project.org/lppl/lppl-1-3c.txt
% and the latest version of this license is in
%    http://www.latex-project.org/lppl.txt
% and version 1.3 or later is part of all distributions of
% LaTeX version 2005/12/01 or later.
%
% This work has the LPPL maintenance status "maintained".
%
% This Current Maintainer of this work is Heiko Oberdiek.
%
% The Base Interpreter refers to any `TeX-Format',
% because some files are installed in TDS:tex/generic//.
%
% This work consists of the main source file ifpdf.dtx
% and the derived files
%    ifpdf.sty, ifpdf.pdf, ifpdf.ins, ifpdf.drv, ifpdf-test1.tex.
%
% Distribution:
%    CTAN:macros/latex/contrib/oberdiek/ifpdf.dtx
%    CTAN:macros/latex/contrib/oberdiek/ifpdf.pdf
%
% Unpacking:
%    (a) If ifpdf.ins is present:
%           tex ifpdf.ins
%    (b) Without ifpdf.ins:
%           tex ifpdf.dtx
%    (c) If you insist on using LaTeX
%           latex \let\install=y% \iffalse meta-comment
%
% File: ifpdf.dtx
% Version: 2011/01/30 v2.3
% Info: Provides the ifpdf switch
%
% Copyright (C) 2001, 2005-2011 by
%    Heiko Oberdiek <heiko.oberdiek at googlemail.com>
%
% This work may be distributed and/or modified under the
% conditions of the LaTeX Project Public License, either
% version 1.3c of this license or (at your option) any later
% version. This version of this license is in
%    http://www.latex-project.org/lppl/lppl-1-3c.txt
% and the latest version of this license is in
%    http://www.latex-project.org/lppl.txt
% and version 1.3 or later is part of all distributions of
% LaTeX version 2005/12/01 or later.
%
% This work has the LPPL maintenance status "maintained".
%
% This Current Maintainer of this work is Heiko Oberdiek.
%
% The Base Interpreter refers to any `TeX-Format',
% because some files are installed in TDS:tex/generic//.
%
% This work consists of the main source file ifpdf.dtx
% and the derived files
%    ifpdf.sty, ifpdf.pdf, ifpdf.ins, ifpdf.drv, ifpdf-test1.tex.
%
% Distribution:
%    CTAN:macros/latex/contrib/oberdiek/ifpdf.dtx
%    CTAN:macros/latex/contrib/oberdiek/ifpdf.pdf
%
% Unpacking:
%    (a) If ifpdf.ins is present:
%           tex ifpdf.ins
%    (b) Without ifpdf.ins:
%           tex ifpdf.dtx
%    (c) If you insist on using LaTeX
%           latex \let\install=y% \iffalse meta-comment
%
% File: ifpdf.dtx
% Version: 2011/01/30 v2.3
% Info: Provides the ifpdf switch
%
% Copyright (C) 2001, 2005-2011 by
%    Heiko Oberdiek <heiko.oberdiek at googlemail.com>
%
% This work may be distributed and/or modified under the
% conditions of the LaTeX Project Public License, either
% version 1.3c of this license or (at your option) any later
% version. This version of this license is in
%    http://www.latex-project.org/lppl/lppl-1-3c.txt
% and the latest version of this license is in
%    http://www.latex-project.org/lppl.txt
% and version 1.3 or later is part of all distributions of
% LaTeX version 2005/12/01 or later.
%
% This work has the LPPL maintenance status "maintained".
%
% This Current Maintainer of this work is Heiko Oberdiek.
%
% The Base Interpreter refers to any `TeX-Format',
% because some files are installed in TDS:tex/generic//.
%
% This work consists of the main source file ifpdf.dtx
% and the derived files
%    ifpdf.sty, ifpdf.pdf, ifpdf.ins, ifpdf.drv, ifpdf-test1.tex.
%
% Distribution:
%    CTAN:macros/latex/contrib/oberdiek/ifpdf.dtx
%    CTAN:macros/latex/contrib/oberdiek/ifpdf.pdf
%
% Unpacking:
%    (a) If ifpdf.ins is present:
%           tex ifpdf.ins
%    (b) Without ifpdf.ins:
%           tex ifpdf.dtx
%    (c) If you insist on using LaTeX
%           latex \let\install=y\input{ifpdf.dtx}
%        (quote the arguments according to the demands of your shell)
%
% Documentation:
%    (a) If ifpdf.drv is present:
%           latex ifpdf.drv
%    (b) Without ifpdf.drv:
%           latex ifpdf.dtx; ...
%    The class ltxdoc loads the configuration file ltxdoc.cfg
%    if available. Here you can specify further options, e.g.
%    use A4 as paper format:
%       \PassOptionsToClass{a4paper}{article}
%
%    Programm calls to get the documentation (example):
%       pdflatex ifpdf.dtx
%       makeindex -s gind.ist ifpdf.idx
%       pdflatex ifpdf.dtx
%       makeindex -s gind.ist ifpdf.idx
%       pdflatex ifpdf.dtx
%
% Installation:
%    TDS:tex/generic/oberdiek/ifpdf.sty
%    TDS:doc/latex/oberdiek/ifpdf.pdf
%    TDS:doc/latex/oberdiek/test/ifpdf-test1.tex
%    TDS:source/latex/oberdiek/ifpdf.dtx
%
%<*ignore>
\begingroup
  \catcode123=1 %
  \catcode125=2 %
  \def\x{LaTeX2e}%
\expandafter\endgroup
\ifcase 0\ifx\install y1\fi\expandafter
         \ifx\csname processbatchFile\endcsname\relax\else1\fi
         \ifx\fmtname\x\else 1\fi\relax
\else\csname fi\endcsname
%</ignore>
%<*install>
\input docstrip.tex
\Msg{************************************************************************}
\Msg{* Installation}
\Msg{* Package: ifpdf 2011/01/30 v2.3 Provides the ifpdf switch (HO)}
\Msg{************************************************************************}

\keepsilent
\askforoverwritefalse

\let\MetaPrefix\relax
\preamble

This is a generated file.

Project: ifpdf
Version: 2011/01/30 v2.3

Copyright (C) 2001, 2005-2011 by
   Heiko Oberdiek <heiko.oberdiek at googlemail.com>

This work may be distributed and/or modified under the
conditions of the LaTeX Project Public License, either
version 1.3c of this license or (at your option) any later
version. This version of this license is in
   http://www.latex-project.org/lppl/lppl-1-3c.txt
and the latest version of this license is in
   http://www.latex-project.org/lppl.txt
and version 1.3 or later is part of all distributions of
LaTeX version 2005/12/01 or later.

This work has the LPPL maintenance status "maintained".

This Current Maintainer of this work is Heiko Oberdiek.

The Base Interpreter refers to any `TeX-Format',
because some files are installed in TDS:tex/generic//.

This work consists of the main source file ifpdf.dtx
and the derived files
   ifpdf.sty, ifpdf.pdf, ifpdf.ins, ifpdf.drv, ifpdf-test1.tex.

\endpreamble
\let\MetaPrefix\DoubleperCent

\generate{%
  \file{ifpdf.ins}{\from{ifpdf.dtx}{install}}%
  \file{ifpdf.drv}{\from{ifpdf.dtx}{driver}}%
  \usedir{tex/generic/oberdiek}%
  \file{ifpdf.sty}{\from{ifpdf.dtx}{package}}%
  \usedir{doc/latex/oberdiek/test}%
  \file{ifpdf-test1.tex}{\from{ifpdf.dtx}{test1}}%
  \nopreamble
  \nopostamble
  \usedir{source/latex/oberdiek/catalogue}%
  \file{ifpdf.xml}{\from{ifpdf.dtx}{catalogue}}%
}

\catcode32=13\relax% active space
\let =\space%
\Msg{************************************************************************}
\Msg{*}
\Msg{* To finish the installation you have to move the following}
\Msg{* file into a directory searched by TeX:}
\Msg{*}
\Msg{*     ifpdf.sty}
\Msg{*}
\Msg{* To produce the documentation run the file `ifpdf.drv'}
\Msg{* through LaTeX.}
\Msg{*}
\Msg{* Happy TeXing!}
\Msg{*}
\Msg{************************************************************************}

\endbatchfile
%</install>
%<*ignore>
\fi
%</ignore>
%<*driver>
\NeedsTeXFormat{LaTeX2e}
\ProvidesFile{ifpdf.drv}%
  [2011/01/30 v2.3 Provides the ifpdf switch (HO)]%
\documentclass{ltxdoc}
\usepackage{holtxdoc}[2011/11/22]
\begin{document}
  \DocInput{ifpdf.dtx}%
\end{document}
%</driver>
% \fi
%
% \CheckSum{600}
%
% \CharacterTable
%  {Upper-case    \A\B\C\D\E\F\G\H\I\J\K\L\M\N\O\P\Q\R\S\T\U\V\W\X\Y\Z
%   Lower-case    \a\b\c\d\e\f\g\h\i\j\k\l\m\n\o\p\q\r\s\t\u\v\w\x\y\z
%   Digits        \0\1\2\3\4\5\6\7\8\9
%   Exclamation   \!     Double quote  \"     Hash (number) \#
%   Dollar        \$     Percent       \%     Ampersand     \&
%   Acute accent  \'     Left paren    \(     Right paren   \)
%   Asterisk      \*     Plus          \+     Comma         \,
%   Minus         \-     Point         \.     Solidus       \/
%   Colon         \:     Semicolon     \;     Less than     \<
%   Equals        \=     Greater than  \>     Question mark \?
%   Commercial at \@     Left bracket  \[     Backslash     \\
%   Right bracket \]     Circumflex    \^     Underscore    \_
%   Grave accent  \`     Left brace    \{     Vertical bar  \|
%   Right brace   \}     Tilde         \~}
%
% \GetFileInfo{ifpdf.drv}
%
% \title{The \xpackage{ifpdf} package}
% \date{2011/01/30 v2.3}
% \author{Heiko Oberdiek\\\xemail{heiko.oberdiek at googlemail.com}}
%
% \maketitle
%
% \begin{abstract}
% This package looks for \pdfTeX\ in pdf mode and
% implements and sets the switch \cs{ifpdf}.
% The detection is based on \cs{pdfoutput} and
% the package will not change this value.
% It works with plain or \LaTeX\ formats.
% \end{abstract}
%
% \tableofcontents
%
% \hypersetup{bookmarksopenlevel=2}
% \section{Documentation}
%
% \subsection{Introduction}
%
% It is commonly known that H\`an Th\^e Th\`anh's \pdfTeX\ generates
% PDF output directly and many people uses \pdfTeX\ for this purpose.
% However the DVI output was never thrown away. In contrary, he
% new features for typesetting that works in both PDF and DVI mode.
%
% In the meantime many \TeX\ distributions replace the traditional
% \TeX\ binary with \pdfTeX. Then, for example, called as |latex|
% \pdfTeX\ works in DVI mode with the \LaTeX\ format preloaded,
% called as |pdflatex| \pdfTeX\ starts in PDF mode.
%
% Often packages or users want to know, whether the current document
% is typset by \pdfTeX\ in PDF mode, because the different modes
% have different capabilities (color setting, graphics inclusion,
% \dots). For this purpose \pdfTeX's
% \cs{pdfoutput} can be asked.
%
% As regulary reader of \TeX\ newsgroups and mailing lists I could
% observe many problems with this task. Common errors are:
% \begin{itemize}
% \item
%   \pdfTeX\ has \emph{two} modes. Using \pdfTeX\ does not mean
%   that the user always want to have PDF mode. For example, the
%   PostScript support is better in DVI mode in conjunction with a
%   PostScript aware DVI driver (e.g. dvips). Also the additional
%   typesetting features are mode independent and also available
%   in DVI mode.
% \item
%   \LaTeX's \cs{@ifundefined} inherited the side effect from
%   \cs{csname}. Unknown commands are defined with the meaning
%   of \cs{relax}. If it is checked, whether \cs{pdfoutput} is
%   defined, then this should not be forgotten.
% \item
%   Having \cs{pdfoutput} does not automatically mean PDF mode.
%   Also the value of \cs{pdfoutput} must be asked.
% \item
%   \cs{pdfoutput} must not be destroyed in some way. Later
%   code and packages are fooled then and will perhaps make
%   wrong decisions. For example they may drop support for
%   PDF mode, because they do not know that \pdfTeX\ is
%   running at all.
% \end{itemize}
%
% Robin Fairbairns provides an entry for this topic in his
% excellent FAQ (\url{http://www.tex.ac.uk/faq}):
% \href{http://www.tex.ac.uk/cgi-bin/texfaq2html?label=ifpdf}^^A
% {Am I using PDFTeX?}
%
% \subsection{Usage}
%
% The package \xpackage{ifpdf} can be used with both \plainTeX\
% and \LaTeX:
% \begin{description}
% \item[\plainTeX:] |\input ifpdf.sty|
% \item[\LaTeXe:]   |\usepackage{ifpdf}|
% \end{description}
%
% \DescribeMacro{\ifpdf}
% The package provides the switch \cs{ifpdf}:
% \begin{quote}
%   |\ifpdf|\\
%   \hspace*{1.5em}\dots\ do things, if \pdfTeX\ is running in pdf mode \dots\\
%   |\else|\\
%   \hspace*{1.5em}\dots\ other \TeX\ or \pdfTeX\ in dvi mode \dots\\
%   |\fi|
% \end{quote}
% Users of the package \xpackage{ifthen} can use the switch as boolean:
% \begin{quote}
%   |\boolean{pdf}|
% \end{quote}
% The package can also be used to set global
% documentclass options:
% \begin{quote}
%   |\RequirePackage{ifpdf}|\\
%   |\ifpdf|\\
%   \hspace*{1.5em}|\documentclass[pdftex,|\dots|]{|\dots|}|\\
%   |\else|\\
%   \hspace*{1.5em}|\documentclass[|\dots|]{|\dots|}|\\
%   |\fi|
% \end{quote}
%
% \subsection{Specification}
%
% The package have the following properties:
% \begin{itemize}
% \item It asks the setting of \cs{pdfoutput} for detecting
%   \pdfTeX\ in PDF mode.
% \item It never changes \cs{pdfoutput}.
% \item If \cs{pdfoutput} is undefined or has the meaning \cs{relax},
%   but the engine provides the primitive \cs{pdfoutput},
%   then \cs{pdfoutput} is enabled or restored if possible
%   (only \LuaTeX, version 0.36.0 or higher).
% \item It can be used with many formats including
%   \plainTeX\ and \LaTeX.
% \end{itemize}
% The mode detection implements the following algorithm:
% \begin{quote}
% \def\={${:}{=}$}
% \def\>{\null\qquad\ignorespaces}
% \def\comment#1{^^A
%   \textrm{\itshape\% #1}^^A
% }
% \sffamily
% if undefined(\cs{pdfoutput})\\
% \> \cs{ifpdf} \= \cs{iffalse} \comment{\pdfTeX is not running}\\
% else\\
% \> if \cs{pdfoutput} $\leq$ 0\\
% \>\> \cs{ifpdf} \= \cs{iffalse} \comment{\pdfTeX\ in DVI mode}\\
% \> else\\
% \>\> \cs{ifpdf} \= \cs{iftrue} \comment{\pdfTeX\ in PDF mode}\\
% \> fi\\
% fi
% \end{quote}
% The function \textsf{undefined} checks both cases, undefined command
% and \cs{relax}.
%
% \hypersetup{bookmarksopenlevel=1}
%
% \StopEventually{
% }
%
% \section{Implementation}
%
%    \begin{macrocode}
%<*package>
%    \end{macrocode}
%
% \subsection{Reload check and package identification}
%    Reload check, especially if the package is not used with \LaTeX.
%    \begin{macrocode}
\begingroup\catcode61\catcode48\catcode32=10\relax%
  \catcode13=5 % ^^M
  \endlinechar=13 %
  \catcode35=6 % #
  \catcode39=12 % '
  \catcode44=12 % ,
  \catcode45=12 % -
  \catcode46=12 % .
  \catcode58=12 % :
  \catcode64=11 % @
  \catcode123=1 % {
  \catcode125=2 % }
  \expandafter\let\expandafter\x\csname ver@ifpdf.sty\endcsname
  \ifx\x\relax % plain-TeX, first loading
  \else
    \def\empty{}%
    \ifx\x\empty % LaTeX, first loading,
      % variable is initialized, but \ProvidesPackage not yet seen
    \else
      \expandafter\ifx\csname PackageInfo\endcsname\relax
        \def\x#1#2{%
          \immediate\write-1{Package #1 Info: #2.}%
        }%
      \else
        \def\x#1#2{\PackageInfo{#1}{#2, stopped}}%
      \fi
      \x{ifpdf}{The package is already loaded}%
      \aftergroup\endinput
    \fi
  \fi
\endgroup%
%    \end{macrocode}
%    Package identification:
%    \begin{macrocode}
\begingroup\catcode61\catcode48\catcode32=10\relax%
  \catcode13=5 % ^^M
  \endlinechar=13 %
  \catcode35=6 % #
  \catcode39=12 % '
  \catcode40=12 % (
  \catcode41=12 % )
  \catcode44=12 % ,
  \catcode45=12 % -
  \catcode46=12 % .
  \catcode47=12 % /
  \catcode58=12 % :
  \catcode64=11 % @
  \catcode91=12 % [
  \catcode93=12 % ]
  \catcode123=1 % {
  \catcode125=2 % }
  \expandafter\ifx\csname ProvidesPackage\endcsname\relax
    \def\x#1#2#3[#4]{\endgroup
      \immediate\write-1{Package: #3 #4}%
      \xdef#1{#4}%
    }%
  \else
    \def\x#1#2[#3]{\endgroup
      #2[{#3}]%
      \ifx#1\@undefined
        \xdef#1{#3}%
      \fi
      \ifx#1\relax
        \xdef#1{#3}%
      \fi
    }%
  \fi
\expandafter\x\csname ver@ifpdf.sty\endcsname
\ProvidesPackage{ifpdf}%
  [2011/01/30 v2.3 Provides the ifpdf switch (HO)]%
%    \end{macrocode}
%
% \subsection{Catcodes}
%
%    \begin{macrocode}
\begingroup\catcode61\catcode48\catcode32=10\relax%
  \catcode13=5 % ^^M
  \endlinechar=13 %
  \catcode123=1 % {
  \catcode125=2 % }
  \catcode64=11 % @
  \def\x{\endgroup
    \expandafter\edef\csname ifpdf@AtEnd\endcsname{%
      \endlinechar=\the\endlinechar\relax
      \catcode13=\the\catcode13\relax
      \catcode32=\the\catcode32\relax
      \catcode35=\the\catcode35\relax
      \catcode61=\the\catcode61\relax
      \catcode64=\the\catcode64\relax
      \catcode123=\the\catcode123\relax
      \catcode125=\the\catcode125\relax
    }%
  }%
\x\catcode61\catcode48\catcode32=10\relax%
\catcode13=5 % ^^M
\endlinechar=13 %
\catcode35=6 % #
\catcode64=11 % @
\catcode123=1 % {
\catcode125=2 % }
\def\TMP@EnsureCode#1#2{%
  \edef\ifpdf@AtEnd{%
    \ifpdf@AtEnd
    \catcode#1=\the\catcode#1\relax
  }%
  \catcode#1=#2\relax
}
\TMP@EnsureCode{10}{12}% ^^J
\TMP@EnsureCode{39}{12}% '
\TMP@EnsureCode{40}{12}% (
\TMP@EnsureCode{41}{12}% )
\TMP@EnsureCode{44}{12}% ,
\TMP@EnsureCode{45}{12}% -
\TMP@EnsureCode{46}{12}% .
\TMP@EnsureCode{47}{12}% /
\TMP@EnsureCode{58}{12}% :
\TMP@EnsureCode{60}{12}% <
\TMP@EnsureCode{91}{12}% [
\TMP@EnsureCode{93}{12}% ]
\TMP@EnsureCode{94}{7}% ^
\TMP@EnsureCode{96}{12}% `
\edef\ifpdf@AtEnd{\ifpdf@AtEnd\noexpand\endinput}
%    \end{macrocode}
%
% \subsection{Check for previously defined \cs{ifpdf}}
%
%    \begin{macrocode}
\begingroup
  \expandafter\ifx\csname ifpdf\endcsname\relax
  \else
    \edef\i/{\expandafter\string\csname ifpdf\endcsname}%
    \expandafter\ifx\csname PackageError\endcsname\relax
      \def\x#1#2{%
        \edef\z{#2}%
        \expandafter\errhelp\expandafter{\z}%
        \errmessage{Package ifpdf Error: #1}%
      }%
      \def\y{^^J}%
      \newlinechar=10 %
    \else
      \def\x#1#2{%
        \PackageError{ifpdf}{#1}{#2}%
      }%
      \def\y{\MessageBreak}%
    \fi
    \x{Name clash, \i/ is already defined}{%
      Incompatible versions of \i/ can cause problems,\y
      therefore package loading is aborted.%
    }%
    \endgroup
    \expandafter\ifpdf@AtEnd
  \fi%
\endgroup
%    \end{macrocode}
%
% \subsection{\cs{pdfoutput} and \LuaTeX}
%
%    It might happen, that \LuaTeX\ is running, but \cs{pdfoutput}
%    does not exist. In version 0.40 only \cs{directlua}
%    is available at startup time. The enabling Lua function was
%    already added in version 0.36. Thus we can ignore older versions,
%    here \cs{pdfoutput} is available at startup time.
%
%    \begin{macrocode}
\begingroup
  \def\skip#1\relax\begingroup{}%
  \expandafter\ifx\csname pdfoutput\endcsname\relax
  \else
    \expandafter\skip
  \fi
  \expandafter\ifx\csname directlua\endcsname\relax
    \expandafter\skip
  \fi
\endgroup
\begingroup\expandafter\expandafter\expandafter\endgroup
\expandafter\ifx\csname RequirePackage\endcsname\relax
  \def\TMP@RequirePackage#1[#2]{%
    \begingroup\expandafter\expandafter\expandafter\endgroup
    \expandafter\ifx\csname ver@#1.sty\endcsname\relax
      \input #1.sty\relax
    \fi
  }%
  \TMP@RequirePackage{ifluatex}[2009/04/10]%
\else
  \RequirePackage{ifluatex}[2009/04/10]%
\fi
\ifluatex
  \ifnum\luatexversion<36 %
%    \end{macrocode}
%    Unhappily \LuaTeX's \cs{primitive} (derived from
%    pdfTeX's \cs{pdfprimitive}) cannot be used:
%    \begin{quote}
%    |\protected\gdef\pdfoutput{\primitive\pdfoutput}|
%    \end{quote}
%    Setting a value works, but getting fails,
%    because \TeX\ does no longer see it as number. It
%    is unexpandable and breaks numerical contexts.
%    This was fixed in \hologo{pdfTeX} 1.40.10
%    (bugfix \#4289: ``|\primitive\pdfoutput| cannot be queried'').
%    \begin{macrocode}
  \else
    \begingroup
      \directlua{tex.enableprimitives('ifpdf', {'pdfoutput'})}%
      \global\let\pdfoutput\ifpdfpdfoutput
    \endgroup
  \fi
\fi
\relax\begingroup\endgroup
%    \end{macrocode}
%
% \subsection{\cs{ifpdf}}
%
%    \begin{macro}{\ifpdf}
%    Create and set the switch. \cs{newif} initializes the
%    switch with \cs{iffalse}. \cs{newif} is \cs{outer}
%    in \plainTeX.
%    \begin{macrocode}
\begingroup\expandafter\expandafter\expandafter\endgroup
\expandafter\ifx\csname newif\endcsname\relax
  \edef\pdffalse{%
    \let
    \expandafter\noexpand\csname ifpdf\endcsname
    \expandafter\noexpand\csname iffalse\endcsname
  }%
  \edef\pdftrue{%
    \let
    \expandafter\noexpand\csname ifpdf\endcsname
    \expandafter\noexpand\csname iftrue\endcsname
  }%
  \pdffalse
\else
  \csname newif\expandafter\endcsname\csname ifpdf\endcsname
\fi
%    \end{macrocode}
%    Test \cs{pdfoutput}. Is it  defined and different from
%    \cs{relax}? Someone could have used \LaTeX\ internal
%    \cs{@ifundefined}, or something else involving.
%    Notice, \cs{csname} is executed inside a group for the test
%    to cancel the side effect of \cs{csname}.
%    \begin{macrocode}
\begingroup\expandafter\expandafter\expandafter\endgroup
\expandafter\ifx\csname pdfoutput\endcsname\relax
\else
  \ifnum\pdfoutput<1 %
%    \end{macrocode}
%    \cs{pdfoutput}|=0| or negative, so not generating pdf.
%    \begin{macrocode}
  \else
    \pdftrue
  \fi
\fi
%    \end{macrocode}
%    \end{macro}
%
% \subsection{Test for fool attempts}
%
%    \begin{macrocode}
\begingroup
  \expandafter\ifx\csname pdfoutput\endcsname\relax
  \else
    \escapechar=92 %
    \edef\m{\meaning\pdfoutput}%
    \edef\p{\string\pdfoutput}%
    \ifx\m\p
    \else
      \expandafter\ifx\csname PackageWarningNoLine\endcsname\relax
        \def\PackageWarningNoLine#1#2{%
          \immediate\write16{%
            Package `#1' Warning: #2.%
          }%
        }%
      \fi
      \PackageWarningNoLine{ifpdf}{%
        Someone has redefined \string\pdfoutput%
      }%
    \fi
  \fi
\endgroup
%    \end{macrocode}
%
% \subsection{Protocol entry}
%
%     Log comment:
%    \begin{macrocode}
\begingroup
  \expandafter\ifx\csname PackageInfo\endcsname\relax
    \def\x#1#2{%
      \immediate\write-1{Package #1 Info: #2.}%
    }%
  \else
    \let\x\PackageInfo
    \expandafter\let\csname on@line\endcsname\empty
  \fi
  \x{ifpdf}{pdfTeX in PDF mode is \ifpdf\else not \fi detected}%
\endgroup
%    \end{macrocode}
%    \begin{macrocode}
\ifpdf@AtEnd%
%    \end{macrocode}
%    \begin{macrocode}
%</package>
%    \end{macrocode}
%
% \section{Test}
%
% \subsection{Catcode checks for loading}
%
%    \begin{macrocode}
%<*test1>
%    \end{macrocode}
%    \begin{macrocode}
\catcode`\{=1 %
\catcode`\}=2 %
\catcode`\#=6 %
\catcode`\@=11 %
\expandafter\ifx\csname count@\endcsname\relax
  \countdef\count@=255 %
\fi
\expandafter\ifx\csname @gobble\endcsname\relax
  \long\def\@gobble#1{}%
\fi
\expandafter\ifx\csname @firstofone\endcsname\relax
  \long\def\@firstofone#1{#1}%
\fi
\expandafter\ifx\csname loop\endcsname\relax
  \expandafter\@firstofone
\else
  \expandafter\@gobble
\fi
{%
  \def\loop#1\repeat{%
    \def\body{#1}%
    \iterate
  }%
  \def\iterate{%
    \body
      \let\next\iterate
    \else
      \let\next\relax
    \fi
    \next
  }%
  \let\repeat=\fi
}%
\def\RestoreCatcodes{}
\count@=0 %
\loop
  \edef\RestoreCatcodes{%
    \RestoreCatcodes
    \catcode\the\count@=\the\catcode\count@\relax
  }%
\ifnum\count@<255 %
  \advance\count@ 1 %
\repeat

\def\RangeCatcodeInvalid#1#2{%
  \count@=#1\relax
  \loop
    \catcode\count@=15 %
  \ifnum\count@<#2\relax
    \advance\count@ 1 %
  \repeat
}
\def\RangeCatcodeCheck#1#2#3{%
  \count@=#1\relax
  \loop
    \ifnum#3=\catcode\count@
    \else
      \errmessage{%
        Character \the\count@\space
        with wrong catcode \the\catcode\count@\space
        instead of \number#3%
      }%
    \fi
  \ifnum\count@<#2\relax
    \advance\count@ 1 %
  \repeat
}
\def\space{ }
\expandafter\ifx\csname LoadCommand\endcsname\relax
  \def\LoadCommand{\input ifpdf.sty\relax}%
\fi
\def\Test{%
  \RangeCatcodeInvalid{0}{47}%
  \RangeCatcodeInvalid{58}{64}%
  \RangeCatcodeInvalid{91}{96}%
  \RangeCatcodeInvalid{123}{255}%
  \catcode`\@=12 %
  \catcode`\\=0 %
  \catcode`\%=14 %
  \LoadCommand
  \RangeCatcodeCheck{0}{36}{15}%
  \RangeCatcodeCheck{37}{37}{14}%
  \RangeCatcodeCheck{38}{47}{15}%
  \RangeCatcodeCheck{48}{57}{12}%
  \RangeCatcodeCheck{58}{63}{15}%
  \RangeCatcodeCheck{64}{64}{12}%
  \RangeCatcodeCheck{65}{90}{11}%
  \RangeCatcodeCheck{91}{91}{15}%
  \RangeCatcodeCheck{92}{92}{0}%
  \RangeCatcodeCheck{93}{96}{15}%
  \RangeCatcodeCheck{97}{122}{11}%
  \RangeCatcodeCheck{123}{255}{15}%
  \RestoreCatcodes
}
\Test
\csname @@end\endcsname
\end
%    \end{macrocode}
%    \begin{macrocode}
%</test1>
%    \end{macrocode}
%
% \section{Installation}
%
% \subsection{Download}
%
% \paragraph{Package.} This package is available on
% CTAN\footnote{\url{ftp://ftp.ctan.org/tex-archive/}}:
% \begin{description}
% \item[\CTAN{macros/latex/contrib/oberdiek/ifpdf.dtx}] The source file.
% \item[\CTAN{macros/latex/contrib/oberdiek/ifpdf.pdf}] Documentation.
% \end{description}
%
%
% \paragraph{Bundle.} All the packages of the bundle `oberdiek'
% are also available in a TDS compliant ZIP archive. There
% the packages are already unpacked and the documentation files
% are generated. The files and directories obey the TDS standard.
% \begin{description}
% \item[\CTAN{install/macros/latex/contrib/oberdiek.tds.zip}]
% \end{description}
% \emph{TDS} refers to the standard ``A Directory Structure
% for \TeX\ Files'' (\CTAN{tds/tds.pdf}). Directories
% with \xfile{texmf} in their name are usually organized this way.
%
% \subsection{Bundle installation}
%
% \paragraph{Unpacking.} Unpack the \xfile{oberdiek.tds.zip} in the
% TDS tree (also known as \xfile{texmf} tree) of your choice.
% Example (linux):
% \begin{quote}
%   |unzip oberdiek.tds.zip -d ~/texmf|
% \end{quote}
%
% \paragraph{Script installation.}
% Check the directory \xfile{TDS:scripts/oberdiek/} for
% scripts that need further installation steps.
% Package \xpackage{attachfile2} comes with the Perl script
% \xfile{pdfatfi.pl} that should be installed in such a way
% that it can be called as \texttt{pdfatfi}.
% Example (linux):
% \begin{quote}
%   |chmod +x scripts/oberdiek/pdfatfi.pl|\\
%   |cp scripts/oberdiek/pdfatfi.pl /usr/local/bin/|
% \end{quote}
%
% \subsection{Package installation}
%
% \paragraph{Unpacking.} The \xfile{.dtx} file is a self-extracting
% \docstrip\ archive. The files are extracted by running the
% \xfile{.dtx} through \plainTeX:
% \begin{quote}
%   \verb|tex ifpdf.dtx|
% \end{quote}
%
% \paragraph{TDS.} Now the different files must be moved into
% the different directories in your installation TDS tree
% (also known as \xfile{texmf} tree):
% \begin{quote}
% \def\t{^^A
% \begin{tabular}{@{}>{\ttfamily}l@{ $\rightarrow$ }>{\ttfamily}l@{}}
%   ifpdf.sty & tex/generic/oberdiek/ifpdf.sty\\
%   ifpdf.pdf & doc/latex/oberdiek/ifpdf.pdf\\
%   test/ifpdf-test1.tex & doc/latex/oberdiek/test/ifpdf-test1.tex\\
%   ifpdf.dtx & source/latex/oberdiek/ifpdf.dtx\\
% \end{tabular}^^A
% }^^A
% \sbox0{\t}^^A
% \ifdim\wd0>\linewidth
%   \begingroup
%     \advance\linewidth by\leftmargin
%     \advance\linewidth by\rightmargin
%   \edef\x{\endgroup
%     \def\noexpand\lw{\the\linewidth}^^A
%   }\x
%   \def\lwbox{^^A
%     \leavevmode
%     \hbox to \linewidth{^^A
%       \kern-\leftmargin\relax
%       \hss
%       \usebox0
%       \hss
%       \kern-\rightmargin\relax
%     }^^A
%   }^^A
%   \ifdim\wd0>\lw
%     \sbox0{\small\t}^^A
%     \ifdim\wd0>\linewidth
%       \ifdim\wd0>\lw
%         \sbox0{\footnotesize\t}^^A
%         \ifdim\wd0>\linewidth
%           \ifdim\wd0>\lw
%             \sbox0{\scriptsize\t}^^A
%             \ifdim\wd0>\linewidth
%               \ifdim\wd0>\lw
%                 \sbox0{\tiny\t}^^A
%                 \ifdim\wd0>\linewidth
%                   \lwbox
%                 \else
%                   \usebox0
%                 \fi
%               \else
%                 \lwbox
%               \fi
%             \else
%               \usebox0
%             \fi
%           \else
%             \lwbox
%           \fi
%         \else
%           \usebox0
%         \fi
%       \else
%         \lwbox
%       \fi
%     \else
%       \usebox0
%     \fi
%   \else
%     \lwbox
%   \fi
% \else
%   \usebox0
% \fi
% \end{quote}
% If you have a \xfile{docstrip.cfg} that configures and enables \docstrip's
% TDS installing feature, then some files can already be in the right
% place, see the documentation of \docstrip.
%
% \subsection{Refresh file name databases}
%
% If your \TeX~distribution
% (\teTeX, \mikTeX, \dots) relies on file name databases, you must refresh
% these. For example, \teTeX\ users run \verb|texhash| or
% \verb|mktexlsr|.
%
% \subsection{Some details for the interested}
%
% \paragraph{Attached source.}
%
% The PDF documentation on CTAN also includes the
% \xfile{.dtx} source file. It can be extracted by
% AcrobatReader 6 or higher. Another option is \textsf{pdftk},
% e.g. unpack the file into the current directory:
% \begin{quote}
%   \verb|pdftk ifpdf.pdf unpack_files output .|
% \end{quote}
%
% \paragraph{Unpacking with \LaTeX.}
% The \xfile{.dtx} chooses its action depending on the format:
% \begin{description}
% \item[\plainTeX:] Run \docstrip\ and extract the files.
% \item[\LaTeX:] Generate the documentation.
% \end{description}
% If you insist on using \LaTeX\ for \docstrip\ (really,
% \docstrip\ does not need \LaTeX), then inform the autodetect routine
% about your intention:
% \begin{quote}
%   \verb|latex \let\install=y\input{ifpdf.dtx}|
% \end{quote}
% Do not forget to quote the argument according to the demands
% of your shell.
%
% \paragraph{Generating the documentation.}
% You can use both the \xfile{.dtx} or the \xfile{.drv} to generate
% the documentation. The process can be configured by the
% configuration file \xfile{ltxdoc.cfg}. For instance, put this
% line into this file, if you want to have A4 as paper format:
% \begin{quote}
%   \verb|\PassOptionsToClass{a4paper}{article}|
% \end{quote}
% An example follows how to generate the
% documentation with pdf\LaTeX:
% \begin{quote}
%\begin{verbatim}
%pdflatex ifpdf.dtx
%makeindex -s gind.ist ifpdf.idx
%pdflatex ifpdf.dtx
%makeindex -s gind.ist ifpdf.idx
%pdflatex ifpdf.dtx
%\end{verbatim}
% \end{quote}
%
% \section{Catalogue}
%
% The following XML file can be used as source for the
% \href{http://mirror.ctan.org/help/Catalogue/catalogue.html}{\TeX\ Catalogue}.
% The elements \texttt{caption} and \texttt{description} are imported
% from the original XML file from the Catalogue.
% The name of the XML file in the Catalogue is \xfile{ifpdf.xml}.
%    \begin{macrocode}
%<*catalogue>
<?xml version='1.0' encoding='us-ascii'?>
<!DOCTYPE entry SYSTEM 'catalogue.dtd'>
<entry datestamp='$Date$' modifier='$Author$' id='ifpdf'>
  <name>ifpdf</name>
  <caption>Provides the \ifpdf conditional.</caption>
  <authorref id='auth:oberdiek'/>
  <copyright owner='Heiko Oberdiek' year='2001,2005-2011'/>
  <license type='lppl1.3'/>
  <version number='2.3'/>
  <description>
      This package looks for pdfTeX in pdf mode and implements and
      sets the switch <tt>\ifpdf</tt>. The detection is based on the
      value of <tt>\pdfoutput</tt> (which the package will not
      change).  The package works with plain or LaTeX formats.  To use
      it with LaTeX simply <tt>\usepackage{ifpdf}</tt>. Then use
      <tt>\ifpdf ... \else ... \fi</tt>.
    <p/>
    The package is part of the <xref refid='oberdiek'>oberdiek</xref> bundle.
  </description>
  <documentation details='Package documentation'
      href='ctan:/macros/latex/contrib/oberdiek/ifpdf.pdf'/>
  <ctan file='true' path='/macros/latex/contrib/oberdiek/ifpdf.dtx'/>
  <miktex location='oberdiek'/>
  <texlive location='oberdiek'/>
  <install path='/macros/latex/contrib/oberdiek/oberdiek.tds.zip'/>
</entry>
%</catalogue>
%    \end{macrocode}
%
% \begin{History}
%   \begin{Version}{2001/06/14 v1.0}
%   \item
%     First public version.
%   \end{Version}
%   \begin{Version}{2001/07/14 v1.1}
%   \item
%     Documentation addition: global options
%   \end{Version}
%   \begin{Version}{2001/09/26 v1.2}
%   \item
%     Documentation typo corrected.
%   \item
%     Version number corrected.
%   \item
%     Line number in log entry removed.
%   \end{Version}
%   \begin{Version}{2005/07/22 v1.3}
%   \item
%     Some source code comments from Robin Fairbairns added.
%   \item
%     Bug fix for negative values of \cs{pdfoutput}
%     (Oleg Katsitadze)
%   \item
%     LPPL 1.3
%   \item
%     Installation section with locations added.
%   \end{Version}
%   \begin{Version}{2006/02/20 v1.4}
%   \item
%     DTX framework.
%   \item
%     More robust check in case of undefined \cs{pdfoutput}.
%   \item
%     Extended documentation.
%   \end{Version}
%   \begin{Version}{2007/09/09 v1.5}
%   \item
%     Catcode settings added.
%   \end{Version}
%   \begin{Version}{2007/12/12 v1.6}
%   \item
%     Minor update.
%   \end{Version}
%   \begin{Version}{2008/12/12 v1.7}
%   \item
%     Fix in documentation for \cs{boolean} (found by S. Venkataraman).
%   \item
%     Code is not changed.
%   \end{Version}
%   \begin{Version}{2009/04/10 v2.0}
%   \item
%     Support for \LuaTeX\ 0.40 added.
%   \item
%     Checks, whether \cs{pdfoutput} was changed.
%   \end{Version}
%   \begin{Version}{2010/01/28 v2.1}
%   \item
%     Compatibility to ini\TeX\ added.
%   \end{Version}
%   \begin{Version}{2010/09/13 v2.2}
%   \item
%     Code with \cs{escapechar} rewritten because of
%     LuaTeX bug: \cs{escapechar} does not work.
%   \end{Version}
%   \begin{Version}{2011/01/30 v2.3}
%   \item
%     Already loaded package files are not input in \hologo{plainTeX}.
%   \end{Version}
% \end{History}
%
% \PrintIndex
%
% \Finale
\endinput

%        (quote the arguments according to the demands of your shell)
%
% Documentation:
%    (a) If ifpdf.drv is present:
%           latex ifpdf.drv
%    (b) Without ifpdf.drv:
%           latex ifpdf.dtx; ...
%    The class ltxdoc loads the configuration file ltxdoc.cfg
%    if available. Here you can specify further options, e.g.
%    use A4 as paper format:
%       \PassOptionsToClass{a4paper}{article}
%
%    Programm calls to get the documentation (example):
%       pdflatex ifpdf.dtx
%       makeindex -s gind.ist ifpdf.idx
%       pdflatex ifpdf.dtx
%       makeindex -s gind.ist ifpdf.idx
%       pdflatex ifpdf.dtx
%
% Installation:
%    TDS:tex/generic/oberdiek/ifpdf.sty
%    TDS:doc/latex/oberdiek/ifpdf.pdf
%    TDS:doc/latex/oberdiek/test/ifpdf-test1.tex
%    TDS:source/latex/oberdiek/ifpdf.dtx
%
%<*ignore>
\begingroup
  \catcode123=1 %
  \catcode125=2 %
  \def\x{LaTeX2e}%
\expandafter\endgroup
\ifcase 0\ifx\install y1\fi\expandafter
         \ifx\csname processbatchFile\endcsname\relax\else1\fi
         \ifx\fmtname\x\else 1\fi\relax
\else\csname fi\endcsname
%</ignore>
%<*install>
\input docstrip.tex
\Msg{************************************************************************}
\Msg{* Installation}
\Msg{* Package: ifpdf 2011/01/30 v2.3 Provides the ifpdf switch (HO)}
\Msg{************************************************************************}

\keepsilent
\askforoverwritefalse

\let\MetaPrefix\relax
\preamble

This is a generated file.

Project: ifpdf
Version: 2011/01/30 v2.3

Copyright (C) 2001, 2005-2011 by
   Heiko Oberdiek <heiko.oberdiek at googlemail.com>

This work may be distributed and/or modified under the
conditions of the LaTeX Project Public License, either
version 1.3c of this license or (at your option) any later
version. This version of this license is in
   http://www.latex-project.org/lppl/lppl-1-3c.txt
and the latest version of this license is in
   http://www.latex-project.org/lppl.txt
and version 1.3 or later is part of all distributions of
LaTeX version 2005/12/01 or later.

This work has the LPPL maintenance status "maintained".

This Current Maintainer of this work is Heiko Oberdiek.

The Base Interpreter refers to any `TeX-Format',
because some files are installed in TDS:tex/generic//.

This work consists of the main source file ifpdf.dtx
and the derived files
   ifpdf.sty, ifpdf.pdf, ifpdf.ins, ifpdf.drv, ifpdf-test1.tex.

\endpreamble
\let\MetaPrefix\DoubleperCent

\generate{%
  \file{ifpdf.ins}{\from{ifpdf.dtx}{install}}%
  \file{ifpdf.drv}{\from{ifpdf.dtx}{driver}}%
  \usedir{tex/generic/oberdiek}%
  \file{ifpdf.sty}{\from{ifpdf.dtx}{package}}%
  \usedir{doc/latex/oberdiek/test}%
  \file{ifpdf-test1.tex}{\from{ifpdf.dtx}{test1}}%
  \nopreamble
  \nopostamble
  \usedir{source/latex/oberdiek/catalogue}%
  \file{ifpdf.xml}{\from{ifpdf.dtx}{catalogue}}%
}

\catcode32=13\relax% active space
\let =\space%
\Msg{************************************************************************}
\Msg{*}
\Msg{* To finish the installation you have to move the following}
\Msg{* file into a directory searched by TeX:}
\Msg{*}
\Msg{*     ifpdf.sty}
\Msg{*}
\Msg{* To produce the documentation run the file `ifpdf.drv'}
\Msg{* through LaTeX.}
\Msg{*}
\Msg{* Happy TeXing!}
\Msg{*}
\Msg{************************************************************************}

\endbatchfile
%</install>
%<*ignore>
\fi
%</ignore>
%<*driver>
\NeedsTeXFormat{LaTeX2e}
\ProvidesFile{ifpdf.drv}%
  [2011/01/30 v2.3 Provides the ifpdf switch (HO)]%
\documentclass{ltxdoc}
\usepackage{holtxdoc}[2011/11/22]
\begin{document}
  \DocInput{ifpdf.dtx}%
\end{document}
%</driver>
% \fi
%
% \CheckSum{600}
%
% \CharacterTable
%  {Upper-case    \A\B\C\D\E\F\G\H\I\J\K\L\M\N\O\P\Q\R\S\T\U\V\W\X\Y\Z
%   Lower-case    \a\b\c\d\e\f\g\h\i\j\k\l\m\n\o\p\q\r\s\t\u\v\w\x\y\z
%   Digits        \0\1\2\3\4\5\6\7\8\9
%   Exclamation   \!     Double quote  \"     Hash (number) \#
%   Dollar        \$     Percent       \%     Ampersand     \&
%   Acute accent  \'     Left paren    \(     Right paren   \)
%   Asterisk      \*     Plus          \+     Comma         \,
%   Minus         \-     Point         \.     Solidus       \/
%   Colon         \:     Semicolon     \;     Less than     \<
%   Equals        \=     Greater than  \>     Question mark \?
%   Commercial at \@     Left bracket  \[     Backslash     \\
%   Right bracket \]     Circumflex    \^     Underscore    \_
%   Grave accent  \`     Left brace    \{     Vertical bar  \|
%   Right brace   \}     Tilde         \~}
%
% \GetFileInfo{ifpdf.drv}
%
% \title{The \xpackage{ifpdf} package}
% \date{2011/01/30 v2.3}
% \author{Heiko Oberdiek\\\xemail{heiko.oberdiek at googlemail.com}}
%
% \maketitle
%
% \begin{abstract}
% This package looks for \pdfTeX\ in pdf mode and
% implements and sets the switch \cs{ifpdf}.
% The detection is based on \cs{pdfoutput} and
% the package will not change this value.
% It works with plain or \LaTeX\ formats.
% \end{abstract}
%
% \tableofcontents
%
% \hypersetup{bookmarksopenlevel=2}
% \section{Documentation}
%
% \subsection{Introduction}
%
% It is commonly known that H\`an Th\^e Th\`anh's \pdfTeX\ generates
% PDF output directly and many people uses \pdfTeX\ for this purpose.
% However the DVI output was never thrown away. In contrary, he
% new features for typesetting that works in both PDF and DVI mode.
%
% In the meantime many \TeX\ distributions replace the traditional
% \TeX\ binary with \pdfTeX. Then, for example, called as |latex|
% \pdfTeX\ works in DVI mode with the \LaTeX\ format preloaded,
% called as |pdflatex| \pdfTeX\ starts in PDF mode.
%
% Often packages or users want to know, whether the current document
% is typset by \pdfTeX\ in PDF mode, because the different modes
% have different capabilities (color setting, graphics inclusion,
% \dots). For this purpose \pdfTeX's
% \cs{pdfoutput} can be asked.
%
% As regulary reader of \TeX\ newsgroups and mailing lists I could
% observe many problems with this task. Common errors are:
% \begin{itemize}
% \item
%   \pdfTeX\ has \emph{two} modes. Using \pdfTeX\ does not mean
%   that the user always want to have PDF mode. For example, the
%   PostScript support is better in DVI mode in conjunction with a
%   PostScript aware DVI driver (e.g. dvips). Also the additional
%   typesetting features are mode independent and also available
%   in DVI mode.
% \item
%   \LaTeX's \cs{@ifundefined} inherited the side effect from
%   \cs{csname}. Unknown commands are defined with the meaning
%   of \cs{relax}. If it is checked, whether \cs{pdfoutput} is
%   defined, then this should not be forgotten.
% \item
%   Having \cs{pdfoutput} does not automatically mean PDF mode.
%   Also the value of \cs{pdfoutput} must be asked.
% \item
%   \cs{pdfoutput} must not be destroyed in some way. Later
%   code and packages are fooled then and will perhaps make
%   wrong decisions. For example they may drop support for
%   PDF mode, because they do not know that \pdfTeX\ is
%   running at all.
% \end{itemize}
%
% Robin Fairbairns provides an entry for this topic in his
% excellent FAQ (\url{http://www.tex.ac.uk/faq}):
% \href{http://www.tex.ac.uk/cgi-bin/texfaq2html?label=ifpdf}^^A
% {Am I using PDFTeX?}
%
% \subsection{Usage}
%
% The package \xpackage{ifpdf} can be used with both \plainTeX\
% and \LaTeX:
% \begin{description}
% \item[\plainTeX:] |\input ifpdf.sty|
% \item[\LaTeXe:]   |\usepackage{ifpdf}|
% \end{description}
%
% \DescribeMacro{\ifpdf}
% The package provides the switch \cs{ifpdf}:
% \begin{quote}
%   |\ifpdf|\\
%   \hspace*{1.5em}\dots\ do things, if \pdfTeX\ is running in pdf mode \dots\\
%   |\else|\\
%   \hspace*{1.5em}\dots\ other \TeX\ or \pdfTeX\ in dvi mode \dots\\
%   |\fi|
% \end{quote}
% Users of the package \xpackage{ifthen} can use the switch as boolean:
% \begin{quote}
%   |\boolean{pdf}|
% \end{quote}
% The package can also be used to set global
% documentclass options:
% \begin{quote}
%   |\RequirePackage{ifpdf}|\\
%   |\ifpdf|\\
%   \hspace*{1.5em}|\documentclass[pdftex,|\dots|]{|\dots|}|\\
%   |\else|\\
%   \hspace*{1.5em}|\documentclass[|\dots|]{|\dots|}|\\
%   |\fi|
% \end{quote}
%
% \subsection{Specification}
%
% The package have the following properties:
% \begin{itemize}
% \item It asks the setting of \cs{pdfoutput} for detecting
%   \pdfTeX\ in PDF mode.
% \item It never changes \cs{pdfoutput}.
% \item If \cs{pdfoutput} is undefined or has the meaning \cs{relax},
%   but the engine provides the primitive \cs{pdfoutput},
%   then \cs{pdfoutput} is enabled or restored if possible
%   (only \LuaTeX, version 0.36.0 or higher).
% \item It can be used with many formats including
%   \plainTeX\ and \LaTeX.
% \end{itemize}
% The mode detection implements the following algorithm:
% \begin{quote}
% \def\={${:}{=}$}
% \def\>{\null\qquad\ignorespaces}
% \def\comment#1{^^A
%   \textrm{\itshape\% #1}^^A
% }
% \sffamily
% if undefined(\cs{pdfoutput})\\
% \> \cs{ifpdf} \= \cs{iffalse} \comment{\pdfTeX is not running}\\
% else\\
% \> if \cs{pdfoutput} $\leq$ 0\\
% \>\> \cs{ifpdf} \= \cs{iffalse} \comment{\pdfTeX\ in DVI mode}\\
% \> else\\
% \>\> \cs{ifpdf} \= \cs{iftrue} \comment{\pdfTeX\ in PDF mode}\\
% \> fi\\
% fi
% \end{quote}
% The function \textsf{undefined} checks both cases, undefined command
% and \cs{relax}.
%
% \hypersetup{bookmarksopenlevel=1}
%
% \StopEventually{
% }
%
% \section{Implementation}
%
%    \begin{macrocode}
%<*package>
%    \end{macrocode}
%
% \subsection{Reload check and package identification}
%    Reload check, especially if the package is not used with \LaTeX.
%    \begin{macrocode}
\begingroup\catcode61\catcode48\catcode32=10\relax%
  \catcode13=5 % ^^M
  \endlinechar=13 %
  \catcode35=6 % #
  \catcode39=12 % '
  \catcode44=12 % ,
  \catcode45=12 % -
  \catcode46=12 % .
  \catcode58=12 % :
  \catcode64=11 % @
  \catcode123=1 % {
  \catcode125=2 % }
  \expandafter\let\expandafter\x\csname ver@ifpdf.sty\endcsname
  \ifx\x\relax % plain-TeX, first loading
  \else
    \def\empty{}%
    \ifx\x\empty % LaTeX, first loading,
      % variable is initialized, but \ProvidesPackage not yet seen
    \else
      \expandafter\ifx\csname PackageInfo\endcsname\relax
        \def\x#1#2{%
          \immediate\write-1{Package #1 Info: #2.}%
        }%
      \else
        \def\x#1#2{\PackageInfo{#1}{#2, stopped}}%
      \fi
      \x{ifpdf}{The package is already loaded}%
      \aftergroup\endinput
    \fi
  \fi
\endgroup%
%    \end{macrocode}
%    Package identification:
%    \begin{macrocode}
\begingroup\catcode61\catcode48\catcode32=10\relax%
  \catcode13=5 % ^^M
  \endlinechar=13 %
  \catcode35=6 % #
  \catcode39=12 % '
  \catcode40=12 % (
  \catcode41=12 % )
  \catcode44=12 % ,
  \catcode45=12 % -
  \catcode46=12 % .
  \catcode47=12 % /
  \catcode58=12 % :
  \catcode64=11 % @
  \catcode91=12 % [
  \catcode93=12 % ]
  \catcode123=1 % {
  \catcode125=2 % }
  \expandafter\ifx\csname ProvidesPackage\endcsname\relax
    \def\x#1#2#3[#4]{\endgroup
      \immediate\write-1{Package: #3 #4}%
      \xdef#1{#4}%
    }%
  \else
    \def\x#1#2[#3]{\endgroup
      #2[{#3}]%
      \ifx#1\@undefined
        \xdef#1{#3}%
      \fi
      \ifx#1\relax
        \xdef#1{#3}%
      \fi
    }%
  \fi
\expandafter\x\csname ver@ifpdf.sty\endcsname
\ProvidesPackage{ifpdf}%
  [2011/01/30 v2.3 Provides the ifpdf switch (HO)]%
%    \end{macrocode}
%
% \subsection{Catcodes}
%
%    \begin{macrocode}
\begingroup\catcode61\catcode48\catcode32=10\relax%
  \catcode13=5 % ^^M
  \endlinechar=13 %
  \catcode123=1 % {
  \catcode125=2 % }
  \catcode64=11 % @
  \def\x{\endgroup
    \expandafter\edef\csname ifpdf@AtEnd\endcsname{%
      \endlinechar=\the\endlinechar\relax
      \catcode13=\the\catcode13\relax
      \catcode32=\the\catcode32\relax
      \catcode35=\the\catcode35\relax
      \catcode61=\the\catcode61\relax
      \catcode64=\the\catcode64\relax
      \catcode123=\the\catcode123\relax
      \catcode125=\the\catcode125\relax
    }%
  }%
\x\catcode61\catcode48\catcode32=10\relax%
\catcode13=5 % ^^M
\endlinechar=13 %
\catcode35=6 % #
\catcode64=11 % @
\catcode123=1 % {
\catcode125=2 % }
\def\TMP@EnsureCode#1#2{%
  \edef\ifpdf@AtEnd{%
    \ifpdf@AtEnd
    \catcode#1=\the\catcode#1\relax
  }%
  \catcode#1=#2\relax
}
\TMP@EnsureCode{10}{12}% ^^J
\TMP@EnsureCode{39}{12}% '
\TMP@EnsureCode{40}{12}% (
\TMP@EnsureCode{41}{12}% )
\TMP@EnsureCode{44}{12}% ,
\TMP@EnsureCode{45}{12}% -
\TMP@EnsureCode{46}{12}% .
\TMP@EnsureCode{47}{12}% /
\TMP@EnsureCode{58}{12}% :
\TMP@EnsureCode{60}{12}% <
\TMP@EnsureCode{91}{12}% [
\TMP@EnsureCode{93}{12}% ]
\TMP@EnsureCode{94}{7}% ^
\TMP@EnsureCode{96}{12}% `
\edef\ifpdf@AtEnd{\ifpdf@AtEnd\noexpand\endinput}
%    \end{macrocode}
%
% \subsection{Check for previously defined \cs{ifpdf}}
%
%    \begin{macrocode}
\begingroup
  \expandafter\ifx\csname ifpdf\endcsname\relax
  \else
    \edef\i/{\expandafter\string\csname ifpdf\endcsname}%
    \expandafter\ifx\csname PackageError\endcsname\relax
      \def\x#1#2{%
        \edef\z{#2}%
        \expandafter\errhelp\expandafter{\z}%
        \errmessage{Package ifpdf Error: #1}%
      }%
      \def\y{^^J}%
      \newlinechar=10 %
    \else
      \def\x#1#2{%
        \PackageError{ifpdf}{#1}{#2}%
      }%
      \def\y{\MessageBreak}%
    \fi
    \x{Name clash, \i/ is already defined}{%
      Incompatible versions of \i/ can cause problems,\y
      therefore package loading is aborted.%
    }%
    \endgroup
    \expandafter\ifpdf@AtEnd
  \fi%
\endgroup
%    \end{macrocode}
%
% \subsection{\cs{pdfoutput} and \LuaTeX}
%
%    It might happen, that \LuaTeX\ is running, but \cs{pdfoutput}
%    does not exist. In version 0.40 only \cs{directlua}
%    is available at startup time. The enabling Lua function was
%    already added in version 0.36. Thus we can ignore older versions,
%    here \cs{pdfoutput} is available at startup time.
%
%    \begin{macrocode}
\begingroup
  \def\skip#1\relax\begingroup{}%
  \expandafter\ifx\csname pdfoutput\endcsname\relax
  \else
    \expandafter\skip
  \fi
  \expandafter\ifx\csname directlua\endcsname\relax
    \expandafter\skip
  \fi
\endgroup
\begingroup\expandafter\expandafter\expandafter\endgroup
\expandafter\ifx\csname RequirePackage\endcsname\relax
  \def\TMP@RequirePackage#1[#2]{%
    \begingroup\expandafter\expandafter\expandafter\endgroup
    \expandafter\ifx\csname ver@#1.sty\endcsname\relax
      \input #1.sty\relax
    \fi
  }%
  \TMP@RequirePackage{ifluatex}[2009/04/10]%
\else
  \RequirePackage{ifluatex}[2009/04/10]%
\fi
\ifluatex
  \ifnum\luatexversion<36 %
%    \end{macrocode}
%    Unhappily \LuaTeX's \cs{primitive} (derived from
%    pdfTeX's \cs{pdfprimitive}) cannot be used:
%    \begin{quote}
%    |\protected\gdef\pdfoutput{\primitive\pdfoutput}|
%    \end{quote}
%    Setting a value works, but getting fails,
%    because \TeX\ does no longer see it as number. It
%    is unexpandable and breaks numerical contexts.
%    This was fixed in \hologo{pdfTeX} 1.40.10
%    (bugfix \#4289: ``|\primitive\pdfoutput| cannot be queried'').
%    \begin{macrocode}
  \else
    \begingroup
      \directlua{tex.enableprimitives('ifpdf', {'pdfoutput'})}%
      \global\let\pdfoutput\ifpdfpdfoutput
    \endgroup
  \fi
\fi
\relax\begingroup\endgroup
%    \end{macrocode}
%
% \subsection{\cs{ifpdf}}
%
%    \begin{macro}{\ifpdf}
%    Create and set the switch. \cs{newif} initializes the
%    switch with \cs{iffalse}. \cs{newif} is \cs{outer}
%    in \plainTeX.
%    \begin{macrocode}
\begingroup\expandafter\expandafter\expandafter\endgroup
\expandafter\ifx\csname newif\endcsname\relax
  \edef\pdffalse{%
    \let
    \expandafter\noexpand\csname ifpdf\endcsname
    \expandafter\noexpand\csname iffalse\endcsname
  }%
  \edef\pdftrue{%
    \let
    \expandafter\noexpand\csname ifpdf\endcsname
    \expandafter\noexpand\csname iftrue\endcsname
  }%
  \pdffalse
\else
  \csname newif\expandafter\endcsname\csname ifpdf\endcsname
\fi
%    \end{macrocode}
%    Test \cs{pdfoutput}. Is it  defined and different from
%    \cs{relax}? Someone could have used \LaTeX\ internal
%    \cs{@ifundefined}, or something else involving.
%    Notice, \cs{csname} is executed inside a group for the test
%    to cancel the side effect of \cs{csname}.
%    \begin{macrocode}
\begingroup\expandafter\expandafter\expandafter\endgroup
\expandafter\ifx\csname pdfoutput\endcsname\relax
\else
  \ifnum\pdfoutput<1 %
%    \end{macrocode}
%    \cs{pdfoutput}|=0| or negative, so not generating pdf.
%    \begin{macrocode}
  \else
    \pdftrue
  \fi
\fi
%    \end{macrocode}
%    \end{macro}
%
% \subsection{Test for fool attempts}
%
%    \begin{macrocode}
\begingroup
  \expandafter\ifx\csname pdfoutput\endcsname\relax
  \else
    \escapechar=92 %
    \edef\m{\meaning\pdfoutput}%
    \edef\p{\string\pdfoutput}%
    \ifx\m\p
    \else
      \expandafter\ifx\csname PackageWarningNoLine\endcsname\relax
        \def\PackageWarningNoLine#1#2{%
          \immediate\write16{%
            Package `#1' Warning: #2.%
          }%
        }%
      \fi
      \PackageWarningNoLine{ifpdf}{%
        Someone has redefined \string\pdfoutput%
      }%
    \fi
  \fi
\endgroup
%    \end{macrocode}
%
% \subsection{Protocol entry}
%
%     Log comment:
%    \begin{macrocode}
\begingroup
  \expandafter\ifx\csname PackageInfo\endcsname\relax
    \def\x#1#2{%
      \immediate\write-1{Package #1 Info: #2.}%
    }%
  \else
    \let\x\PackageInfo
    \expandafter\let\csname on@line\endcsname\empty
  \fi
  \x{ifpdf}{pdfTeX in PDF mode is \ifpdf\else not \fi detected}%
\endgroup
%    \end{macrocode}
%    \begin{macrocode}
\ifpdf@AtEnd%
%    \end{macrocode}
%    \begin{macrocode}
%</package>
%    \end{macrocode}
%
% \section{Test}
%
% \subsection{Catcode checks for loading}
%
%    \begin{macrocode}
%<*test1>
%    \end{macrocode}
%    \begin{macrocode}
\catcode`\{=1 %
\catcode`\}=2 %
\catcode`\#=6 %
\catcode`\@=11 %
\expandafter\ifx\csname count@\endcsname\relax
  \countdef\count@=255 %
\fi
\expandafter\ifx\csname @gobble\endcsname\relax
  \long\def\@gobble#1{}%
\fi
\expandafter\ifx\csname @firstofone\endcsname\relax
  \long\def\@firstofone#1{#1}%
\fi
\expandafter\ifx\csname loop\endcsname\relax
  \expandafter\@firstofone
\else
  \expandafter\@gobble
\fi
{%
  \def\loop#1\repeat{%
    \def\body{#1}%
    \iterate
  }%
  \def\iterate{%
    \body
      \let\next\iterate
    \else
      \let\next\relax
    \fi
    \next
  }%
  \let\repeat=\fi
}%
\def\RestoreCatcodes{}
\count@=0 %
\loop
  \edef\RestoreCatcodes{%
    \RestoreCatcodes
    \catcode\the\count@=\the\catcode\count@\relax
  }%
\ifnum\count@<255 %
  \advance\count@ 1 %
\repeat

\def\RangeCatcodeInvalid#1#2{%
  \count@=#1\relax
  \loop
    \catcode\count@=15 %
  \ifnum\count@<#2\relax
    \advance\count@ 1 %
  \repeat
}
\def\RangeCatcodeCheck#1#2#3{%
  \count@=#1\relax
  \loop
    \ifnum#3=\catcode\count@
    \else
      \errmessage{%
        Character \the\count@\space
        with wrong catcode \the\catcode\count@\space
        instead of \number#3%
      }%
    \fi
  \ifnum\count@<#2\relax
    \advance\count@ 1 %
  \repeat
}
\def\space{ }
\expandafter\ifx\csname LoadCommand\endcsname\relax
  \def\LoadCommand{\input ifpdf.sty\relax}%
\fi
\def\Test{%
  \RangeCatcodeInvalid{0}{47}%
  \RangeCatcodeInvalid{58}{64}%
  \RangeCatcodeInvalid{91}{96}%
  \RangeCatcodeInvalid{123}{255}%
  \catcode`\@=12 %
  \catcode`\\=0 %
  \catcode`\%=14 %
  \LoadCommand
  \RangeCatcodeCheck{0}{36}{15}%
  \RangeCatcodeCheck{37}{37}{14}%
  \RangeCatcodeCheck{38}{47}{15}%
  \RangeCatcodeCheck{48}{57}{12}%
  \RangeCatcodeCheck{58}{63}{15}%
  \RangeCatcodeCheck{64}{64}{12}%
  \RangeCatcodeCheck{65}{90}{11}%
  \RangeCatcodeCheck{91}{91}{15}%
  \RangeCatcodeCheck{92}{92}{0}%
  \RangeCatcodeCheck{93}{96}{15}%
  \RangeCatcodeCheck{97}{122}{11}%
  \RangeCatcodeCheck{123}{255}{15}%
  \RestoreCatcodes
}
\Test
\csname @@end\endcsname
\end
%    \end{macrocode}
%    \begin{macrocode}
%</test1>
%    \end{macrocode}
%
% \section{Installation}
%
% \subsection{Download}
%
% \paragraph{Package.} This package is available on
% CTAN\footnote{\url{ftp://ftp.ctan.org/tex-archive/}}:
% \begin{description}
% \item[\CTAN{macros/latex/contrib/oberdiek/ifpdf.dtx}] The source file.
% \item[\CTAN{macros/latex/contrib/oberdiek/ifpdf.pdf}] Documentation.
% \end{description}
%
%
% \paragraph{Bundle.} All the packages of the bundle `oberdiek'
% are also available in a TDS compliant ZIP archive. There
% the packages are already unpacked and the documentation files
% are generated. The files and directories obey the TDS standard.
% \begin{description}
% \item[\CTAN{install/macros/latex/contrib/oberdiek.tds.zip}]
% \end{description}
% \emph{TDS} refers to the standard ``A Directory Structure
% for \TeX\ Files'' (\CTAN{tds/tds.pdf}). Directories
% with \xfile{texmf} in their name are usually organized this way.
%
% \subsection{Bundle installation}
%
% \paragraph{Unpacking.} Unpack the \xfile{oberdiek.tds.zip} in the
% TDS tree (also known as \xfile{texmf} tree) of your choice.
% Example (linux):
% \begin{quote}
%   |unzip oberdiek.tds.zip -d ~/texmf|
% \end{quote}
%
% \paragraph{Script installation.}
% Check the directory \xfile{TDS:scripts/oberdiek/} for
% scripts that need further installation steps.
% Package \xpackage{attachfile2} comes with the Perl script
% \xfile{pdfatfi.pl} that should be installed in such a way
% that it can be called as \texttt{pdfatfi}.
% Example (linux):
% \begin{quote}
%   |chmod +x scripts/oberdiek/pdfatfi.pl|\\
%   |cp scripts/oberdiek/pdfatfi.pl /usr/local/bin/|
% \end{quote}
%
% \subsection{Package installation}
%
% \paragraph{Unpacking.} The \xfile{.dtx} file is a self-extracting
% \docstrip\ archive. The files are extracted by running the
% \xfile{.dtx} through \plainTeX:
% \begin{quote}
%   \verb|tex ifpdf.dtx|
% \end{quote}
%
% \paragraph{TDS.} Now the different files must be moved into
% the different directories in your installation TDS tree
% (also known as \xfile{texmf} tree):
% \begin{quote}
% \def\t{^^A
% \begin{tabular}{@{}>{\ttfamily}l@{ $\rightarrow$ }>{\ttfamily}l@{}}
%   ifpdf.sty & tex/generic/oberdiek/ifpdf.sty\\
%   ifpdf.pdf & doc/latex/oberdiek/ifpdf.pdf\\
%   test/ifpdf-test1.tex & doc/latex/oberdiek/test/ifpdf-test1.tex\\
%   ifpdf.dtx & source/latex/oberdiek/ifpdf.dtx\\
% \end{tabular}^^A
% }^^A
% \sbox0{\t}^^A
% \ifdim\wd0>\linewidth
%   \begingroup
%     \advance\linewidth by\leftmargin
%     \advance\linewidth by\rightmargin
%   \edef\x{\endgroup
%     \def\noexpand\lw{\the\linewidth}^^A
%   }\x
%   \def\lwbox{^^A
%     \leavevmode
%     \hbox to \linewidth{^^A
%       \kern-\leftmargin\relax
%       \hss
%       \usebox0
%       \hss
%       \kern-\rightmargin\relax
%     }^^A
%   }^^A
%   \ifdim\wd0>\lw
%     \sbox0{\small\t}^^A
%     \ifdim\wd0>\linewidth
%       \ifdim\wd0>\lw
%         \sbox0{\footnotesize\t}^^A
%         \ifdim\wd0>\linewidth
%           \ifdim\wd0>\lw
%             \sbox0{\scriptsize\t}^^A
%             \ifdim\wd0>\linewidth
%               \ifdim\wd0>\lw
%                 \sbox0{\tiny\t}^^A
%                 \ifdim\wd0>\linewidth
%                   \lwbox
%                 \else
%                   \usebox0
%                 \fi
%               \else
%                 \lwbox
%               \fi
%             \else
%               \usebox0
%             \fi
%           \else
%             \lwbox
%           \fi
%         \else
%           \usebox0
%         \fi
%       \else
%         \lwbox
%       \fi
%     \else
%       \usebox0
%     \fi
%   \else
%     \lwbox
%   \fi
% \else
%   \usebox0
% \fi
% \end{quote}
% If you have a \xfile{docstrip.cfg} that configures and enables \docstrip's
% TDS installing feature, then some files can already be in the right
% place, see the documentation of \docstrip.
%
% \subsection{Refresh file name databases}
%
% If your \TeX~distribution
% (\teTeX, \mikTeX, \dots) relies on file name databases, you must refresh
% these. For example, \teTeX\ users run \verb|texhash| or
% \verb|mktexlsr|.
%
% \subsection{Some details for the interested}
%
% \paragraph{Attached source.}
%
% The PDF documentation on CTAN also includes the
% \xfile{.dtx} source file. It can be extracted by
% AcrobatReader 6 or higher. Another option is \textsf{pdftk},
% e.g. unpack the file into the current directory:
% \begin{quote}
%   \verb|pdftk ifpdf.pdf unpack_files output .|
% \end{quote}
%
% \paragraph{Unpacking with \LaTeX.}
% The \xfile{.dtx} chooses its action depending on the format:
% \begin{description}
% \item[\plainTeX:] Run \docstrip\ and extract the files.
% \item[\LaTeX:] Generate the documentation.
% \end{description}
% If you insist on using \LaTeX\ for \docstrip\ (really,
% \docstrip\ does not need \LaTeX), then inform the autodetect routine
% about your intention:
% \begin{quote}
%   \verb|latex \let\install=y% \iffalse meta-comment
%
% File: ifpdf.dtx
% Version: 2011/01/30 v2.3
% Info: Provides the ifpdf switch
%
% Copyright (C) 2001, 2005-2011 by
%    Heiko Oberdiek <heiko.oberdiek at googlemail.com>
%
% This work may be distributed and/or modified under the
% conditions of the LaTeX Project Public License, either
% version 1.3c of this license or (at your option) any later
% version. This version of this license is in
%    http://www.latex-project.org/lppl/lppl-1-3c.txt
% and the latest version of this license is in
%    http://www.latex-project.org/lppl.txt
% and version 1.3 or later is part of all distributions of
% LaTeX version 2005/12/01 or later.
%
% This work has the LPPL maintenance status "maintained".
%
% This Current Maintainer of this work is Heiko Oberdiek.
%
% The Base Interpreter refers to any `TeX-Format',
% because some files are installed in TDS:tex/generic//.
%
% This work consists of the main source file ifpdf.dtx
% and the derived files
%    ifpdf.sty, ifpdf.pdf, ifpdf.ins, ifpdf.drv, ifpdf-test1.tex.
%
% Distribution:
%    CTAN:macros/latex/contrib/oberdiek/ifpdf.dtx
%    CTAN:macros/latex/contrib/oberdiek/ifpdf.pdf
%
% Unpacking:
%    (a) If ifpdf.ins is present:
%           tex ifpdf.ins
%    (b) Without ifpdf.ins:
%           tex ifpdf.dtx
%    (c) If you insist on using LaTeX
%           latex \let\install=y\input{ifpdf.dtx}
%        (quote the arguments according to the demands of your shell)
%
% Documentation:
%    (a) If ifpdf.drv is present:
%           latex ifpdf.drv
%    (b) Without ifpdf.drv:
%           latex ifpdf.dtx; ...
%    The class ltxdoc loads the configuration file ltxdoc.cfg
%    if available. Here you can specify further options, e.g.
%    use A4 as paper format:
%       \PassOptionsToClass{a4paper}{article}
%
%    Programm calls to get the documentation (example):
%       pdflatex ifpdf.dtx
%       makeindex -s gind.ist ifpdf.idx
%       pdflatex ifpdf.dtx
%       makeindex -s gind.ist ifpdf.idx
%       pdflatex ifpdf.dtx
%
% Installation:
%    TDS:tex/generic/oberdiek/ifpdf.sty
%    TDS:doc/latex/oberdiek/ifpdf.pdf
%    TDS:doc/latex/oberdiek/test/ifpdf-test1.tex
%    TDS:source/latex/oberdiek/ifpdf.dtx
%
%<*ignore>
\begingroup
  \catcode123=1 %
  \catcode125=2 %
  \def\x{LaTeX2e}%
\expandafter\endgroup
\ifcase 0\ifx\install y1\fi\expandafter
         \ifx\csname processbatchFile\endcsname\relax\else1\fi
         \ifx\fmtname\x\else 1\fi\relax
\else\csname fi\endcsname
%</ignore>
%<*install>
\input docstrip.tex
\Msg{************************************************************************}
\Msg{* Installation}
\Msg{* Package: ifpdf 2011/01/30 v2.3 Provides the ifpdf switch (HO)}
\Msg{************************************************************************}

\keepsilent
\askforoverwritefalse

\let\MetaPrefix\relax
\preamble

This is a generated file.

Project: ifpdf
Version: 2011/01/30 v2.3

Copyright (C) 2001, 2005-2011 by
   Heiko Oberdiek <heiko.oberdiek at googlemail.com>

This work may be distributed and/or modified under the
conditions of the LaTeX Project Public License, either
version 1.3c of this license or (at your option) any later
version. This version of this license is in
   http://www.latex-project.org/lppl/lppl-1-3c.txt
and the latest version of this license is in
   http://www.latex-project.org/lppl.txt
and version 1.3 or later is part of all distributions of
LaTeX version 2005/12/01 or later.

This work has the LPPL maintenance status "maintained".

This Current Maintainer of this work is Heiko Oberdiek.

The Base Interpreter refers to any `TeX-Format',
because some files are installed in TDS:tex/generic//.

This work consists of the main source file ifpdf.dtx
and the derived files
   ifpdf.sty, ifpdf.pdf, ifpdf.ins, ifpdf.drv, ifpdf-test1.tex.

\endpreamble
\let\MetaPrefix\DoubleperCent

\generate{%
  \file{ifpdf.ins}{\from{ifpdf.dtx}{install}}%
  \file{ifpdf.drv}{\from{ifpdf.dtx}{driver}}%
  \usedir{tex/generic/oberdiek}%
  \file{ifpdf.sty}{\from{ifpdf.dtx}{package}}%
  \usedir{doc/latex/oberdiek/test}%
  \file{ifpdf-test1.tex}{\from{ifpdf.dtx}{test1}}%
  \nopreamble
  \nopostamble
  \usedir{source/latex/oberdiek/catalogue}%
  \file{ifpdf.xml}{\from{ifpdf.dtx}{catalogue}}%
}

\catcode32=13\relax% active space
\let =\space%
\Msg{************************************************************************}
\Msg{*}
\Msg{* To finish the installation you have to move the following}
\Msg{* file into a directory searched by TeX:}
\Msg{*}
\Msg{*     ifpdf.sty}
\Msg{*}
\Msg{* To produce the documentation run the file `ifpdf.drv'}
\Msg{* through LaTeX.}
\Msg{*}
\Msg{* Happy TeXing!}
\Msg{*}
\Msg{************************************************************************}

\endbatchfile
%</install>
%<*ignore>
\fi
%</ignore>
%<*driver>
\NeedsTeXFormat{LaTeX2e}
\ProvidesFile{ifpdf.drv}%
  [2011/01/30 v2.3 Provides the ifpdf switch (HO)]%
\documentclass{ltxdoc}
\usepackage{holtxdoc}[2011/11/22]
\begin{document}
  \DocInput{ifpdf.dtx}%
\end{document}
%</driver>
% \fi
%
% \CheckSum{600}
%
% \CharacterTable
%  {Upper-case    \A\B\C\D\E\F\G\H\I\J\K\L\M\N\O\P\Q\R\S\T\U\V\W\X\Y\Z
%   Lower-case    \a\b\c\d\e\f\g\h\i\j\k\l\m\n\o\p\q\r\s\t\u\v\w\x\y\z
%   Digits        \0\1\2\3\4\5\6\7\8\9
%   Exclamation   \!     Double quote  \"     Hash (number) \#
%   Dollar        \$     Percent       \%     Ampersand     \&
%   Acute accent  \'     Left paren    \(     Right paren   \)
%   Asterisk      \*     Plus          \+     Comma         \,
%   Minus         \-     Point         \.     Solidus       \/
%   Colon         \:     Semicolon     \;     Less than     \<
%   Equals        \=     Greater than  \>     Question mark \?
%   Commercial at \@     Left bracket  \[     Backslash     \\
%   Right bracket \]     Circumflex    \^     Underscore    \_
%   Grave accent  \`     Left brace    \{     Vertical bar  \|
%   Right brace   \}     Tilde         \~}
%
% \GetFileInfo{ifpdf.drv}
%
% \title{The \xpackage{ifpdf} package}
% \date{2011/01/30 v2.3}
% \author{Heiko Oberdiek\\\xemail{heiko.oberdiek at googlemail.com}}
%
% \maketitle
%
% \begin{abstract}
% This package looks for \pdfTeX\ in pdf mode and
% implements and sets the switch \cs{ifpdf}.
% The detection is based on \cs{pdfoutput} and
% the package will not change this value.
% It works with plain or \LaTeX\ formats.
% \end{abstract}
%
% \tableofcontents
%
% \hypersetup{bookmarksopenlevel=2}
% \section{Documentation}
%
% \subsection{Introduction}
%
% It is commonly known that H\`an Th\^e Th\`anh's \pdfTeX\ generates
% PDF output directly and many people uses \pdfTeX\ for this purpose.
% However the DVI output was never thrown away. In contrary, he
% new features for typesetting that works in both PDF and DVI mode.
%
% In the meantime many \TeX\ distributions replace the traditional
% \TeX\ binary with \pdfTeX. Then, for example, called as |latex|
% \pdfTeX\ works in DVI mode with the \LaTeX\ format preloaded,
% called as |pdflatex| \pdfTeX\ starts in PDF mode.
%
% Often packages or users want to know, whether the current document
% is typset by \pdfTeX\ in PDF mode, because the different modes
% have different capabilities (color setting, graphics inclusion,
% \dots). For this purpose \pdfTeX's
% \cs{pdfoutput} can be asked.
%
% As regulary reader of \TeX\ newsgroups and mailing lists I could
% observe many problems with this task. Common errors are:
% \begin{itemize}
% \item
%   \pdfTeX\ has \emph{two} modes. Using \pdfTeX\ does not mean
%   that the user always want to have PDF mode. For example, the
%   PostScript support is better in DVI mode in conjunction with a
%   PostScript aware DVI driver (e.g. dvips). Also the additional
%   typesetting features are mode independent and also available
%   in DVI mode.
% \item
%   \LaTeX's \cs{@ifundefined} inherited the side effect from
%   \cs{csname}. Unknown commands are defined with the meaning
%   of \cs{relax}. If it is checked, whether \cs{pdfoutput} is
%   defined, then this should not be forgotten.
% \item
%   Having \cs{pdfoutput} does not automatically mean PDF mode.
%   Also the value of \cs{pdfoutput} must be asked.
% \item
%   \cs{pdfoutput} must not be destroyed in some way. Later
%   code and packages are fooled then and will perhaps make
%   wrong decisions. For example they may drop support for
%   PDF mode, because they do not know that \pdfTeX\ is
%   running at all.
% \end{itemize}
%
% Robin Fairbairns provides an entry for this topic in his
% excellent FAQ (\url{http://www.tex.ac.uk/faq}):
% \href{http://www.tex.ac.uk/cgi-bin/texfaq2html?label=ifpdf}^^A
% {Am I using PDFTeX?}
%
% \subsection{Usage}
%
% The package \xpackage{ifpdf} can be used with both \plainTeX\
% and \LaTeX:
% \begin{description}
% \item[\plainTeX:] |\input ifpdf.sty|
% \item[\LaTeXe:]   |\usepackage{ifpdf}|
% \end{description}
%
% \DescribeMacro{\ifpdf}
% The package provides the switch \cs{ifpdf}:
% \begin{quote}
%   |\ifpdf|\\
%   \hspace*{1.5em}\dots\ do things, if \pdfTeX\ is running in pdf mode \dots\\
%   |\else|\\
%   \hspace*{1.5em}\dots\ other \TeX\ or \pdfTeX\ in dvi mode \dots\\
%   |\fi|
% \end{quote}
% Users of the package \xpackage{ifthen} can use the switch as boolean:
% \begin{quote}
%   |\boolean{pdf}|
% \end{quote}
% The package can also be used to set global
% documentclass options:
% \begin{quote}
%   |\RequirePackage{ifpdf}|\\
%   |\ifpdf|\\
%   \hspace*{1.5em}|\documentclass[pdftex,|\dots|]{|\dots|}|\\
%   |\else|\\
%   \hspace*{1.5em}|\documentclass[|\dots|]{|\dots|}|\\
%   |\fi|
% \end{quote}
%
% \subsection{Specification}
%
% The package have the following properties:
% \begin{itemize}
% \item It asks the setting of \cs{pdfoutput} for detecting
%   \pdfTeX\ in PDF mode.
% \item It never changes \cs{pdfoutput}.
% \item If \cs{pdfoutput} is undefined or has the meaning \cs{relax},
%   but the engine provides the primitive \cs{pdfoutput},
%   then \cs{pdfoutput} is enabled or restored if possible
%   (only \LuaTeX, version 0.36.0 or higher).
% \item It can be used with many formats including
%   \plainTeX\ and \LaTeX.
% \end{itemize}
% The mode detection implements the following algorithm:
% \begin{quote}
% \def\={${:}{=}$}
% \def\>{\null\qquad\ignorespaces}
% \def\comment#1{^^A
%   \textrm{\itshape\% #1}^^A
% }
% \sffamily
% if undefined(\cs{pdfoutput})\\
% \> \cs{ifpdf} \= \cs{iffalse} \comment{\pdfTeX is not running}\\
% else\\
% \> if \cs{pdfoutput} $\leq$ 0\\
% \>\> \cs{ifpdf} \= \cs{iffalse} \comment{\pdfTeX\ in DVI mode}\\
% \> else\\
% \>\> \cs{ifpdf} \= \cs{iftrue} \comment{\pdfTeX\ in PDF mode}\\
% \> fi\\
% fi
% \end{quote}
% The function \textsf{undefined} checks both cases, undefined command
% and \cs{relax}.
%
% \hypersetup{bookmarksopenlevel=1}
%
% \StopEventually{
% }
%
% \section{Implementation}
%
%    \begin{macrocode}
%<*package>
%    \end{macrocode}
%
% \subsection{Reload check and package identification}
%    Reload check, especially if the package is not used with \LaTeX.
%    \begin{macrocode}
\begingroup\catcode61\catcode48\catcode32=10\relax%
  \catcode13=5 % ^^M
  \endlinechar=13 %
  \catcode35=6 % #
  \catcode39=12 % '
  \catcode44=12 % ,
  \catcode45=12 % -
  \catcode46=12 % .
  \catcode58=12 % :
  \catcode64=11 % @
  \catcode123=1 % {
  \catcode125=2 % }
  \expandafter\let\expandafter\x\csname ver@ifpdf.sty\endcsname
  \ifx\x\relax % plain-TeX, first loading
  \else
    \def\empty{}%
    \ifx\x\empty % LaTeX, first loading,
      % variable is initialized, but \ProvidesPackage not yet seen
    \else
      \expandafter\ifx\csname PackageInfo\endcsname\relax
        \def\x#1#2{%
          \immediate\write-1{Package #1 Info: #2.}%
        }%
      \else
        \def\x#1#2{\PackageInfo{#1}{#2, stopped}}%
      \fi
      \x{ifpdf}{The package is already loaded}%
      \aftergroup\endinput
    \fi
  \fi
\endgroup%
%    \end{macrocode}
%    Package identification:
%    \begin{macrocode}
\begingroup\catcode61\catcode48\catcode32=10\relax%
  \catcode13=5 % ^^M
  \endlinechar=13 %
  \catcode35=6 % #
  \catcode39=12 % '
  \catcode40=12 % (
  \catcode41=12 % )
  \catcode44=12 % ,
  \catcode45=12 % -
  \catcode46=12 % .
  \catcode47=12 % /
  \catcode58=12 % :
  \catcode64=11 % @
  \catcode91=12 % [
  \catcode93=12 % ]
  \catcode123=1 % {
  \catcode125=2 % }
  \expandafter\ifx\csname ProvidesPackage\endcsname\relax
    \def\x#1#2#3[#4]{\endgroup
      \immediate\write-1{Package: #3 #4}%
      \xdef#1{#4}%
    }%
  \else
    \def\x#1#2[#3]{\endgroup
      #2[{#3}]%
      \ifx#1\@undefined
        \xdef#1{#3}%
      \fi
      \ifx#1\relax
        \xdef#1{#3}%
      \fi
    }%
  \fi
\expandafter\x\csname ver@ifpdf.sty\endcsname
\ProvidesPackage{ifpdf}%
  [2011/01/30 v2.3 Provides the ifpdf switch (HO)]%
%    \end{macrocode}
%
% \subsection{Catcodes}
%
%    \begin{macrocode}
\begingroup\catcode61\catcode48\catcode32=10\relax%
  \catcode13=5 % ^^M
  \endlinechar=13 %
  \catcode123=1 % {
  \catcode125=2 % }
  \catcode64=11 % @
  \def\x{\endgroup
    \expandafter\edef\csname ifpdf@AtEnd\endcsname{%
      \endlinechar=\the\endlinechar\relax
      \catcode13=\the\catcode13\relax
      \catcode32=\the\catcode32\relax
      \catcode35=\the\catcode35\relax
      \catcode61=\the\catcode61\relax
      \catcode64=\the\catcode64\relax
      \catcode123=\the\catcode123\relax
      \catcode125=\the\catcode125\relax
    }%
  }%
\x\catcode61\catcode48\catcode32=10\relax%
\catcode13=5 % ^^M
\endlinechar=13 %
\catcode35=6 % #
\catcode64=11 % @
\catcode123=1 % {
\catcode125=2 % }
\def\TMP@EnsureCode#1#2{%
  \edef\ifpdf@AtEnd{%
    \ifpdf@AtEnd
    \catcode#1=\the\catcode#1\relax
  }%
  \catcode#1=#2\relax
}
\TMP@EnsureCode{10}{12}% ^^J
\TMP@EnsureCode{39}{12}% '
\TMP@EnsureCode{40}{12}% (
\TMP@EnsureCode{41}{12}% )
\TMP@EnsureCode{44}{12}% ,
\TMP@EnsureCode{45}{12}% -
\TMP@EnsureCode{46}{12}% .
\TMP@EnsureCode{47}{12}% /
\TMP@EnsureCode{58}{12}% :
\TMP@EnsureCode{60}{12}% <
\TMP@EnsureCode{91}{12}% [
\TMP@EnsureCode{93}{12}% ]
\TMP@EnsureCode{94}{7}% ^
\TMP@EnsureCode{96}{12}% `
\edef\ifpdf@AtEnd{\ifpdf@AtEnd\noexpand\endinput}
%    \end{macrocode}
%
% \subsection{Check for previously defined \cs{ifpdf}}
%
%    \begin{macrocode}
\begingroup
  \expandafter\ifx\csname ifpdf\endcsname\relax
  \else
    \edef\i/{\expandafter\string\csname ifpdf\endcsname}%
    \expandafter\ifx\csname PackageError\endcsname\relax
      \def\x#1#2{%
        \edef\z{#2}%
        \expandafter\errhelp\expandafter{\z}%
        \errmessage{Package ifpdf Error: #1}%
      }%
      \def\y{^^J}%
      \newlinechar=10 %
    \else
      \def\x#1#2{%
        \PackageError{ifpdf}{#1}{#2}%
      }%
      \def\y{\MessageBreak}%
    \fi
    \x{Name clash, \i/ is already defined}{%
      Incompatible versions of \i/ can cause problems,\y
      therefore package loading is aborted.%
    }%
    \endgroup
    \expandafter\ifpdf@AtEnd
  \fi%
\endgroup
%    \end{macrocode}
%
% \subsection{\cs{pdfoutput} and \LuaTeX}
%
%    It might happen, that \LuaTeX\ is running, but \cs{pdfoutput}
%    does not exist. In version 0.40 only \cs{directlua}
%    is available at startup time. The enabling Lua function was
%    already added in version 0.36. Thus we can ignore older versions,
%    here \cs{pdfoutput} is available at startup time.
%
%    \begin{macrocode}
\begingroup
  \def\skip#1\relax\begingroup{}%
  \expandafter\ifx\csname pdfoutput\endcsname\relax
  \else
    \expandafter\skip
  \fi
  \expandafter\ifx\csname directlua\endcsname\relax
    \expandafter\skip
  \fi
\endgroup
\begingroup\expandafter\expandafter\expandafter\endgroup
\expandafter\ifx\csname RequirePackage\endcsname\relax
  \def\TMP@RequirePackage#1[#2]{%
    \begingroup\expandafter\expandafter\expandafter\endgroup
    \expandafter\ifx\csname ver@#1.sty\endcsname\relax
      \input #1.sty\relax
    \fi
  }%
  \TMP@RequirePackage{ifluatex}[2009/04/10]%
\else
  \RequirePackage{ifluatex}[2009/04/10]%
\fi
\ifluatex
  \ifnum\luatexversion<36 %
%    \end{macrocode}
%    Unhappily \LuaTeX's \cs{primitive} (derived from
%    pdfTeX's \cs{pdfprimitive}) cannot be used:
%    \begin{quote}
%    |\protected\gdef\pdfoutput{\primitive\pdfoutput}|
%    \end{quote}
%    Setting a value works, but getting fails,
%    because \TeX\ does no longer see it as number. It
%    is unexpandable and breaks numerical contexts.
%    This was fixed in \hologo{pdfTeX} 1.40.10
%    (bugfix \#4289: ``|\primitive\pdfoutput| cannot be queried'').
%    \begin{macrocode}
  \else
    \begingroup
      \directlua{tex.enableprimitives('ifpdf', {'pdfoutput'})}%
      \global\let\pdfoutput\ifpdfpdfoutput
    \endgroup
  \fi
\fi
\relax\begingroup\endgroup
%    \end{macrocode}
%
% \subsection{\cs{ifpdf}}
%
%    \begin{macro}{\ifpdf}
%    Create and set the switch. \cs{newif} initializes the
%    switch with \cs{iffalse}. \cs{newif} is \cs{outer}
%    in \plainTeX.
%    \begin{macrocode}
\begingroup\expandafter\expandafter\expandafter\endgroup
\expandafter\ifx\csname newif\endcsname\relax
  \edef\pdffalse{%
    \let
    \expandafter\noexpand\csname ifpdf\endcsname
    \expandafter\noexpand\csname iffalse\endcsname
  }%
  \edef\pdftrue{%
    \let
    \expandafter\noexpand\csname ifpdf\endcsname
    \expandafter\noexpand\csname iftrue\endcsname
  }%
  \pdffalse
\else
  \csname newif\expandafter\endcsname\csname ifpdf\endcsname
\fi
%    \end{macrocode}
%    Test \cs{pdfoutput}. Is it  defined and different from
%    \cs{relax}? Someone could have used \LaTeX\ internal
%    \cs{@ifundefined}, or something else involving.
%    Notice, \cs{csname} is executed inside a group for the test
%    to cancel the side effect of \cs{csname}.
%    \begin{macrocode}
\begingroup\expandafter\expandafter\expandafter\endgroup
\expandafter\ifx\csname pdfoutput\endcsname\relax
\else
  \ifnum\pdfoutput<1 %
%    \end{macrocode}
%    \cs{pdfoutput}|=0| or negative, so not generating pdf.
%    \begin{macrocode}
  \else
    \pdftrue
  \fi
\fi
%    \end{macrocode}
%    \end{macro}
%
% \subsection{Test for fool attempts}
%
%    \begin{macrocode}
\begingroup
  \expandafter\ifx\csname pdfoutput\endcsname\relax
  \else
    \escapechar=92 %
    \edef\m{\meaning\pdfoutput}%
    \edef\p{\string\pdfoutput}%
    \ifx\m\p
    \else
      \expandafter\ifx\csname PackageWarningNoLine\endcsname\relax
        \def\PackageWarningNoLine#1#2{%
          \immediate\write16{%
            Package `#1' Warning: #2.%
          }%
        }%
      \fi
      \PackageWarningNoLine{ifpdf}{%
        Someone has redefined \string\pdfoutput%
      }%
    \fi
  \fi
\endgroup
%    \end{macrocode}
%
% \subsection{Protocol entry}
%
%     Log comment:
%    \begin{macrocode}
\begingroup
  \expandafter\ifx\csname PackageInfo\endcsname\relax
    \def\x#1#2{%
      \immediate\write-1{Package #1 Info: #2.}%
    }%
  \else
    \let\x\PackageInfo
    \expandafter\let\csname on@line\endcsname\empty
  \fi
  \x{ifpdf}{pdfTeX in PDF mode is \ifpdf\else not \fi detected}%
\endgroup
%    \end{macrocode}
%    \begin{macrocode}
\ifpdf@AtEnd%
%    \end{macrocode}
%    \begin{macrocode}
%</package>
%    \end{macrocode}
%
% \section{Test}
%
% \subsection{Catcode checks for loading}
%
%    \begin{macrocode}
%<*test1>
%    \end{macrocode}
%    \begin{macrocode}
\catcode`\{=1 %
\catcode`\}=2 %
\catcode`\#=6 %
\catcode`\@=11 %
\expandafter\ifx\csname count@\endcsname\relax
  \countdef\count@=255 %
\fi
\expandafter\ifx\csname @gobble\endcsname\relax
  \long\def\@gobble#1{}%
\fi
\expandafter\ifx\csname @firstofone\endcsname\relax
  \long\def\@firstofone#1{#1}%
\fi
\expandafter\ifx\csname loop\endcsname\relax
  \expandafter\@firstofone
\else
  \expandafter\@gobble
\fi
{%
  \def\loop#1\repeat{%
    \def\body{#1}%
    \iterate
  }%
  \def\iterate{%
    \body
      \let\next\iterate
    \else
      \let\next\relax
    \fi
    \next
  }%
  \let\repeat=\fi
}%
\def\RestoreCatcodes{}
\count@=0 %
\loop
  \edef\RestoreCatcodes{%
    \RestoreCatcodes
    \catcode\the\count@=\the\catcode\count@\relax
  }%
\ifnum\count@<255 %
  \advance\count@ 1 %
\repeat

\def\RangeCatcodeInvalid#1#2{%
  \count@=#1\relax
  \loop
    \catcode\count@=15 %
  \ifnum\count@<#2\relax
    \advance\count@ 1 %
  \repeat
}
\def\RangeCatcodeCheck#1#2#3{%
  \count@=#1\relax
  \loop
    \ifnum#3=\catcode\count@
    \else
      \errmessage{%
        Character \the\count@\space
        with wrong catcode \the\catcode\count@\space
        instead of \number#3%
      }%
    \fi
  \ifnum\count@<#2\relax
    \advance\count@ 1 %
  \repeat
}
\def\space{ }
\expandafter\ifx\csname LoadCommand\endcsname\relax
  \def\LoadCommand{\input ifpdf.sty\relax}%
\fi
\def\Test{%
  \RangeCatcodeInvalid{0}{47}%
  \RangeCatcodeInvalid{58}{64}%
  \RangeCatcodeInvalid{91}{96}%
  \RangeCatcodeInvalid{123}{255}%
  \catcode`\@=12 %
  \catcode`\\=0 %
  \catcode`\%=14 %
  \LoadCommand
  \RangeCatcodeCheck{0}{36}{15}%
  \RangeCatcodeCheck{37}{37}{14}%
  \RangeCatcodeCheck{38}{47}{15}%
  \RangeCatcodeCheck{48}{57}{12}%
  \RangeCatcodeCheck{58}{63}{15}%
  \RangeCatcodeCheck{64}{64}{12}%
  \RangeCatcodeCheck{65}{90}{11}%
  \RangeCatcodeCheck{91}{91}{15}%
  \RangeCatcodeCheck{92}{92}{0}%
  \RangeCatcodeCheck{93}{96}{15}%
  \RangeCatcodeCheck{97}{122}{11}%
  \RangeCatcodeCheck{123}{255}{15}%
  \RestoreCatcodes
}
\Test
\csname @@end\endcsname
\end
%    \end{macrocode}
%    \begin{macrocode}
%</test1>
%    \end{macrocode}
%
% \section{Installation}
%
% \subsection{Download}
%
% \paragraph{Package.} This package is available on
% CTAN\footnote{\url{ftp://ftp.ctan.org/tex-archive/}}:
% \begin{description}
% \item[\CTAN{macros/latex/contrib/oberdiek/ifpdf.dtx}] The source file.
% \item[\CTAN{macros/latex/contrib/oberdiek/ifpdf.pdf}] Documentation.
% \end{description}
%
%
% \paragraph{Bundle.} All the packages of the bundle `oberdiek'
% are also available in a TDS compliant ZIP archive. There
% the packages are already unpacked and the documentation files
% are generated. The files and directories obey the TDS standard.
% \begin{description}
% \item[\CTAN{install/macros/latex/contrib/oberdiek.tds.zip}]
% \end{description}
% \emph{TDS} refers to the standard ``A Directory Structure
% for \TeX\ Files'' (\CTAN{tds/tds.pdf}). Directories
% with \xfile{texmf} in their name are usually organized this way.
%
% \subsection{Bundle installation}
%
% \paragraph{Unpacking.} Unpack the \xfile{oberdiek.tds.zip} in the
% TDS tree (also known as \xfile{texmf} tree) of your choice.
% Example (linux):
% \begin{quote}
%   |unzip oberdiek.tds.zip -d ~/texmf|
% \end{quote}
%
% \paragraph{Script installation.}
% Check the directory \xfile{TDS:scripts/oberdiek/} for
% scripts that need further installation steps.
% Package \xpackage{attachfile2} comes with the Perl script
% \xfile{pdfatfi.pl} that should be installed in such a way
% that it can be called as \texttt{pdfatfi}.
% Example (linux):
% \begin{quote}
%   |chmod +x scripts/oberdiek/pdfatfi.pl|\\
%   |cp scripts/oberdiek/pdfatfi.pl /usr/local/bin/|
% \end{quote}
%
% \subsection{Package installation}
%
% \paragraph{Unpacking.} The \xfile{.dtx} file is a self-extracting
% \docstrip\ archive. The files are extracted by running the
% \xfile{.dtx} through \plainTeX:
% \begin{quote}
%   \verb|tex ifpdf.dtx|
% \end{quote}
%
% \paragraph{TDS.} Now the different files must be moved into
% the different directories in your installation TDS tree
% (also known as \xfile{texmf} tree):
% \begin{quote}
% \def\t{^^A
% \begin{tabular}{@{}>{\ttfamily}l@{ $\rightarrow$ }>{\ttfamily}l@{}}
%   ifpdf.sty & tex/generic/oberdiek/ifpdf.sty\\
%   ifpdf.pdf & doc/latex/oberdiek/ifpdf.pdf\\
%   test/ifpdf-test1.tex & doc/latex/oberdiek/test/ifpdf-test1.tex\\
%   ifpdf.dtx & source/latex/oberdiek/ifpdf.dtx\\
% \end{tabular}^^A
% }^^A
% \sbox0{\t}^^A
% \ifdim\wd0>\linewidth
%   \begingroup
%     \advance\linewidth by\leftmargin
%     \advance\linewidth by\rightmargin
%   \edef\x{\endgroup
%     \def\noexpand\lw{\the\linewidth}^^A
%   }\x
%   \def\lwbox{^^A
%     \leavevmode
%     \hbox to \linewidth{^^A
%       \kern-\leftmargin\relax
%       \hss
%       \usebox0
%       \hss
%       \kern-\rightmargin\relax
%     }^^A
%   }^^A
%   \ifdim\wd0>\lw
%     \sbox0{\small\t}^^A
%     \ifdim\wd0>\linewidth
%       \ifdim\wd0>\lw
%         \sbox0{\footnotesize\t}^^A
%         \ifdim\wd0>\linewidth
%           \ifdim\wd0>\lw
%             \sbox0{\scriptsize\t}^^A
%             \ifdim\wd0>\linewidth
%               \ifdim\wd0>\lw
%                 \sbox0{\tiny\t}^^A
%                 \ifdim\wd0>\linewidth
%                   \lwbox
%                 \else
%                   \usebox0
%                 \fi
%               \else
%                 \lwbox
%               \fi
%             \else
%               \usebox0
%             \fi
%           \else
%             \lwbox
%           \fi
%         \else
%           \usebox0
%         \fi
%       \else
%         \lwbox
%       \fi
%     \else
%       \usebox0
%     \fi
%   \else
%     \lwbox
%   \fi
% \else
%   \usebox0
% \fi
% \end{quote}
% If you have a \xfile{docstrip.cfg} that configures and enables \docstrip's
% TDS installing feature, then some files can already be in the right
% place, see the documentation of \docstrip.
%
% \subsection{Refresh file name databases}
%
% If your \TeX~distribution
% (\teTeX, \mikTeX, \dots) relies on file name databases, you must refresh
% these. For example, \teTeX\ users run \verb|texhash| or
% \verb|mktexlsr|.
%
% \subsection{Some details for the interested}
%
% \paragraph{Attached source.}
%
% The PDF documentation on CTAN also includes the
% \xfile{.dtx} source file. It can be extracted by
% AcrobatReader 6 or higher. Another option is \textsf{pdftk},
% e.g. unpack the file into the current directory:
% \begin{quote}
%   \verb|pdftk ifpdf.pdf unpack_files output .|
% \end{quote}
%
% \paragraph{Unpacking with \LaTeX.}
% The \xfile{.dtx} chooses its action depending on the format:
% \begin{description}
% \item[\plainTeX:] Run \docstrip\ and extract the files.
% \item[\LaTeX:] Generate the documentation.
% \end{description}
% If you insist on using \LaTeX\ for \docstrip\ (really,
% \docstrip\ does not need \LaTeX), then inform the autodetect routine
% about your intention:
% \begin{quote}
%   \verb|latex \let\install=y\input{ifpdf.dtx}|
% \end{quote}
% Do not forget to quote the argument according to the demands
% of your shell.
%
% \paragraph{Generating the documentation.}
% You can use both the \xfile{.dtx} or the \xfile{.drv} to generate
% the documentation. The process can be configured by the
% configuration file \xfile{ltxdoc.cfg}. For instance, put this
% line into this file, if you want to have A4 as paper format:
% \begin{quote}
%   \verb|\PassOptionsToClass{a4paper}{article}|
% \end{quote}
% An example follows how to generate the
% documentation with pdf\LaTeX:
% \begin{quote}
%\begin{verbatim}
%pdflatex ifpdf.dtx
%makeindex -s gind.ist ifpdf.idx
%pdflatex ifpdf.dtx
%makeindex -s gind.ist ifpdf.idx
%pdflatex ifpdf.dtx
%\end{verbatim}
% \end{quote}
%
% \section{Catalogue}
%
% The following XML file can be used as source for the
% \href{http://mirror.ctan.org/help/Catalogue/catalogue.html}{\TeX\ Catalogue}.
% The elements \texttt{caption} and \texttt{description} are imported
% from the original XML file from the Catalogue.
% The name of the XML file in the Catalogue is \xfile{ifpdf.xml}.
%    \begin{macrocode}
%<*catalogue>
<?xml version='1.0' encoding='us-ascii'?>
<!DOCTYPE entry SYSTEM 'catalogue.dtd'>
<entry datestamp='$Date$' modifier='$Author$' id='ifpdf'>
  <name>ifpdf</name>
  <caption>Provides the \ifpdf conditional.</caption>
  <authorref id='auth:oberdiek'/>
  <copyright owner='Heiko Oberdiek' year='2001,2005-2011'/>
  <license type='lppl1.3'/>
  <version number='2.3'/>
  <description>
      This package looks for pdfTeX in pdf mode and implements and
      sets the switch <tt>\ifpdf</tt>. The detection is based on the
      value of <tt>\pdfoutput</tt> (which the package will not
      change).  The package works with plain or LaTeX formats.  To use
      it with LaTeX simply <tt>\usepackage{ifpdf}</tt>. Then use
      <tt>\ifpdf ... \else ... \fi</tt>.
    <p/>
    The package is part of the <xref refid='oberdiek'>oberdiek</xref> bundle.
  </description>
  <documentation details='Package documentation'
      href='ctan:/macros/latex/contrib/oberdiek/ifpdf.pdf'/>
  <ctan file='true' path='/macros/latex/contrib/oberdiek/ifpdf.dtx'/>
  <miktex location='oberdiek'/>
  <texlive location='oberdiek'/>
  <install path='/macros/latex/contrib/oberdiek/oberdiek.tds.zip'/>
</entry>
%</catalogue>
%    \end{macrocode}
%
% \begin{History}
%   \begin{Version}{2001/06/14 v1.0}
%   \item
%     First public version.
%   \end{Version}
%   \begin{Version}{2001/07/14 v1.1}
%   \item
%     Documentation addition: global options
%   \end{Version}
%   \begin{Version}{2001/09/26 v1.2}
%   \item
%     Documentation typo corrected.
%   \item
%     Version number corrected.
%   \item
%     Line number in log entry removed.
%   \end{Version}
%   \begin{Version}{2005/07/22 v1.3}
%   \item
%     Some source code comments from Robin Fairbairns added.
%   \item
%     Bug fix for negative values of \cs{pdfoutput}
%     (Oleg Katsitadze)
%   \item
%     LPPL 1.3
%   \item
%     Installation section with locations added.
%   \end{Version}
%   \begin{Version}{2006/02/20 v1.4}
%   \item
%     DTX framework.
%   \item
%     More robust check in case of undefined \cs{pdfoutput}.
%   \item
%     Extended documentation.
%   \end{Version}
%   \begin{Version}{2007/09/09 v1.5}
%   \item
%     Catcode settings added.
%   \end{Version}
%   \begin{Version}{2007/12/12 v1.6}
%   \item
%     Minor update.
%   \end{Version}
%   \begin{Version}{2008/12/12 v1.7}
%   \item
%     Fix in documentation for \cs{boolean} (found by S. Venkataraman).
%   \item
%     Code is not changed.
%   \end{Version}
%   \begin{Version}{2009/04/10 v2.0}
%   \item
%     Support for \LuaTeX\ 0.40 added.
%   \item
%     Checks, whether \cs{pdfoutput} was changed.
%   \end{Version}
%   \begin{Version}{2010/01/28 v2.1}
%   \item
%     Compatibility to ini\TeX\ added.
%   \end{Version}
%   \begin{Version}{2010/09/13 v2.2}
%   \item
%     Code with \cs{escapechar} rewritten because of
%     LuaTeX bug: \cs{escapechar} does not work.
%   \end{Version}
%   \begin{Version}{2011/01/30 v2.3}
%   \item
%     Already loaded package files are not input in \hologo{plainTeX}.
%   \end{Version}
% \end{History}
%
% \PrintIndex
%
% \Finale
\endinput
|
% \end{quote}
% Do not forget to quote the argument according to the demands
% of your shell.
%
% \paragraph{Generating the documentation.}
% You can use both the \xfile{.dtx} or the \xfile{.drv} to generate
% the documentation. The process can be configured by the
% configuration file \xfile{ltxdoc.cfg}. For instance, put this
% line into this file, if you want to have A4 as paper format:
% \begin{quote}
%   \verb|\PassOptionsToClass{a4paper}{article}|
% \end{quote}
% An example follows how to generate the
% documentation with pdf\LaTeX:
% \begin{quote}
%\begin{verbatim}
%pdflatex ifpdf.dtx
%makeindex -s gind.ist ifpdf.idx
%pdflatex ifpdf.dtx
%makeindex -s gind.ist ifpdf.idx
%pdflatex ifpdf.dtx
%\end{verbatim}
% \end{quote}
%
% \section{Catalogue}
%
% The following XML file can be used as source for the
% \href{http://mirror.ctan.org/help/Catalogue/catalogue.html}{\TeX\ Catalogue}.
% The elements \texttt{caption} and \texttt{description} are imported
% from the original XML file from the Catalogue.
% The name of the XML file in the Catalogue is \xfile{ifpdf.xml}.
%    \begin{macrocode}
%<*catalogue>
<?xml version='1.0' encoding='us-ascii'?>
<!DOCTYPE entry SYSTEM 'catalogue.dtd'>
<entry datestamp='$Date$' modifier='$Author$' id='ifpdf'>
  <name>ifpdf</name>
  <caption>Provides the \ifpdf conditional.</caption>
  <authorref id='auth:oberdiek'/>
  <copyright owner='Heiko Oberdiek' year='2001,2005-2011'/>
  <license type='lppl1.3'/>
  <version number='2.3'/>
  <description>
      This package looks for pdfTeX in pdf mode and implements and
      sets the switch <tt>\ifpdf</tt>. The detection is based on the
      value of <tt>\pdfoutput</tt> (which the package will not
      change).  The package works with plain or LaTeX formats.  To use
      it with LaTeX simply <tt>\usepackage{ifpdf}</tt>. Then use
      <tt>\ifpdf ... \else ... \fi</tt>.
    <p/>
    The package is part of the <xref refid='oberdiek'>oberdiek</xref> bundle.
  </description>
  <documentation details='Package documentation'
      href='ctan:/macros/latex/contrib/oberdiek/ifpdf.pdf'/>
  <ctan file='true' path='/macros/latex/contrib/oberdiek/ifpdf.dtx'/>
  <miktex location='oberdiek'/>
  <texlive location='oberdiek'/>
  <install path='/macros/latex/contrib/oberdiek/oberdiek.tds.zip'/>
</entry>
%</catalogue>
%    \end{macrocode}
%
% \begin{History}
%   \begin{Version}{2001/06/14 v1.0}
%   \item
%     First public version.
%   \end{Version}
%   \begin{Version}{2001/07/14 v1.1}
%   \item
%     Documentation addition: global options
%   \end{Version}
%   \begin{Version}{2001/09/26 v1.2}
%   \item
%     Documentation typo corrected.
%   \item
%     Version number corrected.
%   \item
%     Line number in log entry removed.
%   \end{Version}
%   \begin{Version}{2005/07/22 v1.3}
%   \item
%     Some source code comments from Robin Fairbairns added.
%   \item
%     Bug fix for negative values of \cs{pdfoutput}
%     (Oleg Katsitadze)
%   \item
%     LPPL 1.3
%   \item
%     Installation section with locations added.
%   \end{Version}
%   \begin{Version}{2006/02/20 v1.4}
%   \item
%     DTX framework.
%   \item
%     More robust check in case of undefined \cs{pdfoutput}.
%   \item
%     Extended documentation.
%   \end{Version}
%   \begin{Version}{2007/09/09 v1.5}
%   \item
%     Catcode settings added.
%   \end{Version}
%   \begin{Version}{2007/12/12 v1.6}
%   \item
%     Minor update.
%   \end{Version}
%   \begin{Version}{2008/12/12 v1.7}
%   \item
%     Fix in documentation for \cs{boolean} (found by S. Venkataraman).
%   \item
%     Code is not changed.
%   \end{Version}
%   \begin{Version}{2009/04/10 v2.0}
%   \item
%     Support for \LuaTeX\ 0.40 added.
%   \item
%     Checks, whether \cs{pdfoutput} was changed.
%   \end{Version}
%   \begin{Version}{2010/01/28 v2.1}
%   \item
%     Compatibility to ini\TeX\ added.
%   \end{Version}
%   \begin{Version}{2010/09/13 v2.2}
%   \item
%     Code with \cs{escapechar} rewritten because of
%     LuaTeX bug: \cs{escapechar} does not work.
%   \end{Version}
%   \begin{Version}{2011/01/30 v2.3}
%   \item
%     Already loaded package files are not input in \hologo{plainTeX}.
%   \end{Version}
% \end{History}
%
% \PrintIndex
%
% \Finale
\endinput

%        (quote the arguments according to the demands of your shell)
%
% Documentation:
%    (a) If ifpdf.drv is present:
%           latex ifpdf.drv
%    (b) Without ifpdf.drv:
%           latex ifpdf.dtx; ...
%    The class ltxdoc loads the configuration file ltxdoc.cfg
%    if available. Here you can specify further options, e.g.
%    use A4 as paper format:
%       \PassOptionsToClass{a4paper}{article}
%
%    Programm calls to get the documentation (example):
%       pdflatex ifpdf.dtx
%       makeindex -s gind.ist ifpdf.idx
%       pdflatex ifpdf.dtx
%       makeindex -s gind.ist ifpdf.idx
%       pdflatex ifpdf.dtx
%
% Installation:
%    TDS:tex/generic/oberdiek/ifpdf.sty
%    TDS:doc/latex/oberdiek/ifpdf.pdf
%    TDS:doc/latex/oberdiek/test/ifpdf-test1.tex
%    TDS:source/latex/oberdiek/ifpdf.dtx
%
%<*ignore>
\begingroup
  \catcode123=1 %
  \catcode125=2 %
  \def\x{LaTeX2e}%
\expandafter\endgroup
\ifcase 0\ifx\install y1\fi\expandafter
         \ifx\csname processbatchFile\endcsname\relax\else1\fi
         \ifx\fmtname\x\else 1\fi\relax
\else\csname fi\endcsname
%</ignore>
%<*install>
\input docstrip.tex
\Msg{************************************************************************}
\Msg{* Installation}
\Msg{* Package: ifpdf 2011/01/30 v2.3 Provides the ifpdf switch (HO)}
\Msg{************************************************************************}

\keepsilent
\askforoverwritefalse

\let\MetaPrefix\relax
\preamble

This is a generated file.

Project: ifpdf
Version: 2011/01/30 v2.3

Copyright (C) 2001, 2005-2011 by
   Heiko Oberdiek <heiko.oberdiek at googlemail.com>

This work may be distributed and/or modified under the
conditions of the LaTeX Project Public License, either
version 1.3c of this license or (at your option) any later
version. This version of this license is in
   http://www.latex-project.org/lppl/lppl-1-3c.txt
and the latest version of this license is in
   http://www.latex-project.org/lppl.txt
and version 1.3 or later is part of all distributions of
LaTeX version 2005/12/01 or later.

This work has the LPPL maintenance status "maintained".

This Current Maintainer of this work is Heiko Oberdiek.

The Base Interpreter refers to any `TeX-Format',
because some files are installed in TDS:tex/generic//.

This work consists of the main source file ifpdf.dtx
and the derived files
   ifpdf.sty, ifpdf.pdf, ifpdf.ins, ifpdf.drv, ifpdf-test1.tex.

\endpreamble
\let\MetaPrefix\DoubleperCent

\generate{%
  \file{ifpdf.ins}{\from{ifpdf.dtx}{install}}%
  \file{ifpdf.drv}{\from{ifpdf.dtx}{driver}}%
  \usedir{tex/generic/oberdiek}%
  \file{ifpdf.sty}{\from{ifpdf.dtx}{package}}%
  \usedir{doc/latex/oberdiek/test}%
  \file{ifpdf-test1.tex}{\from{ifpdf.dtx}{test1}}%
  \nopreamble
  \nopostamble
  \usedir{source/latex/oberdiek/catalogue}%
  \file{ifpdf.xml}{\from{ifpdf.dtx}{catalogue}}%
}

\catcode32=13\relax% active space
\let =\space%
\Msg{************************************************************************}
\Msg{*}
\Msg{* To finish the installation you have to move the following}
\Msg{* file into a directory searched by TeX:}
\Msg{*}
\Msg{*     ifpdf.sty}
\Msg{*}
\Msg{* To produce the documentation run the file `ifpdf.drv'}
\Msg{* through LaTeX.}
\Msg{*}
\Msg{* Happy TeXing!}
\Msg{*}
\Msg{************************************************************************}

\endbatchfile
%</install>
%<*ignore>
\fi
%</ignore>
%<*driver>
\NeedsTeXFormat{LaTeX2e}
\ProvidesFile{ifpdf.drv}%
  [2011/01/30 v2.3 Provides the ifpdf switch (HO)]%
\documentclass{ltxdoc}
\usepackage{holtxdoc}[2011/11/22]
\begin{document}
  \DocInput{ifpdf.dtx}%
\end{document}
%</driver>
% \fi
%
% \CheckSum{600}
%
% \CharacterTable
%  {Upper-case    \A\B\C\D\E\F\G\H\I\J\K\L\M\N\O\P\Q\R\S\T\U\V\W\X\Y\Z
%   Lower-case    \a\b\c\d\e\f\g\h\i\j\k\l\m\n\o\p\q\r\s\t\u\v\w\x\y\z
%   Digits        \0\1\2\3\4\5\6\7\8\9
%   Exclamation   \!     Double quote  \"     Hash (number) \#
%   Dollar        \$     Percent       \%     Ampersand     \&
%   Acute accent  \'     Left paren    \(     Right paren   \)
%   Asterisk      \*     Plus          \+     Comma         \,
%   Minus         \-     Point         \.     Solidus       \/
%   Colon         \:     Semicolon     \;     Less than     \<
%   Equals        \=     Greater than  \>     Question mark \?
%   Commercial at \@     Left bracket  \[     Backslash     \\
%   Right bracket \]     Circumflex    \^     Underscore    \_
%   Grave accent  \`     Left brace    \{     Vertical bar  \|
%   Right brace   \}     Tilde         \~}
%
% \GetFileInfo{ifpdf.drv}
%
% \title{The \xpackage{ifpdf} package}
% \date{2011/01/30 v2.3}
% \author{Heiko Oberdiek\\\xemail{heiko.oberdiek at googlemail.com}}
%
% \maketitle
%
% \begin{abstract}
% This package looks for \pdfTeX\ in pdf mode and
% implements and sets the switch \cs{ifpdf}.
% The detection is based on \cs{pdfoutput} and
% the package will not change this value.
% It works with plain or \LaTeX\ formats.
% \end{abstract}
%
% \tableofcontents
%
% \hypersetup{bookmarksopenlevel=2}
% \section{Documentation}
%
% \subsection{Introduction}
%
% It is commonly known that H\`an Th\^e Th\`anh's \pdfTeX\ generates
% PDF output directly and many people uses \pdfTeX\ for this purpose.
% However the DVI output was never thrown away. In contrary, he
% new features for typesetting that works in both PDF and DVI mode.
%
% In the meantime many \TeX\ distributions replace the traditional
% \TeX\ binary with \pdfTeX. Then, for example, called as |latex|
% \pdfTeX\ works in DVI mode with the \LaTeX\ format preloaded,
% called as |pdflatex| \pdfTeX\ starts in PDF mode.
%
% Often packages or users want to know, whether the current document
% is typset by \pdfTeX\ in PDF mode, because the different modes
% have different capabilities (color setting, graphics inclusion,
% \dots). For this purpose \pdfTeX's
% \cs{pdfoutput} can be asked.
%
% As regulary reader of \TeX\ newsgroups and mailing lists I could
% observe many problems with this task. Common errors are:
% \begin{itemize}
% \item
%   \pdfTeX\ has \emph{two} modes. Using \pdfTeX\ does not mean
%   that the user always want to have PDF mode. For example, the
%   PostScript support is better in DVI mode in conjunction with a
%   PostScript aware DVI driver (e.g. dvips). Also the additional
%   typesetting features are mode independent and also available
%   in DVI mode.
% \item
%   \LaTeX's \cs{@ifundefined} inherited the side effect from
%   \cs{csname}. Unknown commands are defined with the meaning
%   of \cs{relax}. If it is checked, whether \cs{pdfoutput} is
%   defined, then this should not be forgotten.
% \item
%   Having \cs{pdfoutput} does not automatically mean PDF mode.
%   Also the value of \cs{pdfoutput} must be asked.
% \item
%   \cs{pdfoutput} must not be destroyed in some way. Later
%   code and packages are fooled then and will perhaps make
%   wrong decisions. For example they may drop support for
%   PDF mode, because they do not know that \pdfTeX\ is
%   running at all.
% \end{itemize}
%
% Robin Fairbairns provides an entry for this topic in his
% excellent FAQ (\url{http://www.tex.ac.uk/faq}):
% \href{http://www.tex.ac.uk/cgi-bin/texfaq2html?label=ifpdf}^^A
% {Am I using PDFTeX?}
%
% \subsection{Usage}
%
% The package \xpackage{ifpdf} can be used with both \plainTeX\
% and \LaTeX:
% \begin{description}
% \item[\plainTeX:] |\input ifpdf.sty|
% \item[\LaTeXe:]   |\usepackage{ifpdf}|
% \end{description}
%
% \DescribeMacro{\ifpdf}
% The package provides the switch \cs{ifpdf}:
% \begin{quote}
%   |\ifpdf|\\
%   \hspace*{1.5em}\dots\ do things, if \pdfTeX\ is running in pdf mode \dots\\
%   |\else|\\
%   \hspace*{1.5em}\dots\ other \TeX\ or \pdfTeX\ in dvi mode \dots\\
%   |\fi|
% \end{quote}
% Users of the package \xpackage{ifthen} can use the switch as boolean:
% \begin{quote}
%   |\boolean{pdf}|
% \end{quote}
% The package can also be used to set global
% documentclass options:
% \begin{quote}
%   |\RequirePackage{ifpdf}|\\
%   |\ifpdf|\\
%   \hspace*{1.5em}|\documentclass[pdftex,|\dots|]{|\dots|}|\\
%   |\else|\\
%   \hspace*{1.5em}|\documentclass[|\dots|]{|\dots|}|\\
%   |\fi|
% \end{quote}
%
% \subsection{Specification}
%
% The package have the following properties:
% \begin{itemize}
% \item It asks the setting of \cs{pdfoutput} for detecting
%   \pdfTeX\ in PDF mode.
% \item It never changes \cs{pdfoutput}.
% \item If \cs{pdfoutput} is undefined or has the meaning \cs{relax},
%   but the engine provides the primitive \cs{pdfoutput},
%   then \cs{pdfoutput} is enabled or restored if possible
%   (only \LuaTeX, version 0.36.0 or higher).
% \item It can be used with many formats including
%   \plainTeX\ and \LaTeX.
% \end{itemize}
% The mode detection implements the following algorithm:
% \begin{quote}
% \def\={${:}{=}$}
% \def\>{\null\qquad\ignorespaces}
% \def\comment#1{^^A
%   \textrm{\itshape\% #1}^^A
% }
% \sffamily
% if undefined(\cs{pdfoutput})\\
% \> \cs{ifpdf} \= \cs{iffalse} \comment{\pdfTeX is not running}\\
% else\\
% \> if \cs{pdfoutput} $\leq$ 0\\
% \>\> \cs{ifpdf} \= \cs{iffalse} \comment{\pdfTeX\ in DVI mode}\\
% \> else\\
% \>\> \cs{ifpdf} \= \cs{iftrue} \comment{\pdfTeX\ in PDF mode}\\
% \> fi\\
% fi
% \end{quote}
% The function \textsf{undefined} checks both cases, undefined command
% and \cs{relax}.
%
% \hypersetup{bookmarksopenlevel=1}
%
% \StopEventually{
% }
%
% \section{Implementation}
%
%    \begin{macrocode}
%<*package>
%    \end{macrocode}
%
% \subsection{Reload check and package identification}
%    Reload check, especially if the package is not used with \LaTeX.
%    \begin{macrocode}
\begingroup\catcode61\catcode48\catcode32=10\relax%
  \catcode13=5 % ^^M
  \endlinechar=13 %
  \catcode35=6 % #
  \catcode39=12 % '
  \catcode44=12 % ,
  \catcode45=12 % -
  \catcode46=12 % .
  \catcode58=12 % :
  \catcode64=11 % @
  \catcode123=1 % {
  \catcode125=2 % }
  \expandafter\let\expandafter\x\csname ver@ifpdf.sty\endcsname
  \ifx\x\relax % plain-TeX, first loading
  \else
    \def\empty{}%
    \ifx\x\empty % LaTeX, first loading,
      % variable is initialized, but \ProvidesPackage not yet seen
    \else
      \expandafter\ifx\csname PackageInfo\endcsname\relax
        \def\x#1#2{%
          \immediate\write-1{Package #1 Info: #2.}%
        }%
      \else
        \def\x#1#2{\PackageInfo{#1}{#2, stopped}}%
      \fi
      \x{ifpdf}{The package is already loaded}%
      \aftergroup\endinput
    \fi
  \fi
\endgroup%
%    \end{macrocode}
%    Package identification:
%    \begin{macrocode}
\begingroup\catcode61\catcode48\catcode32=10\relax%
  \catcode13=5 % ^^M
  \endlinechar=13 %
  \catcode35=6 % #
  \catcode39=12 % '
  \catcode40=12 % (
  \catcode41=12 % )
  \catcode44=12 % ,
  \catcode45=12 % -
  \catcode46=12 % .
  \catcode47=12 % /
  \catcode58=12 % :
  \catcode64=11 % @
  \catcode91=12 % [
  \catcode93=12 % ]
  \catcode123=1 % {
  \catcode125=2 % }
  \expandafter\ifx\csname ProvidesPackage\endcsname\relax
    \def\x#1#2#3[#4]{\endgroup
      \immediate\write-1{Package: #3 #4}%
      \xdef#1{#4}%
    }%
  \else
    \def\x#1#2[#3]{\endgroup
      #2[{#3}]%
      \ifx#1\@undefined
        \xdef#1{#3}%
      \fi
      \ifx#1\relax
        \xdef#1{#3}%
      \fi
    }%
  \fi
\expandafter\x\csname ver@ifpdf.sty\endcsname
\ProvidesPackage{ifpdf}%
  [2011/01/30 v2.3 Provides the ifpdf switch (HO)]%
%    \end{macrocode}
%
% \subsection{Catcodes}
%
%    \begin{macrocode}
\begingroup\catcode61\catcode48\catcode32=10\relax%
  \catcode13=5 % ^^M
  \endlinechar=13 %
  \catcode123=1 % {
  \catcode125=2 % }
  \catcode64=11 % @
  \def\x{\endgroup
    \expandafter\edef\csname ifpdf@AtEnd\endcsname{%
      \endlinechar=\the\endlinechar\relax
      \catcode13=\the\catcode13\relax
      \catcode32=\the\catcode32\relax
      \catcode35=\the\catcode35\relax
      \catcode61=\the\catcode61\relax
      \catcode64=\the\catcode64\relax
      \catcode123=\the\catcode123\relax
      \catcode125=\the\catcode125\relax
    }%
  }%
\x\catcode61\catcode48\catcode32=10\relax%
\catcode13=5 % ^^M
\endlinechar=13 %
\catcode35=6 % #
\catcode64=11 % @
\catcode123=1 % {
\catcode125=2 % }
\def\TMP@EnsureCode#1#2{%
  \edef\ifpdf@AtEnd{%
    \ifpdf@AtEnd
    \catcode#1=\the\catcode#1\relax
  }%
  \catcode#1=#2\relax
}
\TMP@EnsureCode{10}{12}% ^^J
\TMP@EnsureCode{39}{12}% '
\TMP@EnsureCode{40}{12}% (
\TMP@EnsureCode{41}{12}% )
\TMP@EnsureCode{44}{12}% ,
\TMP@EnsureCode{45}{12}% -
\TMP@EnsureCode{46}{12}% .
\TMP@EnsureCode{47}{12}% /
\TMP@EnsureCode{58}{12}% :
\TMP@EnsureCode{60}{12}% <
\TMP@EnsureCode{91}{12}% [
\TMP@EnsureCode{93}{12}% ]
\TMP@EnsureCode{94}{7}% ^
\TMP@EnsureCode{96}{12}% `
\edef\ifpdf@AtEnd{\ifpdf@AtEnd\noexpand\endinput}
%    \end{macrocode}
%
% \subsection{Check for previously defined \cs{ifpdf}}
%
%    \begin{macrocode}
\begingroup
  \expandafter\ifx\csname ifpdf\endcsname\relax
  \else
    \edef\i/{\expandafter\string\csname ifpdf\endcsname}%
    \expandafter\ifx\csname PackageError\endcsname\relax
      \def\x#1#2{%
        \edef\z{#2}%
        \expandafter\errhelp\expandafter{\z}%
        \errmessage{Package ifpdf Error: #1}%
      }%
      \def\y{^^J}%
      \newlinechar=10 %
    \else
      \def\x#1#2{%
        \PackageError{ifpdf}{#1}{#2}%
      }%
      \def\y{\MessageBreak}%
    \fi
    \x{Name clash, \i/ is already defined}{%
      Incompatible versions of \i/ can cause problems,\y
      therefore package loading is aborted.%
    }%
    \endgroup
    \expandafter\ifpdf@AtEnd
  \fi%
\endgroup
%    \end{macrocode}
%
% \subsection{\cs{pdfoutput} and \LuaTeX}
%
%    It might happen, that \LuaTeX\ is running, but \cs{pdfoutput}
%    does not exist. In version 0.40 only \cs{directlua}
%    is available at startup time. The enabling Lua function was
%    already added in version 0.36. Thus we can ignore older versions,
%    here \cs{pdfoutput} is available at startup time.
%
%    \begin{macrocode}
\begingroup
  \def\skip#1\relax\begingroup{}%
  \expandafter\ifx\csname pdfoutput\endcsname\relax
  \else
    \expandafter\skip
  \fi
  \expandafter\ifx\csname directlua\endcsname\relax
    \expandafter\skip
  \fi
\endgroup
\begingroup\expandafter\expandafter\expandafter\endgroup
\expandafter\ifx\csname RequirePackage\endcsname\relax
  \def\TMP@RequirePackage#1[#2]{%
    \begingroup\expandafter\expandafter\expandafter\endgroup
    \expandafter\ifx\csname ver@#1.sty\endcsname\relax
      \input #1.sty\relax
    \fi
  }%
  \TMP@RequirePackage{ifluatex}[2009/04/10]%
\else
  \RequirePackage{ifluatex}[2009/04/10]%
\fi
\ifluatex
  \ifnum\luatexversion<36 %
%    \end{macrocode}
%    Unhappily \LuaTeX's \cs{primitive} (derived from
%    pdfTeX's \cs{pdfprimitive}) cannot be used:
%    \begin{quote}
%    |\protected\gdef\pdfoutput{\primitive\pdfoutput}|
%    \end{quote}
%    Setting a value works, but getting fails,
%    because \TeX\ does no longer see it as number. It
%    is unexpandable and breaks numerical contexts.
%    This was fixed in \hologo{pdfTeX} 1.40.10
%    (bugfix \#4289: ``|\primitive\pdfoutput| cannot be queried'').
%    \begin{macrocode}
  \else
    \begingroup
      \directlua{tex.enableprimitives('ifpdf', {'pdfoutput'})}%
      \global\let\pdfoutput\ifpdfpdfoutput
    \endgroup
  \fi
\fi
\relax\begingroup\endgroup
%    \end{macrocode}
%
% \subsection{\cs{ifpdf}}
%
%    \begin{macro}{\ifpdf}
%    Create and set the switch. \cs{newif} initializes the
%    switch with \cs{iffalse}. \cs{newif} is \cs{outer}
%    in \plainTeX.
%    \begin{macrocode}
\begingroup\expandafter\expandafter\expandafter\endgroup
\expandafter\ifx\csname newif\endcsname\relax
  \edef\pdffalse{%
    \let
    \expandafter\noexpand\csname ifpdf\endcsname
    \expandafter\noexpand\csname iffalse\endcsname
  }%
  \edef\pdftrue{%
    \let
    \expandafter\noexpand\csname ifpdf\endcsname
    \expandafter\noexpand\csname iftrue\endcsname
  }%
  \pdffalse
\else
  \csname newif\expandafter\endcsname\csname ifpdf\endcsname
\fi
%    \end{macrocode}
%    Test \cs{pdfoutput}. Is it  defined and different from
%    \cs{relax}? Someone could have used \LaTeX\ internal
%    \cs{@ifundefined}, or something else involving.
%    Notice, \cs{csname} is executed inside a group for the test
%    to cancel the side effect of \cs{csname}.
%    \begin{macrocode}
\begingroup\expandafter\expandafter\expandafter\endgroup
\expandafter\ifx\csname pdfoutput\endcsname\relax
\else
  \ifnum\pdfoutput<1 %
%    \end{macrocode}
%    \cs{pdfoutput}|=0| or negative, so not generating pdf.
%    \begin{macrocode}
  \else
    \pdftrue
  \fi
\fi
%    \end{macrocode}
%    \end{macro}
%
% \subsection{Test for fool attempts}
%
%    \begin{macrocode}
\begingroup
  \expandafter\ifx\csname pdfoutput\endcsname\relax
  \else
    \escapechar=92 %
    \edef\m{\meaning\pdfoutput}%
    \edef\p{\string\pdfoutput}%
    \ifx\m\p
    \else
      \expandafter\ifx\csname PackageWarningNoLine\endcsname\relax
        \def\PackageWarningNoLine#1#2{%
          \immediate\write16{%
            Package `#1' Warning: #2.%
          }%
        }%
      \fi
      \PackageWarningNoLine{ifpdf}{%
        Someone has redefined \string\pdfoutput%
      }%
    \fi
  \fi
\endgroup
%    \end{macrocode}
%
% \subsection{Protocol entry}
%
%     Log comment:
%    \begin{macrocode}
\begingroup
  \expandafter\ifx\csname PackageInfo\endcsname\relax
    \def\x#1#2{%
      \immediate\write-1{Package #1 Info: #2.}%
    }%
  \else
    \let\x\PackageInfo
    \expandafter\let\csname on@line\endcsname\empty
  \fi
  \x{ifpdf}{pdfTeX in PDF mode is \ifpdf\else not \fi detected}%
\endgroup
%    \end{macrocode}
%    \begin{macrocode}
\ifpdf@AtEnd%
%    \end{macrocode}
%    \begin{macrocode}
%</package>
%    \end{macrocode}
%
% \section{Test}
%
% \subsection{Catcode checks for loading}
%
%    \begin{macrocode}
%<*test1>
%    \end{macrocode}
%    \begin{macrocode}
\catcode`\{=1 %
\catcode`\}=2 %
\catcode`\#=6 %
\catcode`\@=11 %
\expandafter\ifx\csname count@\endcsname\relax
  \countdef\count@=255 %
\fi
\expandafter\ifx\csname @gobble\endcsname\relax
  \long\def\@gobble#1{}%
\fi
\expandafter\ifx\csname @firstofone\endcsname\relax
  \long\def\@firstofone#1{#1}%
\fi
\expandafter\ifx\csname loop\endcsname\relax
  \expandafter\@firstofone
\else
  \expandafter\@gobble
\fi
{%
  \def\loop#1\repeat{%
    \def\body{#1}%
    \iterate
  }%
  \def\iterate{%
    \body
      \let\next\iterate
    \else
      \let\next\relax
    \fi
    \next
  }%
  \let\repeat=\fi
}%
\def\RestoreCatcodes{}
\count@=0 %
\loop
  \edef\RestoreCatcodes{%
    \RestoreCatcodes
    \catcode\the\count@=\the\catcode\count@\relax
  }%
\ifnum\count@<255 %
  \advance\count@ 1 %
\repeat

\def\RangeCatcodeInvalid#1#2{%
  \count@=#1\relax
  \loop
    \catcode\count@=15 %
  \ifnum\count@<#2\relax
    \advance\count@ 1 %
  \repeat
}
\def\RangeCatcodeCheck#1#2#3{%
  \count@=#1\relax
  \loop
    \ifnum#3=\catcode\count@
    \else
      \errmessage{%
        Character \the\count@\space
        with wrong catcode \the\catcode\count@\space
        instead of \number#3%
      }%
    \fi
  \ifnum\count@<#2\relax
    \advance\count@ 1 %
  \repeat
}
\def\space{ }
\expandafter\ifx\csname LoadCommand\endcsname\relax
  \def\LoadCommand{\input ifpdf.sty\relax}%
\fi
\def\Test{%
  \RangeCatcodeInvalid{0}{47}%
  \RangeCatcodeInvalid{58}{64}%
  \RangeCatcodeInvalid{91}{96}%
  \RangeCatcodeInvalid{123}{255}%
  \catcode`\@=12 %
  \catcode`\\=0 %
  \catcode`\%=14 %
  \LoadCommand
  \RangeCatcodeCheck{0}{36}{15}%
  \RangeCatcodeCheck{37}{37}{14}%
  \RangeCatcodeCheck{38}{47}{15}%
  \RangeCatcodeCheck{48}{57}{12}%
  \RangeCatcodeCheck{58}{63}{15}%
  \RangeCatcodeCheck{64}{64}{12}%
  \RangeCatcodeCheck{65}{90}{11}%
  \RangeCatcodeCheck{91}{91}{15}%
  \RangeCatcodeCheck{92}{92}{0}%
  \RangeCatcodeCheck{93}{96}{15}%
  \RangeCatcodeCheck{97}{122}{11}%
  \RangeCatcodeCheck{123}{255}{15}%
  \RestoreCatcodes
}
\Test
\csname @@end\endcsname
\end
%    \end{macrocode}
%    \begin{macrocode}
%</test1>
%    \end{macrocode}
%
% \section{Installation}
%
% \subsection{Download}
%
% \paragraph{Package.} This package is available on
% CTAN\footnote{\url{ftp://ftp.ctan.org/tex-archive/}}:
% \begin{description}
% \item[\CTAN{macros/latex/contrib/oberdiek/ifpdf.dtx}] The source file.
% \item[\CTAN{macros/latex/contrib/oberdiek/ifpdf.pdf}] Documentation.
% \end{description}
%
%
% \paragraph{Bundle.} All the packages of the bundle `oberdiek'
% are also available in a TDS compliant ZIP archive. There
% the packages are already unpacked and the documentation files
% are generated. The files and directories obey the TDS standard.
% \begin{description}
% \item[\CTAN{install/macros/latex/contrib/oberdiek.tds.zip}]
% \end{description}
% \emph{TDS} refers to the standard ``A Directory Structure
% for \TeX\ Files'' (\CTAN{tds/tds.pdf}). Directories
% with \xfile{texmf} in their name are usually organized this way.
%
% \subsection{Bundle installation}
%
% \paragraph{Unpacking.} Unpack the \xfile{oberdiek.tds.zip} in the
% TDS tree (also known as \xfile{texmf} tree) of your choice.
% Example (linux):
% \begin{quote}
%   |unzip oberdiek.tds.zip -d ~/texmf|
% \end{quote}
%
% \paragraph{Script installation.}
% Check the directory \xfile{TDS:scripts/oberdiek/} for
% scripts that need further installation steps.
% Package \xpackage{attachfile2} comes with the Perl script
% \xfile{pdfatfi.pl} that should be installed in such a way
% that it can be called as \texttt{pdfatfi}.
% Example (linux):
% \begin{quote}
%   |chmod +x scripts/oberdiek/pdfatfi.pl|\\
%   |cp scripts/oberdiek/pdfatfi.pl /usr/local/bin/|
% \end{quote}
%
% \subsection{Package installation}
%
% \paragraph{Unpacking.} The \xfile{.dtx} file is a self-extracting
% \docstrip\ archive. The files are extracted by running the
% \xfile{.dtx} through \plainTeX:
% \begin{quote}
%   \verb|tex ifpdf.dtx|
% \end{quote}
%
% \paragraph{TDS.} Now the different files must be moved into
% the different directories in your installation TDS tree
% (also known as \xfile{texmf} tree):
% \begin{quote}
% \def\t{^^A
% \begin{tabular}{@{}>{\ttfamily}l@{ $\rightarrow$ }>{\ttfamily}l@{}}
%   ifpdf.sty & tex/generic/oberdiek/ifpdf.sty\\
%   ifpdf.pdf & doc/latex/oberdiek/ifpdf.pdf\\
%   test/ifpdf-test1.tex & doc/latex/oberdiek/test/ifpdf-test1.tex\\
%   ifpdf.dtx & source/latex/oberdiek/ifpdf.dtx\\
% \end{tabular}^^A
% }^^A
% \sbox0{\t}^^A
% \ifdim\wd0>\linewidth
%   \begingroup
%     \advance\linewidth by\leftmargin
%     \advance\linewidth by\rightmargin
%   \edef\x{\endgroup
%     \def\noexpand\lw{\the\linewidth}^^A
%   }\x
%   \def\lwbox{^^A
%     \leavevmode
%     \hbox to \linewidth{^^A
%       \kern-\leftmargin\relax
%       \hss
%       \usebox0
%       \hss
%       \kern-\rightmargin\relax
%     }^^A
%   }^^A
%   \ifdim\wd0>\lw
%     \sbox0{\small\t}^^A
%     \ifdim\wd0>\linewidth
%       \ifdim\wd0>\lw
%         \sbox0{\footnotesize\t}^^A
%         \ifdim\wd0>\linewidth
%           \ifdim\wd0>\lw
%             \sbox0{\scriptsize\t}^^A
%             \ifdim\wd0>\linewidth
%               \ifdim\wd0>\lw
%                 \sbox0{\tiny\t}^^A
%                 \ifdim\wd0>\linewidth
%                   \lwbox
%                 \else
%                   \usebox0
%                 \fi
%               \else
%                 \lwbox
%               \fi
%             \else
%               \usebox0
%             \fi
%           \else
%             \lwbox
%           \fi
%         \else
%           \usebox0
%         \fi
%       \else
%         \lwbox
%       \fi
%     \else
%       \usebox0
%     \fi
%   \else
%     \lwbox
%   \fi
% \else
%   \usebox0
% \fi
% \end{quote}
% If you have a \xfile{docstrip.cfg} that configures and enables \docstrip's
% TDS installing feature, then some files can already be in the right
% place, see the documentation of \docstrip.
%
% \subsection{Refresh file name databases}
%
% If your \TeX~distribution
% (\teTeX, \mikTeX, \dots) relies on file name databases, you must refresh
% these. For example, \teTeX\ users run \verb|texhash| or
% \verb|mktexlsr|.
%
% \subsection{Some details for the interested}
%
% \paragraph{Attached source.}
%
% The PDF documentation on CTAN also includes the
% \xfile{.dtx} source file. It can be extracted by
% AcrobatReader 6 or higher. Another option is \textsf{pdftk},
% e.g. unpack the file into the current directory:
% \begin{quote}
%   \verb|pdftk ifpdf.pdf unpack_files output .|
% \end{quote}
%
% \paragraph{Unpacking with \LaTeX.}
% The \xfile{.dtx} chooses its action depending on the format:
% \begin{description}
% \item[\plainTeX:] Run \docstrip\ and extract the files.
% \item[\LaTeX:] Generate the documentation.
% \end{description}
% If you insist on using \LaTeX\ for \docstrip\ (really,
% \docstrip\ does not need \LaTeX), then inform the autodetect routine
% about your intention:
% \begin{quote}
%   \verb|latex \let\install=y% \iffalse meta-comment
%
% File: ifpdf.dtx
% Version: 2011/01/30 v2.3
% Info: Provides the ifpdf switch
%
% Copyright (C) 2001, 2005-2011 by
%    Heiko Oberdiek <heiko.oberdiek at googlemail.com>
%
% This work may be distributed and/or modified under the
% conditions of the LaTeX Project Public License, either
% version 1.3c of this license or (at your option) any later
% version. This version of this license is in
%    http://www.latex-project.org/lppl/lppl-1-3c.txt
% and the latest version of this license is in
%    http://www.latex-project.org/lppl.txt
% and version 1.3 or later is part of all distributions of
% LaTeX version 2005/12/01 or later.
%
% This work has the LPPL maintenance status "maintained".
%
% This Current Maintainer of this work is Heiko Oberdiek.
%
% The Base Interpreter refers to any `TeX-Format',
% because some files are installed in TDS:tex/generic//.
%
% This work consists of the main source file ifpdf.dtx
% and the derived files
%    ifpdf.sty, ifpdf.pdf, ifpdf.ins, ifpdf.drv, ifpdf-test1.tex.
%
% Distribution:
%    CTAN:macros/latex/contrib/oberdiek/ifpdf.dtx
%    CTAN:macros/latex/contrib/oberdiek/ifpdf.pdf
%
% Unpacking:
%    (a) If ifpdf.ins is present:
%           tex ifpdf.ins
%    (b) Without ifpdf.ins:
%           tex ifpdf.dtx
%    (c) If you insist on using LaTeX
%           latex \let\install=y% \iffalse meta-comment
%
% File: ifpdf.dtx
% Version: 2011/01/30 v2.3
% Info: Provides the ifpdf switch
%
% Copyright (C) 2001, 2005-2011 by
%    Heiko Oberdiek <heiko.oberdiek at googlemail.com>
%
% This work may be distributed and/or modified under the
% conditions of the LaTeX Project Public License, either
% version 1.3c of this license or (at your option) any later
% version. This version of this license is in
%    http://www.latex-project.org/lppl/lppl-1-3c.txt
% and the latest version of this license is in
%    http://www.latex-project.org/lppl.txt
% and version 1.3 or later is part of all distributions of
% LaTeX version 2005/12/01 or later.
%
% This work has the LPPL maintenance status "maintained".
%
% This Current Maintainer of this work is Heiko Oberdiek.
%
% The Base Interpreter refers to any `TeX-Format',
% because some files are installed in TDS:tex/generic//.
%
% This work consists of the main source file ifpdf.dtx
% and the derived files
%    ifpdf.sty, ifpdf.pdf, ifpdf.ins, ifpdf.drv, ifpdf-test1.tex.
%
% Distribution:
%    CTAN:macros/latex/contrib/oberdiek/ifpdf.dtx
%    CTAN:macros/latex/contrib/oberdiek/ifpdf.pdf
%
% Unpacking:
%    (a) If ifpdf.ins is present:
%           tex ifpdf.ins
%    (b) Without ifpdf.ins:
%           tex ifpdf.dtx
%    (c) If you insist on using LaTeX
%           latex \let\install=y\input{ifpdf.dtx}
%        (quote the arguments according to the demands of your shell)
%
% Documentation:
%    (a) If ifpdf.drv is present:
%           latex ifpdf.drv
%    (b) Without ifpdf.drv:
%           latex ifpdf.dtx; ...
%    The class ltxdoc loads the configuration file ltxdoc.cfg
%    if available. Here you can specify further options, e.g.
%    use A4 as paper format:
%       \PassOptionsToClass{a4paper}{article}
%
%    Programm calls to get the documentation (example):
%       pdflatex ifpdf.dtx
%       makeindex -s gind.ist ifpdf.idx
%       pdflatex ifpdf.dtx
%       makeindex -s gind.ist ifpdf.idx
%       pdflatex ifpdf.dtx
%
% Installation:
%    TDS:tex/generic/oberdiek/ifpdf.sty
%    TDS:doc/latex/oberdiek/ifpdf.pdf
%    TDS:doc/latex/oberdiek/test/ifpdf-test1.tex
%    TDS:source/latex/oberdiek/ifpdf.dtx
%
%<*ignore>
\begingroup
  \catcode123=1 %
  \catcode125=2 %
  \def\x{LaTeX2e}%
\expandafter\endgroup
\ifcase 0\ifx\install y1\fi\expandafter
         \ifx\csname processbatchFile\endcsname\relax\else1\fi
         \ifx\fmtname\x\else 1\fi\relax
\else\csname fi\endcsname
%</ignore>
%<*install>
\input docstrip.tex
\Msg{************************************************************************}
\Msg{* Installation}
\Msg{* Package: ifpdf 2011/01/30 v2.3 Provides the ifpdf switch (HO)}
\Msg{************************************************************************}

\keepsilent
\askforoverwritefalse

\let\MetaPrefix\relax
\preamble

This is a generated file.

Project: ifpdf
Version: 2011/01/30 v2.3

Copyright (C) 2001, 2005-2011 by
   Heiko Oberdiek <heiko.oberdiek at googlemail.com>

This work may be distributed and/or modified under the
conditions of the LaTeX Project Public License, either
version 1.3c of this license or (at your option) any later
version. This version of this license is in
   http://www.latex-project.org/lppl/lppl-1-3c.txt
and the latest version of this license is in
   http://www.latex-project.org/lppl.txt
and version 1.3 or later is part of all distributions of
LaTeX version 2005/12/01 or later.

This work has the LPPL maintenance status "maintained".

This Current Maintainer of this work is Heiko Oberdiek.

The Base Interpreter refers to any `TeX-Format',
because some files are installed in TDS:tex/generic//.

This work consists of the main source file ifpdf.dtx
and the derived files
   ifpdf.sty, ifpdf.pdf, ifpdf.ins, ifpdf.drv, ifpdf-test1.tex.

\endpreamble
\let\MetaPrefix\DoubleperCent

\generate{%
  \file{ifpdf.ins}{\from{ifpdf.dtx}{install}}%
  \file{ifpdf.drv}{\from{ifpdf.dtx}{driver}}%
  \usedir{tex/generic/oberdiek}%
  \file{ifpdf.sty}{\from{ifpdf.dtx}{package}}%
  \usedir{doc/latex/oberdiek/test}%
  \file{ifpdf-test1.tex}{\from{ifpdf.dtx}{test1}}%
  \nopreamble
  \nopostamble
  \usedir{source/latex/oberdiek/catalogue}%
  \file{ifpdf.xml}{\from{ifpdf.dtx}{catalogue}}%
}

\catcode32=13\relax% active space
\let =\space%
\Msg{************************************************************************}
\Msg{*}
\Msg{* To finish the installation you have to move the following}
\Msg{* file into a directory searched by TeX:}
\Msg{*}
\Msg{*     ifpdf.sty}
\Msg{*}
\Msg{* To produce the documentation run the file `ifpdf.drv'}
\Msg{* through LaTeX.}
\Msg{*}
\Msg{* Happy TeXing!}
\Msg{*}
\Msg{************************************************************************}

\endbatchfile
%</install>
%<*ignore>
\fi
%</ignore>
%<*driver>
\NeedsTeXFormat{LaTeX2e}
\ProvidesFile{ifpdf.drv}%
  [2011/01/30 v2.3 Provides the ifpdf switch (HO)]%
\documentclass{ltxdoc}
\usepackage{holtxdoc}[2011/11/22]
\begin{document}
  \DocInput{ifpdf.dtx}%
\end{document}
%</driver>
% \fi
%
% \CheckSum{600}
%
% \CharacterTable
%  {Upper-case    \A\B\C\D\E\F\G\H\I\J\K\L\M\N\O\P\Q\R\S\T\U\V\W\X\Y\Z
%   Lower-case    \a\b\c\d\e\f\g\h\i\j\k\l\m\n\o\p\q\r\s\t\u\v\w\x\y\z
%   Digits        \0\1\2\3\4\5\6\7\8\9
%   Exclamation   \!     Double quote  \"     Hash (number) \#
%   Dollar        \$     Percent       \%     Ampersand     \&
%   Acute accent  \'     Left paren    \(     Right paren   \)
%   Asterisk      \*     Plus          \+     Comma         \,
%   Minus         \-     Point         \.     Solidus       \/
%   Colon         \:     Semicolon     \;     Less than     \<
%   Equals        \=     Greater than  \>     Question mark \?
%   Commercial at \@     Left bracket  \[     Backslash     \\
%   Right bracket \]     Circumflex    \^     Underscore    \_
%   Grave accent  \`     Left brace    \{     Vertical bar  \|
%   Right brace   \}     Tilde         \~}
%
% \GetFileInfo{ifpdf.drv}
%
% \title{The \xpackage{ifpdf} package}
% \date{2011/01/30 v2.3}
% \author{Heiko Oberdiek\\\xemail{heiko.oberdiek at googlemail.com}}
%
% \maketitle
%
% \begin{abstract}
% This package looks for \pdfTeX\ in pdf mode and
% implements and sets the switch \cs{ifpdf}.
% The detection is based on \cs{pdfoutput} and
% the package will not change this value.
% It works with plain or \LaTeX\ formats.
% \end{abstract}
%
% \tableofcontents
%
% \hypersetup{bookmarksopenlevel=2}
% \section{Documentation}
%
% \subsection{Introduction}
%
% It is commonly known that H\`an Th\^e Th\`anh's \pdfTeX\ generates
% PDF output directly and many people uses \pdfTeX\ for this purpose.
% However the DVI output was never thrown away. In contrary, he
% new features for typesetting that works in both PDF and DVI mode.
%
% In the meantime many \TeX\ distributions replace the traditional
% \TeX\ binary with \pdfTeX. Then, for example, called as |latex|
% \pdfTeX\ works in DVI mode with the \LaTeX\ format preloaded,
% called as |pdflatex| \pdfTeX\ starts in PDF mode.
%
% Often packages or users want to know, whether the current document
% is typset by \pdfTeX\ in PDF mode, because the different modes
% have different capabilities (color setting, graphics inclusion,
% \dots). For this purpose \pdfTeX's
% \cs{pdfoutput} can be asked.
%
% As regulary reader of \TeX\ newsgroups and mailing lists I could
% observe many problems with this task. Common errors are:
% \begin{itemize}
% \item
%   \pdfTeX\ has \emph{two} modes. Using \pdfTeX\ does not mean
%   that the user always want to have PDF mode. For example, the
%   PostScript support is better in DVI mode in conjunction with a
%   PostScript aware DVI driver (e.g. dvips). Also the additional
%   typesetting features are mode independent and also available
%   in DVI mode.
% \item
%   \LaTeX's \cs{@ifundefined} inherited the side effect from
%   \cs{csname}. Unknown commands are defined with the meaning
%   of \cs{relax}. If it is checked, whether \cs{pdfoutput} is
%   defined, then this should not be forgotten.
% \item
%   Having \cs{pdfoutput} does not automatically mean PDF mode.
%   Also the value of \cs{pdfoutput} must be asked.
% \item
%   \cs{pdfoutput} must not be destroyed in some way. Later
%   code and packages are fooled then and will perhaps make
%   wrong decisions. For example they may drop support for
%   PDF mode, because they do not know that \pdfTeX\ is
%   running at all.
% \end{itemize}
%
% Robin Fairbairns provides an entry for this topic in his
% excellent FAQ (\url{http://www.tex.ac.uk/faq}):
% \href{http://www.tex.ac.uk/cgi-bin/texfaq2html?label=ifpdf}^^A
% {Am I using PDFTeX?}
%
% \subsection{Usage}
%
% The package \xpackage{ifpdf} can be used with both \plainTeX\
% and \LaTeX:
% \begin{description}
% \item[\plainTeX:] |\input ifpdf.sty|
% \item[\LaTeXe:]   |\usepackage{ifpdf}|
% \end{description}
%
% \DescribeMacro{\ifpdf}
% The package provides the switch \cs{ifpdf}:
% \begin{quote}
%   |\ifpdf|\\
%   \hspace*{1.5em}\dots\ do things, if \pdfTeX\ is running in pdf mode \dots\\
%   |\else|\\
%   \hspace*{1.5em}\dots\ other \TeX\ or \pdfTeX\ in dvi mode \dots\\
%   |\fi|
% \end{quote}
% Users of the package \xpackage{ifthen} can use the switch as boolean:
% \begin{quote}
%   |\boolean{pdf}|
% \end{quote}
% The package can also be used to set global
% documentclass options:
% \begin{quote}
%   |\RequirePackage{ifpdf}|\\
%   |\ifpdf|\\
%   \hspace*{1.5em}|\documentclass[pdftex,|\dots|]{|\dots|}|\\
%   |\else|\\
%   \hspace*{1.5em}|\documentclass[|\dots|]{|\dots|}|\\
%   |\fi|
% \end{quote}
%
% \subsection{Specification}
%
% The package have the following properties:
% \begin{itemize}
% \item It asks the setting of \cs{pdfoutput} for detecting
%   \pdfTeX\ in PDF mode.
% \item It never changes \cs{pdfoutput}.
% \item If \cs{pdfoutput} is undefined or has the meaning \cs{relax},
%   but the engine provides the primitive \cs{pdfoutput},
%   then \cs{pdfoutput} is enabled or restored if possible
%   (only \LuaTeX, version 0.36.0 or higher).
% \item It can be used with many formats including
%   \plainTeX\ and \LaTeX.
% \end{itemize}
% The mode detection implements the following algorithm:
% \begin{quote}
% \def\={${:}{=}$}
% \def\>{\null\qquad\ignorespaces}
% \def\comment#1{^^A
%   \textrm{\itshape\% #1}^^A
% }
% \sffamily
% if undefined(\cs{pdfoutput})\\
% \> \cs{ifpdf} \= \cs{iffalse} \comment{\pdfTeX is not running}\\
% else\\
% \> if \cs{pdfoutput} $\leq$ 0\\
% \>\> \cs{ifpdf} \= \cs{iffalse} \comment{\pdfTeX\ in DVI mode}\\
% \> else\\
% \>\> \cs{ifpdf} \= \cs{iftrue} \comment{\pdfTeX\ in PDF mode}\\
% \> fi\\
% fi
% \end{quote}
% The function \textsf{undefined} checks both cases, undefined command
% and \cs{relax}.
%
% \hypersetup{bookmarksopenlevel=1}
%
% \StopEventually{
% }
%
% \section{Implementation}
%
%    \begin{macrocode}
%<*package>
%    \end{macrocode}
%
% \subsection{Reload check and package identification}
%    Reload check, especially if the package is not used with \LaTeX.
%    \begin{macrocode}
\begingroup\catcode61\catcode48\catcode32=10\relax%
  \catcode13=5 % ^^M
  \endlinechar=13 %
  \catcode35=6 % #
  \catcode39=12 % '
  \catcode44=12 % ,
  \catcode45=12 % -
  \catcode46=12 % .
  \catcode58=12 % :
  \catcode64=11 % @
  \catcode123=1 % {
  \catcode125=2 % }
  \expandafter\let\expandafter\x\csname ver@ifpdf.sty\endcsname
  \ifx\x\relax % plain-TeX, first loading
  \else
    \def\empty{}%
    \ifx\x\empty % LaTeX, first loading,
      % variable is initialized, but \ProvidesPackage not yet seen
    \else
      \expandafter\ifx\csname PackageInfo\endcsname\relax
        \def\x#1#2{%
          \immediate\write-1{Package #1 Info: #2.}%
        }%
      \else
        \def\x#1#2{\PackageInfo{#1}{#2, stopped}}%
      \fi
      \x{ifpdf}{The package is already loaded}%
      \aftergroup\endinput
    \fi
  \fi
\endgroup%
%    \end{macrocode}
%    Package identification:
%    \begin{macrocode}
\begingroup\catcode61\catcode48\catcode32=10\relax%
  \catcode13=5 % ^^M
  \endlinechar=13 %
  \catcode35=6 % #
  \catcode39=12 % '
  \catcode40=12 % (
  \catcode41=12 % )
  \catcode44=12 % ,
  \catcode45=12 % -
  \catcode46=12 % .
  \catcode47=12 % /
  \catcode58=12 % :
  \catcode64=11 % @
  \catcode91=12 % [
  \catcode93=12 % ]
  \catcode123=1 % {
  \catcode125=2 % }
  \expandafter\ifx\csname ProvidesPackage\endcsname\relax
    \def\x#1#2#3[#4]{\endgroup
      \immediate\write-1{Package: #3 #4}%
      \xdef#1{#4}%
    }%
  \else
    \def\x#1#2[#3]{\endgroup
      #2[{#3}]%
      \ifx#1\@undefined
        \xdef#1{#3}%
      \fi
      \ifx#1\relax
        \xdef#1{#3}%
      \fi
    }%
  \fi
\expandafter\x\csname ver@ifpdf.sty\endcsname
\ProvidesPackage{ifpdf}%
  [2011/01/30 v2.3 Provides the ifpdf switch (HO)]%
%    \end{macrocode}
%
% \subsection{Catcodes}
%
%    \begin{macrocode}
\begingroup\catcode61\catcode48\catcode32=10\relax%
  \catcode13=5 % ^^M
  \endlinechar=13 %
  \catcode123=1 % {
  \catcode125=2 % }
  \catcode64=11 % @
  \def\x{\endgroup
    \expandafter\edef\csname ifpdf@AtEnd\endcsname{%
      \endlinechar=\the\endlinechar\relax
      \catcode13=\the\catcode13\relax
      \catcode32=\the\catcode32\relax
      \catcode35=\the\catcode35\relax
      \catcode61=\the\catcode61\relax
      \catcode64=\the\catcode64\relax
      \catcode123=\the\catcode123\relax
      \catcode125=\the\catcode125\relax
    }%
  }%
\x\catcode61\catcode48\catcode32=10\relax%
\catcode13=5 % ^^M
\endlinechar=13 %
\catcode35=6 % #
\catcode64=11 % @
\catcode123=1 % {
\catcode125=2 % }
\def\TMP@EnsureCode#1#2{%
  \edef\ifpdf@AtEnd{%
    \ifpdf@AtEnd
    \catcode#1=\the\catcode#1\relax
  }%
  \catcode#1=#2\relax
}
\TMP@EnsureCode{10}{12}% ^^J
\TMP@EnsureCode{39}{12}% '
\TMP@EnsureCode{40}{12}% (
\TMP@EnsureCode{41}{12}% )
\TMP@EnsureCode{44}{12}% ,
\TMP@EnsureCode{45}{12}% -
\TMP@EnsureCode{46}{12}% .
\TMP@EnsureCode{47}{12}% /
\TMP@EnsureCode{58}{12}% :
\TMP@EnsureCode{60}{12}% <
\TMP@EnsureCode{91}{12}% [
\TMP@EnsureCode{93}{12}% ]
\TMP@EnsureCode{94}{7}% ^
\TMP@EnsureCode{96}{12}% `
\edef\ifpdf@AtEnd{\ifpdf@AtEnd\noexpand\endinput}
%    \end{macrocode}
%
% \subsection{Check for previously defined \cs{ifpdf}}
%
%    \begin{macrocode}
\begingroup
  \expandafter\ifx\csname ifpdf\endcsname\relax
  \else
    \edef\i/{\expandafter\string\csname ifpdf\endcsname}%
    \expandafter\ifx\csname PackageError\endcsname\relax
      \def\x#1#2{%
        \edef\z{#2}%
        \expandafter\errhelp\expandafter{\z}%
        \errmessage{Package ifpdf Error: #1}%
      }%
      \def\y{^^J}%
      \newlinechar=10 %
    \else
      \def\x#1#2{%
        \PackageError{ifpdf}{#1}{#2}%
      }%
      \def\y{\MessageBreak}%
    \fi
    \x{Name clash, \i/ is already defined}{%
      Incompatible versions of \i/ can cause problems,\y
      therefore package loading is aborted.%
    }%
    \endgroup
    \expandafter\ifpdf@AtEnd
  \fi%
\endgroup
%    \end{macrocode}
%
% \subsection{\cs{pdfoutput} and \LuaTeX}
%
%    It might happen, that \LuaTeX\ is running, but \cs{pdfoutput}
%    does not exist. In version 0.40 only \cs{directlua}
%    is available at startup time. The enabling Lua function was
%    already added in version 0.36. Thus we can ignore older versions,
%    here \cs{pdfoutput} is available at startup time.
%
%    \begin{macrocode}
\begingroup
  \def\skip#1\relax\begingroup{}%
  \expandafter\ifx\csname pdfoutput\endcsname\relax
  \else
    \expandafter\skip
  \fi
  \expandafter\ifx\csname directlua\endcsname\relax
    \expandafter\skip
  \fi
\endgroup
\begingroup\expandafter\expandafter\expandafter\endgroup
\expandafter\ifx\csname RequirePackage\endcsname\relax
  \def\TMP@RequirePackage#1[#2]{%
    \begingroup\expandafter\expandafter\expandafter\endgroup
    \expandafter\ifx\csname ver@#1.sty\endcsname\relax
      \input #1.sty\relax
    \fi
  }%
  \TMP@RequirePackage{ifluatex}[2009/04/10]%
\else
  \RequirePackage{ifluatex}[2009/04/10]%
\fi
\ifluatex
  \ifnum\luatexversion<36 %
%    \end{macrocode}
%    Unhappily \LuaTeX's \cs{primitive} (derived from
%    pdfTeX's \cs{pdfprimitive}) cannot be used:
%    \begin{quote}
%    |\protected\gdef\pdfoutput{\primitive\pdfoutput}|
%    \end{quote}
%    Setting a value works, but getting fails,
%    because \TeX\ does no longer see it as number. It
%    is unexpandable and breaks numerical contexts.
%    This was fixed in \hologo{pdfTeX} 1.40.10
%    (bugfix \#4289: ``|\primitive\pdfoutput| cannot be queried'').
%    \begin{macrocode}
  \else
    \begingroup
      \directlua{tex.enableprimitives('ifpdf', {'pdfoutput'})}%
      \global\let\pdfoutput\ifpdfpdfoutput
    \endgroup
  \fi
\fi
\relax\begingroup\endgroup
%    \end{macrocode}
%
% \subsection{\cs{ifpdf}}
%
%    \begin{macro}{\ifpdf}
%    Create and set the switch. \cs{newif} initializes the
%    switch with \cs{iffalse}. \cs{newif} is \cs{outer}
%    in \plainTeX.
%    \begin{macrocode}
\begingroup\expandafter\expandafter\expandafter\endgroup
\expandafter\ifx\csname newif\endcsname\relax
  \edef\pdffalse{%
    \let
    \expandafter\noexpand\csname ifpdf\endcsname
    \expandafter\noexpand\csname iffalse\endcsname
  }%
  \edef\pdftrue{%
    \let
    \expandafter\noexpand\csname ifpdf\endcsname
    \expandafter\noexpand\csname iftrue\endcsname
  }%
  \pdffalse
\else
  \csname newif\expandafter\endcsname\csname ifpdf\endcsname
\fi
%    \end{macrocode}
%    Test \cs{pdfoutput}. Is it  defined and different from
%    \cs{relax}? Someone could have used \LaTeX\ internal
%    \cs{@ifundefined}, or something else involving.
%    Notice, \cs{csname} is executed inside a group for the test
%    to cancel the side effect of \cs{csname}.
%    \begin{macrocode}
\begingroup\expandafter\expandafter\expandafter\endgroup
\expandafter\ifx\csname pdfoutput\endcsname\relax
\else
  \ifnum\pdfoutput<1 %
%    \end{macrocode}
%    \cs{pdfoutput}|=0| or negative, so not generating pdf.
%    \begin{macrocode}
  \else
    \pdftrue
  \fi
\fi
%    \end{macrocode}
%    \end{macro}
%
% \subsection{Test for fool attempts}
%
%    \begin{macrocode}
\begingroup
  \expandafter\ifx\csname pdfoutput\endcsname\relax
  \else
    \escapechar=92 %
    \edef\m{\meaning\pdfoutput}%
    \edef\p{\string\pdfoutput}%
    \ifx\m\p
    \else
      \expandafter\ifx\csname PackageWarningNoLine\endcsname\relax
        \def\PackageWarningNoLine#1#2{%
          \immediate\write16{%
            Package `#1' Warning: #2.%
          }%
        }%
      \fi
      \PackageWarningNoLine{ifpdf}{%
        Someone has redefined \string\pdfoutput%
      }%
    \fi
  \fi
\endgroup
%    \end{macrocode}
%
% \subsection{Protocol entry}
%
%     Log comment:
%    \begin{macrocode}
\begingroup
  \expandafter\ifx\csname PackageInfo\endcsname\relax
    \def\x#1#2{%
      \immediate\write-1{Package #1 Info: #2.}%
    }%
  \else
    \let\x\PackageInfo
    \expandafter\let\csname on@line\endcsname\empty
  \fi
  \x{ifpdf}{pdfTeX in PDF mode is \ifpdf\else not \fi detected}%
\endgroup
%    \end{macrocode}
%    \begin{macrocode}
\ifpdf@AtEnd%
%    \end{macrocode}
%    \begin{macrocode}
%</package>
%    \end{macrocode}
%
% \section{Test}
%
% \subsection{Catcode checks for loading}
%
%    \begin{macrocode}
%<*test1>
%    \end{macrocode}
%    \begin{macrocode}
\catcode`\{=1 %
\catcode`\}=2 %
\catcode`\#=6 %
\catcode`\@=11 %
\expandafter\ifx\csname count@\endcsname\relax
  \countdef\count@=255 %
\fi
\expandafter\ifx\csname @gobble\endcsname\relax
  \long\def\@gobble#1{}%
\fi
\expandafter\ifx\csname @firstofone\endcsname\relax
  \long\def\@firstofone#1{#1}%
\fi
\expandafter\ifx\csname loop\endcsname\relax
  \expandafter\@firstofone
\else
  \expandafter\@gobble
\fi
{%
  \def\loop#1\repeat{%
    \def\body{#1}%
    \iterate
  }%
  \def\iterate{%
    \body
      \let\next\iterate
    \else
      \let\next\relax
    \fi
    \next
  }%
  \let\repeat=\fi
}%
\def\RestoreCatcodes{}
\count@=0 %
\loop
  \edef\RestoreCatcodes{%
    \RestoreCatcodes
    \catcode\the\count@=\the\catcode\count@\relax
  }%
\ifnum\count@<255 %
  \advance\count@ 1 %
\repeat

\def\RangeCatcodeInvalid#1#2{%
  \count@=#1\relax
  \loop
    \catcode\count@=15 %
  \ifnum\count@<#2\relax
    \advance\count@ 1 %
  \repeat
}
\def\RangeCatcodeCheck#1#2#3{%
  \count@=#1\relax
  \loop
    \ifnum#3=\catcode\count@
    \else
      \errmessage{%
        Character \the\count@\space
        with wrong catcode \the\catcode\count@\space
        instead of \number#3%
      }%
    \fi
  \ifnum\count@<#2\relax
    \advance\count@ 1 %
  \repeat
}
\def\space{ }
\expandafter\ifx\csname LoadCommand\endcsname\relax
  \def\LoadCommand{\input ifpdf.sty\relax}%
\fi
\def\Test{%
  \RangeCatcodeInvalid{0}{47}%
  \RangeCatcodeInvalid{58}{64}%
  \RangeCatcodeInvalid{91}{96}%
  \RangeCatcodeInvalid{123}{255}%
  \catcode`\@=12 %
  \catcode`\\=0 %
  \catcode`\%=14 %
  \LoadCommand
  \RangeCatcodeCheck{0}{36}{15}%
  \RangeCatcodeCheck{37}{37}{14}%
  \RangeCatcodeCheck{38}{47}{15}%
  \RangeCatcodeCheck{48}{57}{12}%
  \RangeCatcodeCheck{58}{63}{15}%
  \RangeCatcodeCheck{64}{64}{12}%
  \RangeCatcodeCheck{65}{90}{11}%
  \RangeCatcodeCheck{91}{91}{15}%
  \RangeCatcodeCheck{92}{92}{0}%
  \RangeCatcodeCheck{93}{96}{15}%
  \RangeCatcodeCheck{97}{122}{11}%
  \RangeCatcodeCheck{123}{255}{15}%
  \RestoreCatcodes
}
\Test
\csname @@end\endcsname
\end
%    \end{macrocode}
%    \begin{macrocode}
%</test1>
%    \end{macrocode}
%
% \section{Installation}
%
% \subsection{Download}
%
% \paragraph{Package.} This package is available on
% CTAN\footnote{\url{ftp://ftp.ctan.org/tex-archive/}}:
% \begin{description}
% \item[\CTAN{macros/latex/contrib/oberdiek/ifpdf.dtx}] The source file.
% \item[\CTAN{macros/latex/contrib/oberdiek/ifpdf.pdf}] Documentation.
% \end{description}
%
%
% \paragraph{Bundle.} All the packages of the bundle `oberdiek'
% are also available in a TDS compliant ZIP archive. There
% the packages are already unpacked and the documentation files
% are generated. The files and directories obey the TDS standard.
% \begin{description}
% \item[\CTAN{install/macros/latex/contrib/oberdiek.tds.zip}]
% \end{description}
% \emph{TDS} refers to the standard ``A Directory Structure
% for \TeX\ Files'' (\CTAN{tds/tds.pdf}). Directories
% with \xfile{texmf} in their name are usually organized this way.
%
% \subsection{Bundle installation}
%
% \paragraph{Unpacking.} Unpack the \xfile{oberdiek.tds.zip} in the
% TDS tree (also known as \xfile{texmf} tree) of your choice.
% Example (linux):
% \begin{quote}
%   |unzip oberdiek.tds.zip -d ~/texmf|
% \end{quote}
%
% \paragraph{Script installation.}
% Check the directory \xfile{TDS:scripts/oberdiek/} for
% scripts that need further installation steps.
% Package \xpackage{attachfile2} comes with the Perl script
% \xfile{pdfatfi.pl} that should be installed in such a way
% that it can be called as \texttt{pdfatfi}.
% Example (linux):
% \begin{quote}
%   |chmod +x scripts/oberdiek/pdfatfi.pl|\\
%   |cp scripts/oberdiek/pdfatfi.pl /usr/local/bin/|
% \end{quote}
%
% \subsection{Package installation}
%
% \paragraph{Unpacking.} The \xfile{.dtx} file is a self-extracting
% \docstrip\ archive. The files are extracted by running the
% \xfile{.dtx} through \plainTeX:
% \begin{quote}
%   \verb|tex ifpdf.dtx|
% \end{quote}
%
% \paragraph{TDS.} Now the different files must be moved into
% the different directories in your installation TDS tree
% (also known as \xfile{texmf} tree):
% \begin{quote}
% \def\t{^^A
% \begin{tabular}{@{}>{\ttfamily}l@{ $\rightarrow$ }>{\ttfamily}l@{}}
%   ifpdf.sty & tex/generic/oberdiek/ifpdf.sty\\
%   ifpdf.pdf & doc/latex/oberdiek/ifpdf.pdf\\
%   test/ifpdf-test1.tex & doc/latex/oberdiek/test/ifpdf-test1.tex\\
%   ifpdf.dtx & source/latex/oberdiek/ifpdf.dtx\\
% \end{tabular}^^A
% }^^A
% \sbox0{\t}^^A
% \ifdim\wd0>\linewidth
%   \begingroup
%     \advance\linewidth by\leftmargin
%     \advance\linewidth by\rightmargin
%   \edef\x{\endgroup
%     \def\noexpand\lw{\the\linewidth}^^A
%   }\x
%   \def\lwbox{^^A
%     \leavevmode
%     \hbox to \linewidth{^^A
%       \kern-\leftmargin\relax
%       \hss
%       \usebox0
%       \hss
%       \kern-\rightmargin\relax
%     }^^A
%   }^^A
%   \ifdim\wd0>\lw
%     \sbox0{\small\t}^^A
%     \ifdim\wd0>\linewidth
%       \ifdim\wd0>\lw
%         \sbox0{\footnotesize\t}^^A
%         \ifdim\wd0>\linewidth
%           \ifdim\wd0>\lw
%             \sbox0{\scriptsize\t}^^A
%             \ifdim\wd0>\linewidth
%               \ifdim\wd0>\lw
%                 \sbox0{\tiny\t}^^A
%                 \ifdim\wd0>\linewidth
%                   \lwbox
%                 \else
%                   \usebox0
%                 \fi
%               \else
%                 \lwbox
%               \fi
%             \else
%               \usebox0
%             \fi
%           \else
%             \lwbox
%           \fi
%         \else
%           \usebox0
%         \fi
%       \else
%         \lwbox
%       \fi
%     \else
%       \usebox0
%     \fi
%   \else
%     \lwbox
%   \fi
% \else
%   \usebox0
% \fi
% \end{quote}
% If you have a \xfile{docstrip.cfg} that configures and enables \docstrip's
% TDS installing feature, then some files can already be in the right
% place, see the documentation of \docstrip.
%
% \subsection{Refresh file name databases}
%
% If your \TeX~distribution
% (\teTeX, \mikTeX, \dots) relies on file name databases, you must refresh
% these. For example, \teTeX\ users run \verb|texhash| or
% \verb|mktexlsr|.
%
% \subsection{Some details for the interested}
%
% \paragraph{Attached source.}
%
% The PDF documentation on CTAN also includes the
% \xfile{.dtx} source file. It can be extracted by
% AcrobatReader 6 or higher. Another option is \textsf{pdftk},
% e.g. unpack the file into the current directory:
% \begin{quote}
%   \verb|pdftk ifpdf.pdf unpack_files output .|
% \end{quote}
%
% \paragraph{Unpacking with \LaTeX.}
% The \xfile{.dtx} chooses its action depending on the format:
% \begin{description}
% \item[\plainTeX:] Run \docstrip\ and extract the files.
% \item[\LaTeX:] Generate the documentation.
% \end{description}
% If you insist on using \LaTeX\ for \docstrip\ (really,
% \docstrip\ does not need \LaTeX), then inform the autodetect routine
% about your intention:
% \begin{quote}
%   \verb|latex \let\install=y\input{ifpdf.dtx}|
% \end{quote}
% Do not forget to quote the argument according to the demands
% of your shell.
%
% \paragraph{Generating the documentation.}
% You can use both the \xfile{.dtx} or the \xfile{.drv} to generate
% the documentation. The process can be configured by the
% configuration file \xfile{ltxdoc.cfg}. For instance, put this
% line into this file, if you want to have A4 as paper format:
% \begin{quote}
%   \verb|\PassOptionsToClass{a4paper}{article}|
% \end{quote}
% An example follows how to generate the
% documentation with pdf\LaTeX:
% \begin{quote}
%\begin{verbatim}
%pdflatex ifpdf.dtx
%makeindex -s gind.ist ifpdf.idx
%pdflatex ifpdf.dtx
%makeindex -s gind.ist ifpdf.idx
%pdflatex ifpdf.dtx
%\end{verbatim}
% \end{quote}
%
% \section{Catalogue}
%
% The following XML file can be used as source for the
% \href{http://mirror.ctan.org/help/Catalogue/catalogue.html}{\TeX\ Catalogue}.
% The elements \texttt{caption} and \texttt{description} are imported
% from the original XML file from the Catalogue.
% The name of the XML file in the Catalogue is \xfile{ifpdf.xml}.
%    \begin{macrocode}
%<*catalogue>
<?xml version='1.0' encoding='us-ascii'?>
<!DOCTYPE entry SYSTEM 'catalogue.dtd'>
<entry datestamp='$Date$' modifier='$Author$' id='ifpdf'>
  <name>ifpdf</name>
  <caption>Provides the \ifpdf conditional.</caption>
  <authorref id='auth:oberdiek'/>
  <copyright owner='Heiko Oberdiek' year='2001,2005-2011'/>
  <license type='lppl1.3'/>
  <version number='2.3'/>
  <description>
      This package looks for pdfTeX in pdf mode and implements and
      sets the switch <tt>\ifpdf</tt>. The detection is based on the
      value of <tt>\pdfoutput</tt> (which the package will not
      change).  The package works with plain or LaTeX formats.  To use
      it with LaTeX simply <tt>\usepackage{ifpdf}</tt>. Then use
      <tt>\ifpdf ... \else ... \fi</tt>.
    <p/>
    The package is part of the <xref refid='oberdiek'>oberdiek</xref> bundle.
  </description>
  <documentation details='Package documentation'
      href='ctan:/macros/latex/contrib/oberdiek/ifpdf.pdf'/>
  <ctan file='true' path='/macros/latex/contrib/oberdiek/ifpdf.dtx'/>
  <miktex location='oberdiek'/>
  <texlive location='oberdiek'/>
  <install path='/macros/latex/contrib/oberdiek/oberdiek.tds.zip'/>
</entry>
%</catalogue>
%    \end{macrocode}
%
% \begin{History}
%   \begin{Version}{2001/06/14 v1.0}
%   \item
%     First public version.
%   \end{Version}
%   \begin{Version}{2001/07/14 v1.1}
%   \item
%     Documentation addition: global options
%   \end{Version}
%   \begin{Version}{2001/09/26 v1.2}
%   \item
%     Documentation typo corrected.
%   \item
%     Version number corrected.
%   \item
%     Line number in log entry removed.
%   \end{Version}
%   \begin{Version}{2005/07/22 v1.3}
%   \item
%     Some source code comments from Robin Fairbairns added.
%   \item
%     Bug fix for negative values of \cs{pdfoutput}
%     (Oleg Katsitadze)
%   \item
%     LPPL 1.3
%   \item
%     Installation section with locations added.
%   \end{Version}
%   \begin{Version}{2006/02/20 v1.4}
%   \item
%     DTX framework.
%   \item
%     More robust check in case of undefined \cs{pdfoutput}.
%   \item
%     Extended documentation.
%   \end{Version}
%   \begin{Version}{2007/09/09 v1.5}
%   \item
%     Catcode settings added.
%   \end{Version}
%   \begin{Version}{2007/12/12 v1.6}
%   \item
%     Minor update.
%   \end{Version}
%   \begin{Version}{2008/12/12 v1.7}
%   \item
%     Fix in documentation for \cs{boolean} (found by S. Venkataraman).
%   \item
%     Code is not changed.
%   \end{Version}
%   \begin{Version}{2009/04/10 v2.0}
%   \item
%     Support for \LuaTeX\ 0.40 added.
%   \item
%     Checks, whether \cs{pdfoutput} was changed.
%   \end{Version}
%   \begin{Version}{2010/01/28 v2.1}
%   \item
%     Compatibility to ini\TeX\ added.
%   \end{Version}
%   \begin{Version}{2010/09/13 v2.2}
%   \item
%     Code with \cs{escapechar} rewritten because of
%     LuaTeX bug: \cs{escapechar} does not work.
%   \end{Version}
%   \begin{Version}{2011/01/30 v2.3}
%   \item
%     Already loaded package files are not input in \hologo{plainTeX}.
%   \end{Version}
% \end{History}
%
% \PrintIndex
%
% \Finale
\endinput

%        (quote the arguments according to the demands of your shell)
%
% Documentation:
%    (a) If ifpdf.drv is present:
%           latex ifpdf.drv
%    (b) Without ifpdf.drv:
%           latex ifpdf.dtx; ...
%    The class ltxdoc loads the configuration file ltxdoc.cfg
%    if available. Here you can specify further options, e.g.
%    use A4 as paper format:
%       \PassOptionsToClass{a4paper}{article}
%
%    Programm calls to get the documentation (example):
%       pdflatex ifpdf.dtx
%       makeindex -s gind.ist ifpdf.idx
%       pdflatex ifpdf.dtx
%       makeindex -s gind.ist ifpdf.idx
%       pdflatex ifpdf.dtx
%
% Installation:
%    TDS:tex/generic/oberdiek/ifpdf.sty
%    TDS:doc/latex/oberdiek/ifpdf.pdf
%    TDS:doc/latex/oberdiek/test/ifpdf-test1.tex
%    TDS:source/latex/oberdiek/ifpdf.dtx
%
%<*ignore>
\begingroup
  \catcode123=1 %
  \catcode125=2 %
  \def\x{LaTeX2e}%
\expandafter\endgroup
\ifcase 0\ifx\install y1\fi\expandafter
         \ifx\csname processbatchFile\endcsname\relax\else1\fi
         \ifx\fmtname\x\else 1\fi\relax
\else\csname fi\endcsname
%</ignore>
%<*install>
\input docstrip.tex
\Msg{************************************************************************}
\Msg{* Installation}
\Msg{* Package: ifpdf 2011/01/30 v2.3 Provides the ifpdf switch (HO)}
\Msg{************************************************************************}

\keepsilent
\askforoverwritefalse

\let\MetaPrefix\relax
\preamble

This is a generated file.

Project: ifpdf
Version: 2011/01/30 v2.3

Copyright (C) 2001, 2005-2011 by
   Heiko Oberdiek <heiko.oberdiek at googlemail.com>

This work may be distributed and/or modified under the
conditions of the LaTeX Project Public License, either
version 1.3c of this license or (at your option) any later
version. This version of this license is in
   http://www.latex-project.org/lppl/lppl-1-3c.txt
and the latest version of this license is in
   http://www.latex-project.org/lppl.txt
and version 1.3 or later is part of all distributions of
LaTeX version 2005/12/01 or later.

This work has the LPPL maintenance status "maintained".

This Current Maintainer of this work is Heiko Oberdiek.

The Base Interpreter refers to any `TeX-Format',
because some files are installed in TDS:tex/generic//.

This work consists of the main source file ifpdf.dtx
and the derived files
   ifpdf.sty, ifpdf.pdf, ifpdf.ins, ifpdf.drv, ifpdf-test1.tex.

\endpreamble
\let\MetaPrefix\DoubleperCent

\generate{%
  \file{ifpdf.ins}{\from{ifpdf.dtx}{install}}%
  \file{ifpdf.drv}{\from{ifpdf.dtx}{driver}}%
  \usedir{tex/generic/oberdiek}%
  \file{ifpdf.sty}{\from{ifpdf.dtx}{package}}%
  \usedir{doc/latex/oberdiek/test}%
  \file{ifpdf-test1.tex}{\from{ifpdf.dtx}{test1}}%
  \nopreamble
  \nopostamble
  \usedir{source/latex/oberdiek/catalogue}%
  \file{ifpdf.xml}{\from{ifpdf.dtx}{catalogue}}%
}

\catcode32=13\relax% active space
\let =\space%
\Msg{************************************************************************}
\Msg{*}
\Msg{* To finish the installation you have to move the following}
\Msg{* file into a directory searched by TeX:}
\Msg{*}
\Msg{*     ifpdf.sty}
\Msg{*}
\Msg{* To produce the documentation run the file `ifpdf.drv'}
\Msg{* through LaTeX.}
\Msg{*}
\Msg{* Happy TeXing!}
\Msg{*}
\Msg{************************************************************************}

\endbatchfile
%</install>
%<*ignore>
\fi
%</ignore>
%<*driver>
\NeedsTeXFormat{LaTeX2e}
\ProvidesFile{ifpdf.drv}%
  [2011/01/30 v2.3 Provides the ifpdf switch (HO)]%
\documentclass{ltxdoc}
\usepackage{holtxdoc}[2011/11/22]
\begin{document}
  \DocInput{ifpdf.dtx}%
\end{document}
%</driver>
% \fi
%
% \CheckSum{600}
%
% \CharacterTable
%  {Upper-case    \A\B\C\D\E\F\G\H\I\J\K\L\M\N\O\P\Q\R\S\T\U\V\W\X\Y\Z
%   Lower-case    \a\b\c\d\e\f\g\h\i\j\k\l\m\n\o\p\q\r\s\t\u\v\w\x\y\z
%   Digits        \0\1\2\3\4\5\6\7\8\9
%   Exclamation   \!     Double quote  \"     Hash (number) \#
%   Dollar        \$     Percent       \%     Ampersand     \&
%   Acute accent  \'     Left paren    \(     Right paren   \)
%   Asterisk      \*     Plus          \+     Comma         \,
%   Minus         \-     Point         \.     Solidus       \/
%   Colon         \:     Semicolon     \;     Less than     \<
%   Equals        \=     Greater than  \>     Question mark \?
%   Commercial at \@     Left bracket  \[     Backslash     \\
%   Right bracket \]     Circumflex    \^     Underscore    \_
%   Grave accent  \`     Left brace    \{     Vertical bar  \|
%   Right brace   \}     Tilde         \~}
%
% \GetFileInfo{ifpdf.drv}
%
% \title{The \xpackage{ifpdf} package}
% \date{2011/01/30 v2.3}
% \author{Heiko Oberdiek\\\xemail{heiko.oberdiek at googlemail.com}}
%
% \maketitle
%
% \begin{abstract}
% This package looks for \pdfTeX\ in pdf mode and
% implements and sets the switch \cs{ifpdf}.
% The detection is based on \cs{pdfoutput} and
% the package will not change this value.
% It works with plain or \LaTeX\ formats.
% \end{abstract}
%
% \tableofcontents
%
% \hypersetup{bookmarksopenlevel=2}
% \section{Documentation}
%
% \subsection{Introduction}
%
% It is commonly known that H\`an Th\^e Th\`anh's \pdfTeX\ generates
% PDF output directly and many people uses \pdfTeX\ for this purpose.
% However the DVI output was never thrown away. In contrary, he
% new features for typesetting that works in both PDF and DVI mode.
%
% In the meantime many \TeX\ distributions replace the traditional
% \TeX\ binary with \pdfTeX. Then, for example, called as |latex|
% \pdfTeX\ works in DVI mode with the \LaTeX\ format preloaded,
% called as |pdflatex| \pdfTeX\ starts in PDF mode.
%
% Often packages or users want to know, whether the current document
% is typset by \pdfTeX\ in PDF mode, because the different modes
% have different capabilities (color setting, graphics inclusion,
% \dots). For this purpose \pdfTeX's
% \cs{pdfoutput} can be asked.
%
% As regulary reader of \TeX\ newsgroups and mailing lists I could
% observe many problems with this task. Common errors are:
% \begin{itemize}
% \item
%   \pdfTeX\ has \emph{two} modes. Using \pdfTeX\ does not mean
%   that the user always want to have PDF mode. For example, the
%   PostScript support is better in DVI mode in conjunction with a
%   PostScript aware DVI driver (e.g. dvips). Also the additional
%   typesetting features are mode independent and also available
%   in DVI mode.
% \item
%   \LaTeX's \cs{@ifundefined} inherited the side effect from
%   \cs{csname}. Unknown commands are defined with the meaning
%   of \cs{relax}. If it is checked, whether \cs{pdfoutput} is
%   defined, then this should not be forgotten.
% \item
%   Having \cs{pdfoutput} does not automatically mean PDF mode.
%   Also the value of \cs{pdfoutput} must be asked.
% \item
%   \cs{pdfoutput} must not be destroyed in some way. Later
%   code and packages are fooled then and will perhaps make
%   wrong decisions. For example they may drop support for
%   PDF mode, because they do not know that \pdfTeX\ is
%   running at all.
% \end{itemize}
%
% Robin Fairbairns provides an entry for this topic in his
% excellent FAQ (\url{http://www.tex.ac.uk/faq}):
% \href{http://www.tex.ac.uk/cgi-bin/texfaq2html?label=ifpdf}^^A
% {Am I using PDFTeX?}
%
% \subsection{Usage}
%
% The package \xpackage{ifpdf} can be used with both \plainTeX\
% and \LaTeX:
% \begin{description}
% \item[\plainTeX:] |\input ifpdf.sty|
% \item[\LaTeXe:]   |\usepackage{ifpdf}|
% \end{description}
%
% \DescribeMacro{\ifpdf}
% The package provides the switch \cs{ifpdf}:
% \begin{quote}
%   |\ifpdf|\\
%   \hspace*{1.5em}\dots\ do things, if \pdfTeX\ is running in pdf mode \dots\\
%   |\else|\\
%   \hspace*{1.5em}\dots\ other \TeX\ or \pdfTeX\ in dvi mode \dots\\
%   |\fi|
% \end{quote}
% Users of the package \xpackage{ifthen} can use the switch as boolean:
% \begin{quote}
%   |\boolean{pdf}|
% \end{quote}
% The package can also be used to set global
% documentclass options:
% \begin{quote}
%   |\RequirePackage{ifpdf}|\\
%   |\ifpdf|\\
%   \hspace*{1.5em}|\documentclass[pdftex,|\dots|]{|\dots|}|\\
%   |\else|\\
%   \hspace*{1.5em}|\documentclass[|\dots|]{|\dots|}|\\
%   |\fi|
% \end{quote}
%
% \subsection{Specification}
%
% The package have the following properties:
% \begin{itemize}
% \item It asks the setting of \cs{pdfoutput} for detecting
%   \pdfTeX\ in PDF mode.
% \item It never changes \cs{pdfoutput}.
% \item If \cs{pdfoutput} is undefined or has the meaning \cs{relax},
%   but the engine provides the primitive \cs{pdfoutput},
%   then \cs{pdfoutput} is enabled or restored if possible
%   (only \LuaTeX, version 0.36.0 or higher).
% \item It can be used with many formats including
%   \plainTeX\ and \LaTeX.
% \end{itemize}
% The mode detection implements the following algorithm:
% \begin{quote}
% \def\={${:}{=}$}
% \def\>{\null\qquad\ignorespaces}
% \def\comment#1{^^A
%   \textrm{\itshape\% #1}^^A
% }
% \sffamily
% if undefined(\cs{pdfoutput})\\
% \> \cs{ifpdf} \= \cs{iffalse} \comment{\pdfTeX is not running}\\
% else\\
% \> if \cs{pdfoutput} $\leq$ 0\\
% \>\> \cs{ifpdf} \= \cs{iffalse} \comment{\pdfTeX\ in DVI mode}\\
% \> else\\
% \>\> \cs{ifpdf} \= \cs{iftrue} \comment{\pdfTeX\ in PDF mode}\\
% \> fi\\
% fi
% \end{quote}
% The function \textsf{undefined} checks both cases, undefined command
% and \cs{relax}.
%
% \hypersetup{bookmarksopenlevel=1}
%
% \StopEventually{
% }
%
% \section{Implementation}
%
%    \begin{macrocode}
%<*package>
%    \end{macrocode}
%
% \subsection{Reload check and package identification}
%    Reload check, especially if the package is not used with \LaTeX.
%    \begin{macrocode}
\begingroup\catcode61\catcode48\catcode32=10\relax%
  \catcode13=5 % ^^M
  \endlinechar=13 %
  \catcode35=6 % #
  \catcode39=12 % '
  \catcode44=12 % ,
  \catcode45=12 % -
  \catcode46=12 % .
  \catcode58=12 % :
  \catcode64=11 % @
  \catcode123=1 % {
  \catcode125=2 % }
  \expandafter\let\expandafter\x\csname ver@ifpdf.sty\endcsname
  \ifx\x\relax % plain-TeX, first loading
  \else
    \def\empty{}%
    \ifx\x\empty % LaTeX, first loading,
      % variable is initialized, but \ProvidesPackage not yet seen
    \else
      \expandafter\ifx\csname PackageInfo\endcsname\relax
        \def\x#1#2{%
          \immediate\write-1{Package #1 Info: #2.}%
        }%
      \else
        \def\x#1#2{\PackageInfo{#1}{#2, stopped}}%
      \fi
      \x{ifpdf}{The package is already loaded}%
      \aftergroup\endinput
    \fi
  \fi
\endgroup%
%    \end{macrocode}
%    Package identification:
%    \begin{macrocode}
\begingroup\catcode61\catcode48\catcode32=10\relax%
  \catcode13=5 % ^^M
  \endlinechar=13 %
  \catcode35=6 % #
  \catcode39=12 % '
  \catcode40=12 % (
  \catcode41=12 % )
  \catcode44=12 % ,
  \catcode45=12 % -
  \catcode46=12 % .
  \catcode47=12 % /
  \catcode58=12 % :
  \catcode64=11 % @
  \catcode91=12 % [
  \catcode93=12 % ]
  \catcode123=1 % {
  \catcode125=2 % }
  \expandafter\ifx\csname ProvidesPackage\endcsname\relax
    \def\x#1#2#3[#4]{\endgroup
      \immediate\write-1{Package: #3 #4}%
      \xdef#1{#4}%
    }%
  \else
    \def\x#1#2[#3]{\endgroup
      #2[{#3}]%
      \ifx#1\@undefined
        \xdef#1{#3}%
      \fi
      \ifx#1\relax
        \xdef#1{#3}%
      \fi
    }%
  \fi
\expandafter\x\csname ver@ifpdf.sty\endcsname
\ProvidesPackage{ifpdf}%
  [2011/01/30 v2.3 Provides the ifpdf switch (HO)]%
%    \end{macrocode}
%
% \subsection{Catcodes}
%
%    \begin{macrocode}
\begingroup\catcode61\catcode48\catcode32=10\relax%
  \catcode13=5 % ^^M
  \endlinechar=13 %
  \catcode123=1 % {
  \catcode125=2 % }
  \catcode64=11 % @
  \def\x{\endgroup
    \expandafter\edef\csname ifpdf@AtEnd\endcsname{%
      \endlinechar=\the\endlinechar\relax
      \catcode13=\the\catcode13\relax
      \catcode32=\the\catcode32\relax
      \catcode35=\the\catcode35\relax
      \catcode61=\the\catcode61\relax
      \catcode64=\the\catcode64\relax
      \catcode123=\the\catcode123\relax
      \catcode125=\the\catcode125\relax
    }%
  }%
\x\catcode61\catcode48\catcode32=10\relax%
\catcode13=5 % ^^M
\endlinechar=13 %
\catcode35=6 % #
\catcode64=11 % @
\catcode123=1 % {
\catcode125=2 % }
\def\TMP@EnsureCode#1#2{%
  \edef\ifpdf@AtEnd{%
    \ifpdf@AtEnd
    \catcode#1=\the\catcode#1\relax
  }%
  \catcode#1=#2\relax
}
\TMP@EnsureCode{10}{12}% ^^J
\TMP@EnsureCode{39}{12}% '
\TMP@EnsureCode{40}{12}% (
\TMP@EnsureCode{41}{12}% )
\TMP@EnsureCode{44}{12}% ,
\TMP@EnsureCode{45}{12}% -
\TMP@EnsureCode{46}{12}% .
\TMP@EnsureCode{47}{12}% /
\TMP@EnsureCode{58}{12}% :
\TMP@EnsureCode{60}{12}% <
\TMP@EnsureCode{91}{12}% [
\TMP@EnsureCode{93}{12}% ]
\TMP@EnsureCode{94}{7}% ^
\TMP@EnsureCode{96}{12}% `
\edef\ifpdf@AtEnd{\ifpdf@AtEnd\noexpand\endinput}
%    \end{macrocode}
%
% \subsection{Check for previously defined \cs{ifpdf}}
%
%    \begin{macrocode}
\begingroup
  \expandafter\ifx\csname ifpdf\endcsname\relax
  \else
    \edef\i/{\expandafter\string\csname ifpdf\endcsname}%
    \expandafter\ifx\csname PackageError\endcsname\relax
      \def\x#1#2{%
        \edef\z{#2}%
        \expandafter\errhelp\expandafter{\z}%
        \errmessage{Package ifpdf Error: #1}%
      }%
      \def\y{^^J}%
      \newlinechar=10 %
    \else
      \def\x#1#2{%
        \PackageError{ifpdf}{#1}{#2}%
      }%
      \def\y{\MessageBreak}%
    \fi
    \x{Name clash, \i/ is already defined}{%
      Incompatible versions of \i/ can cause problems,\y
      therefore package loading is aborted.%
    }%
    \endgroup
    \expandafter\ifpdf@AtEnd
  \fi%
\endgroup
%    \end{macrocode}
%
% \subsection{\cs{pdfoutput} and \LuaTeX}
%
%    It might happen, that \LuaTeX\ is running, but \cs{pdfoutput}
%    does not exist. In version 0.40 only \cs{directlua}
%    is available at startup time. The enabling Lua function was
%    already added in version 0.36. Thus we can ignore older versions,
%    here \cs{pdfoutput} is available at startup time.
%
%    \begin{macrocode}
\begingroup
  \def\skip#1\relax\begingroup{}%
  \expandafter\ifx\csname pdfoutput\endcsname\relax
  \else
    \expandafter\skip
  \fi
  \expandafter\ifx\csname directlua\endcsname\relax
    \expandafter\skip
  \fi
\endgroup
\begingroup\expandafter\expandafter\expandafter\endgroup
\expandafter\ifx\csname RequirePackage\endcsname\relax
  \def\TMP@RequirePackage#1[#2]{%
    \begingroup\expandafter\expandafter\expandafter\endgroup
    \expandafter\ifx\csname ver@#1.sty\endcsname\relax
      \input #1.sty\relax
    \fi
  }%
  \TMP@RequirePackage{ifluatex}[2009/04/10]%
\else
  \RequirePackage{ifluatex}[2009/04/10]%
\fi
\ifluatex
  \ifnum\luatexversion<36 %
%    \end{macrocode}
%    Unhappily \LuaTeX's \cs{primitive} (derived from
%    pdfTeX's \cs{pdfprimitive}) cannot be used:
%    \begin{quote}
%    |\protected\gdef\pdfoutput{\primitive\pdfoutput}|
%    \end{quote}
%    Setting a value works, but getting fails,
%    because \TeX\ does no longer see it as number. It
%    is unexpandable and breaks numerical contexts.
%    This was fixed in \hologo{pdfTeX} 1.40.10
%    (bugfix \#4289: ``|\primitive\pdfoutput| cannot be queried'').
%    \begin{macrocode}
  \else
    \begingroup
      \directlua{tex.enableprimitives('ifpdf', {'pdfoutput'})}%
      \global\let\pdfoutput\ifpdfpdfoutput
    \endgroup
  \fi
\fi
\relax\begingroup\endgroup
%    \end{macrocode}
%
% \subsection{\cs{ifpdf}}
%
%    \begin{macro}{\ifpdf}
%    Create and set the switch. \cs{newif} initializes the
%    switch with \cs{iffalse}. \cs{newif} is \cs{outer}
%    in \plainTeX.
%    \begin{macrocode}
\begingroup\expandafter\expandafter\expandafter\endgroup
\expandafter\ifx\csname newif\endcsname\relax
  \edef\pdffalse{%
    \let
    \expandafter\noexpand\csname ifpdf\endcsname
    \expandafter\noexpand\csname iffalse\endcsname
  }%
  \edef\pdftrue{%
    \let
    \expandafter\noexpand\csname ifpdf\endcsname
    \expandafter\noexpand\csname iftrue\endcsname
  }%
  \pdffalse
\else
  \csname newif\expandafter\endcsname\csname ifpdf\endcsname
\fi
%    \end{macrocode}
%    Test \cs{pdfoutput}. Is it  defined and different from
%    \cs{relax}? Someone could have used \LaTeX\ internal
%    \cs{@ifundefined}, or something else involving.
%    Notice, \cs{csname} is executed inside a group for the test
%    to cancel the side effect of \cs{csname}.
%    \begin{macrocode}
\begingroup\expandafter\expandafter\expandafter\endgroup
\expandafter\ifx\csname pdfoutput\endcsname\relax
\else
  \ifnum\pdfoutput<1 %
%    \end{macrocode}
%    \cs{pdfoutput}|=0| or negative, so not generating pdf.
%    \begin{macrocode}
  \else
    \pdftrue
  \fi
\fi
%    \end{macrocode}
%    \end{macro}
%
% \subsection{Test for fool attempts}
%
%    \begin{macrocode}
\begingroup
  \expandafter\ifx\csname pdfoutput\endcsname\relax
  \else
    \escapechar=92 %
    \edef\m{\meaning\pdfoutput}%
    \edef\p{\string\pdfoutput}%
    \ifx\m\p
    \else
      \expandafter\ifx\csname PackageWarningNoLine\endcsname\relax
        \def\PackageWarningNoLine#1#2{%
          \immediate\write16{%
            Package `#1' Warning: #2.%
          }%
        }%
      \fi
      \PackageWarningNoLine{ifpdf}{%
        Someone has redefined \string\pdfoutput%
      }%
    \fi
  \fi
\endgroup
%    \end{macrocode}
%
% \subsection{Protocol entry}
%
%     Log comment:
%    \begin{macrocode}
\begingroup
  \expandafter\ifx\csname PackageInfo\endcsname\relax
    \def\x#1#2{%
      \immediate\write-1{Package #1 Info: #2.}%
    }%
  \else
    \let\x\PackageInfo
    \expandafter\let\csname on@line\endcsname\empty
  \fi
  \x{ifpdf}{pdfTeX in PDF mode is \ifpdf\else not \fi detected}%
\endgroup
%    \end{macrocode}
%    \begin{macrocode}
\ifpdf@AtEnd%
%    \end{macrocode}
%    \begin{macrocode}
%</package>
%    \end{macrocode}
%
% \section{Test}
%
% \subsection{Catcode checks for loading}
%
%    \begin{macrocode}
%<*test1>
%    \end{macrocode}
%    \begin{macrocode}
\catcode`\{=1 %
\catcode`\}=2 %
\catcode`\#=6 %
\catcode`\@=11 %
\expandafter\ifx\csname count@\endcsname\relax
  \countdef\count@=255 %
\fi
\expandafter\ifx\csname @gobble\endcsname\relax
  \long\def\@gobble#1{}%
\fi
\expandafter\ifx\csname @firstofone\endcsname\relax
  \long\def\@firstofone#1{#1}%
\fi
\expandafter\ifx\csname loop\endcsname\relax
  \expandafter\@firstofone
\else
  \expandafter\@gobble
\fi
{%
  \def\loop#1\repeat{%
    \def\body{#1}%
    \iterate
  }%
  \def\iterate{%
    \body
      \let\next\iterate
    \else
      \let\next\relax
    \fi
    \next
  }%
  \let\repeat=\fi
}%
\def\RestoreCatcodes{}
\count@=0 %
\loop
  \edef\RestoreCatcodes{%
    \RestoreCatcodes
    \catcode\the\count@=\the\catcode\count@\relax
  }%
\ifnum\count@<255 %
  \advance\count@ 1 %
\repeat

\def\RangeCatcodeInvalid#1#2{%
  \count@=#1\relax
  \loop
    \catcode\count@=15 %
  \ifnum\count@<#2\relax
    \advance\count@ 1 %
  \repeat
}
\def\RangeCatcodeCheck#1#2#3{%
  \count@=#1\relax
  \loop
    \ifnum#3=\catcode\count@
    \else
      \errmessage{%
        Character \the\count@\space
        with wrong catcode \the\catcode\count@\space
        instead of \number#3%
      }%
    \fi
  \ifnum\count@<#2\relax
    \advance\count@ 1 %
  \repeat
}
\def\space{ }
\expandafter\ifx\csname LoadCommand\endcsname\relax
  \def\LoadCommand{\input ifpdf.sty\relax}%
\fi
\def\Test{%
  \RangeCatcodeInvalid{0}{47}%
  \RangeCatcodeInvalid{58}{64}%
  \RangeCatcodeInvalid{91}{96}%
  \RangeCatcodeInvalid{123}{255}%
  \catcode`\@=12 %
  \catcode`\\=0 %
  \catcode`\%=14 %
  \LoadCommand
  \RangeCatcodeCheck{0}{36}{15}%
  \RangeCatcodeCheck{37}{37}{14}%
  \RangeCatcodeCheck{38}{47}{15}%
  \RangeCatcodeCheck{48}{57}{12}%
  \RangeCatcodeCheck{58}{63}{15}%
  \RangeCatcodeCheck{64}{64}{12}%
  \RangeCatcodeCheck{65}{90}{11}%
  \RangeCatcodeCheck{91}{91}{15}%
  \RangeCatcodeCheck{92}{92}{0}%
  \RangeCatcodeCheck{93}{96}{15}%
  \RangeCatcodeCheck{97}{122}{11}%
  \RangeCatcodeCheck{123}{255}{15}%
  \RestoreCatcodes
}
\Test
\csname @@end\endcsname
\end
%    \end{macrocode}
%    \begin{macrocode}
%</test1>
%    \end{macrocode}
%
% \section{Installation}
%
% \subsection{Download}
%
% \paragraph{Package.} This package is available on
% CTAN\footnote{\url{ftp://ftp.ctan.org/tex-archive/}}:
% \begin{description}
% \item[\CTAN{macros/latex/contrib/oberdiek/ifpdf.dtx}] The source file.
% \item[\CTAN{macros/latex/contrib/oberdiek/ifpdf.pdf}] Documentation.
% \end{description}
%
%
% \paragraph{Bundle.} All the packages of the bundle `oberdiek'
% are also available in a TDS compliant ZIP archive. There
% the packages are already unpacked and the documentation files
% are generated. The files and directories obey the TDS standard.
% \begin{description}
% \item[\CTAN{install/macros/latex/contrib/oberdiek.tds.zip}]
% \end{description}
% \emph{TDS} refers to the standard ``A Directory Structure
% for \TeX\ Files'' (\CTAN{tds/tds.pdf}). Directories
% with \xfile{texmf} in their name are usually organized this way.
%
% \subsection{Bundle installation}
%
% \paragraph{Unpacking.} Unpack the \xfile{oberdiek.tds.zip} in the
% TDS tree (also known as \xfile{texmf} tree) of your choice.
% Example (linux):
% \begin{quote}
%   |unzip oberdiek.tds.zip -d ~/texmf|
% \end{quote}
%
% \paragraph{Script installation.}
% Check the directory \xfile{TDS:scripts/oberdiek/} for
% scripts that need further installation steps.
% Package \xpackage{attachfile2} comes with the Perl script
% \xfile{pdfatfi.pl} that should be installed in such a way
% that it can be called as \texttt{pdfatfi}.
% Example (linux):
% \begin{quote}
%   |chmod +x scripts/oberdiek/pdfatfi.pl|\\
%   |cp scripts/oberdiek/pdfatfi.pl /usr/local/bin/|
% \end{quote}
%
% \subsection{Package installation}
%
% \paragraph{Unpacking.} The \xfile{.dtx} file is a self-extracting
% \docstrip\ archive. The files are extracted by running the
% \xfile{.dtx} through \plainTeX:
% \begin{quote}
%   \verb|tex ifpdf.dtx|
% \end{quote}
%
% \paragraph{TDS.} Now the different files must be moved into
% the different directories in your installation TDS tree
% (also known as \xfile{texmf} tree):
% \begin{quote}
% \def\t{^^A
% \begin{tabular}{@{}>{\ttfamily}l@{ $\rightarrow$ }>{\ttfamily}l@{}}
%   ifpdf.sty & tex/generic/oberdiek/ifpdf.sty\\
%   ifpdf.pdf & doc/latex/oberdiek/ifpdf.pdf\\
%   test/ifpdf-test1.tex & doc/latex/oberdiek/test/ifpdf-test1.tex\\
%   ifpdf.dtx & source/latex/oberdiek/ifpdf.dtx\\
% \end{tabular}^^A
% }^^A
% \sbox0{\t}^^A
% \ifdim\wd0>\linewidth
%   \begingroup
%     \advance\linewidth by\leftmargin
%     \advance\linewidth by\rightmargin
%   \edef\x{\endgroup
%     \def\noexpand\lw{\the\linewidth}^^A
%   }\x
%   \def\lwbox{^^A
%     \leavevmode
%     \hbox to \linewidth{^^A
%       \kern-\leftmargin\relax
%       \hss
%       \usebox0
%       \hss
%       \kern-\rightmargin\relax
%     }^^A
%   }^^A
%   \ifdim\wd0>\lw
%     \sbox0{\small\t}^^A
%     \ifdim\wd0>\linewidth
%       \ifdim\wd0>\lw
%         \sbox0{\footnotesize\t}^^A
%         \ifdim\wd0>\linewidth
%           \ifdim\wd0>\lw
%             \sbox0{\scriptsize\t}^^A
%             \ifdim\wd0>\linewidth
%               \ifdim\wd0>\lw
%                 \sbox0{\tiny\t}^^A
%                 \ifdim\wd0>\linewidth
%                   \lwbox
%                 \else
%                   \usebox0
%                 \fi
%               \else
%                 \lwbox
%               \fi
%             \else
%               \usebox0
%             \fi
%           \else
%             \lwbox
%           \fi
%         \else
%           \usebox0
%         \fi
%       \else
%         \lwbox
%       \fi
%     \else
%       \usebox0
%     \fi
%   \else
%     \lwbox
%   \fi
% \else
%   \usebox0
% \fi
% \end{quote}
% If you have a \xfile{docstrip.cfg} that configures and enables \docstrip's
% TDS installing feature, then some files can already be in the right
% place, see the documentation of \docstrip.
%
% \subsection{Refresh file name databases}
%
% If your \TeX~distribution
% (\teTeX, \mikTeX, \dots) relies on file name databases, you must refresh
% these. For example, \teTeX\ users run \verb|texhash| or
% \verb|mktexlsr|.
%
% \subsection{Some details for the interested}
%
% \paragraph{Attached source.}
%
% The PDF documentation on CTAN also includes the
% \xfile{.dtx} source file. It can be extracted by
% AcrobatReader 6 or higher. Another option is \textsf{pdftk},
% e.g. unpack the file into the current directory:
% \begin{quote}
%   \verb|pdftk ifpdf.pdf unpack_files output .|
% \end{quote}
%
% \paragraph{Unpacking with \LaTeX.}
% The \xfile{.dtx} chooses its action depending on the format:
% \begin{description}
% \item[\plainTeX:] Run \docstrip\ and extract the files.
% \item[\LaTeX:] Generate the documentation.
% \end{description}
% If you insist on using \LaTeX\ for \docstrip\ (really,
% \docstrip\ does not need \LaTeX), then inform the autodetect routine
% about your intention:
% \begin{quote}
%   \verb|latex \let\install=y% \iffalse meta-comment
%
% File: ifpdf.dtx
% Version: 2011/01/30 v2.3
% Info: Provides the ifpdf switch
%
% Copyright (C) 2001, 2005-2011 by
%    Heiko Oberdiek <heiko.oberdiek at googlemail.com>
%
% This work may be distributed and/or modified under the
% conditions of the LaTeX Project Public License, either
% version 1.3c of this license or (at your option) any later
% version. This version of this license is in
%    http://www.latex-project.org/lppl/lppl-1-3c.txt
% and the latest version of this license is in
%    http://www.latex-project.org/lppl.txt
% and version 1.3 or later is part of all distributions of
% LaTeX version 2005/12/01 or later.
%
% This work has the LPPL maintenance status "maintained".
%
% This Current Maintainer of this work is Heiko Oberdiek.
%
% The Base Interpreter refers to any `TeX-Format',
% because some files are installed in TDS:tex/generic//.
%
% This work consists of the main source file ifpdf.dtx
% and the derived files
%    ifpdf.sty, ifpdf.pdf, ifpdf.ins, ifpdf.drv, ifpdf-test1.tex.
%
% Distribution:
%    CTAN:macros/latex/contrib/oberdiek/ifpdf.dtx
%    CTAN:macros/latex/contrib/oberdiek/ifpdf.pdf
%
% Unpacking:
%    (a) If ifpdf.ins is present:
%           tex ifpdf.ins
%    (b) Without ifpdf.ins:
%           tex ifpdf.dtx
%    (c) If you insist on using LaTeX
%           latex \let\install=y\input{ifpdf.dtx}
%        (quote the arguments according to the demands of your shell)
%
% Documentation:
%    (a) If ifpdf.drv is present:
%           latex ifpdf.drv
%    (b) Without ifpdf.drv:
%           latex ifpdf.dtx; ...
%    The class ltxdoc loads the configuration file ltxdoc.cfg
%    if available. Here you can specify further options, e.g.
%    use A4 as paper format:
%       \PassOptionsToClass{a4paper}{article}
%
%    Programm calls to get the documentation (example):
%       pdflatex ifpdf.dtx
%       makeindex -s gind.ist ifpdf.idx
%       pdflatex ifpdf.dtx
%       makeindex -s gind.ist ifpdf.idx
%       pdflatex ifpdf.dtx
%
% Installation:
%    TDS:tex/generic/oberdiek/ifpdf.sty
%    TDS:doc/latex/oberdiek/ifpdf.pdf
%    TDS:doc/latex/oberdiek/test/ifpdf-test1.tex
%    TDS:source/latex/oberdiek/ifpdf.dtx
%
%<*ignore>
\begingroup
  \catcode123=1 %
  \catcode125=2 %
  \def\x{LaTeX2e}%
\expandafter\endgroup
\ifcase 0\ifx\install y1\fi\expandafter
         \ifx\csname processbatchFile\endcsname\relax\else1\fi
         \ifx\fmtname\x\else 1\fi\relax
\else\csname fi\endcsname
%</ignore>
%<*install>
\input docstrip.tex
\Msg{************************************************************************}
\Msg{* Installation}
\Msg{* Package: ifpdf 2011/01/30 v2.3 Provides the ifpdf switch (HO)}
\Msg{************************************************************************}

\keepsilent
\askforoverwritefalse

\let\MetaPrefix\relax
\preamble

This is a generated file.

Project: ifpdf
Version: 2011/01/30 v2.3

Copyright (C) 2001, 2005-2011 by
   Heiko Oberdiek <heiko.oberdiek at googlemail.com>

This work may be distributed and/or modified under the
conditions of the LaTeX Project Public License, either
version 1.3c of this license or (at your option) any later
version. This version of this license is in
   http://www.latex-project.org/lppl/lppl-1-3c.txt
and the latest version of this license is in
   http://www.latex-project.org/lppl.txt
and version 1.3 or later is part of all distributions of
LaTeX version 2005/12/01 or later.

This work has the LPPL maintenance status "maintained".

This Current Maintainer of this work is Heiko Oberdiek.

The Base Interpreter refers to any `TeX-Format',
because some files are installed in TDS:tex/generic//.

This work consists of the main source file ifpdf.dtx
and the derived files
   ifpdf.sty, ifpdf.pdf, ifpdf.ins, ifpdf.drv, ifpdf-test1.tex.

\endpreamble
\let\MetaPrefix\DoubleperCent

\generate{%
  \file{ifpdf.ins}{\from{ifpdf.dtx}{install}}%
  \file{ifpdf.drv}{\from{ifpdf.dtx}{driver}}%
  \usedir{tex/generic/oberdiek}%
  \file{ifpdf.sty}{\from{ifpdf.dtx}{package}}%
  \usedir{doc/latex/oberdiek/test}%
  \file{ifpdf-test1.tex}{\from{ifpdf.dtx}{test1}}%
  \nopreamble
  \nopostamble
  \usedir{source/latex/oberdiek/catalogue}%
  \file{ifpdf.xml}{\from{ifpdf.dtx}{catalogue}}%
}

\catcode32=13\relax% active space
\let =\space%
\Msg{************************************************************************}
\Msg{*}
\Msg{* To finish the installation you have to move the following}
\Msg{* file into a directory searched by TeX:}
\Msg{*}
\Msg{*     ifpdf.sty}
\Msg{*}
\Msg{* To produce the documentation run the file `ifpdf.drv'}
\Msg{* through LaTeX.}
\Msg{*}
\Msg{* Happy TeXing!}
\Msg{*}
\Msg{************************************************************************}

\endbatchfile
%</install>
%<*ignore>
\fi
%</ignore>
%<*driver>
\NeedsTeXFormat{LaTeX2e}
\ProvidesFile{ifpdf.drv}%
  [2011/01/30 v2.3 Provides the ifpdf switch (HO)]%
\documentclass{ltxdoc}
\usepackage{holtxdoc}[2011/11/22]
\begin{document}
  \DocInput{ifpdf.dtx}%
\end{document}
%</driver>
% \fi
%
% \CheckSum{600}
%
% \CharacterTable
%  {Upper-case    \A\B\C\D\E\F\G\H\I\J\K\L\M\N\O\P\Q\R\S\T\U\V\W\X\Y\Z
%   Lower-case    \a\b\c\d\e\f\g\h\i\j\k\l\m\n\o\p\q\r\s\t\u\v\w\x\y\z
%   Digits        \0\1\2\3\4\5\6\7\8\9
%   Exclamation   \!     Double quote  \"     Hash (number) \#
%   Dollar        \$     Percent       \%     Ampersand     \&
%   Acute accent  \'     Left paren    \(     Right paren   \)
%   Asterisk      \*     Plus          \+     Comma         \,
%   Minus         \-     Point         \.     Solidus       \/
%   Colon         \:     Semicolon     \;     Less than     \<
%   Equals        \=     Greater than  \>     Question mark \?
%   Commercial at \@     Left bracket  \[     Backslash     \\
%   Right bracket \]     Circumflex    \^     Underscore    \_
%   Grave accent  \`     Left brace    \{     Vertical bar  \|
%   Right brace   \}     Tilde         \~}
%
% \GetFileInfo{ifpdf.drv}
%
% \title{The \xpackage{ifpdf} package}
% \date{2011/01/30 v2.3}
% \author{Heiko Oberdiek\\\xemail{heiko.oberdiek at googlemail.com}}
%
% \maketitle
%
% \begin{abstract}
% This package looks for \pdfTeX\ in pdf mode and
% implements and sets the switch \cs{ifpdf}.
% The detection is based on \cs{pdfoutput} and
% the package will not change this value.
% It works with plain or \LaTeX\ formats.
% \end{abstract}
%
% \tableofcontents
%
% \hypersetup{bookmarksopenlevel=2}
% \section{Documentation}
%
% \subsection{Introduction}
%
% It is commonly known that H\`an Th\^e Th\`anh's \pdfTeX\ generates
% PDF output directly and many people uses \pdfTeX\ for this purpose.
% However the DVI output was never thrown away. In contrary, he
% new features for typesetting that works in both PDF and DVI mode.
%
% In the meantime many \TeX\ distributions replace the traditional
% \TeX\ binary with \pdfTeX. Then, for example, called as |latex|
% \pdfTeX\ works in DVI mode with the \LaTeX\ format preloaded,
% called as |pdflatex| \pdfTeX\ starts in PDF mode.
%
% Often packages or users want to know, whether the current document
% is typset by \pdfTeX\ in PDF mode, because the different modes
% have different capabilities (color setting, graphics inclusion,
% \dots). For this purpose \pdfTeX's
% \cs{pdfoutput} can be asked.
%
% As regulary reader of \TeX\ newsgroups and mailing lists I could
% observe many problems with this task. Common errors are:
% \begin{itemize}
% \item
%   \pdfTeX\ has \emph{two} modes. Using \pdfTeX\ does not mean
%   that the user always want to have PDF mode. For example, the
%   PostScript support is better in DVI mode in conjunction with a
%   PostScript aware DVI driver (e.g. dvips). Also the additional
%   typesetting features are mode independent and also available
%   in DVI mode.
% \item
%   \LaTeX's \cs{@ifundefined} inherited the side effect from
%   \cs{csname}. Unknown commands are defined with the meaning
%   of \cs{relax}. If it is checked, whether \cs{pdfoutput} is
%   defined, then this should not be forgotten.
% \item
%   Having \cs{pdfoutput} does not automatically mean PDF mode.
%   Also the value of \cs{pdfoutput} must be asked.
% \item
%   \cs{pdfoutput} must not be destroyed in some way. Later
%   code and packages are fooled then and will perhaps make
%   wrong decisions. For example they may drop support for
%   PDF mode, because they do not know that \pdfTeX\ is
%   running at all.
% \end{itemize}
%
% Robin Fairbairns provides an entry for this topic in his
% excellent FAQ (\url{http://www.tex.ac.uk/faq}):
% \href{http://www.tex.ac.uk/cgi-bin/texfaq2html?label=ifpdf}^^A
% {Am I using PDFTeX?}
%
% \subsection{Usage}
%
% The package \xpackage{ifpdf} can be used with both \plainTeX\
% and \LaTeX:
% \begin{description}
% \item[\plainTeX:] |\input ifpdf.sty|
% \item[\LaTeXe:]   |\usepackage{ifpdf}|
% \end{description}
%
% \DescribeMacro{\ifpdf}
% The package provides the switch \cs{ifpdf}:
% \begin{quote}
%   |\ifpdf|\\
%   \hspace*{1.5em}\dots\ do things, if \pdfTeX\ is running in pdf mode \dots\\
%   |\else|\\
%   \hspace*{1.5em}\dots\ other \TeX\ or \pdfTeX\ in dvi mode \dots\\
%   |\fi|
% \end{quote}
% Users of the package \xpackage{ifthen} can use the switch as boolean:
% \begin{quote}
%   |\boolean{pdf}|
% \end{quote}
% The package can also be used to set global
% documentclass options:
% \begin{quote}
%   |\RequirePackage{ifpdf}|\\
%   |\ifpdf|\\
%   \hspace*{1.5em}|\documentclass[pdftex,|\dots|]{|\dots|}|\\
%   |\else|\\
%   \hspace*{1.5em}|\documentclass[|\dots|]{|\dots|}|\\
%   |\fi|
% \end{quote}
%
% \subsection{Specification}
%
% The package have the following properties:
% \begin{itemize}
% \item It asks the setting of \cs{pdfoutput} for detecting
%   \pdfTeX\ in PDF mode.
% \item It never changes \cs{pdfoutput}.
% \item If \cs{pdfoutput} is undefined or has the meaning \cs{relax},
%   but the engine provides the primitive \cs{pdfoutput},
%   then \cs{pdfoutput} is enabled or restored if possible
%   (only \LuaTeX, version 0.36.0 or higher).
% \item It can be used with many formats including
%   \plainTeX\ and \LaTeX.
% \end{itemize}
% The mode detection implements the following algorithm:
% \begin{quote}
% \def\={${:}{=}$}
% \def\>{\null\qquad\ignorespaces}
% \def\comment#1{^^A
%   \textrm{\itshape\% #1}^^A
% }
% \sffamily
% if undefined(\cs{pdfoutput})\\
% \> \cs{ifpdf} \= \cs{iffalse} \comment{\pdfTeX is not running}\\
% else\\
% \> if \cs{pdfoutput} $\leq$ 0\\
% \>\> \cs{ifpdf} \= \cs{iffalse} \comment{\pdfTeX\ in DVI mode}\\
% \> else\\
% \>\> \cs{ifpdf} \= \cs{iftrue} \comment{\pdfTeX\ in PDF mode}\\
% \> fi\\
% fi
% \end{quote}
% The function \textsf{undefined} checks both cases, undefined command
% and \cs{relax}.
%
% \hypersetup{bookmarksopenlevel=1}
%
% \StopEventually{
% }
%
% \section{Implementation}
%
%    \begin{macrocode}
%<*package>
%    \end{macrocode}
%
% \subsection{Reload check and package identification}
%    Reload check, especially if the package is not used with \LaTeX.
%    \begin{macrocode}
\begingroup\catcode61\catcode48\catcode32=10\relax%
  \catcode13=5 % ^^M
  \endlinechar=13 %
  \catcode35=6 % #
  \catcode39=12 % '
  \catcode44=12 % ,
  \catcode45=12 % -
  \catcode46=12 % .
  \catcode58=12 % :
  \catcode64=11 % @
  \catcode123=1 % {
  \catcode125=2 % }
  \expandafter\let\expandafter\x\csname ver@ifpdf.sty\endcsname
  \ifx\x\relax % plain-TeX, first loading
  \else
    \def\empty{}%
    \ifx\x\empty % LaTeX, first loading,
      % variable is initialized, but \ProvidesPackage not yet seen
    \else
      \expandafter\ifx\csname PackageInfo\endcsname\relax
        \def\x#1#2{%
          \immediate\write-1{Package #1 Info: #2.}%
        }%
      \else
        \def\x#1#2{\PackageInfo{#1}{#2, stopped}}%
      \fi
      \x{ifpdf}{The package is already loaded}%
      \aftergroup\endinput
    \fi
  \fi
\endgroup%
%    \end{macrocode}
%    Package identification:
%    \begin{macrocode}
\begingroup\catcode61\catcode48\catcode32=10\relax%
  \catcode13=5 % ^^M
  \endlinechar=13 %
  \catcode35=6 % #
  \catcode39=12 % '
  \catcode40=12 % (
  \catcode41=12 % )
  \catcode44=12 % ,
  \catcode45=12 % -
  \catcode46=12 % .
  \catcode47=12 % /
  \catcode58=12 % :
  \catcode64=11 % @
  \catcode91=12 % [
  \catcode93=12 % ]
  \catcode123=1 % {
  \catcode125=2 % }
  \expandafter\ifx\csname ProvidesPackage\endcsname\relax
    \def\x#1#2#3[#4]{\endgroup
      \immediate\write-1{Package: #3 #4}%
      \xdef#1{#4}%
    }%
  \else
    \def\x#1#2[#3]{\endgroup
      #2[{#3}]%
      \ifx#1\@undefined
        \xdef#1{#3}%
      \fi
      \ifx#1\relax
        \xdef#1{#3}%
      \fi
    }%
  \fi
\expandafter\x\csname ver@ifpdf.sty\endcsname
\ProvidesPackage{ifpdf}%
  [2011/01/30 v2.3 Provides the ifpdf switch (HO)]%
%    \end{macrocode}
%
% \subsection{Catcodes}
%
%    \begin{macrocode}
\begingroup\catcode61\catcode48\catcode32=10\relax%
  \catcode13=5 % ^^M
  \endlinechar=13 %
  \catcode123=1 % {
  \catcode125=2 % }
  \catcode64=11 % @
  \def\x{\endgroup
    \expandafter\edef\csname ifpdf@AtEnd\endcsname{%
      \endlinechar=\the\endlinechar\relax
      \catcode13=\the\catcode13\relax
      \catcode32=\the\catcode32\relax
      \catcode35=\the\catcode35\relax
      \catcode61=\the\catcode61\relax
      \catcode64=\the\catcode64\relax
      \catcode123=\the\catcode123\relax
      \catcode125=\the\catcode125\relax
    }%
  }%
\x\catcode61\catcode48\catcode32=10\relax%
\catcode13=5 % ^^M
\endlinechar=13 %
\catcode35=6 % #
\catcode64=11 % @
\catcode123=1 % {
\catcode125=2 % }
\def\TMP@EnsureCode#1#2{%
  \edef\ifpdf@AtEnd{%
    \ifpdf@AtEnd
    \catcode#1=\the\catcode#1\relax
  }%
  \catcode#1=#2\relax
}
\TMP@EnsureCode{10}{12}% ^^J
\TMP@EnsureCode{39}{12}% '
\TMP@EnsureCode{40}{12}% (
\TMP@EnsureCode{41}{12}% )
\TMP@EnsureCode{44}{12}% ,
\TMP@EnsureCode{45}{12}% -
\TMP@EnsureCode{46}{12}% .
\TMP@EnsureCode{47}{12}% /
\TMP@EnsureCode{58}{12}% :
\TMP@EnsureCode{60}{12}% <
\TMP@EnsureCode{91}{12}% [
\TMP@EnsureCode{93}{12}% ]
\TMP@EnsureCode{94}{7}% ^
\TMP@EnsureCode{96}{12}% `
\edef\ifpdf@AtEnd{\ifpdf@AtEnd\noexpand\endinput}
%    \end{macrocode}
%
% \subsection{Check for previously defined \cs{ifpdf}}
%
%    \begin{macrocode}
\begingroup
  \expandafter\ifx\csname ifpdf\endcsname\relax
  \else
    \edef\i/{\expandafter\string\csname ifpdf\endcsname}%
    \expandafter\ifx\csname PackageError\endcsname\relax
      \def\x#1#2{%
        \edef\z{#2}%
        \expandafter\errhelp\expandafter{\z}%
        \errmessage{Package ifpdf Error: #1}%
      }%
      \def\y{^^J}%
      \newlinechar=10 %
    \else
      \def\x#1#2{%
        \PackageError{ifpdf}{#1}{#2}%
      }%
      \def\y{\MessageBreak}%
    \fi
    \x{Name clash, \i/ is already defined}{%
      Incompatible versions of \i/ can cause problems,\y
      therefore package loading is aborted.%
    }%
    \endgroup
    \expandafter\ifpdf@AtEnd
  \fi%
\endgroup
%    \end{macrocode}
%
% \subsection{\cs{pdfoutput} and \LuaTeX}
%
%    It might happen, that \LuaTeX\ is running, but \cs{pdfoutput}
%    does not exist. In version 0.40 only \cs{directlua}
%    is available at startup time. The enabling Lua function was
%    already added in version 0.36. Thus we can ignore older versions,
%    here \cs{pdfoutput} is available at startup time.
%
%    \begin{macrocode}
\begingroup
  \def\skip#1\relax\begingroup{}%
  \expandafter\ifx\csname pdfoutput\endcsname\relax
  \else
    \expandafter\skip
  \fi
  \expandafter\ifx\csname directlua\endcsname\relax
    \expandafter\skip
  \fi
\endgroup
\begingroup\expandafter\expandafter\expandafter\endgroup
\expandafter\ifx\csname RequirePackage\endcsname\relax
  \def\TMP@RequirePackage#1[#2]{%
    \begingroup\expandafter\expandafter\expandafter\endgroup
    \expandafter\ifx\csname ver@#1.sty\endcsname\relax
      \input #1.sty\relax
    \fi
  }%
  \TMP@RequirePackage{ifluatex}[2009/04/10]%
\else
  \RequirePackage{ifluatex}[2009/04/10]%
\fi
\ifluatex
  \ifnum\luatexversion<36 %
%    \end{macrocode}
%    Unhappily \LuaTeX's \cs{primitive} (derived from
%    pdfTeX's \cs{pdfprimitive}) cannot be used:
%    \begin{quote}
%    |\protected\gdef\pdfoutput{\primitive\pdfoutput}|
%    \end{quote}
%    Setting a value works, but getting fails,
%    because \TeX\ does no longer see it as number. It
%    is unexpandable and breaks numerical contexts.
%    This was fixed in \hologo{pdfTeX} 1.40.10
%    (bugfix \#4289: ``|\primitive\pdfoutput| cannot be queried'').
%    \begin{macrocode}
  \else
    \begingroup
      \directlua{tex.enableprimitives('ifpdf', {'pdfoutput'})}%
      \global\let\pdfoutput\ifpdfpdfoutput
    \endgroup
  \fi
\fi
\relax\begingroup\endgroup
%    \end{macrocode}
%
% \subsection{\cs{ifpdf}}
%
%    \begin{macro}{\ifpdf}
%    Create and set the switch. \cs{newif} initializes the
%    switch with \cs{iffalse}. \cs{newif} is \cs{outer}
%    in \plainTeX.
%    \begin{macrocode}
\begingroup\expandafter\expandafter\expandafter\endgroup
\expandafter\ifx\csname newif\endcsname\relax
  \edef\pdffalse{%
    \let
    \expandafter\noexpand\csname ifpdf\endcsname
    \expandafter\noexpand\csname iffalse\endcsname
  }%
  \edef\pdftrue{%
    \let
    \expandafter\noexpand\csname ifpdf\endcsname
    \expandafter\noexpand\csname iftrue\endcsname
  }%
  \pdffalse
\else
  \csname newif\expandafter\endcsname\csname ifpdf\endcsname
\fi
%    \end{macrocode}
%    Test \cs{pdfoutput}. Is it  defined and different from
%    \cs{relax}? Someone could have used \LaTeX\ internal
%    \cs{@ifundefined}, or something else involving.
%    Notice, \cs{csname} is executed inside a group for the test
%    to cancel the side effect of \cs{csname}.
%    \begin{macrocode}
\begingroup\expandafter\expandafter\expandafter\endgroup
\expandafter\ifx\csname pdfoutput\endcsname\relax
\else
  \ifnum\pdfoutput<1 %
%    \end{macrocode}
%    \cs{pdfoutput}|=0| or negative, so not generating pdf.
%    \begin{macrocode}
  \else
    \pdftrue
  \fi
\fi
%    \end{macrocode}
%    \end{macro}
%
% \subsection{Test for fool attempts}
%
%    \begin{macrocode}
\begingroup
  \expandafter\ifx\csname pdfoutput\endcsname\relax
  \else
    \escapechar=92 %
    \edef\m{\meaning\pdfoutput}%
    \edef\p{\string\pdfoutput}%
    \ifx\m\p
    \else
      \expandafter\ifx\csname PackageWarningNoLine\endcsname\relax
        \def\PackageWarningNoLine#1#2{%
          \immediate\write16{%
            Package `#1' Warning: #2.%
          }%
        }%
      \fi
      \PackageWarningNoLine{ifpdf}{%
        Someone has redefined \string\pdfoutput%
      }%
    \fi
  \fi
\endgroup
%    \end{macrocode}
%
% \subsection{Protocol entry}
%
%     Log comment:
%    \begin{macrocode}
\begingroup
  \expandafter\ifx\csname PackageInfo\endcsname\relax
    \def\x#1#2{%
      \immediate\write-1{Package #1 Info: #2.}%
    }%
  \else
    \let\x\PackageInfo
    \expandafter\let\csname on@line\endcsname\empty
  \fi
  \x{ifpdf}{pdfTeX in PDF mode is \ifpdf\else not \fi detected}%
\endgroup
%    \end{macrocode}
%    \begin{macrocode}
\ifpdf@AtEnd%
%    \end{macrocode}
%    \begin{macrocode}
%</package>
%    \end{macrocode}
%
% \section{Test}
%
% \subsection{Catcode checks for loading}
%
%    \begin{macrocode}
%<*test1>
%    \end{macrocode}
%    \begin{macrocode}
\catcode`\{=1 %
\catcode`\}=2 %
\catcode`\#=6 %
\catcode`\@=11 %
\expandafter\ifx\csname count@\endcsname\relax
  \countdef\count@=255 %
\fi
\expandafter\ifx\csname @gobble\endcsname\relax
  \long\def\@gobble#1{}%
\fi
\expandafter\ifx\csname @firstofone\endcsname\relax
  \long\def\@firstofone#1{#1}%
\fi
\expandafter\ifx\csname loop\endcsname\relax
  \expandafter\@firstofone
\else
  \expandafter\@gobble
\fi
{%
  \def\loop#1\repeat{%
    \def\body{#1}%
    \iterate
  }%
  \def\iterate{%
    \body
      \let\next\iterate
    \else
      \let\next\relax
    \fi
    \next
  }%
  \let\repeat=\fi
}%
\def\RestoreCatcodes{}
\count@=0 %
\loop
  \edef\RestoreCatcodes{%
    \RestoreCatcodes
    \catcode\the\count@=\the\catcode\count@\relax
  }%
\ifnum\count@<255 %
  \advance\count@ 1 %
\repeat

\def\RangeCatcodeInvalid#1#2{%
  \count@=#1\relax
  \loop
    \catcode\count@=15 %
  \ifnum\count@<#2\relax
    \advance\count@ 1 %
  \repeat
}
\def\RangeCatcodeCheck#1#2#3{%
  \count@=#1\relax
  \loop
    \ifnum#3=\catcode\count@
    \else
      \errmessage{%
        Character \the\count@\space
        with wrong catcode \the\catcode\count@\space
        instead of \number#3%
      }%
    \fi
  \ifnum\count@<#2\relax
    \advance\count@ 1 %
  \repeat
}
\def\space{ }
\expandafter\ifx\csname LoadCommand\endcsname\relax
  \def\LoadCommand{\input ifpdf.sty\relax}%
\fi
\def\Test{%
  \RangeCatcodeInvalid{0}{47}%
  \RangeCatcodeInvalid{58}{64}%
  \RangeCatcodeInvalid{91}{96}%
  \RangeCatcodeInvalid{123}{255}%
  \catcode`\@=12 %
  \catcode`\\=0 %
  \catcode`\%=14 %
  \LoadCommand
  \RangeCatcodeCheck{0}{36}{15}%
  \RangeCatcodeCheck{37}{37}{14}%
  \RangeCatcodeCheck{38}{47}{15}%
  \RangeCatcodeCheck{48}{57}{12}%
  \RangeCatcodeCheck{58}{63}{15}%
  \RangeCatcodeCheck{64}{64}{12}%
  \RangeCatcodeCheck{65}{90}{11}%
  \RangeCatcodeCheck{91}{91}{15}%
  \RangeCatcodeCheck{92}{92}{0}%
  \RangeCatcodeCheck{93}{96}{15}%
  \RangeCatcodeCheck{97}{122}{11}%
  \RangeCatcodeCheck{123}{255}{15}%
  \RestoreCatcodes
}
\Test
\csname @@end\endcsname
\end
%    \end{macrocode}
%    \begin{macrocode}
%</test1>
%    \end{macrocode}
%
% \section{Installation}
%
% \subsection{Download}
%
% \paragraph{Package.} This package is available on
% CTAN\footnote{\url{ftp://ftp.ctan.org/tex-archive/}}:
% \begin{description}
% \item[\CTAN{macros/latex/contrib/oberdiek/ifpdf.dtx}] The source file.
% \item[\CTAN{macros/latex/contrib/oberdiek/ifpdf.pdf}] Documentation.
% \end{description}
%
%
% \paragraph{Bundle.} All the packages of the bundle `oberdiek'
% are also available in a TDS compliant ZIP archive. There
% the packages are already unpacked and the documentation files
% are generated. The files and directories obey the TDS standard.
% \begin{description}
% \item[\CTAN{install/macros/latex/contrib/oberdiek.tds.zip}]
% \end{description}
% \emph{TDS} refers to the standard ``A Directory Structure
% for \TeX\ Files'' (\CTAN{tds/tds.pdf}). Directories
% with \xfile{texmf} in their name are usually organized this way.
%
% \subsection{Bundle installation}
%
% \paragraph{Unpacking.} Unpack the \xfile{oberdiek.tds.zip} in the
% TDS tree (also known as \xfile{texmf} tree) of your choice.
% Example (linux):
% \begin{quote}
%   |unzip oberdiek.tds.zip -d ~/texmf|
% \end{quote}
%
% \paragraph{Script installation.}
% Check the directory \xfile{TDS:scripts/oberdiek/} for
% scripts that need further installation steps.
% Package \xpackage{attachfile2} comes with the Perl script
% \xfile{pdfatfi.pl} that should be installed in such a way
% that it can be called as \texttt{pdfatfi}.
% Example (linux):
% \begin{quote}
%   |chmod +x scripts/oberdiek/pdfatfi.pl|\\
%   |cp scripts/oberdiek/pdfatfi.pl /usr/local/bin/|
% \end{quote}
%
% \subsection{Package installation}
%
% \paragraph{Unpacking.} The \xfile{.dtx} file is a self-extracting
% \docstrip\ archive. The files are extracted by running the
% \xfile{.dtx} through \plainTeX:
% \begin{quote}
%   \verb|tex ifpdf.dtx|
% \end{quote}
%
% \paragraph{TDS.} Now the different files must be moved into
% the different directories in your installation TDS tree
% (also known as \xfile{texmf} tree):
% \begin{quote}
% \def\t{^^A
% \begin{tabular}{@{}>{\ttfamily}l@{ $\rightarrow$ }>{\ttfamily}l@{}}
%   ifpdf.sty & tex/generic/oberdiek/ifpdf.sty\\
%   ifpdf.pdf & doc/latex/oberdiek/ifpdf.pdf\\
%   test/ifpdf-test1.tex & doc/latex/oberdiek/test/ifpdf-test1.tex\\
%   ifpdf.dtx & source/latex/oberdiek/ifpdf.dtx\\
% \end{tabular}^^A
% }^^A
% \sbox0{\t}^^A
% \ifdim\wd0>\linewidth
%   \begingroup
%     \advance\linewidth by\leftmargin
%     \advance\linewidth by\rightmargin
%   \edef\x{\endgroup
%     \def\noexpand\lw{\the\linewidth}^^A
%   }\x
%   \def\lwbox{^^A
%     \leavevmode
%     \hbox to \linewidth{^^A
%       \kern-\leftmargin\relax
%       \hss
%       \usebox0
%       \hss
%       \kern-\rightmargin\relax
%     }^^A
%   }^^A
%   \ifdim\wd0>\lw
%     \sbox0{\small\t}^^A
%     \ifdim\wd0>\linewidth
%       \ifdim\wd0>\lw
%         \sbox0{\footnotesize\t}^^A
%         \ifdim\wd0>\linewidth
%           \ifdim\wd0>\lw
%             \sbox0{\scriptsize\t}^^A
%             \ifdim\wd0>\linewidth
%               \ifdim\wd0>\lw
%                 \sbox0{\tiny\t}^^A
%                 \ifdim\wd0>\linewidth
%                   \lwbox
%                 \else
%                   \usebox0
%                 \fi
%               \else
%                 \lwbox
%               \fi
%             \else
%               \usebox0
%             \fi
%           \else
%             \lwbox
%           \fi
%         \else
%           \usebox0
%         \fi
%       \else
%         \lwbox
%       \fi
%     \else
%       \usebox0
%     \fi
%   \else
%     \lwbox
%   \fi
% \else
%   \usebox0
% \fi
% \end{quote}
% If you have a \xfile{docstrip.cfg} that configures and enables \docstrip's
% TDS installing feature, then some files can already be in the right
% place, see the documentation of \docstrip.
%
% \subsection{Refresh file name databases}
%
% If your \TeX~distribution
% (\teTeX, \mikTeX, \dots) relies on file name databases, you must refresh
% these. For example, \teTeX\ users run \verb|texhash| or
% \verb|mktexlsr|.
%
% \subsection{Some details for the interested}
%
% \paragraph{Attached source.}
%
% The PDF documentation on CTAN also includes the
% \xfile{.dtx} source file. It can be extracted by
% AcrobatReader 6 or higher. Another option is \textsf{pdftk},
% e.g. unpack the file into the current directory:
% \begin{quote}
%   \verb|pdftk ifpdf.pdf unpack_files output .|
% \end{quote}
%
% \paragraph{Unpacking with \LaTeX.}
% The \xfile{.dtx} chooses its action depending on the format:
% \begin{description}
% \item[\plainTeX:] Run \docstrip\ and extract the files.
% \item[\LaTeX:] Generate the documentation.
% \end{description}
% If you insist on using \LaTeX\ for \docstrip\ (really,
% \docstrip\ does not need \LaTeX), then inform the autodetect routine
% about your intention:
% \begin{quote}
%   \verb|latex \let\install=y\input{ifpdf.dtx}|
% \end{quote}
% Do not forget to quote the argument according to the demands
% of your shell.
%
% \paragraph{Generating the documentation.}
% You can use both the \xfile{.dtx} or the \xfile{.drv} to generate
% the documentation. The process can be configured by the
% configuration file \xfile{ltxdoc.cfg}. For instance, put this
% line into this file, if you want to have A4 as paper format:
% \begin{quote}
%   \verb|\PassOptionsToClass{a4paper}{article}|
% \end{quote}
% An example follows how to generate the
% documentation with pdf\LaTeX:
% \begin{quote}
%\begin{verbatim}
%pdflatex ifpdf.dtx
%makeindex -s gind.ist ifpdf.idx
%pdflatex ifpdf.dtx
%makeindex -s gind.ist ifpdf.idx
%pdflatex ifpdf.dtx
%\end{verbatim}
% \end{quote}
%
% \section{Catalogue}
%
% The following XML file can be used as source for the
% \href{http://mirror.ctan.org/help/Catalogue/catalogue.html}{\TeX\ Catalogue}.
% The elements \texttt{caption} and \texttt{description} are imported
% from the original XML file from the Catalogue.
% The name of the XML file in the Catalogue is \xfile{ifpdf.xml}.
%    \begin{macrocode}
%<*catalogue>
<?xml version='1.0' encoding='us-ascii'?>
<!DOCTYPE entry SYSTEM 'catalogue.dtd'>
<entry datestamp='$Date$' modifier='$Author$' id='ifpdf'>
  <name>ifpdf</name>
  <caption>Provides the \ifpdf conditional.</caption>
  <authorref id='auth:oberdiek'/>
  <copyright owner='Heiko Oberdiek' year='2001,2005-2011'/>
  <license type='lppl1.3'/>
  <version number='2.3'/>
  <description>
      This package looks for pdfTeX in pdf mode and implements and
      sets the switch <tt>\ifpdf</tt>. The detection is based on the
      value of <tt>\pdfoutput</tt> (which the package will not
      change).  The package works with plain or LaTeX formats.  To use
      it with LaTeX simply <tt>\usepackage{ifpdf}</tt>. Then use
      <tt>\ifpdf ... \else ... \fi</tt>.
    <p/>
    The package is part of the <xref refid='oberdiek'>oberdiek</xref> bundle.
  </description>
  <documentation details='Package documentation'
      href='ctan:/macros/latex/contrib/oberdiek/ifpdf.pdf'/>
  <ctan file='true' path='/macros/latex/contrib/oberdiek/ifpdf.dtx'/>
  <miktex location='oberdiek'/>
  <texlive location='oberdiek'/>
  <install path='/macros/latex/contrib/oberdiek/oberdiek.tds.zip'/>
</entry>
%</catalogue>
%    \end{macrocode}
%
% \begin{History}
%   \begin{Version}{2001/06/14 v1.0}
%   \item
%     First public version.
%   \end{Version}
%   \begin{Version}{2001/07/14 v1.1}
%   \item
%     Documentation addition: global options
%   \end{Version}
%   \begin{Version}{2001/09/26 v1.2}
%   \item
%     Documentation typo corrected.
%   \item
%     Version number corrected.
%   \item
%     Line number in log entry removed.
%   \end{Version}
%   \begin{Version}{2005/07/22 v1.3}
%   \item
%     Some source code comments from Robin Fairbairns added.
%   \item
%     Bug fix for negative values of \cs{pdfoutput}
%     (Oleg Katsitadze)
%   \item
%     LPPL 1.3
%   \item
%     Installation section with locations added.
%   \end{Version}
%   \begin{Version}{2006/02/20 v1.4}
%   \item
%     DTX framework.
%   \item
%     More robust check in case of undefined \cs{pdfoutput}.
%   \item
%     Extended documentation.
%   \end{Version}
%   \begin{Version}{2007/09/09 v1.5}
%   \item
%     Catcode settings added.
%   \end{Version}
%   \begin{Version}{2007/12/12 v1.6}
%   \item
%     Minor update.
%   \end{Version}
%   \begin{Version}{2008/12/12 v1.7}
%   \item
%     Fix in documentation for \cs{boolean} (found by S. Venkataraman).
%   \item
%     Code is not changed.
%   \end{Version}
%   \begin{Version}{2009/04/10 v2.0}
%   \item
%     Support for \LuaTeX\ 0.40 added.
%   \item
%     Checks, whether \cs{pdfoutput} was changed.
%   \end{Version}
%   \begin{Version}{2010/01/28 v2.1}
%   \item
%     Compatibility to ini\TeX\ added.
%   \end{Version}
%   \begin{Version}{2010/09/13 v2.2}
%   \item
%     Code with \cs{escapechar} rewritten because of
%     LuaTeX bug: \cs{escapechar} does not work.
%   \end{Version}
%   \begin{Version}{2011/01/30 v2.3}
%   \item
%     Already loaded package files are not input in \hologo{plainTeX}.
%   \end{Version}
% \end{History}
%
% \PrintIndex
%
% \Finale
\endinput
|
% \end{quote}
% Do not forget to quote the argument according to the demands
% of your shell.
%
% \paragraph{Generating the documentation.}
% You can use both the \xfile{.dtx} or the \xfile{.drv} to generate
% the documentation. The process can be configured by the
% configuration file \xfile{ltxdoc.cfg}. For instance, put this
% line into this file, if you want to have A4 as paper format:
% \begin{quote}
%   \verb|\PassOptionsToClass{a4paper}{article}|
% \end{quote}
% An example follows how to generate the
% documentation with pdf\LaTeX:
% \begin{quote}
%\begin{verbatim}
%pdflatex ifpdf.dtx
%makeindex -s gind.ist ifpdf.idx
%pdflatex ifpdf.dtx
%makeindex -s gind.ist ifpdf.idx
%pdflatex ifpdf.dtx
%\end{verbatim}
% \end{quote}
%
% \section{Catalogue}
%
% The following XML file can be used as source for the
% \href{http://mirror.ctan.org/help/Catalogue/catalogue.html}{\TeX\ Catalogue}.
% The elements \texttt{caption} and \texttt{description} are imported
% from the original XML file from the Catalogue.
% The name of the XML file in the Catalogue is \xfile{ifpdf.xml}.
%    \begin{macrocode}
%<*catalogue>
<?xml version='1.0' encoding='us-ascii'?>
<!DOCTYPE entry SYSTEM 'catalogue.dtd'>
<entry datestamp='$Date$' modifier='$Author$' id='ifpdf'>
  <name>ifpdf</name>
  <caption>Provides the \ifpdf conditional.</caption>
  <authorref id='auth:oberdiek'/>
  <copyright owner='Heiko Oberdiek' year='2001,2005-2011'/>
  <license type='lppl1.3'/>
  <version number='2.3'/>
  <description>
      This package looks for pdfTeX in pdf mode and implements and
      sets the switch <tt>\ifpdf</tt>. The detection is based on the
      value of <tt>\pdfoutput</tt> (which the package will not
      change).  The package works with plain or LaTeX formats.  To use
      it with LaTeX simply <tt>\usepackage{ifpdf}</tt>. Then use
      <tt>\ifpdf ... \else ... \fi</tt>.
    <p/>
    The package is part of the <xref refid='oberdiek'>oberdiek</xref> bundle.
  </description>
  <documentation details='Package documentation'
      href='ctan:/macros/latex/contrib/oberdiek/ifpdf.pdf'/>
  <ctan file='true' path='/macros/latex/contrib/oberdiek/ifpdf.dtx'/>
  <miktex location='oberdiek'/>
  <texlive location='oberdiek'/>
  <install path='/macros/latex/contrib/oberdiek/oberdiek.tds.zip'/>
</entry>
%</catalogue>
%    \end{macrocode}
%
% \begin{History}
%   \begin{Version}{2001/06/14 v1.0}
%   \item
%     First public version.
%   \end{Version}
%   \begin{Version}{2001/07/14 v1.1}
%   \item
%     Documentation addition: global options
%   \end{Version}
%   \begin{Version}{2001/09/26 v1.2}
%   \item
%     Documentation typo corrected.
%   \item
%     Version number corrected.
%   \item
%     Line number in log entry removed.
%   \end{Version}
%   \begin{Version}{2005/07/22 v1.3}
%   \item
%     Some source code comments from Robin Fairbairns added.
%   \item
%     Bug fix for negative values of \cs{pdfoutput}
%     (Oleg Katsitadze)
%   \item
%     LPPL 1.3
%   \item
%     Installation section with locations added.
%   \end{Version}
%   \begin{Version}{2006/02/20 v1.4}
%   \item
%     DTX framework.
%   \item
%     More robust check in case of undefined \cs{pdfoutput}.
%   \item
%     Extended documentation.
%   \end{Version}
%   \begin{Version}{2007/09/09 v1.5}
%   \item
%     Catcode settings added.
%   \end{Version}
%   \begin{Version}{2007/12/12 v1.6}
%   \item
%     Minor update.
%   \end{Version}
%   \begin{Version}{2008/12/12 v1.7}
%   \item
%     Fix in documentation for \cs{boolean} (found by S. Venkataraman).
%   \item
%     Code is not changed.
%   \end{Version}
%   \begin{Version}{2009/04/10 v2.0}
%   \item
%     Support for \LuaTeX\ 0.40 added.
%   \item
%     Checks, whether \cs{pdfoutput} was changed.
%   \end{Version}
%   \begin{Version}{2010/01/28 v2.1}
%   \item
%     Compatibility to ini\TeX\ added.
%   \end{Version}
%   \begin{Version}{2010/09/13 v2.2}
%   \item
%     Code with \cs{escapechar} rewritten because of
%     LuaTeX bug: \cs{escapechar} does not work.
%   \end{Version}
%   \begin{Version}{2011/01/30 v2.3}
%   \item
%     Already loaded package files are not input in \hologo{plainTeX}.
%   \end{Version}
% \end{History}
%
% \PrintIndex
%
% \Finale
\endinput
|
% \end{quote}
% Do not forget to quote the argument according to the demands
% of your shell.
%
% \paragraph{Generating the documentation.}
% You can use both the \xfile{.dtx} or the \xfile{.drv} to generate
% the documentation. The process can be configured by the
% configuration file \xfile{ltxdoc.cfg}. For instance, put this
% line into this file, if you want to have A4 as paper format:
% \begin{quote}
%   \verb|\PassOptionsToClass{a4paper}{article}|
% \end{quote}
% An example follows how to generate the
% documentation with pdf\LaTeX:
% \begin{quote}
%\begin{verbatim}
%pdflatex ifpdf.dtx
%makeindex -s gind.ist ifpdf.idx
%pdflatex ifpdf.dtx
%makeindex -s gind.ist ifpdf.idx
%pdflatex ifpdf.dtx
%\end{verbatim}
% \end{quote}
%
% \section{Catalogue}
%
% The following XML file can be used as source for the
% \href{http://mirror.ctan.org/help/Catalogue/catalogue.html}{\TeX\ Catalogue}.
% The elements \texttt{caption} and \texttt{description} are imported
% from the original XML file from the Catalogue.
% The name of the XML file in the Catalogue is \xfile{ifpdf.xml}.
%    \begin{macrocode}
%<*catalogue>
<?xml version='1.0' encoding='us-ascii'?>
<!DOCTYPE entry SYSTEM 'catalogue.dtd'>
<entry datestamp='$Date$' modifier='$Author$' id='ifpdf'>
  <name>ifpdf</name>
  <caption>Provides the \ifpdf conditional.</caption>
  <authorref id='auth:oberdiek'/>
  <copyright owner='Heiko Oberdiek' year='2001,2005-2011'/>
  <license type='lppl1.3'/>
  <version number='2.3'/>
  <description>
      This package looks for pdfTeX in pdf mode and implements and
      sets the switch <tt>\ifpdf</tt>. The detection is based on the
      value of <tt>\pdfoutput</tt> (which the package will not
      change).  The package works with plain or LaTeX formats.  To use
      it with LaTeX simply <tt>\usepackage{ifpdf}</tt>. Then use
      <tt>\ifpdf ... \else ... \fi</tt>.
    <p/>
    The package is part of the <xref refid='oberdiek'>oberdiek</xref> bundle.
  </description>
  <documentation details='Package documentation'
      href='ctan:/macros/latex/contrib/oberdiek/ifpdf.pdf'/>
  <ctan file='true' path='/macros/latex/contrib/oberdiek/ifpdf.dtx'/>
  <miktex location='oberdiek'/>
  <texlive location='oberdiek'/>
  <install path='/macros/latex/contrib/oberdiek/oberdiek.tds.zip'/>
</entry>
%</catalogue>
%    \end{macrocode}
%
% \begin{History}
%   \begin{Version}{2001/06/14 v1.0}
%   \item
%     First public version.
%   \end{Version}
%   \begin{Version}{2001/07/14 v1.1}
%   \item
%     Documentation addition: global options
%   \end{Version}
%   \begin{Version}{2001/09/26 v1.2}
%   \item
%     Documentation typo corrected.
%   \item
%     Version number corrected.
%   \item
%     Line number in log entry removed.
%   \end{Version}
%   \begin{Version}{2005/07/22 v1.3}
%   \item
%     Some source code comments from Robin Fairbairns added.
%   \item
%     Bug fix for negative values of \cs{pdfoutput}
%     (Oleg Katsitadze)
%   \item
%     LPPL 1.3
%   \item
%     Installation section with locations added.
%   \end{Version}
%   \begin{Version}{2006/02/20 v1.4}
%   \item
%     DTX framework.
%   \item
%     More robust check in case of undefined \cs{pdfoutput}.
%   \item
%     Extended documentation.
%   \end{Version}
%   \begin{Version}{2007/09/09 v1.5}
%   \item
%     Catcode settings added.
%   \end{Version}
%   \begin{Version}{2007/12/12 v1.6}
%   \item
%     Minor update.
%   \end{Version}
%   \begin{Version}{2008/12/12 v1.7}
%   \item
%     Fix in documentation for \cs{boolean} (found by S. Venkataraman).
%   \item
%     Code is not changed.
%   \end{Version}
%   \begin{Version}{2009/04/10 v2.0}
%   \item
%     Support for \LuaTeX\ 0.40 added.
%   \item
%     Checks, whether \cs{pdfoutput} was changed.
%   \end{Version}
%   \begin{Version}{2010/01/28 v2.1}
%   \item
%     Compatibility to ini\TeX\ added.
%   \end{Version}
%   \begin{Version}{2010/09/13 v2.2}
%   \item
%     Code with \cs{escapechar} rewritten because of
%     LuaTeX bug: \cs{escapechar} does not work.
%   \end{Version}
%   \begin{Version}{2011/01/30 v2.3}
%   \item
%     Already loaded package files are not input in \hologo{plainTeX}.
%   \end{Version}
% \end{History}
%
% \PrintIndex
%
% \Finale
\endinput

%        (quote the arguments according to the demands of your shell)
%
% Documentation:
%    (a) If ifpdf.drv is present:
%           latex ifpdf.drv
%    (b) Without ifpdf.drv:
%           latex ifpdf.dtx; ...
%    The class ltxdoc loads the configuration file ltxdoc.cfg
%    if available. Here you can specify further options, e.g.
%    use A4 as paper format:
%       \PassOptionsToClass{a4paper}{article}
%
%    Programm calls to get the documentation (example):
%       pdflatex ifpdf.dtx
%       makeindex -s gind.ist ifpdf.idx
%       pdflatex ifpdf.dtx
%       makeindex -s gind.ist ifpdf.idx
%       pdflatex ifpdf.dtx
%
% Installation:
%    TDS:tex/generic/oberdiek/ifpdf.sty
%    TDS:doc/latex/oberdiek/ifpdf.pdf
%    TDS:doc/latex/oberdiek/test/ifpdf-test1.tex
%    TDS:source/latex/oberdiek/ifpdf.dtx
%
%<*ignore>
\begingroup
  \catcode123=1 %
  \catcode125=2 %
  \def\x{LaTeX2e}%
\expandafter\endgroup
\ifcase 0\ifx\install y1\fi\expandafter
         \ifx\csname processbatchFile\endcsname\relax\else1\fi
         \ifx\fmtname\x\else 1\fi\relax
\else\csname fi\endcsname
%</ignore>
%<*install>
\input docstrip.tex
\Msg{************************************************************************}
\Msg{* Installation}
\Msg{* Package: ifpdf 2011/01/30 v2.3 Provides the ifpdf switch (HO)}
\Msg{************************************************************************}

\keepsilent
\askforoverwritefalse

\let\MetaPrefix\relax
\preamble

This is a generated file.

Project: ifpdf
Version: 2011/01/30 v2.3

Copyright (C) 2001, 2005-2011 by
   Heiko Oberdiek <heiko.oberdiek at googlemail.com>

This work may be distributed and/or modified under the
conditions of the LaTeX Project Public License, either
version 1.3c of this license or (at your option) any later
version. This version of this license is in
   http://www.latex-project.org/lppl/lppl-1-3c.txt
and the latest version of this license is in
   http://www.latex-project.org/lppl.txt
and version 1.3 or later is part of all distributions of
LaTeX version 2005/12/01 or later.

This work has the LPPL maintenance status "maintained".

This Current Maintainer of this work is Heiko Oberdiek.

The Base Interpreter refers to any `TeX-Format',
because some files are installed in TDS:tex/generic//.

This work consists of the main source file ifpdf.dtx
and the derived files
   ifpdf.sty, ifpdf.pdf, ifpdf.ins, ifpdf.drv, ifpdf-test1.tex.

\endpreamble
\let\MetaPrefix\DoubleperCent

\generate{%
  \file{ifpdf.ins}{\from{ifpdf.dtx}{install}}%
  \file{ifpdf.drv}{\from{ifpdf.dtx}{driver}}%
  \usedir{tex/generic/oberdiek}%
  \file{ifpdf.sty}{\from{ifpdf.dtx}{package}}%
  \usedir{doc/latex/oberdiek/test}%
  \file{ifpdf-test1.tex}{\from{ifpdf.dtx}{test1}}%
  \nopreamble
  \nopostamble
  \usedir{source/latex/oberdiek/catalogue}%
  \file{ifpdf.xml}{\from{ifpdf.dtx}{catalogue}}%
}

\catcode32=13\relax% active space
\let =\space%
\Msg{************************************************************************}
\Msg{*}
\Msg{* To finish the installation you have to move the following}
\Msg{* file into a directory searched by TeX:}
\Msg{*}
\Msg{*     ifpdf.sty}
\Msg{*}
\Msg{* To produce the documentation run the file `ifpdf.drv'}
\Msg{* through LaTeX.}
\Msg{*}
\Msg{* Happy TeXing!}
\Msg{*}
\Msg{************************************************************************}

\endbatchfile
%</install>
%<*ignore>
\fi
%</ignore>
%<*driver>
\NeedsTeXFormat{LaTeX2e}
\ProvidesFile{ifpdf.drv}%
  [2011/01/30 v2.3 Provides the ifpdf switch (HO)]%
\documentclass{ltxdoc}
\usepackage{holtxdoc}[2011/11/22]
\begin{document}
  \DocInput{ifpdf.dtx}%
\end{document}
%</driver>
% \fi
%
% \CheckSum{600}
%
% \CharacterTable
%  {Upper-case    \A\B\C\D\E\F\G\H\I\J\K\L\M\N\O\P\Q\R\S\T\U\V\W\X\Y\Z
%   Lower-case    \a\b\c\d\e\f\g\h\i\j\k\l\m\n\o\p\q\r\s\t\u\v\w\x\y\z
%   Digits        \0\1\2\3\4\5\6\7\8\9
%   Exclamation   \!     Double quote  \"     Hash (number) \#
%   Dollar        \$     Percent       \%     Ampersand     \&
%   Acute accent  \'     Left paren    \(     Right paren   \)
%   Asterisk      \*     Plus          \+     Comma         \,
%   Minus         \-     Point         \.     Solidus       \/
%   Colon         \:     Semicolon     \;     Less than     \<
%   Equals        \=     Greater than  \>     Question mark \?
%   Commercial at \@     Left bracket  \[     Backslash     \\
%   Right bracket \]     Circumflex    \^     Underscore    \_
%   Grave accent  \`     Left brace    \{     Vertical bar  \|
%   Right brace   \}     Tilde         \~}
%
% \GetFileInfo{ifpdf.drv}
%
% \title{The \xpackage{ifpdf} package}
% \date{2011/01/30 v2.3}
% \author{Heiko Oberdiek\\\xemail{heiko.oberdiek at googlemail.com}}
%
% \maketitle
%
% \begin{abstract}
% This package looks for \pdfTeX\ in pdf mode and
% implements and sets the switch \cs{ifpdf}.
% The detection is based on \cs{pdfoutput} and
% the package will not change this value.
% It works with plain or \LaTeX\ formats.
% \end{abstract}
%
% \tableofcontents
%
% \hypersetup{bookmarksopenlevel=2}
% \section{Documentation}
%
% \subsection{Introduction}
%
% It is commonly known that H\`an Th\^e Th\`anh's \pdfTeX\ generates
% PDF output directly and many people uses \pdfTeX\ for this purpose.
% However the DVI output was never thrown away. In contrary, he
% new features for typesetting that works in both PDF and DVI mode.
%
% In the meantime many \TeX\ distributions replace the traditional
% \TeX\ binary with \pdfTeX. Then, for example, called as |latex|
% \pdfTeX\ works in DVI mode with the \LaTeX\ format preloaded,
% called as |pdflatex| \pdfTeX\ starts in PDF mode.
%
% Often packages or users want to know, whether the current document
% is typset by \pdfTeX\ in PDF mode, because the different modes
% have different capabilities (color setting, graphics inclusion,
% \dots). For this purpose \pdfTeX's
% \cs{pdfoutput} can be asked.
%
% As regulary reader of \TeX\ newsgroups and mailing lists I could
% observe many problems with this task. Common errors are:
% \begin{itemize}
% \item
%   \pdfTeX\ has \emph{two} modes. Using \pdfTeX\ does not mean
%   that the user always want to have PDF mode. For example, the
%   PostScript support is better in DVI mode in conjunction with a
%   PostScript aware DVI driver (e.g. dvips). Also the additional
%   typesetting features are mode independent and also available
%   in DVI mode.
% \item
%   \LaTeX's \cs{@ifundefined} inherited the side effect from
%   \cs{csname}. Unknown commands are defined with the meaning
%   of \cs{relax}. If it is checked, whether \cs{pdfoutput} is
%   defined, then this should not be forgotten.
% \item
%   Having \cs{pdfoutput} does not automatically mean PDF mode.
%   Also the value of \cs{pdfoutput} must be asked.
% \item
%   \cs{pdfoutput} must not be destroyed in some way. Later
%   code and packages are fooled then and will perhaps make
%   wrong decisions. For example they may drop support for
%   PDF mode, because they do not know that \pdfTeX\ is
%   running at all.
% \end{itemize}
%
% Robin Fairbairns provides an entry for this topic in his
% excellent FAQ (\url{http://www.tex.ac.uk/faq}):
% \href{http://www.tex.ac.uk/cgi-bin/texfaq2html?label=ifpdf}^^A
% {Am I using PDFTeX?}
%
% \subsection{Usage}
%
% The package \xpackage{ifpdf} can be used with both \plainTeX\
% and \LaTeX:
% \begin{description}
% \item[\plainTeX:] |\input ifpdf.sty|
% \item[\LaTeXe:]   |\usepackage{ifpdf}|
% \end{description}
%
% \DescribeMacro{\ifpdf}
% The package provides the switch \cs{ifpdf}:
% \begin{quote}
%   |\ifpdf|\\
%   \hspace*{1.5em}\dots\ do things, if \pdfTeX\ is running in pdf mode \dots\\
%   |\else|\\
%   \hspace*{1.5em}\dots\ other \TeX\ or \pdfTeX\ in dvi mode \dots\\
%   |\fi|
% \end{quote}
% Users of the package \xpackage{ifthen} can use the switch as boolean:
% \begin{quote}
%   |\boolean{pdf}|
% \end{quote}
% The package can also be used to set global
% documentclass options:
% \begin{quote}
%   |\RequirePackage{ifpdf}|\\
%   |\ifpdf|\\
%   \hspace*{1.5em}|\documentclass[pdftex,|\dots|]{|\dots|}|\\
%   |\else|\\
%   \hspace*{1.5em}|\documentclass[|\dots|]{|\dots|}|\\
%   |\fi|
% \end{quote}
%
% \subsection{Specification}
%
% The package have the following properties:
% \begin{itemize}
% \item It asks the setting of \cs{pdfoutput} for detecting
%   \pdfTeX\ in PDF mode.
% \item It never changes \cs{pdfoutput}.
% \item If \cs{pdfoutput} is undefined or has the meaning \cs{relax},
%   but the engine provides the primitive \cs{pdfoutput},
%   then \cs{pdfoutput} is enabled or restored if possible
%   (only \LuaTeX, version 0.36.0 or higher).
% \item It can be used with many formats including
%   \plainTeX\ and \LaTeX.
% \end{itemize}
% The mode detection implements the following algorithm:
% \begin{quote}
% \def\={${:}{=}$}
% \def\>{\null\qquad\ignorespaces}
% \def\comment#1{^^A
%   \textrm{\itshape\% #1}^^A
% }
% \sffamily
% if undefined(\cs{pdfoutput})\\
% \> \cs{ifpdf} \= \cs{iffalse} \comment{\pdfTeX is not running}\\
% else\\
% \> if \cs{pdfoutput} $\leq$ 0\\
% \>\> \cs{ifpdf} \= \cs{iffalse} \comment{\pdfTeX\ in DVI mode}\\
% \> else\\
% \>\> \cs{ifpdf} \= \cs{iftrue} \comment{\pdfTeX\ in PDF mode}\\
% \> fi\\
% fi
% \end{quote}
% The function \textsf{undefined} checks both cases, undefined command
% and \cs{relax}.
%
% \hypersetup{bookmarksopenlevel=1}
%
% \StopEventually{
% }
%
% \section{Implementation}
%
%    \begin{macrocode}
%<*package>
%    \end{macrocode}
%
% \subsection{Reload check and package identification}
%    Reload check, especially if the package is not used with \LaTeX.
%    \begin{macrocode}
\begingroup\catcode61\catcode48\catcode32=10\relax%
  \catcode13=5 % ^^M
  \endlinechar=13 %
  \catcode35=6 % #
  \catcode39=12 % '
  \catcode44=12 % ,
  \catcode45=12 % -
  \catcode46=12 % .
  \catcode58=12 % :
  \catcode64=11 % @
  \catcode123=1 % {
  \catcode125=2 % }
  \expandafter\let\expandafter\x\csname ver@ifpdf.sty\endcsname
  \ifx\x\relax % plain-TeX, first loading
  \else
    \def\empty{}%
    \ifx\x\empty % LaTeX, first loading,
      % variable is initialized, but \ProvidesPackage not yet seen
    \else
      \expandafter\ifx\csname PackageInfo\endcsname\relax
        \def\x#1#2{%
          \immediate\write-1{Package #1 Info: #2.}%
        }%
      \else
        \def\x#1#2{\PackageInfo{#1}{#2, stopped}}%
      \fi
      \x{ifpdf}{The package is already loaded}%
      \aftergroup\endinput
    \fi
  \fi
\endgroup%
%    \end{macrocode}
%    Package identification:
%    \begin{macrocode}
\begingroup\catcode61\catcode48\catcode32=10\relax%
  \catcode13=5 % ^^M
  \endlinechar=13 %
  \catcode35=6 % #
  \catcode39=12 % '
  \catcode40=12 % (
  \catcode41=12 % )
  \catcode44=12 % ,
  \catcode45=12 % -
  \catcode46=12 % .
  \catcode47=12 % /
  \catcode58=12 % :
  \catcode64=11 % @
  \catcode91=12 % [
  \catcode93=12 % ]
  \catcode123=1 % {
  \catcode125=2 % }
  \expandafter\ifx\csname ProvidesPackage\endcsname\relax
    \def\x#1#2#3[#4]{\endgroup
      \immediate\write-1{Package: #3 #4}%
      \xdef#1{#4}%
    }%
  \else
    \def\x#1#2[#3]{\endgroup
      #2[{#3}]%
      \ifx#1\@undefined
        \xdef#1{#3}%
      \fi
      \ifx#1\relax
        \xdef#1{#3}%
      \fi
    }%
  \fi
\expandafter\x\csname ver@ifpdf.sty\endcsname
\ProvidesPackage{ifpdf}%
  [2011/01/30 v2.3 Provides the ifpdf switch (HO)]%
%    \end{macrocode}
%
% \subsection{Catcodes}
%
%    \begin{macrocode}
\begingroup\catcode61\catcode48\catcode32=10\relax%
  \catcode13=5 % ^^M
  \endlinechar=13 %
  \catcode123=1 % {
  \catcode125=2 % }
  \catcode64=11 % @
  \def\x{\endgroup
    \expandafter\edef\csname ifpdf@AtEnd\endcsname{%
      \endlinechar=\the\endlinechar\relax
      \catcode13=\the\catcode13\relax
      \catcode32=\the\catcode32\relax
      \catcode35=\the\catcode35\relax
      \catcode61=\the\catcode61\relax
      \catcode64=\the\catcode64\relax
      \catcode123=\the\catcode123\relax
      \catcode125=\the\catcode125\relax
    }%
  }%
\x\catcode61\catcode48\catcode32=10\relax%
\catcode13=5 % ^^M
\endlinechar=13 %
\catcode35=6 % #
\catcode64=11 % @
\catcode123=1 % {
\catcode125=2 % }
\def\TMP@EnsureCode#1#2{%
  \edef\ifpdf@AtEnd{%
    \ifpdf@AtEnd
    \catcode#1=\the\catcode#1\relax
  }%
  \catcode#1=#2\relax
}
\TMP@EnsureCode{10}{12}% ^^J
\TMP@EnsureCode{39}{12}% '
\TMP@EnsureCode{40}{12}% (
\TMP@EnsureCode{41}{12}% )
\TMP@EnsureCode{44}{12}% ,
\TMP@EnsureCode{45}{12}% -
\TMP@EnsureCode{46}{12}% .
\TMP@EnsureCode{47}{12}% /
\TMP@EnsureCode{58}{12}% :
\TMP@EnsureCode{60}{12}% <
\TMP@EnsureCode{91}{12}% [
\TMP@EnsureCode{93}{12}% ]
\TMP@EnsureCode{94}{7}% ^
\TMP@EnsureCode{96}{12}% `
\edef\ifpdf@AtEnd{\ifpdf@AtEnd\noexpand\endinput}
%    \end{macrocode}
%
% \subsection{Check for previously defined \cs{ifpdf}}
%
%    \begin{macrocode}
\begingroup
  \expandafter\ifx\csname ifpdf\endcsname\relax
  \else
    \edef\i/{\expandafter\string\csname ifpdf\endcsname}%
    \expandafter\ifx\csname PackageError\endcsname\relax
      \def\x#1#2{%
        \edef\z{#2}%
        \expandafter\errhelp\expandafter{\z}%
        \errmessage{Package ifpdf Error: #1}%
      }%
      \def\y{^^J}%
      \newlinechar=10 %
    \else
      \def\x#1#2{%
        \PackageError{ifpdf}{#1}{#2}%
      }%
      \def\y{\MessageBreak}%
    \fi
    \x{Name clash, \i/ is already defined}{%
      Incompatible versions of \i/ can cause problems,\y
      therefore package loading is aborted.%
    }%
    \endgroup
    \expandafter\ifpdf@AtEnd
  \fi%
\endgroup
%    \end{macrocode}
%
% \subsection{\cs{pdfoutput} and \LuaTeX}
%
%    It might happen, that \LuaTeX\ is running, but \cs{pdfoutput}
%    does not exist. In version 0.40 only \cs{directlua}
%    is available at startup time. The enabling Lua function was
%    already added in version 0.36. Thus we can ignore older versions,
%    here \cs{pdfoutput} is available at startup time.
%
%    \begin{macrocode}
\begingroup
  \def\skip#1\relax\begingroup{}%
  \expandafter\ifx\csname pdfoutput\endcsname\relax
  \else
    \expandafter\skip
  \fi
  \expandafter\ifx\csname directlua\endcsname\relax
    \expandafter\skip
  \fi
\endgroup
\begingroup\expandafter\expandafter\expandafter\endgroup
\expandafter\ifx\csname RequirePackage\endcsname\relax
  \def\TMP@RequirePackage#1[#2]{%
    \begingroup\expandafter\expandafter\expandafter\endgroup
    \expandafter\ifx\csname ver@#1.sty\endcsname\relax
      \input #1.sty\relax
    \fi
  }%
  \TMP@RequirePackage{ifluatex}[2009/04/10]%
\else
  \RequirePackage{ifluatex}[2009/04/10]%
\fi
\ifluatex
  \ifnum\luatexversion<36 %
%    \end{macrocode}
%    Unhappily \LuaTeX's \cs{primitive} (derived from
%    pdfTeX's \cs{pdfprimitive}) cannot be used:
%    \begin{quote}
%    |\protected\gdef\pdfoutput{\primitive\pdfoutput}|
%    \end{quote}
%    Setting a value works, but getting fails,
%    because \TeX\ does no longer see it as number. It
%    is unexpandable and breaks numerical contexts.
%    This was fixed in \hologo{pdfTeX} 1.40.10
%    (bugfix \#4289: ``|\primitive\pdfoutput| cannot be queried'').
%    \begin{macrocode}
  \else
    \begingroup
      \directlua{tex.enableprimitives('ifpdf', {'pdfoutput'})}%
      \global\let\pdfoutput\ifpdfpdfoutput
    \endgroup
  \fi
\fi
\relax\begingroup\endgroup
%    \end{macrocode}
%
% \subsection{\cs{ifpdf}}
%
%    \begin{macro}{\ifpdf}
%    Create and set the switch. \cs{newif} initializes the
%    switch with \cs{iffalse}. \cs{newif} is \cs{outer}
%    in \plainTeX.
%    \begin{macrocode}
\begingroup\expandafter\expandafter\expandafter\endgroup
\expandafter\ifx\csname newif\endcsname\relax
  \edef\pdffalse{%
    \let
    \expandafter\noexpand\csname ifpdf\endcsname
    \expandafter\noexpand\csname iffalse\endcsname
  }%
  \edef\pdftrue{%
    \let
    \expandafter\noexpand\csname ifpdf\endcsname
    \expandafter\noexpand\csname iftrue\endcsname
  }%
  \pdffalse
\else
  \csname newif\expandafter\endcsname\csname ifpdf\endcsname
\fi
%    \end{macrocode}
%    Test \cs{pdfoutput}. Is it  defined and different from
%    \cs{relax}? Someone could have used \LaTeX\ internal
%    \cs{@ifundefined}, or something else involving.
%    Notice, \cs{csname} is executed inside a group for the test
%    to cancel the side effect of \cs{csname}.
%    \begin{macrocode}
\begingroup\expandafter\expandafter\expandafter\endgroup
\expandafter\ifx\csname pdfoutput\endcsname\relax
\else
  \ifnum\pdfoutput<1 %
%    \end{macrocode}
%    \cs{pdfoutput}|=0| or negative, so not generating pdf.
%    \begin{macrocode}
  \else
    \pdftrue
  \fi
\fi
%    \end{macrocode}
%    \end{macro}
%
% \subsection{Test for fool attempts}
%
%    \begin{macrocode}
\begingroup
  \expandafter\ifx\csname pdfoutput\endcsname\relax
  \else
    \escapechar=92 %
    \edef\m{\meaning\pdfoutput}%
    \edef\p{\string\pdfoutput}%
    \ifx\m\p
    \else
      \expandafter\ifx\csname PackageWarningNoLine\endcsname\relax
        \def\PackageWarningNoLine#1#2{%
          \immediate\write16{%
            Package `#1' Warning: #2.%
          }%
        }%
      \fi
      \PackageWarningNoLine{ifpdf}{%
        Someone has redefined \string\pdfoutput%
      }%
    \fi
  \fi
\endgroup
%    \end{macrocode}
%
% \subsection{Protocol entry}
%
%     Log comment:
%    \begin{macrocode}
\begingroup
  \expandafter\ifx\csname PackageInfo\endcsname\relax
    \def\x#1#2{%
      \immediate\write-1{Package #1 Info: #2.}%
    }%
  \else
    \let\x\PackageInfo
    \expandafter\let\csname on@line\endcsname\empty
  \fi
  \x{ifpdf}{pdfTeX in PDF mode is \ifpdf\else not \fi detected}%
\endgroup
%    \end{macrocode}
%    \begin{macrocode}
\ifpdf@AtEnd%
%    \end{macrocode}
%    \begin{macrocode}
%</package>
%    \end{macrocode}
%
% \section{Test}
%
% \subsection{Catcode checks for loading}
%
%    \begin{macrocode}
%<*test1>
%    \end{macrocode}
%    \begin{macrocode}
\catcode`\{=1 %
\catcode`\}=2 %
\catcode`\#=6 %
\catcode`\@=11 %
\expandafter\ifx\csname count@\endcsname\relax
  \countdef\count@=255 %
\fi
\expandafter\ifx\csname @gobble\endcsname\relax
  \long\def\@gobble#1{}%
\fi
\expandafter\ifx\csname @firstofone\endcsname\relax
  \long\def\@firstofone#1{#1}%
\fi
\expandafter\ifx\csname loop\endcsname\relax
  \expandafter\@firstofone
\else
  \expandafter\@gobble
\fi
{%
  \def\loop#1\repeat{%
    \def\body{#1}%
    \iterate
  }%
  \def\iterate{%
    \body
      \let\next\iterate
    \else
      \let\next\relax
    \fi
    \next
  }%
  \let\repeat=\fi
}%
\def\RestoreCatcodes{}
\count@=0 %
\loop
  \edef\RestoreCatcodes{%
    \RestoreCatcodes
    \catcode\the\count@=\the\catcode\count@\relax
  }%
\ifnum\count@<255 %
  \advance\count@ 1 %
\repeat

\def\RangeCatcodeInvalid#1#2{%
  \count@=#1\relax
  \loop
    \catcode\count@=15 %
  \ifnum\count@<#2\relax
    \advance\count@ 1 %
  \repeat
}
\def\RangeCatcodeCheck#1#2#3{%
  \count@=#1\relax
  \loop
    \ifnum#3=\catcode\count@
    \else
      \errmessage{%
        Character \the\count@\space
        with wrong catcode \the\catcode\count@\space
        instead of \number#3%
      }%
    \fi
  \ifnum\count@<#2\relax
    \advance\count@ 1 %
  \repeat
}
\def\space{ }
\expandafter\ifx\csname LoadCommand\endcsname\relax
  \def\LoadCommand{\input ifpdf.sty\relax}%
\fi
\def\Test{%
  \RangeCatcodeInvalid{0}{47}%
  \RangeCatcodeInvalid{58}{64}%
  \RangeCatcodeInvalid{91}{96}%
  \RangeCatcodeInvalid{123}{255}%
  \catcode`\@=12 %
  \catcode`\\=0 %
  \catcode`\%=14 %
  \LoadCommand
  \RangeCatcodeCheck{0}{36}{15}%
  \RangeCatcodeCheck{37}{37}{14}%
  \RangeCatcodeCheck{38}{47}{15}%
  \RangeCatcodeCheck{48}{57}{12}%
  \RangeCatcodeCheck{58}{63}{15}%
  \RangeCatcodeCheck{64}{64}{12}%
  \RangeCatcodeCheck{65}{90}{11}%
  \RangeCatcodeCheck{91}{91}{15}%
  \RangeCatcodeCheck{92}{92}{0}%
  \RangeCatcodeCheck{93}{96}{15}%
  \RangeCatcodeCheck{97}{122}{11}%
  \RangeCatcodeCheck{123}{255}{15}%
  \RestoreCatcodes
}
\Test
\csname @@end\endcsname
\end
%    \end{macrocode}
%    \begin{macrocode}
%</test1>
%    \end{macrocode}
%
% \section{Installation}
%
% \subsection{Download}
%
% \paragraph{Package.} This package is available on
% CTAN\footnote{\url{ftp://ftp.ctan.org/tex-archive/}}:
% \begin{description}
% \item[\CTAN{macros/latex/contrib/oberdiek/ifpdf.dtx}] The source file.
% \item[\CTAN{macros/latex/contrib/oberdiek/ifpdf.pdf}] Documentation.
% \end{description}
%
%
% \paragraph{Bundle.} All the packages of the bundle `oberdiek'
% are also available in a TDS compliant ZIP archive. There
% the packages are already unpacked and the documentation files
% are generated. The files and directories obey the TDS standard.
% \begin{description}
% \item[\CTAN{install/macros/latex/contrib/oberdiek.tds.zip}]
% \end{description}
% \emph{TDS} refers to the standard ``A Directory Structure
% for \TeX\ Files'' (\CTAN{tds/tds.pdf}). Directories
% with \xfile{texmf} in their name are usually organized this way.
%
% \subsection{Bundle installation}
%
% \paragraph{Unpacking.} Unpack the \xfile{oberdiek.tds.zip} in the
% TDS tree (also known as \xfile{texmf} tree) of your choice.
% Example (linux):
% \begin{quote}
%   |unzip oberdiek.tds.zip -d ~/texmf|
% \end{quote}
%
% \paragraph{Script installation.}
% Check the directory \xfile{TDS:scripts/oberdiek/} for
% scripts that need further installation steps.
% Package \xpackage{attachfile2} comes with the Perl script
% \xfile{pdfatfi.pl} that should be installed in such a way
% that it can be called as \texttt{pdfatfi}.
% Example (linux):
% \begin{quote}
%   |chmod +x scripts/oberdiek/pdfatfi.pl|\\
%   |cp scripts/oberdiek/pdfatfi.pl /usr/local/bin/|
% \end{quote}
%
% \subsection{Package installation}
%
% \paragraph{Unpacking.} The \xfile{.dtx} file is a self-extracting
% \docstrip\ archive. The files are extracted by running the
% \xfile{.dtx} through \plainTeX:
% \begin{quote}
%   \verb|tex ifpdf.dtx|
% \end{quote}
%
% \paragraph{TDS.} Now the different files must be moved into
% the different directories in your installation TDS tree
% (also known as \xfile{texmf} tree):
% \begin{quote}
% \def\t{^^A
% \begin{tabular}{@{}>{\ttfamily}l@{ $\rightarrow$ }>{\ttfamily}l@{}}
%   ifpdf.sty & tex/generic/oberdiek/ifpdf.sty\\
%   ifpdf.pdf & doc/latex/oberdiek/ifpdf.pdf\\
%   test/ifpdf-test1.tex & doc/latex/oberdiek/test/ifpdf-test1.tex\\
%   ifpdf.dtx & source/latex/oberdiek/ifpdf.dtx\\
% \end{tabular}^^A
% }^^A
% \sbox0{\t}^^A
% \ifdim\wd0>\linewidth
%   \begingroup
%     \advance\linewidth by\leftmargin
%     \advance\linewidth by\rightmargin
%   \edef\x{\endgroup
%     \def\noexpand\lw{\the\linewidth}^^A
%   }\x
%   \def\lwbox{^^A
%     \leavevmode
%     \hbox to \linewidth{^^A
%       \kern-\leftmargin\relax
%       \hss
%       \usebox0
%       \hss
%       \kern-\rightmargin\relax
%     }^^A
%   }^^A
%   \ifdim\wd0>\lw
%     \sbox0{\small\t}^^A
%     \ifdim\wd0>\linewidth
%       \ifdim\wd0>\lw
%         \sbox0{\footnotesize\t}^^A
%         \ifdim\wd0>\linewidth
%           \ifdim\wd0>\lw
%             \sbox0{\scriptsize\t}^^A
%             \ifdim\wd0>\linewidth
%               \ifdim\wd0>\lw
%                 \sbox0{\tiny\t}^^A
%                 \ifdim\wd0>\linewidth
%                   \lwbox
%                 \else
%                   \usebox0
%                 \fi
%               \else
%                 \lwbox
%               \fi
%             \else
%               \usebox0
%             \fi
%           \else
%             \lwbox
%           \fi
%         \else
%           \usebox0
%         \fi
%       \else
%         \lwbox
%       \fi
%     \else
%       \usebox0
%     \fi
%   \else
%     \lwbox
%   \fi
% \else
%   \usebox0
% \fi
% \end{quote}
% If you have a \xfile{docstrip.cfg} that configures and enables \docstrip's
% TDS installing feature, then some files can already be in the right
% place, see the documentation of \docstrip.
%
% \subsection{Refresh file name databases}
%
% If your \TeX~distribution
% (\teTeX, \mikTeX, \dots) relies on file name databases, you must refresh
% these. For example, \teTeX\ users run \verb|texhash| or
% \verb|mktexlsr|.
%
% \subsection{Some details for the interested}
%
% \paragraph{Attached source.}
%
% The PDF documentation on CTAN also includes the
% \xfile{.dtx} source file. It can be extracted by
% AcrobatReader 6 or higher. Another option is \textsf{pdftk},
% e.g. unpack the file into the current directory:
% \begin{quote}
%   \verb|pdftk ifpdf.pdf unpack_files output .|
% \end{quote}
%
% \paragraph{Unpacking with \LaTeX.}
% The \xfile{.dtx} chooses its action depending on the format:
% \begin{description}
% \item[\plainTeX:] Run \docstrip\ and extract the files.
% \item[\LaTeX:] Generate the documentation.
% \end{description}
% If you insist on using \LaTeX\ for \docstrip\ (really,
% \docstrip\ does not need \LaTeX), then inform the autodetect routine
% about your intention:
% \begin{quote}
%   \verb|latex \let\install=y% \iffalse meta-comment
%
% File: ifpdf.dtx
% Version: 2011/01/30 v2.3
% Info: Provides the ifpdf switch
%
% Copyright (C) 2001, 2005-2011 by
%    Heiko Oberdiek <heiko.oberdiek at googlemail.com>
%
% This work may be distributed and/or modified under the
% conditions of the LaTeX Project Public License, either
% version 1.3c of this license or (at your option) any later
% version. This version of this license is in
%    http://www.latex-project.org/lppl/lppl-1-3c.txt
% and the latest version of this license is in
%    http://www.latex-project.org/lppl.txt
% and version 1.3 or later is part of all distributions of
% LaTeX version 2005/12/01 or later.
%
% This work has the LPPL maintenance status "maintained".
%
% This Current Maintainer of this work is Heiko Oberdiek.
%
% The Base Interpreter refers to any `TeX-Format',
% because some files are installed in TDS:tex/generic//.
%
% This work consists of the main source file ifpdf.dtx
% and the derived files
%    ifpdf.sty, ifpdf.pdf, ifpdf.ins, ifpdf.drv, ifpdf-test1.tex.
%
% Distribution:
%    CTAN:macros/latex/contrib/oberdiek/ifpdf.dtx
%    CTAN:macros/latex/contrib/oberdiek/ifpdf.pdf
%
% Unpacking:
%    (a) If ifpdf.ins is present:
%           tex ifpdf.ins
%    (b) Without ifpdf.ins:
%           tex ifpdf.dtx
%    (c) If you insist on using LaTeX
%           latex \let\install=y% \iffalse meta-comment
%
% File: ifpdf.dtx
% Version: 2011/01/30 v2.3
% Info: Provides the ifpdf switch
%
% Copyright (C) 2001, 2005-2011 by
%    Heiko Oberdiek <heiko.oberdiek at googlemail.com>
%
% This work may be distributed and/or modified under the
% conditions of the LaTeX Project Public License, either
% version 1.3c of this license or (at your option) any later
% version. This version of this license is in
%    http://www.latex-project.org/lppl/lppl-1-3c.txt
% and the latest version of this license is in
%    http://www.latex-project.org/lppl.txt
% and version 1.3 or later is part of all distributions of
% LaTeX version 2005/12/01 or later.
%
% This work has the LPPL maintenance status "maintained".
%
% This Current Maintainer of this work is Heiko Oberdiek.
%
% The Base Interpreter refers to any `TeX-Format',
% because some files are installed in TDS:tex/generic//.
%
% This work consists of the main source file ifpdf.dtx
% and the derived files
%    ifpdf.sty, ifpdf.pdf, ifpdf.ins, ifpdf.drv, ifpdf-test1.tex.
%
% Distribution:
%    CTAN:macros/latex/contrib/oberdiek/ifpdf.dtx
%    CTAN:macros/latex/contrib/oberdiek/ifpdf.pdf
%
% Unpacking:
%    (a) If ifpdf.ins is present:
%           tex ifpdf.ins
%    (b) Without ifpdf.ins:
%           tex ifpdf.dtx
%    (c) If you insist on using LaTeX
%           latex \let\install=y% \iffalse meta-comment
%
% File: ifpdf.dtx
% Version: 2011/01/30 v2.3
% Info: Provides the ifpdf switch
%
% Copyright (C) 2001, 2005-2011 by
%    Heiko Oberdiek <heiko.oberdiek at googlemail.com>
%
% This work may be distributed and/or modified under the
% conditions of the LaTeX Project Public License, either
% version 1.3c of this license or (at your option) any later
% version. This version of this license is in
%    http://www.latex-project.org/lppl/lppl-1-3c.txt
% and the latest version of this license is in
%    http://www.latex-project.org/lppl.txt
% and version 1.3 or later is part of all distributions of
% LaTeX version 2005/12/01 or later.
%
% This work has the LPPL maintenance status "maintained".
%
% This Current Maintainer of this work is Heiko Oberdiek.
%
% The Base Interpreter refers to any `TeX-Format',
% because some files are installed in TDS:tex/generic//.
%
% This work consists of the main source file ifpdf.dtx
% and the derived files
%    ifpdf.sty, ifpdf.pdf, ifpdf.ins, ifpdf.drv, ifpdf-test1.tex.
%
% Distribution:
%    CTAN:macros/latex/contrib/oberdiek/ifpdf.dtx
%    CTAN:macros/latex/contrib/oberdiek/ifpdf.pdf
%
% Unpacking:
%    (a) If ifpdf.ins is present:
%           tex ifpdf.ins
%    (b) Without ifpdf.ins:
%           tex ifpdf.dtx
%    (c) If you insist on using LaTeX
%           latex \let\install=y\input{ifpdf.dtx}
%        (quote the arguments according to the demands of your shell)
%
% Documentation:
%    (a) If ifpdf.drv is present:
%           latex ifpdf.drv
%    (b) Without ifpdf.drv:
%           latex ifpdf.dtx; ...
%    The class ltxdoc loads the configuration file ltxdoc.cfg
%    if available. Here you can specify further options, e.g.
%    use A4 as paper format:
%       \PassOptionsToClass{a4paper}{article}
%
%    Programm calls to get the documentation (example):
%       pdflatex ifpdf.dtx
%       makeindex -s gind.ist ifpdf.idx
%       pdflatex ifpdf.dtx
%       makeindex -s gind.ist ifpdf.idx
%       pdflatex ifpdf.dtx
%
% Installation:
%    TDS:tex/generic/oberdiek/ifpdf.sty
%    TDS:doc/latex/oberdiek/ifpdf.pdf
%    TDS:doc/latex/oberdiek/test/ifpdf-test1.tex
%    TDS:source/latex/oberdiek/ifpdf.dtx
%
%<*ignore>
\begingroup
  \catcode123=1 %
  \catcode125=2 %
  \def\x{LaTeX2e}%
\expandafter\endgroup
\ifcase 0\ifx\install y1\fi\expandafter
         \ifx\csname processbatchFile\endcsname\relax\else1\fi
         \ifx\fmtname\x\else 1\fi\relax
\else\csname fi\endcsname
%</ignore>
%<*install>
\input docstrip.tex
\Msg{************************************************************************}
\Msg{* Installation}
\Msg{* Package: ifpdf 2011/01/30 v2.3 Provides the ifpdf switch (HO)}
\Msg{************************************************************************}

\keepsilent
\askforoverwritefalse

\let\MetaPrefix\relax
\preamble

This is a generated file.

Project: ifpdf
Version: 2011/01/30 v2.3

Copyright (C) 2001, 2005-2011 by
   Heiko Oberdiek <heiko.oberdiek at googlemail.com>

This work may be distributed and/or modified under the
conditions of the LaTeX Project Public License, either
version 1.3c of this license or (at your option) any later
version. This version of this license is in
   http://www.latex-project.org/lppl/lppl-1-3c.txt
and the latest version of this license is in
   http://www.latex-project.org/lppl.txt
and version 1.3 or later is part of all distributions of
LaTeX version 2005/12/01 or later.

This work has the LPPL maintenance status "maintained".

This Current Maintainer of this work is Heiko Oberdiek.

The Base Interpreter refers to any `TeX-Format',
because some files are installed in TDS:tex/generic//.

This work consists of the main source file ifpdf.dtx
and the derived files
   ifpdf.sty, ifpdf.pdf, ifpdf.ins, ifpdf.drv, ifpdf-test1.tex.

\endpreamble
\let\MetaPrefix\DoubleperCent

\generate{%
  \file{ifpdf.ins}{\from{ifpdf.dtx}{install}}%
  \file{ifpdf.drv}{\from{ifpdf.dtx}{driver}}%
  \usedir{tex/generic/oberdiek}%
  \file{ifpdf.sty}{\from{ifpdf.dtx}{package}}%
  \usedir{doc/latex/oberdiek/test}%
  \file{ifpdf-test1.tex}{\from{ifpdf.dtx}{test1}}%
  \nopreamble
  \nopostamble
  \usedir{source/latex/oberdiek/catalogue}%
  \file{ifpdf.xml}{\from{ifpdf.dtx}{catalogue}}%
}

\catcode32=13\relax% active space
\let =\space%
\Msg{************************************************************************}
\Msg{*}
\Msg{* To finish the installation you have to move the following}
\Msg{* file into a directory searched by TeX:}
\Msg{*}
\Msg{*     ifpdf.sty}
\Msg{*}
\Msg{* To produce the documentation run the file `ifpdf.drv'}
\Msg{* through LaTeX.}
\Msg{*}
\Msg{* Happy TeXing!}
\Msg{*}
\Msg{************************************************************************}

\endbatchfile
%</install>
%<*ignore>
\fi
%</ignore>
%<*driver>
\NeedsTeXFormat{LaTeX2e}
\ProvidesFile{ifpdf.drv}%
  [2011/01/30 v2.3 Provides the ifpdf switch (HO)]%
\documentclass{ltxdoc}
\usepackage{holtxdoc}[2011/11/22]
\begin{document}
  \DocInput{ifpdf.dtx}%
\end{document}
%</driver>
% \fi
%
% \CheckSum{600}
%
% \CharacterTable
%  {Upper-case    \A\B\C\D\E\F\G\H\I\J\K\L\M\N\O\P\Q\R\S\T\U\V\W\X\Y\Z
%   Lower-case    \a\b\c\d\e\f\g\h\i\j\k\l\m\n\o\p\q\r\s\t\u\v\w\x\y\z
%   Digits        \0\1\2\3\4\5\6\7\8\9
%   Exclamation   \!     Double quote  \"     Hash (number) \#
%   Dollar        \$     Percent       \%     Ampersand     \&
%   Acute accent  \'     Left paren    \(     Right paren   \)
%   Asterisk      \*     Plus          \+     Comma         \,
%   Minus         \-     Point         \.     Solidus       \/
%   Colon         \:     Semicolon     \;     Less than     \<
%   Equals        \=     Greater than  \>     Question mark \?
%   Commercial at \@     Left bracket  \[     Backslash     \\
%   Right bracket \]     Circumflex    \^     Underscore    \_
%   Grave accent  \`     Left brace    \{     Vertical bar  \|
%   Right brace   \}     Tilde         \~}
%
% \GetFileInfo{ifpdf.drv}
%
% \title{The \xpackage{ifpdf} package}
% \date{2011/01/30 v2.3}
% \author{Heiko Oberdiek\\\xemail{heiko.oberdiek at googlemail.com}}
%
% \maketitle
%
% \begin{abstract}
% This package looks for \pdfTeX\ in pdf mode and
% implements and sets the switch \cs{ifpdf}.
% The detection is based on \cs{pdfoutput} and
% the package will not change this value.
% It works with plain or \LaTeX\ formats.
% \end{abstract}
%
% \tableofcontents
%
% \hypersetup{bookmarksopenlevel=2}
% \section{Documentation}
%
% \subsection{Introduction}
%
% It is commonly known that H\`an Th\^e Th\`anh's \pdfTeX\ generates
% PDF output directly and many people uses \pdfTeX\ for this purpose.
% However the DVI output was never thrown away. In contrary, he
% new features for typesetting that works in both PDF and DVI mode.
%
% In the meantime many \TeX\ distributions replace the traditional
% \TeX\ binary with \pdfTeX. Then, for example, called as |latex|
% \pdfTeX\ works in DVI mode with the \LaTeX\ format preloaded,
% called as |pdflatex| \pdfTeX\ starts in PDF mode.
%
% Often packages or users want to know, whether the current document
% is typset by \pdfTeX\ in PDF mode, because the different modes
% have different capabilities (color setting, graphics inclusion,
% \dots). For this purpose \pdfTeX's
% \cs{pdfoutput} can be asked.
%
% As regulary reader of \TeX\ newsgroups and mailing lists I could
% observe many problems with this task. Common errors are:
% \begin{itemize}
% \item
%   \pdfTeX\ has \emph{two} modes. Using \pdfTeX\ does not mean
%   that the user always want to have PDF mode. For example, the
%   PostScript support is better in DVI mode in conjunction with a
%   PostScript aware DVI driver (e.g. dvips). Also the additional
%   typesetting features are mode independent and also available
%   in DVI mode.
% \item
%   \LaTeX's \cs{@ifundefined} inherited the side effect from
%   \cs{csname}. Unknown commands are defined with the meaning
%   of \cs{relax}. If it is checked, whether \cs{pdfoutput} is
%   defined, then this should not be forgotten.
% \item
%   Having \cs{pdfoutput} does not automatically mean PDF mode.
%   Also the value of \cs{pdfoutput} must be asked.
% \item
%   \cs{pdfoutput} must not be destroyed in some way. Later
%   code and packages are fooled then and will perhaps make
%   wrong decisions. For example they may drop support for
%   PDF mode, because they do not know that \pdfTeX\ is
%   running at all.
% \end{itemize}
%
% Robin Fairbairns provides an entry for this topic in his
% excellent FAQ (\url{http://www.tex.ac.uk/faq}):
% \href{http://www.tex.ac.uk/cgi-bin/texfaq2html?label=ifpdf}^^A
% {Am I using PDFTeX?}
%
% \subsection{Usage}
%
% The package \xpackage{ifpdf} can be used with both \plainTeX\
% and \LaTeX:
% \begin{description}
% \item[\plainTeX:] |\input ifpdf.sty|
% \item[\LaTeXe:]   |\usepackage{ifpdf}|
% \end{description}
%
% \DescribeMacro{\ifpdf}
% The package provides the switch \cs{ifpdf}:
% \begin{quote}
%   |\ifpdf|\\
%   \hspace*{1.5em}\dots\ do things, if \pdfTeX\ is running in pdf mode \dots\\
%   |\else|\\
%   \hspace*{1.5em}\dots\ other \TeX\ or \pdfTeX\ in dvi mode \dots\\
%   |\fi|
% \end{quote}
% Users of the package \xpackage{ifthen} can use the switch as boolean:
% \begin{quote}
%   |\boolean{pdf}|
% \end{quote}
% The package can also be used to set global
% documentclass options:
% \begin{quote}
%   |\RequirePackage{ifpdf}|\\
%   |\ifpdf|\\
%   \hspace*{1.5em}|\documentclass[pdftex,|\dots|]{|\dots|}|\\
%   |\else|\\
%   \hspace*{1.5em}|\documentclass[|\dots|]{|\dots|}|\\
%   |\fi|
% \end{quote}
%
% \subsection{Specification}
%
% The package have the following properties:
% \begin{itemize}
% \item It asks the setting of \cs{pdfoutput} for detecting
%   \pdfTeX\ in PDF mode.
% \item It never changes \cs{pdfoutput}.
% \item If \cs{pdfoutput} is undefined or has the meaning \cs{relax},
%   but the engine provides the primitive \cs{pdfoutput},
%   then \cs{pdfoutput} is enabled or restored if possible
%   (only \LuaTeX, version 0.36.0 or higher).
% \item It can be used with many formats including
%   \plainTeX\ and \LaTeX.
% \end{itemize}
% The mode detection implements the following algorithm:
% \begin{quote}
% \def\={${:}{=}$}
% \def\>{\null\qquad\ignorespaces}
% \def\comment#1{^^A
%   \textrm{\itshape\% #1}^^A
% }
% \sffamily
% if undefined(\cs{pdfoutput})\\
% \> \cs{ifpdf} \= \cs{iffalse} \comment{\pdfTeX is not running}\\
% else\\
% \> if \cs{pdfoutput} $\leq$ 0\\
% \>\> \cs{ifpdf} \= \cs{iffalse} \comment{\pdfTeX\ in DVI mode}\\
% \> else\\
% \>\> \cs{ifpdf} \= \cs{iftrue} \comment{\pdfTeX\ in PDF mode}\\
% \> fi\\
% fi
% \end{quote}
% The function \textsf{undefined} checks both cases, undefined command
% and \cs{relax}.
%
% \hypersetup{bookmarksopenlevel=1}
%
% \StopEventually{
% }
%
% \section{Implementation}
%
%    \begin{macrocode}
%<*package>
%    \end{macrocode}
%
% \subsection{Reload check and package identification}
%    Reload check, especially if the package is not used with \LaTeX.
%    \begin{macrocode}
\begingroup\catcode61\catcode48\catcode32=10\relax%
  \catcode13=5 % ^^M
  \endlinechar=13 %
  \catcode35=6 % #
  \catcode39=12 % '
  \catcode44=12 % ,
  \catcode45=12 % -
  \catcode46=12 % .
  \catcode58=12 % :
  \catcode64=11 % @
  \catcode123=1 % {
  \catcode125=2 % }
  \expandafter\let\expandafter\x\csname ver@ifpdf.sty\endcsname
  \ifx\x\relax % plain-TeX, first loading
  \else
    \def\empty{}%
    \ifx\x\empty % LaTeX, first loading,
      % variable is initialized, but \ProvidesPackage not yet seen
    \else
      \expandafter\ifx\csname PackageInfo\endcsname\relax
        \def\x#1#2{%
          \immediate\write-1{Package #1 Info: #2.}%
        }%
      \else
        \def\x#1#2{\PackageInfo{#1}{#2, stopped}}%
      \fi
      \x{ifpdf}{The package is already loaded}%
      \aftergroup\endinput
    \fi
  \fi
\endgroup%
%    \end{macrocode}
%    Package identification:
%    \begin{macrocode}
\begingroup\catcode61\catcode48\catcode32=10\relax%
  \catcode13=5 % ^^M
  \endlinechar=13 %
  \catcode35=6 % #
  \catcode39=12 % '
  \catcode40=12 % (
  \catcode41=12 % )
  \catcode44=12 % ,
  \catcode45=12 % -
  \catcode46=12 % .
  \catcode47=12 % /
  \catcode58=12 % :
  \catcode64=11 % @
  \catcode91=12 % [
  \catcode93=12 % ]
  \catcode123=1 % {
  \catcode125=2 % }
  \expandafter\ifx\csname ProvidesPackage\endcsname\relax
    \def\x#1#2#3[#4]{\endgroup
      \immediate\write-1{Package: #3 #4}%
      \xdef#1{#4}%
    }%
  \else
    \def\x#1#2[#3]{\endgroup
      #2[{#3}]%
      \ifx#1\@undefined
        \xdef#1{#3}%
      \fi
      \ifx#1\relax
        \xdef#1{#3}%
      \fi
    }%
  \fi
\expandafter\x\csname ver@ifpdf.sty\endcsname
\ProvidesPackage{ifpdf}%
  [2011/01/30 v2.3 Provides the ifpdf switch (HO)]%
%    \end{macrocode}
%
% \subsection{Catcodes}
%
%    \begin{macrocode}
\begingroup\catcode61\catcode48\catcode32=10\relax%
  \catcode13=5 % ^^M
  \endlinechar=13 %
  \catcode123=1 % {
  \catcode125=2 % }
  \catcode64=11 % @
  \def\x{\endgroup
    \expandafter\edef\csname ifpdf@AtEnd\endcsname{%
      \endlinechar=\the\endlinechar\relax
      \catcode13=\the\catcode13\relax
      \catcode32=\the\catcode32\relax
      \catcode35=\the\catcode35\relax
      \catcode61=\the\catcode61\relax
      \catcode64=\the\catcode64\relax
      \catcode123=\the\catcode123\relax
      \catcode125=\the\catcode125\relax
    }%
  }%
\x\catcode61\catcode48\catcode32=10\relax%
\catcode13=5 % ^^M
\endlinechar=13 %
\catcode35=6 % #
\catcode64=11 % @
\catcode123=1 % {
\catcode125=2 % }
\def\TMP@EnsureCode#1#2{%
  \edef\ifpdf@AtEnd{%
    \ifpdf@AtEnd
    \catcode#1=\the\catcode#1\relax
  }%
  \catcode#1=#2\relax
}
\TMP@EnsureCode{10}{12}% ^^J
\TMP@EnsureCode{39}{12}% '
\TMP@EnsureCode{40}{12}% (
\TMP@EnsureCode{41}{12}% )
\TMP@EnsureCode{44}{12}% ,
\TMP@EnsureCode{45}{12}% -
\TMP@EnsureCode{46}{12}% .
\TMP@EnsureCode{47}{12}% /
\TMP@EnsureCode{58}{12}% :
\TMP@EnsureCode{60}{12}% <
\TMP@EnsureCode{91}{12}% [
\TMP@EnsureCode{93}{12}% ]
\TMP@EnsureCode{94}{7}% ^
\TMP@EnsureCode{96}{12}% `
\edef\ifpdf@AtEnd{\ifpdf@AtEnd\noexpand\endinput}
%    \end{macrocode}
%
% \subsection{Check for previously defined \cs{ifpdf}}
%
%    \begin{macrocode}
\begingroup
  \expandafter\ifx\csname ifpdf\endcsname\relax
  \else
    \edef\i/{\expandafter\string\csname ifpdf\endcsname}%
    \expandafter\ifx\csname PackageError\endcsname\relax
      \def\x#1#2{%
        \edef\z{#2}%
        \expandafter\errhelp\expandafter{\z}%
        \errmessage{Package ifpdf Error: #1}%
      }%
      \def\y{^^J}%
      \newlinechar=10 %
    \else
      \def\x#1#2{%
        \PackageError{ifpdf}{#1}{#2}%
      }%
      \def\y{\MessageBreak}%
    \fi
    \x{Name clash, \i/ is already defined}{%
      Incompatible versions of \i/ can cause problems,\y
      therefore package loading is aborted.%
    }%
    \endgroup
    \expandafter\ifpdf@AtEnd
  \fi%
\endgroup
%    \end{macrocode}
%
% \subsection{\cs{pdfoutput} and \LuaTeX}
%
%    It might happen, that \LuaTeX\ is running, but \cs{pdfoutput}
%    does not exist. In version 0.40 only \cs{directlua}
%    is available at startup time. The enabling Lua function was
%    already added in version 0.36. Thus we can ignore older versions,
%    here \cs{pdfoutput} is available at startup time.
%
%    \begin{macrocode}
\begingroup
  \def\skip#1\relax\begingroup{}%
  \expandafter\ifx\csname pdfoutput\endcsname\relax
  \else
    \expandafter\skip
  \fi
  \expandafter\ifx\csname directlua\endcsname\relax
    \expandafter\skip
  \fi
\endgroup
\begingroup\expandafter\expandafter\expandafter\endgroup
\expandafter\ifx\csname RequirePackage\endcsname\relax
  \def\TMP@RequirePackage#1[#2]{%
    \begingroup\expandafter\expandafter\expandafter\endgroup
    \expandafter\ifx\csname ver@#1.sty\endcsname\relax
      \input #1.sty\relax
    \fi
  }%
  \TMP@RequirePackage{ifluatex}[2009/04/10]%
\else
  \RequirePackage{ifluatex}[2009/04/10]%
\fi
\ifluatex
  \ifnum\luatexversion<36 %
%    \end{macrocode}
%    Unhappily \LuaTeX's \cs{primitive} (derived from
%    pdfTeX's \cs{pdfprimitive}) cannot be used:
%    \begin{quote}
%    |\protected\gdef\pdfoutput{\primitive\pdfoutput}|
%    \end{quote}
%    Setting a value works, but getting fails,
%    because \TeX\ does no longer see it as number. It
%    is unexpandable and breaks numerical contexts.
%    This was fixed in \hologo{pdfTeX} 1.40.10
%    (bugfix \#4289: ``|\primitive\pdfoutput| cannot be queried'').
%    \begin{macrocode}
  \else
    \begingroup
      \directlua{tex.enableprimitives('ifpdf', {'pdfoutput'})}%
      \global\let\pdfoutput\ifpdfpdfoutput
    \endgroup
  \fi
\fi
\relax\begingroup\endgroup
%    \end{macrocode}
%
% \subsection{\cs{ifpdf}}
%
%    \begin{macro}{\ifpdf}
%    Create and set the switch. \cs{newif} initializes the
%    switch with \cs{iffalse}. \cs{newif} is \cs{outer}
%    in \plainTeX.
%    \begin{macrocode}
\begingroup\expandafter\expandafter\expandafter\endgroup
\expandafter\ifx\csname newif\endcsname\relax
  \edef\pdffalse{%
    \let
    \expandafter\noexpand\csname ifpdf\endcsname
    \expandafter\noexpand\csname iffalse\endcsname
  }%
  \edef\pdftrue{%
    \let
    \expandafter\noexpand\csname ifpdf\endcsname
    \expandafter\noexpand\csname iftrue\endcsname
  }%
  \pdffalse
\else
  \csname newif\expandafter\endcsname\csname ifpdf\endcsname
\fi
%    \end{macrocode}
%    Test \cs{pdfoutput}. Is it  defined and different from
%    \cs{relax}? Someone could have used \LaTeX\ internal
%    \cs{@ifundefined}, or something else involving.
%    Notice, \cs{csname} is executed inside a group for the test
%    to cancel the side effect of \cs{csname}.
%    \begin{macrocode}
\begingroup\expandafter\expandafter\expandafter\endgroup
\expandafter\ifx\csname pdfoutput\endcsname\relax
\else
  \ifnum\pdfoutput<1 %
%    \end{macrocode}
%    \cs{pdfoutput}|=0| or negative, so not generating pdf.
%    \begin{macrocode}
  \else
    \pdftrue
  \fi
\fi
%    \end{macrocode}
%    \end{macro}
%
% \subsection{Test for fool attempts}
%
%    \begin{macrocode}
\begingroup
  \expandafter\ifx\csname pdfoutput\endcsname\relax
  \else
    \escapechar=92 %
    \edef\m{\meaning\pdfoutput}%
    \edef\p{\string\pdfoutput}%
    \ifx\m\p
    \else
      \expandafter\ifx\csname PackageWarningNoLine\endcsname\relax
        \def\PackageWarningNoLine#1#2{%
          \immediate\write16{%
            Package `#1' Warning: #2.%
          }%
        }%
      \fi
      \PackageWarningNoLine{ifpdf}{%
        Someone has redefined \string\pdfoutput%
      }%
    \fi
  \fi
\endgroup
%    \end{macrocode}
%
% \subsection{Protocol entry}
%
%     Log comment:
%    \begin{macrocode}
\begingroup
  \expandafter\ifx\csname PackageInfo\endcsname\relax
    \def\x#1#2{%
      \immediate\write-1{Package #1 Info: #2.}%
    }%
  \else
    \let\x\PackageInfo
    \expandafter\let\csname on@line\endcsname\empty
  \fi
  \x{ifpdf}{pdfTeX in PDF mode is \ifpdf\else not \fi detected}%
\endgroup
%    \end{macrocode}
%    \begin{macrocode}
\ifpdf@AtEnd%
%    \end{macrocode}
%    \begin{macrocode}
%</package>
%    \end{macrocode}
%
% \section{Test}
%
% \subsection{Catcode checks for loading}
%
%    \begin{macrocode}
%<*test1>
%    \end{macrocode}
%    \begin{macrocode}
\catcode`\{=1 %
\catcode`\}=2 %
\catcode`\#=6 %
\catcode`\@=11 %
\expandafter\ifx\csname count@\endcsname\relax
  \countdef\count@=255 %
\fi
\expandafter\ifx\csname @gobble\endcsname\relax
  \long\def\@gobble#1{}%
\fi
\expandafter\ifx\csname @firstofone\endcsname\relax
  \long\def\@firstofone#1{#1}%
\fi
\expandafter\ifx\csname loop\endcsname\relax
  \expandafter\@firstofone
\else
  \expandafter\@gobble
\fi
{%
  \def\loop#1\repeat{%
    \def\body{#1}%
    \iterate
  }%
  \def\iterate{%
    \body
      \let\next\iterate
    \else
      \let\next\relax
    \fi
    \next
  }%
  \let\repeat=\fi
}%
\def\RestoreCatcodes{}
\count@=0 %
\loop
  \edef\RestoreCatcodes{%
    \RestoreCatcodes
    \catcode\the\count@=\the\catcode\count@\relax
  }%
\ifnum\count@<255 %
  \advance\count@ 1 %
\repeat

\def\RangeCatcodeInvalid#1#2{%
  \count@=#1\relax
  \loop
    \catcode\count@=15 %
  \ifnum\count@<#2\relax
    \advance\count@ 1 %
  \repeat
}
\def\RangeCatcodeCheck#1#2#3{%
  \count@=#1\relax
  \loop
    \ifnum#3=\catcode\count@
    \else
      \errmessage{%
        Character \the\count@\space
        with wrong catcode \the\catcode\count@\space
        instead of \number#3%
      }%
    \fi
  \ifnum\count@<#2\relax
    \advance\count@ 1 %
  \repeat
}
\def\space{ }
\expandafter\ifx\csname LoadCommand\endcsname\relax
  \def\LoadCommand{\input ifpdf.sty\relax}%
\fi
\def\Test{%
  \RangeCatcodeInvalid{0}{47}%
  \RangeCatcodeInvalid{58}{64}%
  \RangeCatcodeInvalid{91}{96}%
  \RangeCatcodeInvalid{123}{255}%
  \catcode`\@=12 %
  \catcode`\\=0 %
  \catcode`\%=14 %
  \LoadCommand
  \RangeCatcodeCheck{0}{36}{15}%
  \RangeCatcodeCheck{37}{37}{14}%
  \RangeCatcodeCheck{38}{47}{15}%
  \RangeCatcodeCheck{48}{57}{12}%
  \RangeCatcodeCheck{58}{63}{15}%
  \RangeCatcodeCheck{64}{64}{12}%
  \RangeCatcodeCheck{65}{90}{11}%
  \RangeCatcodeCheck{91}{91}{15}%
  \RangeCatcodeCheck{92}{92}{0}%
  \RangeCatcodeCheck{93}{96}{15}%
  \RangeCatcodeCheck{97}{122}{11}%
  \RangeCatcodeCheck{123}{255}{15}%
  \RestoreCatcodes
}
\Test
\csname @@end\endcsname
\end
%    \end{macrocode}
%    \begin{macrocode}
%</test1>
%    \end{macrocode}
%
% \section{Installation}
%
% \subsection{Download}
%
% \paragraph{Package.} This package is available on
% CTAN\footnote{\url{ftp://ftp.ctan.org/tex-archive/}}:
% \begin{description}
% \item[\CTAN{macros/latex/contrib/oberdiek/ifpdf.dtx}] The source file.
% \item[\CTAN{macros/latex/contrib/oberdiek/ifpdf.pdf}] Documentation.
% \end{description}
%
%
% \paragraph{Bundle.} All the packages of the bundle `oberdiek'
% are also available in a TDS compliant ZIP archive. There
% the packages are already unpacked and the documentation files
% are generated. The files and directories obey the TDS standard.
% \begin{description}
% \item[\CTAN{install/macros/latex/contrib/oberdiek.tds.zip}]
% \end{description}
% \emph{TDS} refers to the standard ``A Directory Structure
% for \TeX\ Files'' (\CTAN{tds/tds.pdf}). Directories
% with \xfile{texmf} in their name are usually organized this way.
%
% \subsection{Bundle installation}
%
% \paragraph{Unpacking.} Unpack the \xfile{oberdiek.tds.zip} in the
% TDS tree (also known as \xfile{texmf} tree) of your choice.
% Example (linux):
% \begin{quote}
%   |unzip oberdiek.tds.zip -d ~/texmf|
% \end{quote}
%
% \paragraph{Script installation.}
% Check the directory \xfile{TDS:scripts/oberdiek/} for
% scripts that need further installation steps.
% Package \xpackage{attachfile2} comes with the Perl script
% \xfile{pdfatfi.pl} that should be installed in such a way
% that it can be called as \texttt{pdfatfi}.
% Example (linux):
% \begin{quote}
%   |chmod +x scripts/oberdiek/pdfatfi.pl|\\
%   |cp scripts/oberdiek/pdfatfi.pl /usr/local/bin/|
% \end{quote}
%
% \subsection{Package installation}
%
% \paragraph{Unpacking.} The \xfile{.dtx} file is a self-extracting
% \docstrip\ archive. The files are extracted by running the
% \xfile{.dtx} through \plainTeX:
% \begin{quote}
%   \verb|tex ifpdf.dtx|
% \end{quote}
%
% \paragraph{TDS.} Now the different files must be moved into
% the different directories in your installation TDS tree
% (also known as \xfile{texmf} tree):
% \begin{quote}
% \def\t{^^A
% \begin{tabular}{@{}>{\ttfamily}l@{ $\rightarrow$ }>{\ttfamily}l@{}}
%   ifpdf.sty & tex/generic/oberdiek/ifpdf.sty\\
%   ifpdf.pdf & doc/latex/oberdiek/ifpdf.pdf\\
%   test/ifpdf-test1.tex & doc/latex/oberdiek/test/ifpdf-test1.tex\\
%   ifpdf.dtx & source/latex/oberdiek/ifpdf.dtx\\
% \end{tabular}^^A
% }^^A
% \sbox0{\t}^^A
% \ifdim\wd0>\linewidth
%   \begingroup
%     \advance\linewidth by\leftmargin
%     \advance\linewidth by\rightmargin
%   \edef\x{\endgroup
%     \def\noexpand\lw{\the\linewidth}^^A
%   }\x
%   \def\lwbox{^^A
%     \leavevmode
%     \hbox to \linewidth{^^A
%       \kern-\leftmargin\relax
%       \hss
%       \usebox0
%       \hss
%       \kern-\rightmargin\relax
%     }^^A
%   }^^A
%   \ifdim\wd0>\lw
%     \sbox0{\small\t}^^A
%     \ifdim\wd0>\linewidth
%       \ifdim\wd0>\lw
%         \sbox0{\footnotesize\t}^^A
%         \ifdim\wd0>\linewidth
%           \ifdim\wd0>\lw
%             \sbox0{\scriptsize\t}^^A
%             \ifdim\wd0>\linewidth
%               \ifdim\wd0>\lw
%                 \sbox0{\tiny\t}^^A
%                 \ifdim\wd0>\linewidth
%                   \lwbox
%                 \else
%                   \usebox0
%                 \fi
%               \else
%                 \lwbox
%               \fi
%             \else
%               \usebox0
%             \fi
%           \else
%             \lwbox
%           \fi
%         \else
%           \usebox0
%         \fi
%       \else
%         \lwbox
%       \fi
%     \else
%       \usebox0
%     \fi
%   \else
%     \lwbox
%   \fi
% \else
%   \usebox0
% \fi
% \end{quote}
% If you have a \xfile{docstrip.cfg} that configures and enables \docstrip's
% TDS installing feature, then some files can already be in the right
% place, see the documentation of \docstrip.
%
% \subsection{Refresh file name databases}
%
% If your \TeX~distribution
% (\teTeX, \mikTeX, \dots) relies on file name databases, you must refresh
% these. For example, \teTeX\ users run \verb|texhash| or
% \verb|mktexlsr|.
%
% \subsection{Some details for the interested}
%
% \paragraph{Attached source.}
%
% The PDF documentation on CTAN also includes the
% \xfile{.dtx} source file. It can be extracted by
% AcrobatReader 6 or higher. Another option is \textsf{pdftk},
% e.g. unpack the file into the current directory:
% \begin{quote}
%   \verb|pdftk ifpdf.pdf unpack_files output .|
% \end{quote}
%
% \paragraph{Unpacking with \LaTeX.}
% The \xfile{.dtx} chooses its action depending on the format:
% \begin{description}
% \item[\plainTeX:] Run \docstrip\ and extract the files.
% \item[\LaTeX:] Generate the documentation.
% \end{description}
% If you insist on using \LaTeX\ for \docstrip\ (really,
% \docstrip\ does not need \LaTeX), then inform the autodetect routine
% about your intention:
% \begin{quote}
%   \verb|latex \let\install=y\input{ifpdf.dtx}|
% \end{quote}
% Do not forget to quote the argument according to the demands
% of your shell.
%
% \paragraph{Generating the documentation.}
% You can use both the \xfile{.dtx} or the \xfile{.drv} to generate
% the documentation. The process can be configured by the
% configuration file \xfile{ltxdoc.cfg}. For instance, put this
% line into this file, if you want to have A4 as paper format:
% \begin{quote}
%   \verb|\PassOptionsToClass{a4paper}{article}|
% \end{quote}
% An example follows how to generate the
% documentation with pdf\LaTeX:
% \begin{quote}
%\begin{verbatim}
%pdflatex ifpdf.dtx
%makeindex -s gind.ist ifpdf.idx
%pdflatex ifpdf.dtx
%makeindex -s gind.ist ifpdf.idx
%pdflatex ifpdf.dtx
%\end{verbatim}
% \end{quote}
%
% \section{Catalogue}
%
% The following XML file can be used as source for the
% \href{http://mirror.ctan.org/help/Catalogue/catalogue.html}{\TeX\ Catalogue}.
% The elements \texttt{caption} and \texttt{description} are imported
% from the original XML file from the Catalogue.
% The name of the XML file in the Catalogue is \xfile{ifpdf.xml}.
%    \begin{macrocode}
%<*catalogue>
<?xml version='1.0' encoding='us-ascii'?>
<!DOCTYPE entry SYSTEM 'catalogue.dtd'>
<entry datestamp='$Date$' modifier='$Author$' id='ifpdf'>
  <name>ifpdf</name>
  <caption>Provides the \ifpdf conditional.</caption>
  <authorref id='auth:oberdiek'/>
  <copyright owner='Heiko Oberdiek' year='2001,2005-2011'/>
  <license type='lppl1.3'/>
  <version number='2.3'/>
  <description>
      This package looks for pdfTeX in pdf mode and implements and
      sets the switch <tt>\ifpdf</tt>. The detection is based on the
      value of <tt>\pdfoutput</tt> (which the package will not
      change).  The package works with plain or LaTeX formats.  To use
      it with LaTeX simply <tt>\usepackage{ifpdf}</tt>. Then use
      <tt>\ifpdf ... \else ... \fi</tt>.
    <p/>
    The package is part of the <xref refid='oberdiek'>oberdiek</xref> bundle.
  </description>
  <documentation details='Package documentation'
      href='ctan:/macros/latex/contrib/oberdiek/ifpdf.pdf'/>
  <ctan file='true' path='/macros/latex/contrib/oberdiek/ifpdf.dtx'/>
  <miktex location='oberdiek'/>
  <texlive location='oberdiek'/>
  <install path='/macros/latex/contrib/oberdiek/oberdiek.tds.zip'/>
</entry>
%</catalogue>
%    \end{macrocode}
%
% \begin{History}
%   \begin{Version}{2001/06/14 v1.0}
%   \item
%     First public version.
%   \end{Version}
%   \begin{Version}{2001/07/14 v1.1}
%   \item
%     Documentation addition: global options
%   \end{Version}
%   \begin{Version}{2001/09/26 v1.2}
%   \item
%     Documentation typo corrected.
%   \item
%     Version number corrected.
%   \item
%     Line number in log entry removed.
%   \end{Version}
%   \begin{Version}{2005/07/22 v1.3}
%   \item
%     Some source code comments from Robin Fairbairns added.
%   \item
%     Bug fix for negative values of \cs{pdfoutput}
%     (Oleg Katsitadze)
%   \item
%     LPPL 1.3
%   \item
%     Installation section with locations added.
%   \end{Version}
%   \begin{Version}{2006/02/20 v1.4}
%   \item
%     DTX framework.
%   \item
%     More robust check in case of undefined \cs{pdfoutput}.
%   \item
%     Extended documentation.
%   \end{Version}
%   \begin{Version}{2007/09/09 v1.5}
%   \item
%     Catcode settings added.
%   \end{Version}
%   \begin{Version}{2007/12/12 v1.6}
%   \item
%     Minor update.
%   \end{Version}
%   \begin{Version}{2008/12/12 v1.7}
%   \item
%     Fix in documentation for \cs{boolean} (found by S. Venkataraman).
%   \item
%     Code is not changed.
%   \end{Version}
%   \begin{Version}{2009/04/10 v2.0}
%   \item
%     Support for \LuaTeX\ 0.40 added.
%   \item
%     Checks, whether \cs{pdfoutput} was changed.
%   \end{Version}
%   \begin{Version}{2010/01/28 v2.1}
%   \item
%     Compatibility to ini\TeX\ added.
%   \end{Version}
%   \begin{Version}{2010/09/13 v2.2}
%   \item
%     Code with \cs{escapechar} rewritten because of
%     LuaTeX bug: \cs{escapechar} does not work.
%   \end{Version}
%   \begin{Version}{2011/01/30 v2.3}
%   \item
%     Already loaded package files are not input in \hologo{plainTeX}.
%   \end{Version}
% \end{History}
%
% \PrintIndex
%
% \Finale
\endinput

%        (quote the arguments according to the demands of your shell)
%
% Documentation:
%    (a) If ifpdf.drv is present:
%           latex ifpdf.drv
%    (b) Without ifpdf.drv:
%           latex ifpdf.dtx; ...
%    The class ltxdoc loads the configuration file ltxdoc.cfg
%    if available. Here you can specify further options, e.g.
%    use A4 as paper format:
%       \PassOptionsToClass{a4paper}{article}
%
%    Programm calls to get the documentation (example):
%       pdflatex ifpdf.dtx
%       makeindex -s gind.ist ifpdf.idx
%       pdflatex ifpdf.dtx
%       makeindex -s gind.ist ifpdf.idx
%       pdflatex ifpdf.dtx
%
% Installation:
%    TDS:tex/generic/oberdiek/ifpdf.sty
%    TDS:doc/latex/oberdiek/ifpdf.pdf
%    TDS:doc/latex/oberdiek/test/ifpdf-test1.tex
%    TDS:source/latex/oberdiek/ifpdf.dtx
%
%<*ignore>
\begingroup
  \catcode123=1 %
  \catcode125=2 %
  \def\x{LaTeX2e}%
\expandafter\endgroup
\ifcase 0\ifx\install y1\fi\expandafter
         \ifx\csname processbatchFile\endcsname\relax\else1\fi
         \ifx\fmtname\x\else 1\fi\relax
\else\csname fi\endcsname
%</ignore>
%<*install>
\input docstrip.tex
\Msg{************************************************************************}
\Msg{* Installation}
\Msg{* Package: ifpdf 2011/01/30 v2.3 Provides the ifpdf switch (HO)}
\Msg{************************************************************************}

\keepsilent
\askforoverwritefalse

\let\MetaPrefix\relax
\preamble

This is a generated file.

Project: ifpdf
Version: 2011/01/30 v2.3

Copyright (C) 2001, 2005-2011 by
   Heiko Oberdiek <heiko.oberdiek at googlemail.com>

This work may be distributed and/or modified under the
conditions of the LaTeX Project Public License, either
version 1.3c of this license or (at your option) any later
version. This version of this license is in
   http://www.latex-project.org/lppl/lppl-1-3c.txt
and the latest version of this license is in
   http://www.latex-project.org/lppl.txt
and version 1.3 or later is part of all distributions of
LaTeX version 2005/12/01 or later.

This work has the LPPL maintenance status "maintained".

This Current Maintainer of this work is Heiko Oberdiek.

The Base Interpreter refers to any `TeX-Format',
because some files are installed in TDS:tex/generic//.

This work consists of the main source file ifpdf.dtx
and the derived files
   ifpdf.sty, ifpdf.pdf, ifpdf.ins, ifpdf.drv, ifpdf-test1.tex.

\endpreamble
\let\MetaPrefix\DoubleperCent

\generate{%
  \file{ifpdf.ins}{\from{ifpdf.dtx}{install}}%
  \file{ifpdf.drv}{\from{ifpdf.dtx}{driver}}%
  \usedir{tex/generic/oberdiek}%
  \file{ifpdf.sty}{\from{ifpdf.dtx}{package}}%
  \usedir{doc/latex/oberdiek/test}%
  \file{ifpdf-test1.tex}{\from{ifpdf.dtx}{test1}}%
  \nopreamble
  \nopostamble
  \usedir{source/latex/oberdiek/catalogue}%
  \file{ifpdf.xml}{\from{ifpdf.dtx}{catalogue}}%
}

\catcode32=13\relax% active space
\let =\space%
\Msg{************************************************************************}
\Msg{*}
\Msg{* To finish the installation you have to move the following}
\Msg{* file into a directory searched by TeX:}
\Msg{*}
\Msg{*     ifpdf.sty}
\Msg{*}
\Msg{* To produce the documentation run the file `ifpdf.drv'}
\Msg{* through LaTeX.}
\Msg{*}
\Msg{* Happy TeXing!}
\Msg{*}
\Msg{************************************************************************}

\endbatchfile
%</install>
%<*ignore>
\fi
%</ignore>
%<*driver>
\NeedsTeXFormat{LaTeX2e}
\ProvidesFile{ifpdf.drv}%
  [2011/01/30 v2.3 Provides the ifpdf switch (HO)]%
\documentclass{ltxdoc}
\usepackage{holtxdoc}[2011/11/22]
\begin{document}
  \DocInput{ifpdf.dtx}%
\end{document}
%</driver>
% \fi
%
% \CheckSum{600}
%
% \CharacterTable
%  {Upper-case    \A\B\C\D\E\F\G\H\I\J\K\L\M\N\O\P\Q\R\S\T\U\V\W\X\Y\Z
%   Lower-case    \a\b\c\d\e\f\g\h\i\j\k\l\m\n\o\p\q\r\s\t\u\v\w\x\y\z
%   Digits        \0\1\2\3\4\5\6\7\8\9
%   Exclamation   \!     Double quote  \"     Hash (number) \#
%   Dollar        \$     Percent       \%     Ampersand     \&
%   Acute accent  \'     Left paren    \(     Right paren   \)
%   Asterisk      \*     Plus          \+     Comma         \,
%   Minus         \-     Point         \.     Solidus       \/
%   Colon         \:     Semicolon     \;     Less than     \<
%   Equals        \=     Greater than  \>     Question mark \?
%   Commercial at \@     Left bracket  \[     Backslash     \\
%   Right bracket \]     Circumflex    \^     Underscore    \_
%   Grave accent  \`     Left brace    \{     Vertical bar  \|
%   Right brace   \}     Tilde         \~}
%
% \GetFileInfo{ifpdf.drv}
%
% \title{The \xpackage{ifpdf} package}
% \date{2011/01/30 v2.3}
% \author{Heiko Oberdiek\\\xemail{heiko.oberdiek at googlemail.com}}
%
% \maketitle
%
% \begin{abstract}
% This package looks for \pdfTeX\ in pdf mode and
% implements and sets the switch \cs{ifpdf}.
% The detection is based on \cs{pdfoutput} and
% the package will not change this value.
% It works with plain or \LaTeX\ formats.
% \end{abstract}
%
% \tableofcontents
%
% \hypersetup{bookmarksopenlevel=2}
% \section{Documentation}
%
% \subsection{Introduction}
%
% It is commonly known that H\`an Th\^e Th\`anh's \pdfTeX\ generates
% PDF output directly and many people uses \pdfTeX\ for this purpose.
% However the DVI output was never thrown away. In contrary, he
% new features for typesetting that works in both PDF and DVI mode.
%
% In the meantime many \TeX\ distributions replace the traditional
% \TeX\ binary with \pdfTeX. Then, for example, called as |latex|
% \pdfTeX\ works in DVI mode with the \LaTeX\ format preloaded,
% called as |pdflatex| \pdfTeX\ starts in PDF mode.
%
% Often packages or users want to know, whether the current document
% is typset by \pdfTeX\ in PDF mode, because the different modes
% have different capabilities (color setting, graphics inclusion,
% \dots). For this purpose \pdfTeX's
% \cs{pdfoutput} can be asked.
%
% As regulary reader of \TeX\ newsgroups and mailing lists I could
% observe many problems with this task. Common errors are:
% \begin{itemize}
% \item
%   \pdfTeX\ has \emph{two} modes. Using \pdfTeX\ does not mean
%   that the user always want to have PDF mode. For example, the
%   PostScript support is better in DVI mode in conjunction with a
%   PostScript aware DVI driver (e.g. dvips). Also the additional
%   typesetting features are mode independent and also available
%   in DVI mode.
% \item
%   \LaTeX's \cs{@ifundefined} inherited the side effect from
%   \cs{csname}. Unknown commands are defined with the meaning
%   of \cs{relax}. If it is checked, whether \cs{pdfoutput} is
%   defined, then this should not be forgotten.
% \item
%   Having \cs{pdfoutput} does not automatically mean PDF mode.
%   Also the value of \cs{pdfoutput} must be asked.
% \item
%   \cs{pdfoutput} must not be destroyed in some way. Later
%   code and packages are fooled then and will perhaps make
%   wrong decisions. For example they may drop support for
%   PDF mode, because they do not know that \pdfTeX\ is
%   running at all.
% \end{itemize}
%
% Robin Fairbairns provides an entry for this topic in his
% excellent FAQ (\url{http://www.tex.ac.uk/faq}):
% \href{http://www.tex.ac.uk/cgi-bin/texfaq2html?label=ifpdf}^^A
% {Am I using PDFTeX?}
%
% \subsection{Usage}
%
% The package \xpackage{ifpdf} can be used with both \plainTeX\
% and \LaTeX:
% \begin{description}
% \item[\plainTeX:] |\input ifpdf.sty|
% \item[\LaTeXe:]   |\usepackage{ifpdf}|
% \end{description}
%
% \DescribeMacro{\ifpdf}
% The package provides the switch \cs{ifpdf}:
% \begin{quote}
%   |\ifpdf|\\
%   \hspace*{1.5em}\dots\ do things, if \pdfTeX\ is running in pdf mode \dots\\
%   |\else|\\
%   \hspace*{1.5em}\dots\ other \TeX\ or \pdfTeX\ in dvi mode \dots\\
%   |\fi|
% \end{quote}
% Users of the package \xpackage{ifthen} can use the switch as boolean:
% \begin{quote}
%   |\boolean{pdf}|
% \end{quote}
% The package can also be used to set global
% documentclass options:
% \begin{quote}
%   |\RequirePackage{ifpdf}|\\
%   |\ifpdf|\\
%   \hspace*{1.5em}|\documentclass[pdftex,|\dots|]{|\dots|}|\\
%   |\else|\\
%   \hspace*{1.5em}|\documentclass[|\dots|]{|\dots|}|\\
%   |\fi|
% \end{quote}
%
% \subsection{Specification}
%
% The package have the following properties:
% \begin{itemize}
% \item It asks the setting of \cs{pdfoutput} for detecting
%   \pdfTeX\ in PDF mode.
% \item It never changes \cs{pdfoutput}.
% \item If \cs{pdfoutput} is undefined or has the meaning \cs{relax},
%   but the engine provides the primitive \cs{pdfoutput},
%   then \cs{pdfoutput} is enabled or restored if possible
%   (only \LuaTeX, version 0.36.0 or higher).
% \item It can be used with many formats including
%   \plainTeX\ and \LaTeX.
% \end{itemize}
% The mode detection implements the following algorithm:
% \begin{quote}
% \def\={${:}{=}$}
% \def\>{\null\qquad\ignorespaces}
% \def\comment#1{^^A
%   \textrm{\itshape\% #1}^^A
% }
% \sffamily
% if undefined(\cs{pdfoutput})\\
% \> \cs{ifpdf} \= \cs{iffalse} \comment{\pdfTeX is not running}\\
% else\\
% \> if \cs{pdfoutput} $\leq$ 0\\
% \>\> \cs{ifpdf} \= \cs{iffalse} \comment{\pdfTeX\ in DVI mode}\\
% \> else\\
% \>\> \cs{ifpdf} \= \cs{iftrue} \comment{\pdfTeX\ in PDF mode}\\
% \> fi\\
% fi
% \end{quote}
% The function \textsf{undefined} checks both cases, undefined command
% and \cs{relax}.
%
% \hypersetup{bookmarksopenlevel=1}
%
% \StopEventually{
% }
%
% \section{Implementation}
%
%    \begin{macrocode}
%<*package>
%    \end{macrocode}
%
% \subsection{Reload check and package identification}
%    Reload check, especially if the package is not used with \LaTeX.
%    \begin{macrocode}
\begingroup\catcode61\catcode48\catcode32=10\relax%
  \catcode13=5 % ^^M
  \endlinechar=13 %
  \catcode35=6 % #
  \catcode39=12 % '
  \catcode44=12 % ,
  \catcode45=12 % -
  \catcode46=12 % .
  \catcode58=12 % :
  \catcode64=11 % @
  \catcode123=1 % {
  \catcode125=2 % }
  \expandafter\let\expandafter\x\csname ver@ifpdf.sty\endcsname
  \ifx\x\relax % plain-TeX, first loading
  \else
    \def\empty{}%
    \ifx\x\empty % LaTeX, first loading,
      % variable is initialized, but \ProvidesPackage not yet seen
    \else
      \expandafter\ifx\csname PackageInfo\endcsname\relax
        \def\x#1#2{%
          \immediate\write-1{Package #1 Info: #2.}%
        }%
      \else
        \def\x#1#2{\PackageInfo{#1}{#2, stopped}}%
      \fi
      \x{ifpdf}{The package is already loaded}%
      \aftergroup\endinput
    \fi
  \fi
\endgroup%
%    \end{macrocode}
%    Package identification:
%    \begin{macrocode}
\begingroup\catcode61\catcode48\catcode32=10\relax%
  \catcode13=5 % ^^M
  \endlinechar=13 %
  \catcode35=6 % #
  \catcode39=12 % '
  \catcode40=12 % (
  \catcode41=12 % )
  \catcode44=12 % ,
  \catcode45=12 % -
  \catcode46=12 % .
  \catcode47=12 % /
  \catcode58=12 % :
  \catcode64=11 % @
  \catcode91=12 % [
  \catcode93=12 % ]
  \catcode123=1 % {
  \catcode125=2 % }
  \expandafter\ifx\csname ProvidesPackage\endcsname\relax
    \def\x#1#2#3[#4]{\endgroup
      \immediate\write-1{Package: #3 #4}%
      \xdef#1{#4}%
    }%
  \else
    \def\x#1#2[#3]{\endgroup
      #2[{#3}]%
      \ifx#1\@undefined
        \xdef#1{#3}%
      \fi
      \ifx#1\relax
        \xdef#1{#3}%
      \fi
    }%
  \fi
\expandafter\x\csname ver@ifpdf.sty\endcsname
\ProvidesPackage{ifpdf}%
  [2011/01/30 v2.3 Provides the ifpdf switch (HO)]%
%    \end{macrocode}
%
% \subsection{Catcodes}
%
%    \begin{macrocode}
\begingroup\catcode61\catcode48\catcode32=10\relax%
  \catcode13=5 % ^^M
  \endlinechar=13 %
  \catcode123=1 % {
  \catcode125=2 % }
  \catcode64=11 % @
  \def\x{\endgroup
    \expandafter\edef\csname ifpdf@AtEnd\endcsname{%
      \endlinechar=\the\endlinechar\relax
      \catcode13=\the\catcode13\relax
      \catcode32=\the\catcode32\relax
      \catcode35=\the\catcode35\relax
      \catcode61=\the\catcode61\relax
      \catcode64=\the\catcode64\relax
      \catcode123=\the\catcode123\relax
      \catcode125=\the\catcode125\relax
    }%
  }%
\x\catcode61\catcode48\catcode32=10\relax%
\catcode13=5 % ^^M
\endlinechar=13 %
\catcode35=6 % #
\catcode64=11 % @
\catcode123=1 % {
\catcode125=2 % }
\def\TMP@EnsureCode#1#2{%
  \edef\ifpdf@AtEnd{%
    \ifpdf@AtEnd
    \catcode#1=\the\catcode#1\relax
  }%
  \catcode#1=#2\relax
}
\TMP@EnsureCode{10}{12}% ^^J
\TMP@EnsureCode{39}{12}% '
\TMP@EnsureCode{40}{12}% (
\TMP@EnsureCode{41}{12}% )
\TMP@EnsureCode{44}{12}% ,
\TMP@EnsureCode{45}{12}% -
\TMP@EnsureCode{46}{12}% .
\TMP@EnsureCode{47}{12}% /
\TMP@EnsureCode{58}{12}% :
\TMP@EnsureCode{60}{12}% <
\TMP@EnsureCode{91}{12}% [
\TMP@EnsureCode{93}{12}% ]
\TMP@EnsureCode{94}{7}% ^
\TMP@EnsureCode{96}{12}% `
\edef\ifpdf@AtEnd{\ifpdf@AtEnd\noexpand\endinput}
%    \end{macrocode}
%
% \subsection{Check for previously defined \cs{ifpdf}}
%
%    \begin{macrocode}
\begingroup
  \expandafter\ifx\csname ifpdf\endcsname\relax
  \else
    \edef\i/{\expandafter\string\csname ifpdf\endcsname}%
    \expandafter\ifx\csname PackageError\endcsname\relax
      \def\x#1#2{%
        \edef\z{#2}%
        \expandafter\errhelp\expandafter{\z}%
        \errmessage{Package ifpdf Error: #1}%
      }%
      \def\y{^^J}%
      \newlinechar=10 %
    \else
      \def\x#1#2{%
        \PackageError{ifpdf}{#1}{#2}%
      }%
      \def\y{\MessageBreak}%
    \fi
    \x{Name clash, \i/ is already defined}{%
      Incompatible versions of \i/ can cause problems,\y
      therefore package loading is aborted.%
    }%
    \endgroup
    \expandafter\ifpdf@AtEnd
  \fi%
\endgroup
%    \end{macrocode}
%
% \subsection{\cs{pdfoutput} and \LuaTeX}
%
%    It might happen, that \LuaTeX\ is running, but \cs{pdfoutput}
%    does not exist. In version 0.40 only \cs{directlua}
%    is available at startup time. The enabling Lua function was
%    already added in version 0.36. Thus we can ignore older versions,
%    here \cs{pdfoutput} is available at startup time.
%
%    \begin{macrocode}
\begingroup
  \def\skip#1\relax\begingroup{}%
  \expandafter\ifx\csname pdfoutput\endcsname\relax
  \else
    \expandafter\skip
  \fi
  \expandafter\ifx\csname directlua\endcsname\relax
    \expandafter\skip
  \fi
\endgroup
\begingroup\expandafter\expandafter\expandafter\endgroup
\expandafter\ifx\csname RequirePackage\endcsname\relax
  \def\TMP@RequirePackage#1[#2]{%
    \begingroup\expandafter\expandafter\expandafter\endgroup
    \expandafter\ifx\csname ver@#1.sty\endcsname\relax
      \input #1.sty\relax
    \fi
  }%
  \TMP@RequirePackage{ifluatex}[2009/04/10]%
\else
  \RequirePackage{ifluatex}[2009/04/10]%
\fi
\ifluatex
  \ifnum\luatexversion<36 %
%    \end{macrocode}
%    Unhappily \LuaTeX's \cs{primitive} (derived from
%    pdfTeX's \cs{pdfprimitive}) cannot be used:
%    \begin{quote}
%    |\protected\gdef\pdfoutput{\primitive\pdfoutput}|
%    \end{quote}
%    Setting a value works, but getting fails,
%    because \TeX\ does no longer see it as number. It
%    is unexpandable and breaks numerical contexts.
%    This was fixed in \hologo{pdfTeX} 1.40.10
%    (bugfix \#4289: ``|\primitive\pdfoutput| cannot be queried'').
%    \begin{macrocode}
  \else
    \begingroup
      \directlua{tex.enableprimitives('ifpdf', {'pdfoutput'})}%
      \global\let\pdfoutput\ifpdfpdfoutput
    \endgroup
  \fi
\fi
\relax\begingroup\endgroup
%    \end{macrocode}
%
% \subsection{\cs{ifpdf}}
%
%    \begin{macro}{\ifpdf}
%    Create and set the switch. \cs{newif} initializes the
%    switch with \cs{iffalse}. \cs{newif} is \cs{outer}
%    in \plainTeX.
%    \begin{macrocode}
\begingroup\expandafter\expandafter\expandafter\endgroup
\expandafter\ifx\csname newif\endcsname\relax
  \edef\pdffalse{%
    \let
    \expandafter\noexpand\csname ifpdf\endcsname
    \expandafter\noexpand\csname iffalse\endcsname
  }%
  \edef\pdftrue{%
    \let
    \expandafter\noexpand\csname ifpdf\endcsname
    \expandafter\noexpand\csname iftrue\endcsname
  }%
  \pdffalse
\else
  \csname newif\expandafter\endcsname\csname ifpdf\endcsname
\fi
%    \end{macrocode}
%    Test \cs{pdfoutput}. Is it  defined and different from
%    \cs{relax}? Someone could have used \LaTeX\ internal
%    \cs{@ifundefined}, or something else involving.
%    Notice, \cs{csname} is executed inside a group for the test
%    to cancel the side effect of \cs{csname}.
%    \begin{macrocode}
\begingroup\expandafter\expandafter\expandafter\endgroup
\expandafter\ifx\csname pdfoutput\endcsname\relax
\else
  \ifnum\pdfoutput<1 %
%    \end{macrocode}
%    \cs{pdfoutput}|=0| or negative, so not generating pdf.
%    \begin{macrocode}
  \else
    \pdftrue
  \fi
\fi
%    \end{macrocode}
%    \end{macro}
%
% \subsection{Test for fool attempts}
%
%    \begin{macrocode}
\begingroup
  \expandafter\ifx\csname pdfoutput\endcsname\relax
  \else
    \escapechar=92 %
    \edef\m{\meaning\pdfoutput}%
    \edef\p{\string\pdfoutput}%
    \ifx\m\p
    \else
      \expandafter\ifx\csname PackageWarningNoLine\endcsname\relax
        \def\PackageWarningNoLine#1#2{%
          \immediate\write16{%
            Package `#1' Warning: #2.%
          }%
        }%
      \fi
      \PackageWarningNoLine{ifpdf}{%
        Someone has redefined \string\pdfoutput%
      }%
    \fi
  \fi
\endgroup
%    \end{macrocode}
%
% \subsection{Protocol entry}
%
%     Log comment:
%    \begin{macrocode}
\begingroup
  \expandafter\ifx\csname PackageInfo\endcsname\relax
    \def\x#1#2{%
      \immediate\write-1{Package #1 Info: #2.}%
    }%
  \else
    \let\x\PackageInfo
    \expandafter\let\csname on@line\endcsname\empty
  \fi
  \x{ifpdf}{pdfTeX in PDF mode is \ifpdf\else not \fi detected}%
\endgroup
%    \end{macrocode}
%    \begin{macrocode}
\ifpdf@AtEnd%
%    \end{macrocode}
%    \begin{macrocode}
%</package>
%    \end{macrocode}
%
% \section{Test}
%
% \subsection{Catcode checks for loading}
%
%    \begin{macrocode}
%<*test1>
%    \end{macrocode}
%    \begin{macrocode}
\catcode`\{=1 %
\catcode`\}=2 %
\catcode`\#=6 %
\catcode`\@=11 %
\expandafter\ifx\csname count@\endcsname\relax
  \countdef\count@=255 %
\fi
\expandafter\ifx\csname @gobble\endcsname\relax
  \long\def\@gobble#1{}%
\fi
\expandafter\ifx\csname @firstofone\endcsname\relax
  \long\def\@firstofone#1{#1}%
\fi
\expandafter\ifx\csname loop\endcsname\relax
  \expandafter\@firstofone
\else
  \expandafter\@gobble
\fi
{%
  \def\loop#1\repeat{%
    \def\body{#1}%
    \iterate
  }%
  \def\iterate{%
    \body
      \let\next\iterate
    \else
      \let\next\relax
    \fi
    \next
  }%
  \let\repeat=\fi
}%
\def\RestoreCatcodes{}
\count@=0 %
\loop
  \edef\RestoreCatcodes{%
    \RestoreCatcodes
    \catcode\the\count@=\the\catcode\count@\relax
  }%
\ifnum\count@<255 %
  \advance\count@ 1 %
\repeat

\def\RangeCatcodeInvalid#1#2{%
  \count@=#1\relax
  \loop
    \catcode\count@=15 %
  \ifnum\count@<#2\relax
    \advance\count@ 1 %
  \repeat
}
\def\RangeCatcodeCheck#1#2#3{%
  \count@=#1\relax
  \loop
    \ifnum#3=\catcode\count@
    \else
      \errmessage{%
        Character \the\count@\space
        with wrong catcode \the\catcode\count@\space
        instead of \number#3%
      }%
    \fi
  \ifnum\count@<#2\relax
    \advance\count@ 1 %
  \repeat
}
\def\space{ }
\expandafter\ifx\csname LoadCommand\endcsname\relax
  \def\LoadCommand{\input ifpdf.sty\relax}%
\fi
\def\Test{%
  \RangeCatcodeInvalid{0}{47}%
  \RangeCatcodeInvalid{58}{64}%
  \RangeCatcodeInvalid{91}{96}%
  \RangeCatcodeInvalid{123}{255}%
  \catcode`\@=12 %
  \catcode`\\=0 %
  \catcode`\%=14 %
  \LoadCommand
  \RangeCatcodeCheck{0}{36}{15}%
  \RangeCatcodeCheck{37}{37}{14}%
  \RangeCatcodeCheck{38}{47}{15}%
  \RangeCatcodeCheck{48}{57}{12}%
  \RangeCatcodeCheck{58}{63}{15}%
  \RangeCatcodeCheck{64}{64}{12}%
  \RangeCatcodeCheck{65}{90}{11}%
  \RangeCatcodeCheck{91}{91}{15}%
  \RangeCatcodeCheck{92}{92}{0}%
  \RangeCatcodeCheck{93}{96}{15}%
  \RangeCatcodeCheck{97}{122}{11}%
  \RangeCatcodeCheck{123}{255}{15}%
  \RestoreCatcodes
}
\Test
\csname @@end\endcsname
\end
%    \end{macrocode}
%    \begin{macrocode}
%</test1>
%    \end{macrocode}
%
% \section{Installation}
%
% \subsection{Download}
%
% \paragraph{Package.} This package is available on
% CTAN\footnote{\url{ftp://ftp.ctan.org/tex-archive/}}:
% \begin{description}
% \item[\CTAN{macros/latex/contrib/oberdiek/ifpdf.dtx}] The source file.
% \item[\CTAN{macros/latex/contrib/oberdiek/ifpdf.pdf}] Documentation.
% \end{description}
%
%
% \paragraph{Bundle.} All the packages of the bundle `oberdiek'
% are also available in a TDS compliant ZIP archive. There
% the packages are already unpacked and the documentation files
% are generated. The files and directories obey the TDS standard.
% \begin{description}
% \item[\CTAN{install/macros/latex/contrib/oberdiek.tds.zip}]
% \end{description}
% \emph{TDS} refers to the standard ``A Directory Structure
% for \TeX\ Files'' (\CTAN{tds/tds.pdf}). Directories
% with \xfile{texmf} in their name are usually organized this way.
%
% \subsection{Bundle installation}
%
% \paragraph{Unpacking.} Unpack the \xfile{oberdiek.tds.zip} in the
% TDS tree (also known as \xfile{texmf} tree) of your choice.
% Example (linux):
% \begin{quote}
%   |unzip oberdiek.tds.zip -d ~/texmf|
% \end{quote}
%
% \paragraph{Script installation.}
% Check the directory \xfile{TDS:scripts/oberdiek/} for
% scripts that need further installation steps.
% Package \xpackage{attachfile2} comes with the Perl script
% \xfile{pdfatfi.pl} that should be installed in such a way
% that it can be called as \texttt{pdfatfi}.
% Example (linux):
% \begin{quote}
%   |chmod +x scripts/oberdiek/pdfatfi.pl|\\
%   |cp scripts/oberdiek/pdfatfi.pl /usr/local/bin/|
% \end{quote}
%
% \subsection{Package installation}
%
% \paragraph{Unpacking.} The \xfile{.dtx} file is a self-extracting
% \docstrip\ archive. The files are extracted by running the
% \xfile{.dtx} through \plainTeX:
% \begin{quote}
%   \verb|tex ifpdf.dtx|
% \end{quote}
%
% \paragraph{TDS.} Now the different files must be moved into
% the different directories in your installation TDS tree
% (also known as \xfile{texmf} tree):
% \begin{quote}
% \def\t{^^A
% \begin{tabular}{@{}>{\ttfamily}l@{ $\rightarrow$ }>{\ttfamily}l@{}}
%   ifpdf.sty & tex/generic/oberdiek/ifpdf.sty\\
%   ifpdf.pdf & doc/latex/oberdiek/ifpdf.pdf\\
%   test/ifpdf-test1.tex & doc/latex/oberdiek/test/ifpdf-test1.tex\\
%   ifpdf.dtx & source/latex/oberdiek/ifpdf.dtx\\
% \end{tabular}^^A
% }^^A
% \sbox0{\t}^^A
% \ifdim\wd0>\linewidth
%   \begingroup
%     \advance\linewidth by\leftmargin
%     \advance\linewidth by\rightmargin
%   \edef\x{\endgroup
%     \def\noexpand\lw{\the\linewidth}^^A
%   }\x
%   \def\lwbox{^^A
%     \leavevmode
%     \hbox to \linewidth{^^A
%       \kern-\leftmargin\relax
%       \hss
%       \usebox0
%       \hss
%       \kern-\rightmargin\relax
%     }^^A
%   }^^A
%   \ifdim\wd0>\lw
%     \sbox0{\small\t}^^A
%     \ifdim\wd0>\linewidth
%       \ifdim\wd0>\lw
%         \sbox0{\footnotesize\t}^^A
%         \ifdim\wd0>\linewidth
%           \ifdim\wd0>\lw
%             \sbox0{\scriptsize\t}^^A
%             \ifdim\wd0>\linewidth
%               \ifdim\wd0>\lw
%                 \sbox0{\tiny\t}^^A
%                 \ifdim\wd0>\linewidth
%                   \lwbox
%                 \else
%                   \usebox0
%                 \fi
%               \else
%                 \lwbox
%               \fi
%             \else
%               \usebox0
%             \fi
%           \else
%             \lwbox
%           \fi
%         \else
%           \usebox0
%         \fi
%       \else
%         \lwbox
%       \fi
%     \else
%       \usebox0
%     \fi
%   \else
%     \lwbox
%   \fi
% \else
%   \usebox0
% \fi
% \end{quote}
% If you have a \xfile{docstrip.cfg} that configures and enables \docstrip's
% TDS installing feature, then some files can already be in the right
% place, see the documentation of \docstrip.
%
% \subsection{Refresh file name databases}
%
% If your \TeX~distribution
% (\teTeX, \mikTeX, \dots) relies on file name databases, you must refresh
% these. For example, \teTeX\ users run \verb|texhash| or
% \verb|mktexlsr|.
%
% \subsection{Some details for the interested}
%
% \paragraph{Attached source.}
%
% The PDF documentation on CTAN also includes the
% \xfile{.dtx} source file. It can be extracted by
% AcrobatReader 6 or higher. Another option is \textsf{pdftk},
% e.g. unpack the file into the current directory:
% \begin{quote}
%   \verb|pdftk ifpdf.pdf unpack_files output .|
% \end{quote}
%
% \paragraph{Unpacking with \LaTeX.}
% The \xfile{.dtx} chooses its action depending on the format:
% \begin{description}
% \item[\plainTeX:] Run \docstrip\ and extract the files.
% \item[\LaTeX:] Generate the documentation.
% \end{description}
% If you insist on using \LaTeX\ for \docstrip\ (really,
% \docstrip\ does not need \LaTeX), then inform the autodetect routine
% about your intention:
% \begin{quote}
%   \verb|latex \let\install=y% \iffalse meta-comment
%
% File: ifpdf.dtx
% Version: 2011/01/30 v2.3
% Info: Provides the ifpdf switch
%
% Copyright (C) 2001, 2005-2011 by
%    Heiko Oberdiek <heiko.oberdiek at googlemail.com>
%
% This work may be distributed and/or modified under the
% conditions of the LaTeX Project Public License, either
% version 1.3c of this license or (at your option) any later
% version. This version of this license is in
%    http://www.latex-project.org/lppl/lppl-1-3c.txt
% and the latest version of this license is in
%    http://www.latex-project.org/lppl.txt
% and version 1.3 or later is part of all distributions of
% LaTeX version 2005/12/01 or later.
%
% This work has the LPPL maintenance status "maintained".
%
% This Current Maintainer of this work is Heiko Oberdiek.
%
% The Base Interpreter refers to any `TeX-Format',
% because some files are installed in TDS:tex/generic//.
%
% This work consists of the main source file ifpdf.dtx
% and the derived files
%    ifpdf.sty, ifpdf.pdf, ifpdf.ins, ifpdf.drv, ifpdf-test1.tex.
%
% Distribution:
%    CTAN:macros/latex/contrib/oberdiek/ifpdf.dtx
%    CTAN:macros/latex/contrib/oberdiek/ifpdf.pdf
%
% Unpacking:
%    (a) If ifpdf.ins is present:
%           tex ifpdf.ins
%    (b) Without ifpdf.ins:
%           tex ifpdf.dtx
%    (c) If you insist on using LaTeX
%           latex \let\install=y\input{ifpdf.dtx}
%        (quote the arguments according to the demands of your shell)
%
% Documentation:
%    (a) If ifpdf.drv is present:
%           latex ifpdf.drv
%    (b) Without ifpdf.drv:
%           latex ifpdf.dtx; ...
%    The class ltxdoc loads the configuration file ltxdoc.cfg
%    if available. Here you can specify further options, e.g.
%    use A4 as paper format:
%       \PassOptionsToClass{a4paper}{article}
%
%    Programm calls to get the documentation (example):
%       pdflatex ifpdf.dtx
%       makeindex -s gind.ist ifpdf.idx
%       pdflatex ifpdf.dtx
%       makeindex -s gind.ist ifpdf.idx
%       pdflatex ifpdf.dtx
%
% Installation:
%    TDS:tex/generic/oberdiek/ifpdf.sty
%    TDS:doc/latex/oberdiek/ifpdf.pdf
%    TDS:doc/latex/oberdiek/test/ifpdf-test1.tex
%    TDS:source/latex/oberdiek/ifpdf.dtx
%
%<*ignore>
\begingroup
  \catcode123=1 %
  \catcode125=2 %
  \def\x{LaTeX2e}%
\expandafter\endgroup
\ifcase 0\ifx\install y1\fi\expandafter
         \ifx\csname processbatchFile\endcsname\relax\else1\fi
         \ifx\fmtname\x\else 1\fi\relax
\else\csname fi\endcsname
%</ignore>
%<*install>
\input docstrip.tex
\Msg{************************************************************************}
\Msg{* Installation}
\Msg{* Package: ifpdf 2011/01/30 v2.3 Provides the ifpdf switch (HO)}
\Msg{************************************************************************}

\keepsilent
\askforoverwritefalse

\let\MetaPrefix\relax
\preamble

This is a generated file.

Project: ifpdf
Version: 2011/01/30 v2.3

Copyright (C) 2001, 2005-2011 by
   Heiko Oberdiek <heiko.oberdiek at googlemail.com>

This work may be distributed and/or modified under the
conditions of the LaTeX Project Public License, either
version 1.3c of this license or (at your option) any later
version. This version of this license is in
   http://www.latex-project.org/lppl/lppl-1-3c.txt
and the latest version of this license is in
   http://www.latex-project.org/lppl.txt
and version 1.3 or later is part of all distributions of
LaTeX version 2005/12/01 or later.

This work has the LPPL maintenance status "maintained".

This Current Maintainer of this work is Heiko Oberdiek.

The Base Interpreter refers to any `TeX-Format',
because some files are installed in TDS:tex/generic//.

This work consists of the main source file ifpdf.dtx
and the derived files
   ifpdf.sty, ifpdf.pdf, ifpdf.ins, ifpdf.drv, ifpdf-test1.tex.

\endpreamble
\let\MetaPrefix\DoubleperCent

\generate{%
  \file{ifpdf.ins}{\from{ifpdf.dtx}{install}}%
  \file{ifpdf.drv}{\from{ifpdf.dtx}{driver}}%
  \usedir{tex/generic/oberdiek}%
  \file{ifpdf.sty}{\from{ifpdf.dtx}{package}}%
  \usedir{doc/latex/oberdiek/test}%
  \file{ifpdf-test1.tex}{\from{ifpdf.dtx}{test1}}%
  \nopreamble
  \nopostamble
  \usedir{source/latex/oberdiek/catalogue}%
  \file{ifpdf.xml}{\from{ifpdf.dtx}{catalogue}}%
}

\catcode32=13\relax% active space
\let =\space%
\Msg{************************************************************************}
\Msg{*}
\Msg{* To finish the installation you have to move the following}
\Msg{* file into a directory searched by TeX:}
\Msg{*}
\Msg{*     ifpdf.sty}
\Msg{*}
\Msg{* To produce the documentation run the file `ifpdf.drv'}
\Msg{* through LaTeX.}
\Msg{*}
\Msg{* Happy TeXing!}
\Msg{*}
\Msg{************************************************************************}

\endbatchfile
%</install>
%<*ignore>
\fi
%</ignore>
%<*driver>
\NeedsTeXFormat{LaTeX2e}
\ProvidesFile{ifpdf.drv}%
  [2011/01/30 v2.3 Provides the ifpdf switch (HO)]%
\documentclass{ltxdoc}
\usepackage{holtxdoc}[2011/11/22]
\begin{document}
  \DocInput{ifpdf.dtx}%
\end{document}
%</driver>
% \fi
%
% \CheckSum{600}
%
% \CharacterTable
%  {Upper-case    \A\B\C\D\E\F\G\H\I\J\K\L\M\N\O\P\Q\R\S\T\U\V\W\X\Y\Z
%   Lower-case    \a\b\c\d\e\f\g\h\i\j\k\l\m\n\o\p\q\r\s\t\u\v\w\x\y\z
%   Digits        \0\1\2\3\4\5\6\7\8\9
%   Exclamation   \!     Double quote  \"     Hash (number) \#
%   Dollar        \$     Percent       \%     Ampersand     \&
%   Acute accent  \'     Left paren    \(     Right paren   \)
%   Asterisk      \*     Plus          \+     Comma         \,
%   Minus         \-     Point         \.     Solidus       \/
%   Colon         \:     Semicolon     \;     Less than     \<
%   Equals        \=     Greater than  \>     Question mark \?
%   Commercial at \@     Left bracket  \[     Backslash     \\
%   Right bracket \]     Circumflex    \^     Underscore    \_
%   Grave accent  \`     Left brace    \{     Vertical bar  \|
%   Right brace   \}     Tilde         \~}
%
% \GetFileInfo{ifpdf.drv}
%
% \title{The \xpackage{ifpdf} package}
% \date{2011/01/30 v2.3}
% \author{Heiko Oberdiek\\\xemail{heiko.oberdiek at googlemail.com}}
%
% \maketitle
%
% \begin{abstract}
% This package looks for \pdfTeX\ in pdf mode and
% implements and sets the switch \cs{ifpdf}.
% The detection is based on \cs{pdfoutput} and
% the package will not change this value.
% It works with plain or \LaTeX\ formats.
% \end{abstract}
%
% \tableofcontents
%
% \hypersetup{bookmarksopenlevel=2}
% \section{Documentation}
%
% \subsection{Introduction}
%
% It is commonly known that H\`an Th\^e Th\`anh's \pdfTeX\ generates
% PDF output directly and many people uses \pdfTeX\ for this purpose.
% However the DVI output was never thrown away. In contrary, he
% new features for typesetting that works in both PDF and DVI mode.
%
% In the meantime many \TeX\ distributions replace the traditional
% \TeX\ binary with \pdfTeX. Then, for example, called as |latex|
% \pdfTeX\ works in DVI mode with the \LaTeX\ format preloaded,
% called as |pdflatex| \pdfTeX\ starts in PDF mode.
%
% Often packages or users want to know, whether the current document
% is typset by \pdfTeX\ in PDF mode, because the different modes
% have different capabilities (color setting, graphics inclusion,
% \dots). For this purpose \pdfTeX's
% \cs{pdfoutput} can be asked.
%
% As regulary reader of \TeX\ newsgroups and mailing lists I could
% observe many problems with this task. Common errors are:
% \begin{itemize}
% \item
%   \pdfTeX\ has \emph{two} modes. Using \pdfTeX\ does not mean
%   that the user always want to have PDF mode. For example, the
%   PostScript support is better in DVI mode in conjunction with a
%   PostScript aware DVI driver (e.g. dvips). Also the additional
%   typesetting features are mode independent and also available
%   in DVI mode.
% \item
%   \LaTeX's \cs{@ifundefined} inherited the side effect from
%   \cs{csname}. Unknown commands are defined with the meaning
%   of \cs{relax}. If it is checked, whether \cs{pdfoutput} is
%   defined, then this should not be forgotten.
% \item
%   Having \cs{pdfoutput} does not automatically mean PDF mode.
%   Also the value of \cs{pdfoutput} must be asked.
% \item
%   \cs{pdfoutput} must not be destroyed in some way. Later
%   code and packages are fooled then and will perhaps make
%   wrong decisions. For example they may drop support for
%   PDF mode, because they do not know that \pdfTeX\ is
%   running at all.
% \end{itemize}
%
% Robin Fairbairns provides an entry for this topic in his
% excellent FAQ (\url{http://www.tex.ac.uk/faq}):
% \href{http://www.tex.ac.uk/cgi-bin/texfaq2html?label=ifpdf}^^A
% {Am I using PDFTeX?}
%
% \subsection{Usage}
%
% The package \xpackage{ifpdf} can be used with both \plainTeX\
% and \LaTeX:
% \begin{description}
% \item[\plainTeX:] |\input ifpdf.sty|
% \item[\LaTeXe:]   |\usepackage{ifpdf}|
% \end{description}
%
% \DescribeMacro{\ifpdf}
% The package provides the switch \cs{ifpdf}:
% \begin{quote}
%   |\ifpdf|\\
%   \hspace*{1.5em}\dots\ do things, if \pdfTeX\ is running in pdf mode \dots\\
%   |\else|\\
%   \hspace*{1.5em}\dots\ other \TeX\ or \pdfTeX\ in dvi mode \dots\\
%   |\fi|
% \end{quote}
% Users of the package \xpackage{ifthen} can use the switch as boolean:
% \begin{quote}
%   |\boolean{pdf}|
% \end{quote}
% The package can also be used to set global
% documentclass options:
% \begin{quote}
%   |\RequirePackage{ifpdf}|\\
%   |\ifpdf|\\
%   \hspace*{1.5em}|\documentclass[pdftex,|\dots|]{|\dots|}|\\
%   |\else|\\
%   \hspace*{1.5em}|\documentclass[|\dots|]{|\dots|}|\\
%   |\fi|
% \end{quote}
%
% \subsection{Specification}
%
% The package have the following properties:
% \begin{itemize}
% \item It asks the setting of \cs{pdfoutput} for detecting
%   \pdfTeX\ in PDF mode.
% \item It never changes \cs{pdfoutput}.
% \item If \cs{pdfoutput} is undefined or has the meaning \cs{relax},
%   but the engine provides the primitive \cs{pdfoutput},
%   then \cs{pdfoutput} is enabled or restored if possible
%   (only \LuaTeX, version 0.36.0 or higher).
% \item It can be used with many formats including
%   \plainTeX\ and \LaTeX.
% \end{itemize}
% The mode detection implements the following algorithm:
% \begin{quote}
% \def\={${:}{=}$}
% \def\>{\null\qquad\ignorespaces}
% \def\comment#1{^^A
%   \textrm{\itshape\% #1}^^A
% }
% \sffamily
% if undefined(\cs{pdfoutput})\\
% \> \cs{ifpdf} \= \cs{iffalse} \comment{\pdfTeX is not running}\\
% else\\
% \> if \cs{pdfoutput} $\leq$ 0\\
% \>\> \cs{ifpdf} \= \cs{iffalse} \comment{\pdfTeX\ in DVI mode}\\
% \> else\\
% \>\> \cs{ifpdf} \= \cs{iftrue} \comment{\pdfTeX\ in PDF mode}\\
% \> fi\\
% fi
% \end{quote}
% The function \textsf{undefined} checks both cases, undefined command
% and \cs{relax}.
%
% \hypersetup{bookmarksopenlevel=1}
%
% \StopEventually{
% }
%
% \section{Implementation}
%
%    \begin{macrocode}
%<*package>
%    \end{macrocode}
%
% \subsection{Reload check and package identification}
%    Reload check, especially if the package is not used with \LaTeX.
%    \begin{macrocode}
\begingroup\catcode61\catcode48\catcode32=10\relax%
  \catcode13=5 % ^^M
  \endlinechar=13 %
  \catcode35=6 % #
  \catcode39=12 % '
  \catcode44=12 % ,
  \catcode45=12 % -
  \catcode46=12 % .
  \catcode58=12 % :
  \catcode64=11 % @
  \catcode123=1 % {
  \catcode125=2 % }
  \expandafter\let\expandafter\x\csname ver@ifpdf.sty\endcsname
  \ifx\x\relax % plain-TeX, first loading
  \else
    \def\empty{}%
    \ifx\x\empty % LaTeX, first loading,
      % variable is initialized, but \ProvidesPackage not yet seen
    \else
      \expandafter\ifx\csname PackageInfo\endcsname\relax
        \def\x#1#2{%
          \immediate\write-1{Package #1 Info: #2.}%
        }%
      \else
        \def\x#1#2{\PackageInfo{#1}{#2, stopped}}%
      \fi
      \x{ifpdf}{The package is already loaded}%
      \aftergroup\endinput
    \fi
  \fi
\endgroup%
%    \end{macrocode}
%    Package identification:
%    \begin{macrocode}
\begingroup\catcode61\catcode48\catcode32=10\relax%
  \catcode13=5 % ^^M
  \endlinechar=13 %
  \catcode35=6 % #
  \catcode39=12 % '
  \catcode40=12 % (
  \catcode41=12 % )
  \catcode44=12 % ,
  \catcode45=12 % -
  \catcode46=12 % .
  \catcode47=12 % /
  \catcode58=12 % :
  \catcode64=11 % @
  \catcode91=12 % [
  \catcode93=12 % ]
  \catcode123=1 % {
  \catcode125=2 % }
  \expandafter\ifx\csname ProvidesPackage\endcsname\relax
    \def\x#1#2#3[#4]{\endgroup
      \immediate\write-1{Package: #3 #4}%
      \xdef#1{#4}%
    }%
  \else
    \def\x#1#2[#3]{\endgroup
      #2[{#3}]%
      \ifx#1\@undefined
        \xdef#1{#3}%
      \fi
      \ifx#1\relax
        \xdef#1{#3}%
      \fi
    }%
  \fi
\expandafter\x\csname ver@ifpdf.sty\endcsname
\ProvidesPackage{ifpdf}%
  [2011/01/30 v2.3 Provides the ifpdf switch (HO)]%
%    \end{macrocode}
%
% \subsection{Catcodes}
%
%    \begin{macrocode}
\begingroup\catcode61\catcode48\catcode32=10\relax%
  \catcode13=5 % ^^M
  \endlinechar=13 %
  \catcode123=1 % {
  \catcode125=2 % }
  \catcode64=11 % @
  \def\x{\endgroup
    \expandafter\edef\csname ifpdf@AtEnd\endcsname{%
      \endlinechar=\the\endlinechar\relax
      \catcode13=\the\catcode13\relax
      \catcode32=\the\catcode32\relax
      \catcode35=\the\catcode35\relax
      \catcode61=\the\catcode61\relax
      \catcode64=\the\catcode64\relax
      \catcode123=\the\catcode123\relax
      \catcode125=\the\catcode125\relax
    }%
  }%
\x\catcode61\catcode48\catcode32=10\relax%
\catcode13=5 % ^^M
\endlinechar=13 %
\catcode35=6 % #
\catcode64=11 % @
\catcode123=1 % {
\catcode125=2 % }
\def\TMP@EnsureCode#1#2{%
  \edef\ifpdf@AtEnd{%
    \ifpdf@AtEnd
    \catcode#1=\the\catcode#1\relax
  }%
  \catcode#1=#2\relax
}
\TMP@EnsureCode{10}{12}% ^^J
\TMP@EnsureCode{39}{12}% '
\TMP@EnsureCode{40}{12}% (
\TMP@EnsureCode{41}{12}% )
\TMP@EnsureCode{44}{12}% ,
\TMP@EnsureCode{45}{12}% -
\TMP@EnsureCode{46}{12}% .
\TMP@EnsureCode{47}{12}% /
\TMP@EnsureCode{58}{12}% :
\TMP@EnsureCode{60}{12}% <
\TMP@EnsureCode{91}{12}% [
\TMP@EnsureCode{93}{12}% ]
\TMP@EnsureCode{94}{7}% ^
\TMP@EnsureCode{96}{12}% `
\edef\ifpdf@AtEnd{\ifpdf@AtEnd\noexpand\endinput}
%    \end{macrocode}
%
% \subsection{Check for previously defined \cs{ifpdf}}
%
%    \begin{macrocode}
\begingroup
  \expandafter\ifx\csname ifpdf\endcsname\relax
  \else
    \edef\i/{\expandafter\string\csname ifpdf\endcsname}%
    \expandafter\ifx\csname PackageError\endcsname\relax
      \def\x#1#2{%
        \edef\z{#2}%
        \expandafter\errhelp\expandafter{\z}%
        \errmessage{Package ifpdf Error: #1}%
      }%
      \def\y{^^J}%
      \newlinechar=10 %
    \else
      \def\x#1#2{%
        \PackageError{ifpdf}{#1}{#2}%
      }%
      \def\y{\MessageBreak}%
    \fi
    \x{Name clash, \i/ is already defined}{%
      Incompatible versions of \i/ can cause problems,\y
      therefore package loading is aborted.%
    }%
    \endgroup
    \expandafter\ifpdf@AtEnd
  \fi%
\endgroup
%    \end{macrocode}
%
% \subsection{\cs{pdfoutput} and \LuaTeX}
%
%    It might happen, that \LuaTeX\ is running, but \cs{pdfoutput}
%    does not exist. In version 0.40 only \cs{directlua}
%    is available at startup time. The enabling Lua function was
%    already added in version 0.36. Thus we can ignore older versions,
%    here \cs{pdfoutput} is available at startup time.
%
%    \begin{macrocode}
\begingroup
  \def\skip#1\relax\begingroup{}%
  \expandafter\ifx\csname pdfoutput\endcsname\relax
  \else
    \expandafter\skip
  \fi
  \expandafter\ifx\csname directlua\endcsname\relax
    \expandafter\skip
  \fi
\endgroup
\begingroup\expandafter\expandafter\expandafter\endgroup
\expandafter\ifx\csname RequirePackage\endcsname\relax
  \def\TMP@RequirePackage#1[#2]{%
    \begingroup\expandafter\expandafter\expandafter\endgroup
    \expandafter\ifx\csname ver@#1.sty\endcsname\relax
      \input #1.sty\relax
    \fi
  }%
  \TMP@RequirePackage{ifluatex}[2009/04/10]%
\else
  \RequirePackage{ifluatex}[2009/04/10]%
\fi
\ifluatex
  \ifnum\luatexversion<36 %
%    \end{macrocode}
%    Unhappily \LuaTeX's \cs{primitive} (derived from
%    pdfTeX's \cs{pdfprimitive}) cannot be used:
%    \begin{quote}
%    |\protected\gdef\pdfoutput{\primitive\pdfoutput}|
%    \end{quote}
%    Setting a value works, but getting fails,
%    because \TeX\ does no longer see it as number. It
%    is unexpandable and breaks numerical contexts.
%    This was fixed in \hologo{pdfTeX} 1.40.10
%    (bugfix \#4289: ``|\primitive\pdfoutput| cannot be queried'').
%    \begin{macrocode}
  \else
    \begingroup
      \directlua{tex.enableprimitives('ifpdf', {'pdfoutput'})}%
      \global\let\pdfoutput\ifpdfpdfoutput
    \endgroup
  \fi
\fi
\relax\begingroup\endgroup
%    \end{macrocode}
%
% \subsection{\cs{ifpdf}}
%
%    \begin{macro}{\ifpdf}
%    Create and set the switch. \cs{newif} initializes the
%    switch with \cs{iffalse}. \cs{newif} is \cs{outer}
%    in \plainTeX.
%    \begin{macrocode}
\begingroup\expandafter\expandafter\expandafter\endgroup
\expandafter\ifx\csname newif\endcsname\relax
  \edef\pdffalse{%
    \let
    \expandafter\noexpand\csname ifpdf\endcsname
    \expandafter\noexpand\csname iffalse\endcsname
  }%
  \edef\pdftrue{%
    \let
    \expandafter\noexpand\csname ifpdf\endcsname
    \expandafter\noexpand\csname iftrue\endcsname
  }%
  \pdffalse
\else
  \csname newif\expandafter\endcsname\csname ifpdf\endcsname
\fi
%    \end{macrocode}
%    Test \cs{pdfoutput}. Is it  defined and different from
%    \cs{relax}? Someone could have used \LaTeX\ internal
%    \cs{@ifundefined}, or something else involving.
%    Notice, \cs{csname} is executed inside a group for the test
%    to cancel the side effect of \cs{csname}.
%    \begin{macrocode}
\begingroup\expandafter\expandafter\expandafter\endgroup
\expandafter\ifx\csname pdfoutput\endcsname\relax
\else
  \ifnum\pdfoutput<1 %
%    \end{macrocode}
%    \cs{pdfoutput}|=0| or negative, so not generating pdf.
%    \begin{macrocode}
  \else
    \pdftrue
  \fi
\fi
%    \end{macrocode}
%    \end{macro}
%
% \subsection{Test for fool attempts}
%
%    \begin{macrocode}
\begingroup
  \expandafter\ifx\csname pdfoutput\endcsname\relax
  \else
    \escapechar=92 %
    \edef\m{\meaning\pdfoutput}%
    \edef\p{\string\pdfoutput}%
    \ifx\m\p
    \else
      \expandafter\ifx\csname PackageWarningNoLine\endcsname\relax
        \def\PackageWarningNoLine#1#2{%
          \immediate\write16{%
            Package `#1' Warning: #2.%
          }%
        }%
      \fi
      \PackageWarningNoLine{ifpdf}{%
        Someone has redefined \string\pdfoutput%
      }%
    \fi
  \fi
\endgroup
%    \end{macrocode}
%
% \subsection{Protocol entry}
%
%     Log comment:
%    \begin{macrocode}
\begingroup
  \expandafter\ifx\csname PackageInfo\endcsname\relax
    \def\x#1#2{%
      \immediate\write-1{Package #1 Info: #2.}%
    }%
  \else
    \let\x\PackageInfo
    \expandafter\let\csname on@line\endcsname\empty
  \fi
  \x{ifpdf}{pdfTeX in PDF mode is \ifpdf\else not \fi detected}%
\endgroup
%    \end{macrocode}
%    \begin{macrocode}
\ifpdf@AtEnd%
%    \end{macrocode}
%    \begin{macrocode}
%</package>
%    \end{macrocode}
%
% \section{Test}
%
% \subsection{Catcode checks for loading}
%
%    \begin{macrocode}
%<*test1>
%    \end{macrocode}
%    \begin{macrocode}
\catcode`\{=1 %
\catcode`\}=2 %
\catcode`\#=6 %
\catcode`\@=11 %
\expandafter\ifx\csname count@\endcsname\relax
  \countdef\count@=255 %
\fi
\expandafter\ifx\csname @gobble\endcsname\relax
  \long\def\@gobble#1{}%
\fi
\expandafter\ifx\csname @firstofone\endcsname\relax
  \long\def\@firstofone#1{#1}%
\fi
\expandafter\ifx\csname loop\endcsname\relax
  \expandafter\@firstofone
\else
  \expandafter\@gobble
\fi
{%
  \def\loop#1\repeat{%
    \def\body{#1}%
    \iterate
  }%
  \def\iterate{%
    \body
      \let\next\iterate
    \else
      \let\next\relax
    \fi
    \next
  }%
  \let\repeat=\fi
}%
\def\RestoreCatcodes{}
\count@=0 %
\loop
  \edef\RestoreCatcodes{%
    \RestoreCatcodes
    \catcode\the\count@=\the\catcode\count@\relax
  }%
\ifnum\count@<255 %
  \advance\count@ 1 %
\repeat

\def\RangeCatcodeInvalid#1#2{%
  \count@=#1\relax
  \loop
    \catcode\count@=15 %
  \ifnum\count@<#2\relax
    \advance\count@ 1 %
  \repeat
}
\def\RangeCatcodeCheck#1#2#3{%
  \count@=#1\relax
  \loop
    \ifnum#3=\catcode\count@
    \else
      \errmessage{%
        Character \the\count@\space
        with wrong catcode \the\catcode\count@\space
        instead of \number#3%
      }%
    \fi
  \ifnum\count@<#2\relax
    \advance\count@ 1 %
  \repeat
}
\def\space{ }
\expandafter\ifx\csname LoadCommand\endcsname\relax
  \def\LoadCommand{\input ifpdf.sty\relax}%
\fi
\def\Test{%
  \RangeCatcodeInvalid{0}{47}%
  \RangeCatcodeInvalid{58}{64}%
  \RangeCatcodeInvalid{91}{96}%
  \RangeCatcodeInvalid{123}{255}%
  \catcode`\@=12 %
  \catcode`\\=0 %
  \catcode`\%=14 %
  \LoadCommand
  \RangeCatcodeCheck{0}{36}{15}%
  \RangeCatcodeCheck{37}{37}{14}%
  \RangeCatcodeCheck{38}{47}{15}%
  \RangeCatcodeCheck{48}{57}{12}%
  \RangeCatcodeCheck{58}{63}{15}%
  \RangeCatcodeCheck{64}{64}{12}%
  \RangeCatcodeCheck{65}{90}{11}%
  \RangeCatcodeCheck{91}{91}{15}%
  \RangeCatcodeCheck{92}{92}{0}%
  \RangeCatcodeCheck{93}{96}{15}%
  \RangeCatcodeCheck{97}{122}{11}%
  \RangeCatcodeCheck{123}{255}{15}%
  \RestoreCatcodes
}
\Test
\csname @@end\endcsname
\end
%    \end{macrocode}
%    \begin{macrocode}
%</test1>
%    \end{macrocode}
%
% \section{Installation}
%
% \subsection{Download}
%
% \paragraph{Package.} This package is available on
% CTAN\footnote{\url{ftp://ftp.ctan.org/tex-archive/}}:
% \begin{description}
% \item[\CTAN{macros/latex/contrib/oberdiek/ifpdf.dtx}] The source file.
% \item[\CTAN{macros/latex/contrib/oberdiek/ifpdf.pdf}] Documentation.
% \end{description}
%
%
% \paragraph{Bundle.} All the packages of the bundle `oberdiek'
% are also available in a TDS compliant ZIP archive. There
% the packages are already unpacked and the documentation files
% are generated. The files and directories obey the TDS standard.
% \begin{description}
% \item[\CTAN{install/macros/latex/contrib/oberdiek.tds.zip}]
% \end{description}
% \emph{TDS} refers to the standard ``A Directory Structure
% for \TeX\ Files'' (\CTAN{tds/tds.pdf}). Directories
% with \xfile{texmf} in their name are usually organized this way.
%
% \subsection{Bundle installation}
%
% \paragraph{Unpacking.} Unpack the \xfile{oberdiek.tds.zip} in the
% TDS tree (also known as \xfile{texmf} tree) of your choice.
% Example (linux):
% \begin{quote}
%   |unzip oberdiek.tds.zip -d ~/texmf|
% \end{quote}
%
% \paragraph{Script installation.}
% Check the directory \xfile{TDS:scripts/oberdiek/} for
% scripts that need further installation steps.
% Package \xpackage{attachfile2} comes with the Perl script
% \xfile{pdfatfi.pl} that should be installed in such a way
% that it can be called as \texttt{pdfatfi}.
% Example (linux):
% \begin{quote}
%   |chmod +x scripts/oberdiek/pdfatfi.pl|\\
%   |cp scripts/oberdiek/pdfatfi.pl /usr/local/bin/|
% \end{quote}
%
% \subsection{Package installation}
%
% \paragraph{Unpacking.} The \xfile{.dtx} file is a self-extracting
% \docstrip\ archive. The files are extracted by running the
% \xfile{.dtx} through \plainTeX:
% \begin{quote}
%   \verb|tex ifpdf.dtx|
% \end{quote}
%
% \paragraph{TDS.} Now the different files must be moved into
% the different directories in your installation TDS tree
% (also known as \xfile{texmf} tree):
% \begin{quote}
% \def\t{^^A
% \begin{tabular}{@{}>{\ttfamily}l@{ $\rightarrow$ }>{\ttfamily}l@{}}
%   ifpdf.sty & tex/generic/oberdiek/ifpdf.sty\\
%   ifpdf.pdf & doc/latex/oberdiek/ifpdf.pdf\\
%   test/ifpdf-test1.tex & doc/latex/oberdiek/test/ifpdf-test1.tex\\
%   ifpdf.dtx & source/latex/oberdiek/ifpdf.dtx\\
% \end{tabular}^^A
% }^^A
% \sbox0{\t}^^A
% \ifdim\wd0>\linewidth
%   \begingroup
%     \advance\linewidth by\leftmargin
%     \advance\linewidth by\rightmargin
%   \edef\x{\endgroup
%     \def\noexpand\lw{\the\linewidth}^^A
%   }\x
%   \def\lwbox{^^A
%     \leavevmode
%     \hbox to \linewidth{^^A
%       \kern-\leftmargin\relax
%       \hss
%       \usebox0
%       \hss
%       \kern-\rightmargin\relax
%     }^^A
%   }^^A
%   \ifdim\wd0>\lw
%     \sbox0{\small\t}^^A
%     \ifdim\wd0>\linewidth
%       \ifdim\wd0>\lw
%         \sbox0{\footnotesize\t}^^A
%         \ifdim\wd0>\linewidth
%           \ifdim\wd0>\lw
%             \sbox0{\scriptsize\t}^^A
%             \ifdim\wd0>\linewidth
%               \ifdim\wd0>\lw
%                 \sbox0{\tiny\t}^^A
%                 \ifdim\wd0>\linewidth
%                   \lwbox
%                 \else
%                   \usebox0
%                 \fi
%               \else
%                 \lwbox
%               \fi
%             \else
%               \usebox0
%             \fi
%           \else
%             \lwbox
%           \fi
%         \else
%           \usebox0
%         \fi
%       \else
%         \lwbox
%       \fi
%     \else
%       \usebox0
%     \fi
%   \else
%     \lwbox
%   \fi
% \else
%   \usebox0
% \fi
% \end{quote}
% If you have a \xfile{docstrip.cfg} that configures and enables \docstrip's
% TDS installing feature, then some files can already be in the right
% place, see the documentation of \docstrip.
%
% \subsection{Refresh file name databases}
%
% If your \TeX~distribution
% (\teTeX, \mikTeX, \dots) relies on file name databases, you must refresh
% these. For example, \teTeX\ users run \verb|texhash| or
% \verb|mktexlsr|.
%
% \subsection{Some details for the interested}
%
% \paragraph{Attached source.}
%
% The PDF documentation on CTAN also includes the
% \xfile{.dtx} source file. It can be extracted by
% AcrobatReader 6 or higher. Another option is \textsf{pdftk},
% e.g. unpack the file into the current directory:
% \begin{quote}
%   \verb|pdftk ifpdf.pdf unpack_files output .|
% \end{quote}
%
% \paragraph{Unpacking with \LaTeX.}
% The \xfile{.dtx} chooses its action depending on the format:
% \begin{description}
% \item[\plainTeX:] Run \docstrip\ and extract the files.
% \item[\LaTeX:] Generate the documentation.
% \end{description}
% If you insist on using \LaTeX\ for \docstrip\ (really,
% \docstrip\ does not need \LaTeX), then inform the autodetect routine
% about your intention:
% \begin{quote}
%   \verb|latex \let\install=y\input{ifpdf.dtx}|
% \end{quote}
% Do not forget to quote the argument according to the demands
% of your shell.
%
% \paragraph{Generating the documentation.}
% You can use both the \xfile{.dtx} or the \xfile{.drv} to generate
% the documentation. The process can be configured by the
% configuration file \xfile{ltxdoc.cfg}. For instance, put this
% line into this file, if you want to have A4 as paper format:
% \begin{quote}
%   \verb|\PassOptionsToClass{a4paper}{article}|
% \end{quote}
% An example follows how to generate the
% documentation with pdf\LaTeX:
% \begin{quote}
%\begin{verbatim}
%pdflatex ifpdf.dtx
%makeindex -s gind.ist ifpdf.idx
%pdflatex ifpdf.dtx
%makeindex -s gind.ist ifpdf.idx
%pdflatex ifpdf.dtx
%\end{verbatim}
% \end{quote}
%
% \section{Catalogue}
%
% The following XML file can be used as source for the
% \href{http://mirror.ctan.org/help/Catalogue/catalogue.html}{\TeX\ Catalogue}.
% The elements \texttt{caption} and \texttt{description} are imported
% from the original XML file from the Catalogue.
% The name of the XML file in the Catalogue is \xfile{ifpdf.xml}.
%    \begin{macrocode}
%<*catalogue>
<?xml version='1.0' encoding='us-ascii'?>
<!DOCTYPE entry SYSTEM 'catalogue.dtd'>
<entry datestamp='$Date$' modifier='$Author$' id='ifpdf'>
  <name>ifpdf</name>
  <caption>Provides the \ifpdf conditional.</caption>
  <authorref id='auth:oberdiek'/>
  <copyright owner='Heiko Oberdiek' year='2001,2005-2011'/>
  <license type='lppl1.3'/>
  <version number='2.3'/>
  <description>
      This package looks for pdfTeX in pdf mode and implements and
      sets the switch <tt>\ifpdf</tt>. The detection is based on the
      value of <tt>\pdfoutput</tt> (which the package will not
      change).  The package works with plain or LaTeX formats.  To use
      it with LaTeX simply <tt>\usepackage{ifpdf}</tt>. Then use
      <tt>\ifpdf ... \else ... \fi</tt>.
    <p/>
    The package is part of the <xref refid='oberdiek'>oberdiek</xref> bundle.
  </description>
  <documentation details='Package documentation'
      href='ctan:/macros/latex/contrib/oberdiek/ifpdf.pdf'/>
  <ctan file='true' path='/macros/latex/contrib/oberdiek/ifpdf.dtx'/>
  <miktex location='oberdiek'/>
  <texlive location='oberdiek'/>
  <install path='/macros/latex/contrib/oberdiek/oberdiek.tds.zip'/>
</entry>
%</catalogue>
%    \end{macrocode}
%
% \begin{History}
%   \begin{Version}{2001/06/14 v1.0}
%   \item
%     First public version.
%   \end{Version}
%   \begin{Version}{2001/07/14 v1.1}
%   \item
%     Documentation addition: global options
%   \end{Version}
%   \begin{Version}{2001/09/26 v1.2}
%   \item
%     Documentation typo corrected.
%   \item
%     Version number corrected.
%   \item
%     Line number in log entry removed.
%   \end{Version}
%   \begin{Version}{2005/07/22 v1.3}
%   \item
%     Some source code comments from Robin Fairbairns added.
%   \item
%     Bug fix for negative values of \cs{pdfoutput}
%     (Oleg Katsitadze)
%   \item
%     LPPL 1.3
%   \item
%     Installation section with locations added.
%   \end{Version}
%   \begin{Version}{2006/02/20 v1.4}
%   \item
%     DTX framework.
%   \item
%     More robust check in case of undefined \cs{pdfoutput}.
%   \item
%     Extended documentation.
%   \end{Version}
%   \begin{Version}{2007/09/09 v1.5}
%   \item
%     Catcode settings added.
%   \end{Version}
%   \begin{Version}{2007/12/12 v1.6}
%   \item
%     Minor update.
%   \end{Version}
%   \begin{Version}{2008/12/12 v1.7}
%   \item
%     Fix in documentation for \cs{boolean} (found by S. Venkataraman).
%   \item
%     Code is not changed.
%   \end{Version}
%   \begin{Version}{2009/04/10 v2.0}
%   \item
%     Support for \LuaTeX\ 0.40 added.
%   \item
%     Checks, whether \cs{pdfoutput} was changed.
%   \end{Version}
%   \begin{Version}{2010/01/28 v2.1}
%   \item
%     Compatibility to ini\TeX\ added.
%   \end{Version}
%   \begin{Version}{2010/09/13 v2.2}
%   \item
%     Code with \cs{escapechar} rewritten because of
%     LuaTeX bug: \cs{escapechar} does not work.
%   \end{Version}
%   \begin{Version}{2011/01/30 v2.3}
%   \item
%     Already loaded package files are not input in \hologo{plainTeX}.
%   \end{Version}
% \end{History}
%
% \PrintIndex
%
% \Finale
\endinput
|
% \end{quote}
% Do not forget to quote the argument according to the demands
% of your shell.
%
% \paragraph{Generating the documentation.}
% You can use both the \xfile{.dtx} or the \xfile{.drv} to generate
% the documentation. The process can be configured by the
% configuration file \xfile{ltxdoc.cfg}. For instance, put this
% line into this file, if you want to have A4 as paper format:
% \begin{quote}
%   \verb|\PassOptionsToClass{a4paper}{article}|
% \end{quote}
% An example follows how to generate the
% documentation with pdf\LaTeX:
% \begin{quote}
%\begin{verbatim}
%pdflatex ifpdf.dtx
%makeindex -s gind.ist ifpdf.idx
%pdflatex ifpdf.dtx
%makeindex -s gind.ist ifpdf.idx
%pdflatex ifpdf.dtx
%\end{verbatim}
% \end{quote}
%
% \section{Catalogue}
%
% The following XML file can be used as source for the
% \href{http://mirror.ctan.org/help/Catalogue/catalogue.html}{\TeX\ Catalogue}.
% The elements \texttt{caption} and \texttt{description} are imported
% from the original XML file from the Catalogue.
% The name of the XML file in the Catalogue is \xfile{ifpdf.xml}.
%    \begin{macrocode}
%<*catalogue>
<?xml version='1.0' encoding='us-ascii'?>
<!DOCTYPE entry SYSTEM 'catalogue.dtd'>
<entry datestamp='$Date$' modifier='$Author$' id='ifpdf'>
  <name>ifpdf</name>
  <caption>Provides the \ifpdf conditional.</caption>
  <authorref id='auth:oberdiek'/>
  <copyright owner='Heiko Oberdiek' year='2001,2005-2011'/>
  <license type='lppl1.3'/>
  <version number='2.3'/>
  <description>
      This package looks for pdfTeX in pdf mode and implements and
      sets the switch <tt>\ifpdf</tt>. The detection is based on the
      value of <tt>\pdfoutput</tt> (which the package will not
      change).  The package works with plain or LaTeX formats.  To use
      it with LaTeX simply <tt>\usepackage{ifpdf}</tt>. Then use
      <tt>\ifpdf ... \else ... \fi</tt>.
    <p/>
    The package is part of the <xref refid='oberdiek'>oberdiek</xref> bundle.
  </description>
  <documentation details='Package documentation'
      href='ctan:/macros/latex/contrib/oberdiek/ifpdf.pdf'/>
  <ctan file='true' path='/macros/latex/contrib/oberdiek/ifpdf.dtx'/>
  <miktex location='oberdiek'/>
  <texlive location='oberdiek'/>
  <install path='/macros/latex/contrib/oberdiek/oberdiek.tds.zip'/>
</entry>
%</catalogue>
%    \end{macrocode}
%
% \begin{History}
%   \begin{Version}{2001/06/14 v1.0}
%   \item
%     First public version.
%   \end{Version}
%   \begin{Version}{2001/07/14 v1.1}
%   \item
%     Documentation addition: global options
%   \end{Version}
%   \begin{Version}{2001/09/26 v1.2}
%   \item
%     Documentation typo corrected.
%   \item
%     Version number corrected.
%   \item
%     Line number in log entry removed.
%   \end{Version}
%   \begin{Version}{2005/07/22 v1.3}
%   \item
%     Some source code comments from Robin Fairbairns added.
%   \item
%     Bug fix for negative values of \cs{pdfoutput}
%     (Oleg Katsitadze)
%   \item
%     LPPL 1.3
%   \item
%     Installation section with locations added.
%   \end{Version}
%   \begin{Version}{2006/02/20 v1.4}
%   \item
%     DTX framework.
%   \item
%     More robust check in case of undefined \cs{pdfoutput}.
%   \item
%     Extended documentation.
%   \end{Version}
%   \begin{Version}{2007/09/09 v1.5}
%   \item
%     Catcode settings added.
%   \end{Version}
%   \begin{Version}{2007/12/12 v1.6}
%   \item
%     Minor update.
%   \end{Version}
%   \begin{Version}{2008/12/12 v1.7}
%   \item
%     Fix in documentation for \cs{boolean} (found by S. Venkataraman).
%   \item
%     Code is not changed.
%   \end{Version}
%   \begin{Version}{2009/04/10 v2.0}
%   \item
%     Support for \LuaTeX\ 0.40 added.
%   \item
%     Checks, whether \cs{pdfoutput} was changed.
%   \end{Version}
%   \begin{Version}{2010/01/28 v2.1}
%   \item
%     Compatibility to ini\TeX\ added.
%   \end{Version}
%   \begin{Version}{2010/09/13 v2.2}
%   \item
%     Code with \cs{escapechar} rewritten because of
%     LuaTeX bug: \cs{escapechar} does not work.
%   \end{Version}
%   \begin{Version}{2011/01/30 v2.3}
%   \item
%     Already loaded package files are not input in \hologo{plainTeX}.
%   \end{Version}
% \end{History}
%
% \PrintIndex
%
% \Finale
\endinput

%        (quote the arguments according to the demands of your shell)
%
% Documentation:
%    (a) If ifpdf.drv is present:
%           latex ifpdf.drv
%    (b) Without ifpdf.drv:
%           latex ifpdf.dtx; ...
%    The class ltxdoc loads the configuration file ltxdoc.cfg
%    if available. Here you can specify further options, e.g.
%    use A4 as paper format:
%       \PassOptionsToClass{a4paper}{article}
%
%    Programm calls to get the documentation (example):
%       pdflatex ifpdf.dtx
%       makeindex -s gind.ist ifpdf.idx
%       pdflatex ifpdf.dtx
%       makeindex -s gind.ist ifpdf.idx
%       pdflatex ifpdf.dtx
%
% Installation:
%    TDS:tex/generic/oberdiek/ifpdf.sty
%    TDS:doc/latex/oberdiek/ifpdf.pdf
%    TDS:doc/latex/oberdiek/test/ifpdf-test1.tex
%    TDS:source/latex/oberdiek/ifpdf.dtx
%
%<*ignore>
\begingroup
  \catcode123=1 %
  \catcode125=2 %
  \def\x{LaTeX2e}%
\expandafter\endgroup
\ifcase 0\ifx\install y1\fi\expandafter
         \ifx\csname processbatchFile\endcsname\relax\else1\fi
         \ifx\fmtname\x\else 1\fi\relax
\else\csname fi\endcsname
%</ignore>
%<*install>
\input docstrip.tex
\Msg{************************************************************************}
\Msg{* Installation}
\Msg{* Package: ifpdf 2011/01/30 v2.3 Provides the ifpdf switch (HO)}
\Msg{************************************************************************}

\keepsilent
\askforoverwritefalse

\let\MetaPrefix\relax
\preamble

This is a generated file.

Project: ifpdf
Version: 2011/01/30 v2.3

Copyright (C) 2001, 2005-2011 by
   Heiko Oberdiek <heiko.oberdiek at googlemail.com>

This work may be distributed and/or modified under the
conditions of the LaTeX Project Public License, either
version 1.3c of this license or (at your option) any later
version. This version of this license is in
   http://www.latex-project.org/lppl/lppl-1-3c.txt
and the latest version of this license is in
   http://www.latex-project.org/lppl.txt
and version 1.3 or later is part of all distributions of
LaTeX version 2005/12/01 or later.

This work has the LPPL maintenance status "maintained".

This Current Maintainer of this work is Heiko Oberdiek.

The Base Interpreter refers to any `TeX-Format',
because some files are installed in TDS:tex/generic//.

This work consists of the main source file ifpdf.dtx
and the derived files
   ifpdf.sty, ifpdf.pdf, ifpdf.ins, ifpdf.drv, ifpdf-test1.tex.

\endpreamble
\let\MetaPrefix\DoubleperCent

\generate{%
  \file{ifpdf.ins}{\from{ifpdf.dtx}{install}}%
  \file{ifpdf.drv}{\from{ifpdf.dtx}{driver}}%
  \usedir{tex/generic/oberdiek}%
  \file{ifpdf.sty}{\from{ifpdf.dtx}{package}}%
  \usedir{doc/latex/oberdiek/test}%
  \file{ifpdf-test1.tex}{\from{ifpdf.dtx}{test1}}%
  \nopreamble
  \nopostamble
  \usedir{source/latex/oberdiek/catalogue}%
  \file{ifpdf.xml}{\from{ifpdf.dtx}{catalogue}}%
}

\catcode32=13\relax% active space
\let =\space%
\Msg{************************************************************************}
\Msg{*}
\Msg{* To finish the installation you have to move the following}
\Msg{* file into a directory searched by TeX:}
\Msg{*}
\Msg{*     ifpdf.sty}
\Msg{*}
\Msg{* To produce the documentation run the file `ifpdf.drv'}
\Msg{* through LaTeX.}
\Msg{*}
\Msg{* Happy TeXing!}
\Msg{*}
\Msg{************************************************************************}

\endbatchfile
%</install>
%<*ignore>
\fi
%</ignore>
%<*driver>
\NeedsTeXFormat{LaTeX2e}
\ProvidesFile{ifpdf.drv}%
  [2011/01/30 v2.3 Provides the ifpdf switch (HO)]%
\documentclass{ltxdoc}
\usepackage{holtxdoc}[2011/11/22]
\begin{document}
  \DocInput{ifpdf.dtx}%
\end{document}
%</driver>
% \fi
%
% \CheckSum{600}
%
% \CharacterTable
%  {Upper-case    \A\B\C\D\E\F\G\H\I\J\K\L\M\N\O\P\Q\R\S\T\U\V\W\X\Y\Z
%   Lower-case    \a\b\c\d\e\f\g\h\i\j\k\l\m\n\o\p\q\r\s\t\u\v\w\x\y\z
%   Digits        \0\1\2\3\4\5\6\7\8\9
%   Exclamation   \!     Double quote  \"     Hash (number) \#
%   Dollar        \$     Percent       \%     Ampersand     \&
%   Acute accent  \'     Left paren    \(     Right paren   \)
%   Asterisk      \*     Plus          \+     Comma         \,
%   Minus         \-     Point         \.     Solidus       \/
%   Colon         \:     Semicolon     \;     Less than     \<
%   Equals        \=     Greater than  \>     Question mark \?
%   Commercial at \@     Left bracket  \[     Backslash     \\
%   Right bracket \]     Circumflex    \^     Underscore    \_
%   Grave accent  \`     Left brace    \{     Vertical bar  \|
%   Right brace   \}     Tilde         \~}
%
% \GetFileInfo{ifpdf.drv}
%
% \title{The \xpackage{ifpdf} package}
% \date{2011/01/30 v2.3}
% \author{Heiko Oberdiek\\\xemail{heiko.oberdiek at googlemail.com}}
%
% \maketitle
%
% \begin{abstract}
% This package looks for \pdfTeX\ in pdf mode and
% implements and sets the switch \cs{ifpdf}.
% The detection is based on \cs{pdfoutput} and
% the package will not change this value.
% It works with plain or \LaTeX\ formats.
% \end{abstract}
%
% \tableofcontents
%
% \hypersetup{bookmarksopenlevel=2}
% \section{Documentation}
%
% \subsection{Introduction}
%
% It is commonly known that H\`an Th\^e Th\`anh's \pdfTeX\ generates
% PDF output directly and many people uses \pdfTeX\ for this purpose.
% However the DVI output was never thrown away. In contrary, he
% new features for typesetting that works in both PDF and DVI mode.
%
% In the meantime many \TeX\ distributions replace the traditional
% \TeX\ binary with \pdfTeX. Then, for example, called as |latex|
% \pdfTeX\ works in DVI mode with the \LaTeX\ format preloaded,
% called as |pdflatex| \pdfTeX\ starts in PDF mode.
%
% Often packages or users want to know, whether the current document
% is typset by \pdfTeX\ in PDF mode, because the different modes
% have different capabilities (color setting, graphics inclusion,
% \dots). For this purpose \pdfTeX's
% \cs{pdfoutput} can be asked.
%
% As regulary reader of \TeX\ newsgroups and mailing lists I could
% observe many problems with this task. Common errors are:
% \begin{itemize}
% \item
%   \pdfTeX\ has \emph{two} modes. Using \pdfTeX\ does not mean
%   that the user always want to have PDF mode. For example, the
%   PostScript support is better in DVI mode in conjunction with a
%   PostScript aware DVI driver (e.g. dvips). Also the additional
%   typesetting features are mode independent and also available
%   in DVI mode.
% \item
%   \LaTeX's \cs{@ifundefined} inherited the side effect from
%   \cs{csname}. Unknown commands are defined with the meaning
%   of \cs{relax}. If it is checked, whether \cs{pdfoutput} is
%   defined, then this should not be forgotten.
% \item
%   Having \cs{pdfoutput} does not automatically mean PDF mode.
%   Also the value of \cs{pdfoutput} must be asked.
% \item
%   \cs{pdfoutput} must not be destroyed in some way. Later
%   code and packages are fooled then and will perhaps make
%   wrong decisions. For example they may drop support for
%   PDF mode, because they do not know that \pdfTeX\ is
%   running at all.
% \end{itemize}
%
% Robin Fairbairns provides an entry for this topic in his
% excellent FAQ (\url{http://www.tex.ac.uk/faq}):
% \href{http://www.tex.ac.uk/cgi-bin/texfaq2html?label=ifpdf}^^A
% {Am I using PDFTeX?}
%
% \subsection{Usage}
%
% The package \xpackage{ifpdf} can be used with both \plainTeX\
% and \LaTeX:
% \begin{description}
% \item[\plainTeX:] |\input ifpdf.sty|
% \item[\LaTeXe:]   |\usepackage{ifpdf}|
% \end{description}
%
% \DescribeMacro{\ifpdf}
% The package provides the switch \cs{ifpdf}:
% \begin{quote}
%   |\ifpdf|\\
%   \hspace*{1.5em}\dots\ do things, if \pdfTeX\ is running in pdf mode \dots\\
%   |\else|\\
%   \hspace*{1.5em}\dots\ other \TeX\ or \pdfTeX\ in dvi mode \dots\\
%   |\fi|
% \end{quote}
% Users of the package \xpackage{ifthen} can use the switch as boolean:
% \begin{quote}
%   |\boolean{pdf}|
% \end{quote}
% The package can also be used to set global
% documentclass options:
% \begin{quote}
%   |\RequirePackage{ifpdf}|\\
%   |\ifpdf|\\
%   \hspace*{1.5em}|\documentclass[pdftex,|\dots|]{|\dots|}|\\
%   |\else|\\
%   \hspace*{1.5em}|\documentclass[|\dots|]{|\dots|}|\\
%   |\fi|
% \end{quote}
%
% \subsection{Specification}
%
% The package have the following properties:
% \begin{itemize}
% \item It asks the setting of \cs{pdfoutput} for detecting
%   \pdfTeX\ in PDF mode.
% \item It never changes \cs{pdfoutput}.
% \item If \cs{pdfoutput} is undefined or has the meaning \cs{relax},
%   but the engine provides the primitive \cs{pdfoutput},
%   then \cs{pdfoutput} is enabled or restored if possible
%   (only \LuaTeX, version 0.36.0 or higher).
% \item It can be used with many formats including
%   \plainTeX\ and \LaTeX.
% \end{itemize}
% The mode detection implements the following algorithm:
% \begin{quote}
% \def\={${:}{=}$}
% \def\>{\null\qquad\ignorespaces}
% \def\comment#1{^^A
%   \textrm{\itshape\% #1}^^A
% }
% \sffamily
% if undefined(\cs{pdfoutput})\\
% \> \cs{ifpdf} \= \cs{iffalse} \comment{\pdfTeX is not running}\\
% else\\
% \> if \cs{pdfoutput} $\leq$ 0\\
% \>\> \cs{ifpdf} \= \cs{iffalse} \comment{\pdfTeX\ in DVI mode}\\
% \> else\\
% \>\> \cs{ifpdf} \= \cs{iftrue} \comment{\pdfTeX\ in PDF mode}\\
% \> fi\\
% fi
% \end{quote}
% The function \textsf{undefined} checks both cases, undefined command
% and \cs{relax}.
%
% \hypersetup{bookmarksopenlevel=1}
%
% \StopEventually{
% }
%
% \section{Implementation}
%
%    \begin{macrocode}
%<*package>
%    \end{macrocode}
%
% \subsection{Reload check and package identification}
%    Reload check, especially if the package is not used with \LaTeX.
%    \begin{macrocode}
\begingroup\catcode61\catcode48\catcode32=10\relax%
  \catcode13=5 % ^^M
  \endlinechar=13 %
  \catcode35=6 % #
  \catcode39=12 % '
  \catcode44=12 % ,
  \catcode45=12 % -
  \catcode46=12 % .
  \catcode58=12 % :
  \catcode64=11 % @
  \catcode123=1 % {
  \catcode125=2 % }
  \expandafter\let\expandafter\x\csname ver@ifpdf.sty\endcsname
  \ifx\x\relax % plain-TeX, first loading
  \else
    \def\empty{}%
    \ifx\x\empty % LaTeX, first loading,
      % variable is initialized, but \ProvidesPackage not yet seen
    \else
      \expandafter\ifx\csname PackageInfo\endcsname\relax
        \def\x#1#2{%
          \immediate\write-1{Package #1 Info: #2.}%
        }%
      \else
        \def\x#1#2{\PackageInfo{#1}{#2, stopped}}%
      \fi
      \x{ifpdf}{The package is already loaded}%
      \aftergroup\endinput
    \fi
  \fi
\endgroup%
%    \end{macrocode}
%    Package identification:
%    \begin{macrocode}
\begingroup\catcode61\catcode48\catcode32=10\relax%
  \catcode13=5 % ^^M
  \endlinechar=13 %
  \catcode35=6 % #
  \catcode39=12 % '
  \catcode40=12 % (
  \catcode41=12 % )
  \catcode44=12 % ,
  \catcode45=12 % -
  \catcode46=12 % .
  \catcode47=12 % /
  \catcode58=12 % :
  \catcode64=11 % @
  \catcode91=12 % [
  \catcode93=12 % ]
  \catcode123=1 % {
  \catcode125=2 % }
  \expandafter\ifx\csname ProvidesPackage\endcsname\relax
    \def\x#1#2#3[#4]{\endgroup
      \immediate\write-1{Package: #3 #4}%
      \xdef#1{#4}%
    }%
  \else
    \def\x#1#2[#3]{\endgroup
      #2[{#3}]%
      \ifx#1\@undefined
        \xdef#1{#3}%
      \fi
      \ifx#1\relax
        \xdef#1{#3}%
      \fi
    }%
  \fi
\expandafter\x\csname ver@ifpdf.sty\endcsname
\ProvidesPackage{ifpdf}%
  [2011/01/30 v2.3 Provides the ifpdf switch (HO)]%
%    \end{macrocode}
%
% \subsection{Catcodes}
%
%    \begin{macrocode}
\begingroup\catcode61\catcode48\catcode32=10\relax%
  \catcode13=5 % ^^M
  \endlinechar=13 %
  \catcode123=1 % {
  \catcode125=2 % }
  \catcode64=11 % @
  \def\x{\endgroup
    \expandafter\edef\csname ifpdf@AtEnd\endcsname{%
      \endlinechar=\the\endlinechar\relax
      \catcode13=\the\catcode13\relax
      \catcode32=\the\catcode32\relax
      \catcode35=\the\catcode35\relax
      \catcode61=\the\catcode61\relax
      \catcode64=\the\catcode64\relax
      \catcode123=\the\catcode123\relax
      \catcode125=\the\catcode125\relax
    }%
  }%
\x\catcode61\catcode48\catcode32=10\relax%
\catcode13=5 % ^^M
\endlinechar=13 %
\catcode35=6 % #
\catcode64=11 % @
\catcode123=1 % {
\catcode125=2 % }
\def\TMP@EnsureCode#1#2{%
  \edef\ifpdf@AtEnd{%
    \ifpdf@AtEnd
    \catcode#1=\the\catcode#1\relax
  }%
  \catcode#1=#2\relax
}
\TMP@EnsureCode{10}{12}% ^^J
\TMP@EnsureCode{39}{12}% '
\TMP@EnsureCode{40}{12}% (
\TMP@EnsureCode{41}{12}% )
\TMP@EnsureCode{44}{12}% ,
\TMP@EnsureCode{45}{12}% -
\TMP@EnsureCode{46}{12}% .
\TMP@EnsureCode{47}{12}% /
\TMP@EnsureCode{58}{12}% :
\TMP@EnsureCode{60}{12}% <
\TMP@EnsureCode{91}{12}% [
\TMP@EnsureCode{93}{12}% ]
\TMP@EnsureCode{94}{7}% ^
\TMP@EnsureCode{96}{12}% `
\edef\ifpdf@AtEnd{\ifpdf@AtEnd\noexpand\endinput}
%    \end{macrocode}
%
% \subsection{Check for previously defined \cs{ifpdf}}
%
%    \begin{macrocode}
\begingroup
  \expandafter\ifx\csname ifpdf\endcsname\relax
  \else
    \edef\i/{\expandafter\string\csname ifpdf\endcsname}%
    \expandafter\ifx\csname PackageError\endcsname\relax
      \def\x#1#2{%
        \edef\z{#2}%
        \expandafter\errhelp\expandafter{\z}%
        \errmessage{Package ifpdf Error: #1}%
      }%
      \def\y{^^J}%
      \newlinechar=10 %
    \else
      \def\x#1#2{%
        \PackageError{ifpdf}{#1}{#2}%
      }%
      \def\y{\MessageBreak}%
    \fi
    \x{Name clash, \i/ is already defined}{%
      Incompatible versions of \i/ can cause problems,\y
      therefore package loading is aborted.%
    }%
    \endgroup
    \expandafter\ifpdf@AtEnd
  \fi%
\endgroup
%    \end{macrocode}
%
% \subsection{\cs{pdfoutput} and \LuaTeX}
%
%    It might happen, that \LuaTeX\ is running, but \cs{pdfoutput}
%    does not exist. In version 0.40 only \cs{directlua}
%    is available at startup time. The enabling Lua function was
%    already added in version 0.36. Thus we can ignore older versions,
%    here \cs{pdfoutput} is available at startup time.
%
%    \begin{macrocode}
\begingroup
  \def\skip#1\relax\begingroup{}%
  \expandafter\ifx\csname pdfoutput\endcsname\relax
  \else
    \expandafter\skip
  \fi
  \expandafter\ifx\csname directlua\endcsname\relax
    \expandafter\skip
  \fi
\endgroup
\begingroup\expandafter\expandafter\expandafter\endgroup
\expandafter\ifx\csname RequirePackage\endcsname\relax
  \def\TMP@RequirePackage#1[#2]{%
    \begingroup\expandafter\expandafter\expandafter\endgroup
    \expandafter\ifx\csname ver@#1.sty\endcsname\relax
      \input #1.sty\relax
    \fi
  }%
  \TMP@RequirePackage{ifluatex}[2009/04/10]%
\else
  \RequirePackage{ifluatex}[2009/04/10]%
\fi
\ifluatex
  \ifnum\luatexversion<36 %
%    \end{macrocode}
%    Unhappily \LuaTeX's \cs{primitive} (derived from
%    pdfTeX's \cs{pdfprimitive}) cannot be used:
%    \begin{quote}
%    |\protected\gdef\pdfoutput{\primitive\pdfoutput}|
%    \end{quote}
%    Setting a value works, but getting fails,
%    because \TeX\ does no longer see it as number. It
%    is unexpandable and breaks numerical contexts.
%    This was fixed in \hologo{pdfTeX} 1.40.10
%    (bugfix \#4289: ``|\primitive\pdfoutput| cannot be queried'').
%    \begin{macrocode}
  \else
    \begingroup
      \directlua{tex.enableprimitives('ifpdf', {'pdfoutput'})}%
      \global\let\pdfoutput\ifpdfpdfoutput
    \endgroup
  \fi
\fi
\relax\begingroup\endgroup
%    \end{macrocode}
%
% \subsection{\cs{ifpdf}}
%
%    \begin{macro}{\ifpdf}
%    Create and set the switch. \cs{newif} initializes the
%    switch with \cs{iffalse}. \cs{newif} is \cs{outer}
%    in \plainTeX.
%    \begin{macrocode}
\begingroup\expandafter\expandafter\expandafter\endgroup
\expandafter\ifx\csname newif\endcsname\relax
  \edef\pdffalse{%
    \let
    \expandafter\noexpand\csname ifpdf\endcsname
    \expandafter\noexpand\csname iffalse\endcsname
  }%
  \edef\pdftrue{%
    \let
    \expandafter\noexpand\csname ifpdf\endcsname
    \expandafter\noexpand\csname iftrue\endcsname
  }%
  \pdffalse
\else
  \csname newif\expandafter\endcsname\csname ifpdf\endcsname
\fi
%    \end{macrocode}
%    Test \cs{pdfoutput}. Is it  defined and different from
%    \cs{relax}? Someone could have used \LaTeX\ internal
%    \cs{@ifundefined}, or something else involving.
%    Notice, \cs{csname} is executed inside a group for the test
%    to cancel the side effect of \cs{csname}.
%    \begin{macrocode}
\begingroup\expandafter\expandafter\expandafter\endgroup
\expandafter\ifx\csname pdfoutput\endcsname\relax
\else
  \ifnum\pdfoutput<1 %
%    \end{macrocode}
%    \cs{pdfoutput}|=0| or negative, so not generating pdf.
%    \begin{macrocode}
  \else
    \pdftrue
  \fi
\fi
%    \end{macrocode}
%    \end{macro}
%
% \subsection{Test for fool attempts}
%
%    \begin{macrocode}
\begingroup
  \expandafter\ifx\csname pdfoutput\endcsname\relax
  \else
    \escapechar=92 %
    \edef\m{\meaning\pdfoutput}%
    \edef\p{\string\pdfoutput}%
    \ifx\m\p
    \else
      \expandafter\ifx\csname PackageWarningNoLine\endcsname\relax
        \def\PackageWarningNoLine#1#2{%
          \immediate\write16{%
            Package `#1' Warning: #2.%
          }%
        }%
      \fi
      \PackageWarningNoLine{ifpdf}{%
        Someone has redefined \string\pdfoutput%
      }%
    \fi
  \fi
\endgroup
%    \end{macrocode}
%
% \subsection{Protocol entry}
%
%     Log comment:
%    \begin{macrocode}
\begingroup
  \expandafter\ifx\csname PackageInfo\endcsname\relax
    \def\x#1#2{%
      \immediate\write-1{Package #1 Info: #2.}%
    }%
  \else
    \let\x\PackageInfo
    \expandafter\let\csname on@line\endcsname\empty
  \fi
  \x{ifpdf}{pdfTeX in PDF mode is \ifpdf\else not \fi detected}%
\endgroup
%    \end{macrocode}
%    \begin{macrocode}
\ifpdf@AtEnd%
%    \end{macrocode}
%    \begin{macrocode}
%</package>
%    \end{macrocode}
%
% \section{Test}
%
% \subsection{Catcode checks for loading}
%
%    \begin{macrocode}
%<*test1>
%    \end{macrocode}
%    \begin{macrocode}
\catcode`\{=1 %
\catcode`\}=2 %
\catcode`\#=6 %
\catcode`\@=11 %
\expandafter\ifx\csname count@\endcsname\relax
  \countdef\count@=255 %
\fi
\expandafter\ifx\csname @gobble\endcsname\relax
  \long\def\@gobble#1{}%
\fi
\expandafter\ifx\csname @firstofone\endcsname\relax
  \long\def\@firstofone#1{#1}%
\fi
\expandafter\ifx\csname loop\endcsname\relax
  \expandafter\@firstofone
\else
  \expandafter\@gobble
\fi
{%
  \def\loop#1\repeat{%
    \def\body{#1}%
    \iterate
  }%
  \def\iterate{%
    \body
      \let\next\iterate
    \else
      \let\next\relax
    \fi
    \next
  }%
  \let\repeat=\fi
}%
\def\RestoreCatcodes{}
\count@=0 %
\loop
  \edef\RestoreCatcodes{%
    \RestoreCatcodes
    \catcode\the\count@=\the\catcode\count@\relax
  }%
\ifnum\count@<255 %
  \advance\count@ 1 %
\repeat

\def\RangeCatcodeInvalid#1#2{%
  \count@=#1\relax
  \loop
    \catcode\count@=15 %
  \ifnum\count@<#2\relax
    \advance\count@ 1 %
  \repeat
}
\def\RangeCatcodeCheck#1#2#3{%
  \count@=#1\relax
  \loop
    \ifnum#3=\catcode\count@
    \else
      \errmessage{%
        Character \the\count@\space
        with wrong catcode \the\catcode\count@\space
        instead of \number#3%
      }%
    \fi
  \ifnum\count@<#2\relax
    \advance\count@ 1 %
  \repeat
}
\def\space{ }
\expandafter\ifx\csname LoadCommand\endcsname\relax
  \def\LoadCommand{\input ifpdf.sty\relax}%
\fi
\def\Test{%
  \RangeCatcodeInvalid{0}{47}%
  \RangeCatcodeInvalid{58}{64}%
  \RangeCatcodeInvalid{91}{96}%
  \RangeCatcodeInvalid{123}{255}%
  \catcode`\@=12 %
  \catcode`\\=0 %
  \catcode`\%=14 %
  \LoadCommand
  \RangeCatcodeCheck{0}{36}{15}%
  \RangeCatcodeCheck{37}{37}{14}%
  \RangeCatcodeCheck{38}{47}{15}%
  \RangeCatcodeCheck{48}{57}{12}%
  \RangeCatcodeCheck{58}{63}{15}%
  \RangeCatcodeCheck{64}{64}{12}%
  \RangeCatcodeCheck{65}{90}{11}%
  \RangeCatcodeCheck{91}{91}{15}%
  \RangeCatcodeCheck{92}{92}{0}%
  \RangeCatcodeCheck{93}{96}{15}%
  \RangeCatcodeCheck{97}{122}{11}%
  \RangeCatcodeCheck{123}{255}{15}%
  \RestoreCatcodes
}
\Test
\csname @@end\endcsname
\end
%    \end{macrocode}
%    \begin{macrocode}
%</test1>
%    \end{macrocode}
%
% \section{Installation}
%
% \subsection{Download}
%
% \paragraph{Package.} This package is available on
% CTAN\footnote{\url{ftp://ftp.ctan.org/tex-archive/}}:
% \begin{description}
% \item[\CTAN{macros/latex/contrib/oberdiek/ifpdf.dtx}] The source file.
% \item[\CTAN{macros/latex/contrib/oberdiek/ifpdf.pdf}] Documentation.
% \end{description}
%
%
% \paragraph{Bundle.} All the packages of the bundle `oberdiek'
% are also available in a TDS compliant ZIP archive. There
% the packages are already unpacked and the documentation files
% are generated. The files and directories obey the TDS standard.
% \begin{description}
% \item[\CTAN{install/macros/latex/contrib/oberdiek.tds.zip}]
% \end{description}
% \emph{TDS} refers to the standard ``A Directory Structure
% for \TeX\ Files'' (\CTAN{tds/tds.pdf}). Directories
% with \xfile{texmf} in their name are usually organized this way.
%
% \subsection{Bundle installation}
%
% \paragraph{Unpacking.} Unpack the \xfile{oberdiek.tds.zip} in the
% TDS tree (also known as \xfile{texmf} tree) of your choice.
% Example (linux):
% \begin{quote}
%   |unzip oberdiek.tds.zip -d ~/texmf|
% \end{quote}
%
% \paragraph{Script installation.}
% Check the directory \xfile{TDS:scripts/oberdiek/} for
% scripts that need further installation steps.
% Package \xpackage{attachfile2} comes with the Perl script
% \xfile{pdfatfi.pl} that should be installed in such a way
% that it can be called as \texttt{pdfatfi}.
% Example (linux):
% \begin{quote}
%   |chmod +x scripts/oberdiek/pdfatfi.pl|\\
%   |cp scripts/oberdiek/pdfatfi.pl /usr/local/bin/|
% \end{quote}
%
% \subsection{Package installation}
%
% \paragraph{Unpacking.} The \xfile{.dtx} file is a self-extracting
% \docstrip\ archive. The files are extracted by running the
% \xfile{.dtx} through \plainTeX:
% \begin{quote}
%   \verb|tex ifpdf.dtx|
% \end{quote}
%
% \paragraph{TDS.} Now the different files must be moved into
% the different directories in your installation TDS tree
% (also known as \xfile{texmf} tree):
% \begin{quote}
% \def\t{^^A
% \begin{tabular}{@{}>{\ttfamily}l@{ $\rightarrow$ }>{\ttfamily}l@{}}
%   ifpdf.sty & tex/generic/oberdiek/ifpdf.sty\\
%   ifpdf.pdf & doc/latex/oberdiek/ifpdf.pdf\\
%   test/ifpdf-test1.tex & doc/latex/oberdiek/test/ifpdf-test1.tex\\
%   ifpdf.dtx & source/latex/oberdiek/ifpdf.dtx\\
% \end{tabular}^^A
% }^^A
% \sbox0{\t}^^A
% \ifdim\wd0>\linewidth
%   \begingroup
%     \advance\linewidth by\leftmargin
%     \advance\linewidth by\rightmargin
%   \edef\x{\endgroup
%     \def\noexpand\lw{\the\linewidth}^^A
%   }\x
%   \def\lwbox{^^A
%     \leavevmode
%     \hbox to \linewidth{^^A
%       \kern-\leftmargin\relax
%       \hss
%       \usebox0
%       \hss
%       \kern-\rightmargin\relax
%     }^^A
%   }^^A
%   \ifdim\wd0>\lw
%     \sbox0{\small\t}^^A
%     \ifdim\wd0>\linewidth
%       \ifdim\wd0>\lw
%         \sbox0{\footnotesize\t}^^A
%         \ifdim\wd0>\linewidth
%           \ifdim\wd0>\lw
%             \sbox0{\scriptsize\t}^^A
%             \ifdim\wd0>\linewidth
%               \ifdim\wd0>\lw
%                 \sbox0{\tiny\t}^^A
%                 \ifdim\wd0>\linewidth
%                   \lwbox
%                 \else
%                   \usebox0
%                 \fi
%               \else
%                 \lwbox
%               \fi
%             \else
%               \usebox0
%             \fi
%           \else
%             \lwbox
%           \fi
%         \else
%           \usebox0
%         \fi
%       \else
%         \lwbox
%       \fi
%     \else
%       \usebox0
%     \fi
%   \else
%     \lwbox
%   \fi
% \else
%   \usebox0
% \fi
% \end{quote}
% If you have a \xfile{docstrip.cfg} that configures and enables \docstrip's
% TDS installing feature, then some files can already be in the right
% place, see the documentation of \docstrip.
%
% \subsection{Refresh file name databases}
%
% If your \TeX~distribution
% (\teTeX, \mikTeX, \dots) relies on file name databases, you must refresh
% these. For example, \teTeX\ users run \verb|texhash| or
% \verb|mktexlsr|.
%
% \subsection{Some details for the interested}
%
% \paragraph{Attached source.}
%
% The PDF documentation on CTAN also includes the
% \xfile{.dtx} source file. It can be extracted by
% AcrobatReader 6 or higher. Another option is \textsf{pdftk},
% e.g. unpack the file into the current directory:
% \begin{quote}
%   \verb|pdftk ifpdf.pdf unpack_files output .|
% \end{quote}
%
% \paragraph{Unpacking with \LaTeX.}
% The \xfile{.dtx} chooses its action depending on the format:
% \begin{description}
% \item[\plainTeX:] Run \docstrip\ and extract the files.
% \item[\LaTeX:] Generate the documentation.
% \end{description}
% If you insist on using \LaTeX\ for \docstrip\ (really,
% \docstrip\ does not need \LaTeX), then inform the autodetect routine
% about your intention:
% \begin{quote}
%   \verb|latex \let\install=y% \iffalse meta-comment
%
% File: ifpdf.dtx
% Version: 2011/01/30 v2.3
% Info: Provides the ifpdf switch
%
% Copyright (C) 2001, 2005-2011 by
%    Heiko Oberdiek <heiko.oberdiek at googlemail.com>
%
% This work may be distributed and/or modified under the
% conditions of the LaTeX Project Public License, either
% version 1.3c of this license or (at your option) any later
% version. This version of this license is in
%    http://www.latex-project.org/lppl/lppl-1-3c.txt
% and the latest version of this license is in
%    http://www.latex-project.org/lppl.txt
% and version 1.3 or later is part of all distributions of
% LaTeX version 2005/12/01 or later.
%
% This work has the LPPL maintenance status "maintained".
%
% This Current Maintainer of this work is Heiko Oberdiek.
%
% The Base Interpreter refers to any `TeX-Format',
% because some files are installed in TDS:tex/generic//.
%
% This work consists of the main source file ifpdf.dtx
% and the derived files
%    ifpdf.sty, ifpdf.pdf, ifpdf.ins, ifpdf.drv, ifpdf-test1.tex.
%
% Distribution:
%    CTAN:macros/latex/contrib/oberdiek/ifpdf.dtx
%    CTAN:macros/latex/contrib/oberdiek/ifpdf.pdf
%
% Unpacking:
%    (a) If ifpdf.ins is present:
%           tex ifpdf.ins
%    (b) Without ifpdf.ins:
%           tex ifpdf.dtx
%    (c) If you insist on using LaTeX
%           latex \let\install=y% \iffalse meta-comment
%
% File: ifpdf.dtx
% Version: 2011/01/30 v2.3
% Info: Provides the ifpdf switch
%
% Copyright (C) 2001, 2005-2011 by
%    Heiko Oberdiek <heiko.oberdiek at googlemail.com>
%
% This work may be distributed and/or modified under the
% conditions of the LaTeX Project Public License, either
% version 1.3c of this license or (at your option) any later
% version. This version of this license is in
%    http://www.latex-project.org/lppl/lppl-1-3c.txt
% and the latest version of this license is in
%    http://www.latex-project.org/lppl.txt
% and version 1.3 or later is part of all distributions of
% LaTeX version 2005/12/01 or later.
%
% This work has the LPPL maintenance status "maintained".
%
% This Current Maintainer of this work is Heiko Oberdiek.
%
% The Base Interpreter refers to any `TeX-Format',
% because some files are installed in TDS:tex/generic//.
%
% This work consists of the main source file ifpdf.dtx
% and the derived files
%    ifpdf.sty, ifpdf.pdf, ifpdf.ins, ifpdf.drv, ifpdf-test1.tex.
%
% Distribution:
%    CTAN:macros/latex/contrib/oberdiek/ifpdf.dtx
%    CTAN:macros/latex/contrib/oberdiek/ifpdf.pdf
%
% Unpacking:
%    (a) If ifpdf.ins is present:
%           tex ifpdf.ins
%    (b) Without ifpdf.ins:
%           tex ifpdf.dtx
%    (c) If you insist on using LaTeX
%           latex \let\install=y\input{ifpdf.dtx}
%        (quote the arguments according to the demands of your shell)
%
% Documentation:
%    (a) If ifpdf.drv is present:
%           latex ifpdf.drv
%    (b) Without ifpdf.drv:
%           latex ifpdf.dtx; ...
%    The class ltxdoc loads the configuration file ltxdoc.cfg
%    if available. Here you can specify further options, e.g.
%    use A4 as paper format:
%       \PassOptionsToClass{a4paper}{article}
%
%    Programm calls to get the documentation (example):
%       pdflatex ifpdf.dtx
%       makeindex -s gind.ist ifpdf.idx
%       pdflatex ifpdf.dtx
%       makeindex -s gind.ist ifpdf.idx
%       pdflatex ifpdf.dtx
%
% Installation:
%    TDS:tex/generic/oberdiek/ifpdf.sty
%    TDS:doc/latex/oberdiek/ifpdf.pdf
%    TDS:doc/latex/oberdiek/test/ifpdf-test1.tex
%    TDS:source/latex/oberdiek/ifpdf.dtx
%
%<*ignore>
\begingroup
  \catcode123=1 %
  \catcode125=2 %
  \def\x{LaTeX2e}%
\expandafter\endgroup
\ifcase 0\ifx\install y1\fi\expandafter
         \ifx\csname processbatchFile\endcsname\relax\else1\fi
         \ifx\fmtname\x\else 1\fi\relax
\else\csname fi\endcsname
%</ignore>
%<*install>
\input docstrip.tex
\Msg{************************************************************************}
\Msg{* Installation}
\Msg{* Package: ifpdf 2011/01/30 v2.3 Provides the ifpdf switch (HO)}
\Msg{************************************************************************}

\keepsilent
\askforoverwritefalse

\let\MetaPrefix\relax
\preamble

This is a generated file.

Project: ifpdf
Version: 2011/01/30 v2.3

Copyright (C) 2001, 2005-2011 by
   Heiko Oberdiek <heiko.oberdiek at googlemail.com>

This work may be distributed and/or modified under the
conditions of the LaTeX Project Public License, either
version 1.3c of this license or (at your option) any later
version. This version of this license is in
   http://www.latex-project.org/lppl/lppl-1-3c.txt
and the latest version of this license is in
   http://www.latex-project.org/lppl.txt
and version 1.3 or later is part of all distributions of
LaTeX version 2005/12/01 or later.

This work has the LPPL maintenance status "maintained".

This Current Maintainer of this work is Heiko Oberdiek.

The Base Interpreter refers to any `TeX-Format',
because some files are installed in TDS:tex/generic//.

This work consists of the main source file ifpdf.dtx
and the derived files
   ifpdf.sty, ifpdf.pdf, ifpdf.ins, ifpdf.drv, ifpdf-test1.tex.

\endpreamble
\let\MetaPrefix\DoubleperCent

\generate{%
  \file{ifpdf.ins}{\from{ifpdf.dtx}{install}}%
  \file{ifpdf.drv}{\from{ifpdf.dtx}{driver}}%
  \usedir{tex/generic/oberdiek}%
  \file{ifpdf.sty}{\from{ifpdf.dtx}{package}}%
  \usedir{doc/latex/oberdiek/test}%
  \file{ifpdf-test1.tex}{\from{ifpdf.dtx}{test1}}%
  \nopreamble
  \nopostamble
  \usedir{source/latex/oberdiek/catalogue}%
  \file{ifpdf.xml}{\from{ifpdf.dtx}{catalogue}}%
}

\catcode32=13\relax% active space
\let =\space%
\Msg{************************************************************************}
\Msg{*}
\Msg{* To finish the installation you have to move the following}
\Msg{* file into a directory searched by TeX:}
\Msg{*}
\Msg{*     ifpdf.sty}
\Msg{*}
\Msg{* To produce the documentation run the file `ifpdf.drv'}
\Msg{* through LaTeX.}
\Msg{*}
\Msg{* Happy TeXing!}
\Msg{*}
\Msg{************************************************************************}

\endbatchfile
%</install>
%<*ignore>
\fi
%</ignore>
%<*driver>
\NeedsTeXFormat{LaTeX2e}
\ProvidesFile{ifpdf.drv}%
  [2011/01/30 v2.3 Provides the ifpdf switch (HO)]%
\documentclass{ltxdoc}
\usepackage{holtxdoc}[2011/11/22]
\begin{document}
  \DocInput{ifpdf.dtx}%
\end{document}
%</driver>
% \fi
%
% \CheckSum{600}
%
% \CharacterTable
%  {Upper-case    \A\B\C\D\E\F\G\H\I\J\K\L\M\N\O\P\Q\R\S\T\U\V\W\X\Y\Z
%   Lower-case    \a\b\c\d\e\f\g\h\i\j\k\l\m\n\o\p\q\r\s\t\u\v\w\x\y\z
%   Digits        \0\1\2\3\4\5\6\7\8\9
%   Exclamation   \!     Double quote  \"     Hash (number) \#
%   Dollar        \$     Percent       \%     Ampersand     \&
%   Acute accent  \'     Left paren    \(     Right paren   \)
%   Asterisk      \*     Plus          \+     Comma         \,
%   Minus         \-     Point         \.     Solidus       \/
%   Colon         \:     Semicolon     \;     Less than     \<
%   Equals        \=     Greater than  \>     Question mark \?
%   Commercial at \@     Left bracket  \[     Backslash     \\
%   Right bracket \]     Circumflex    \^     Underscore    \_
%   Grave accent  \`     Left brace    \{     Vertical bar  \|
%   Right brace   \}     Tilde         \~}
%
% \GetFileInfo{ifpdf.drv}
%
% \title{The \xpackage{ifpdf} package}
% \date{2011/01/30 v2.3}
% \author{Heiko Oberdiek\\\xemail{heiko.oberdiek at googlemail.com}}
%
% \maketitle
%
% \begin{abstract}
% This package looks for \pdfTeX\ in pdf mode and
% implements and sets the switch \cs{ifpdf}.
% The detection is based on \cs{pdfoutput} and
% the package will not change this value.
% It works with plain or \LaTeX\ formats.
% \end{abstract}
%
% \tableofcontents
%
% \hypersetup{bookmarksopenlevel=2}
% \section{Documentation}
%
% \subsection{Introduction}
%
% It is commonly known that H\`an Th\^e Th\`anh's \pdfTeX\ generates
% PDF output directly and many people uses \pdfTeX\ for this purpose.
% However the DVI output was never thrown away. In contrary, he
% new features for typesetting that works in both PDF and DVI mode.
%
% In the meantime many \TeX\ distributions replace the traditional
% \TeX\ binary with \pdfTeX. Then, for example, called as |latex|
% \pdfTeX\ works in DVI mode with the \LaTeX\ format preloaded,
% called as |pdflatex| \pdfTeX\ starts in PDF mode.
%
% Often packages or users want to know, whether the current document
% is typset by \pdfTeX\ in PDF mode, because the different modes
% have different capabilities (color setting, graphics inclusion,
% \dots). For this purpose \pdfTeX's
% \cs{pdfoutput} can be asked.
%
% As regulary reader of \TeX\ newsgroups and mailing lists I could
% observe many problems with this task. Common errors are:
% \begin{itemize}
% \item
%   \pdfTeX\ has \emph{two} modes. Using \pdfTeX\ does not mean
%   that the user always want to have PDF mode. For example, the
%   PostScript support is better in DVI mode in conjunction with a
%   PostScript aware DVI driver (e.g. dvips). Also the additional
%   typesetting features are mode independent and also available
%   in DVI mode.
% \item
%   \LaTeX's \cs{@ifundefined} inherited the side effect from
%   \cs{csname}. Unknown commands are defined with the meaning
%   of \cs{relax}. If it is checked, whether \cs{pdfoutput} is
%   defined, then this should not be forgotten.
% \item
%   Having \cs{pdfoutput} does not automatically mean PDF mode.
%   Also the value of \cs{pdfoutput} must be asked.
% \item
%   \cs{pdfoutput} must not be destroyed in some way. Later
%   code and packages are fooled then and will perhaps make
%   wrong decisions. For example they may drop support for
%   PDF mode, because they do not know that \pdfTeX\ is
%   running at all.
% \end{itemize}
%
% Robin Fairbairns provides an entry for this topic in his
% excellent FAQ (\url{http://www.tex.ac.uk/faq}):
% \href{http://www.tex.ac.uk/cgi-bin/texfaq2html?label=ifpdf}^^A
% {Am I using PDFTeX?}
%
% \subsection{Usage}
%
% The package \xpackage{ifpdf} can be used with both \plainTeX\
% and \LaTeX:
% \begin{description}
% \item[\plainTeX:] |\input ifpdf.sty|
% \item[\LaTeXe:]   |\usepackage{ifpdf}|
% \end{description}
%
% \DescribeMacro{\ifpdf}
% The package provides the switch \cs{ifpdf}:
% \begin{quote}
%   |\ifpdf|\\
%   \hspace*{1.5em}\dots\ do things, if \pdfTeX\ is running in pdf mode \dots\\
%   |\else|\\
%   \hspace*{1.5em}\dots\ other \TeX\ or \pdfTeX\ in dvi mode \dots\\
%   |\fi|
% \end{quote}
% Users of the package \xpackage{ifthen} can use the switch as boolean:
% \begin{quote}
%   |\boolean{pdf}|
% \end{quote}
% The package can also be used to set global
% documentclass options:
% \begin{quote}
%   |\RequirePackage{ifpdf}|\\
%   |\ifpdf|\\
%   \hspace*{1.5em}|\documentclass[pdftex,|\dots|]{|\dots|}|\\
%   |\else|\\
%   \hspace*{1.5em}|\documentclass[|\dots|]{|\dots|}|\\
%   |\fi|
% \end{quote}
%
% \subsection{Specification}
%
% The package have the following properties:
% \begin{itemize}
% \item It asks the setting of \cs{pdfoutput} for detecting
%   \pdfTeX\ in PDF mode.
% \item It never changes \cs{pdfoutput}.
% \item If \cs{pdfoutput} is undefined or has the meaning \cs{relax},
%   but the engine provides the primitive \cs{pdfoutput},
%   then \cs{pdfoutput} is enabled or restored if possible
%   (only \LuaTeX, version 0.36.0 or higher).
% \item It can be used with many formats including
%   \plainTeX\ and \LaTeX.
% \end{itemize}
% The mode detection implements the following algorithm:
% \begin{quote}
% \def\={${:}{=}$}
% \def\>{\null\qquad\ignorespaces}
% \def\comment#1{^^A
%   \textrm{\itshape\% #1}^^A
% }
% \sffamily
% if undefined(\cs{pdfoutput})\\
% \> \cs{ifpdf} \= \cs{iffalse} \comment{\pdfTeX is not running}\\
% else\\
% \> if \cs{pdfoutput} $\leq$ 0\\
% \>\> \cs{ifpdf} \= \cs{iffalse} \comment{\pdfTeX\ in DVI mode}\\
% \> else\\
% \>\> \cs{ifpdf} \= \cs{iftrue} \comment{\pdfTeX\ in PDF mode}\\
% \> fi\\
% fi
% \end{quote}
% The function \textsf{undefined} checks both cases, undefined command
% and \cs{relax}.
%
% \hypersetup{bookmarksopenlevel=1}
%
% \StopEventually{
% }
%
% \section{Implementation}
%
%    \begin{macrocode}
%<*package>
%    \end{macrocode}
%
% \subsection{Reload check and package identification}
%    Reload check, especially if the package is not used with \LaTeX.
%    \begin{macrocode}
\begingroup\catcode61\catcode48\catcode32=10\relax%
  \catcode13=5 % ^^M
  \endlinechar=13 %
  \catcode35=6 % #
  \catcode39=12 % '
  \catcode44=12 % ,
  \catcode45=12 % -
  \catcode46=12 % .
  \catcode58=12 % :
  \catcode64=11 % @
  \catcode123=1 % {
  \catcode125=2 % }
  \expandafter\let\expandafter\x\csname ver@ifpdf.sty\endcsname
  \ifx\x\relax % plain-TeX, first loading
  \else
    \def\empty{}%
    \ifx\x\empty % LaTeX, first loading,
      % variable is initialized, but \ProvidesPackage not yet seen
    \else
      \expandafter\ifx\csname PackageInfo\endcsname\relax
        \def\x#1#2{%
          \immediate\write-1{Package #1 Info: #2.}%
        }%
      \else
        \def\x#1#2{\PackageInfo{#1}{#2, stopped}}%
      \fi
      \x{ifpdf}{The package is already loaded}%
      \aftergroup\endinput
    \fi
  \fi
\endgroup%
%    \end{macrocode}
%    Package identification:
%    \begin{macrocode}
\begingroup\catcode61\catcode48\catcode32=10\relax%
  \catcode13=5 % ^^M
  \endlinechar=13 %
  \catcode35=6 % #
  \catcode39=12 % '
  \catcode40=12 % (
  \catcode41=12 % )
  \catcode44=12 % ,
  \catcode45=12 % -
  \catcode46=12 % .
  \catcode47=12 % /
  \catcode58=12 % :
  \catcode64=11 % @
  \catcode91=12 % [
  \catcode93=12 % ]
  \catcode123=1 % {
  \catcode125=2 % }
  \expandafter\ifx\csname ProvidesPackage\endcsname\relax
    \def\x#1#2#3[#4]{\endgroup
      \immediate\write-1{Package: #3 #4}%
      \xdef#1{#4}%
    }%
  \else
    \def\x#1#2[#3]{\endgroup
      #2[{#3}]%
      \ifx#1\@undefined
        \xdef#1{#3}%
      \fi
      \ifx#1\relax
        \xdef#1{#3}%
      \fi
    }%
  \fi
\expandafter\x\csname ver@ifpdf.sty\endcsname
\ProvidesPackage{ifpdf}%
  [2011/01/30 v2.3 Provides the ifpdf switch (HO)]%
%    \end{macrocode}
%
% \subsection{Catcodes}
%
%    \begin{macrocode}
\begingroup\catcode61\catcode48\catcode32=10\relax%
  \catcode13=5 % ^^M
  \endlinechar=13 %
  \catcode123=1 % {
  \catcode125=2 % }
  \catcode64=11 % @
  \def\x{\endgroup
    \expandafter\edef\csname ifpdf@AtEnd\endcsname{%
      \endlinechar=\the\endlinechar\relax
      \catcode13=\the\catcode13\relax
      \catcode32=\the\catcode32\relax
      \catcode35=\the\catcode35\relax
      \catcode61=\the\catcode61\relax
      \catcode64=\the\catcode64\relax
      \catcode123=\the\catcode123\relax
      \catcode125=\the\catcode125\relax
    }%
  }%
\x\catcode61\catcode48\catcode32=10\relax%
\catcode13=5 % ^^M
\endlinechar=13 %
\catcode35=6 % #
\catcode64=11 % @
\catcode123=1 % {
\catcode125=2 % }
\def\TMP@EnsureCode#1#2{%
  \edef\ifpdf@AtEnd{%
    \ifpdf@AtEnd
    \catcode#1=\the\catcode#1\relax
  }%
  \catcode#1=#2\relax
}
\TMP@EnsureCode{10}{12}% ^^J
\TMP@EnsureCode{39}{12}% '
\TMP@EnsureCode{40}{12}% (
\TMP@EnsureCode{41}{12}% )
\TMP@EnsureCode{44}{12}% ,
\TMP@EnsureCode{45}{12}% -
\TMP@EnsureCode{46}{12}% .
\TMP@EnsureCode{47}{12}% /
\TMP@EnsureCode{58}{12}% :
\TMP@EnsureCode{60}{12}% <
\TMP@EnsureCode{91}{12}% [
\TMP@EnsureCode{93}{12}% ]
\TMP@EnsureCode{94}{7}% ^
\TMP@EnsureCode{96}{12}% `
\edef\ifpdf@AtEnd{\ifpdf@AtEnd\noexpand\endinput}
%    \end{macrocode}
%
% \subsection{Check for previously defined \cs{ifpdf}}
%
%    \begin{macrocode}
\begingroup
  \expandafter\ifx\csname ifpdf\endcsname\relax
  \else
    \edef\i/{\expandafter\string\csname ifpdf\endcsname}%
    \expandafter\ifx\csname PackageError\endcsname\relax
      \def\x#1#2{%
        \edef\z{#2}%
        \expandafter\errhelp\expandafter{\z}%
        \errmessage{Package ifpdf Error: #1}%
      }%
      \def\y{^^J}%
      \newlinechar=10 %
    \else
      \def\x#1#2{%
        \PackageError{ifpdf}{#1}{#2}%
      }%
      \def\y{\MessageBreak}%
    \fi
    \x{Name clash, \i/ is already defined}{%
      Incompatible versions of \i/ can cause problems,\y
      therefore package loading is aborted.%
    }%
    \endgroup
    \expandafter\ifpdf@AtEnd
  \fi%
\endgroup
%    \end{macrocode}
%
% \subsection{\cs{pdfoutput} and \LuaTeX}
%
%    It might happen, that \LuaTeX\ is running, but \cs{pdfoutput}
%    does not exist. In version 0.40 only \cs{directlua}
%    is available at startup time. The enabling Lua function was
%    already added in version 0.36. Thus we can ignore older versions,
%    here \cs{pdfoutput} is available at startup time.
%
%    \begin{macrocode}
\begingroup
  \def\skip#1\relax\begingroup{}%
  \expandafter\ifx\csname pdfoutput\endcsname\relax
  \else
    \expandafter\skip
  \fi
  \expandafter\ifx\csname directlua\endcsname\relax
    \expandafter\skip
  \fi
\endgroup
\begingroup\expandafter\expandafter\expandafter\endgroup
\expandafter\ifx\csname RequirePackage\endcsname\relax
  \def\TMP@RequirePackage#1[#2]{%
    \begingroup\expandafter\expandafter\expandafter\endgroup
    \expandafter\ifx\csname ver@#1.sty\endcsname\relax
      \input #1.sty\relax
    \fi
  }%
  \TMP@RequirePackage{ifluatex}[2009/04/10]%
\else
  \RequirePackage{ifluatex}[2009/04/10]%
\fi
\ifluatex
  \ifnum\luatexversion<36 %
%    \end{macrocode}
%    Unhappily \LuaTeX's \cs{primitive} (derived from
%    pdfTeX's \cs{pdfprimitive}) cannot be used:
%    \begin{quote}
%    |\protected\gdef\pdfoutput{\primitive\pdfoutput}|
%    \end{quote}
%    Setting a value works, but getting fails,
%    because \TeX\ does no longer see it as number. It
%    is unexpandable and breaks numerical contexts.
%    This was fixed in \hologo{pdfTeX} 1.40.10
%    (bugfix \#4289: ``|\primitive\pdfoutput| cannot be queried'').
%    \begin{macrocode}
  \else
    \begingroup
      \directlua{tex.enableprimitives('ifpdf', {'pdfoutput'})}%
      \global\let\pdfoutput\ifpdfpdfoutput
    \endgroup
  \fi
\fi
\relax\begingroup\endgroup
%    \end{macrocode}
%
% \subsection{\cs{ifpdf}}
%
%    \begin{macro}{\ifpdf}
%    Create and set the switch. \cs{newif} initializes the
%    switch with \cs{iffalse}. \cs{newif} is \cs{outer}
%    in \plainTeX.
%    \begin{macrocode}
\begingroup\expandafter\expandafter\expandafter\endgroup
\expandafter\ifx\csname newif\endcsname\relax
  \edef\pdffalse{%
    \let
    \expandafter\noexpand\csname ifpdf\endcsname
    \expandafter\noexpand\csname iffalse\endcsname
  }%
  \edef\pdftrue{%
    \let
    \expandafter\noexpand\csname ifpdf\endcsname
    \expandafter\noexpand\csname iftrue\endcsname
  }%
  \pdffalse
\else
  \csname newif\expandafter\endcsname\csname ifpdf\endcsname
\fi
%    \end{macrocode}
%    Test \cs{pdfoutput}. Is it  defined and different from
%    \cs{relax}? Someone could have used \LaTeX\ internal
%    \cs{@ifundefined}, or something else involving.
%    Notice, \cs{csname} is executed inside a group for the test
%    to cancel the side effect of \cs{csname}.
%    \begin{macrocode}
\begingroup\expandafter\expandafter\expandafter\endgroup
\expandafter\ifx\csname pdfoutput\endcsname\relax
\else
  \ifnum\pdfoutput<1 %
%    \end{macrocode}
%    \cs{pdfoutput}|=0| or negative, so not generating pdf.
%    \begin{macrocode}
  \else
    \pdftrue
  \fi
\fi
%    \end{macrocode}
%    \end{macro}
%
% \subsection{Test for fool attempts}
%
%    \begin{macrocode}
\begingroup
  \expandafter\ifx\csname pdfoutput\endcsname\relax
  \else
    \escapechar=92 %
    \edef\m{\meaning\pdfoutput}%
    \edef\p{\string\pdfoutput}%
    \ifx\m\p
    \else
      \expandafter\ifx\csname PackageWarningNoLine\endcsname\relax
        \def\PackageWarningNoLine#1#2{%
          \immediate\write16{%
            Package `#1' Warning: #2.%
          }%
        }%
      \fi
      \PackageWarningNoLine{ifpdf}{%
        Someone has redefined \string\pdfoutput%
      }%
    \fi
  \fi
\endgroup
%    \end{macrocode}
%
% \subsection{Protocol entry}
%
%     Log comment:
%    \begin{macrocode}
\begingroup
  \expandafter\ifx\csname PackageInfo\endcsname\relax
    \def\x#1#2{%
      \immediate\write-1{Package #1 Info: #2.}%
    }%
  \else
    \let\x\PackageInfo
    \expandafter\let\csname on@line\endcsname\empty
  \fi
  \x{ifpdf}{pdfTeX in PDF mode is \ifpdf\else not \fi detected}%
\endgroup
%    \end{macrocode}
%    \begin{macrocode}
\ifpdf@AtEnd%
%    \end{macrocode}
%    \begin{macrocode}
%</package>
%    \end{macrocode}
%
% \section{Test}
%
% \subsection{Catcode checks for loading}
%
%    \begin{macrocode}
%<*test1>
%    \end{macrocode}
%    \begin{macrocode}
\catcode`\{=1 %
\catcode`\}=2 %
\catcode`\#=6 %
\catcode`\@=11 %
\expandafter\ifx\csname count@\endcsname\relax
  \countdef\count@=255 %
\fi
\expandafter\ifx\csname @gobble\endcsname\relax
  \long\def\@gobble#1{}%
\fi
\expandafter\ifx\csname @firstofone\endcsname\relax
  \long\def\@firstofone#1{#1}%
\fi
\expandafter\ifx\csname loop\endcsname\relax
  \expandafter\@firstofone
\else
  \expandafter\@gobble
\fi
{%
  \def\loop#1\repeat{%
    \def\body{#1}%
    \iterate
  }%
  \def\iterate{%
    \body
      \let\next\iterate
    \else
      \let\next\relax
    \fi
    \next
  }%
  \let\repeat=\fi
}%
\def\RestoreCatcodes{}
\count@=0 %
\loop
  \edef\RestoreCatcodes{%
    \RestoreCatcodes
    \catcode\the\count@=\the\catcode\count@\relax
  }%
\ifnum\count@<255 %
  \advance\count@ 1 %
\repeat

\def\RangeCatcodeInvalid#1#2{%
  \count@=#1\relax
  \loop
    \catcode\count@=15 %
  \ifnum\count@<#2\relax
    \advance\count@ 1 %
  \repeat
}
\def\RangeCatcodeCheck#1#2#3{%
  \count@=#1\relax
  \loop
    \ifnum#3=\catcode\count@
    \else
      \errmessage{%
        Character \the\count@\space
        with wrong catcode \the\catcode\count@\space
        instead of \number#3%
      }%
    \fi
  \ifnum\count@<#2\relax
    \advance\count@ 1 %
  \repeat
}
\def\space{ }
\expandafter\ifx\csname LoadCommand\endcsname\relax
  \def\LoadCommand{\input ifpdf.sty\relax}%
\fi
\def\Test{%
  \RangeCatcodeInvalid{0}{47}%
  \RangeCatcodeInvalid{58}{64}%
  \RangeCatcodeInvalid{91}{96}%
  \RangeCatcodeInvalid{123}{255}%
  \catcode`\@=12 %
  \catcode`\\=0 %
  \catcode`\%=14 %
  \LoadCommand
  \RangeCatcodeCheck{0}{36}{15}%
  \RangeCatcodeCheck{37}{37}{14}%
  \RangeCatcodeCheck{38}{47}{15}%
  \RangeCatcodeCheck{48}{57}{12}%
  \RangeCatcodeCheck{58}{63}{15}%
  \RangeCatcodeCheck{64}{64}{12}%
  \RangeCatcodeCheck{65}{90}{11}%
  \RangeCatcodeCheck{91}{91}{15}%
  \RangeCatcodeCheck{92}{92}{0}%
  \RangeCatcodeCheck{93}{96}{15}%
  \RangeCatcodeCheck{97}{122}{11}%
  \RangeCatcodeCheck{123}{255}{15}%
  \RestoreCatcodes
}
\Test
\csname @@end\endcsname
\end
%    \end{macrocode}
%    \begin{macrocode}
%</test1>
%    \end{macrocode}
%
% \section{Installation}
%
% \subsection{Download}
%
% \paragraph{Package.} This package is available on
% CTAN\footnote{\url{ftp://ftp.ctan.org/tex-archive/}}:
% \begin{description}
% \item[\CTAN{macros/latex/contrib/oberdiek/ifpdf.dtx}] The source file.
% \item[\CTAN{macros/latex/contrib/oberdiek/ifpdf.pdf}] Documentation.
% \end{description}
%
%
% \paragraph{Bundle.} All the packages of the bundle `oberdiek'
% are also available in a TDS compliant ZIP archive. There
% the packages are already unpacked and the documentation files
% are generated. The files and directories obey the TDS standard.
% \begin{description}
% \item[\CTAN{install/macros/latex/contrib/oberdiek.tds.zip}]
% \end{description}
% \emph{TDS} refers to the standard ``A Directory Structure
% for \TeX\ Files'' (\CTAN{tds/tds.pdf}). Directories
% with \xfile{texmf} in their name are usually organized this way.
%
% \subsection{Bundle installation}
%
% \paragraph{Unpacking.} Unpack the \xfile{oberdiek.tds.zip} in the
% TDS tree (also known as \xfile{texmf} tree) of your choice.
% Example (linux):
% \begin{quote}
%   |unzip oberdiek.tds.zip -d ~/texmf|
% \end{quote}
%
% \paragraph{Script installation.}
% Check the directory \xfile{TDS:scripts/oberdiek/} for
% scripts that need further installation steps.
% Package \xpackage{attachfile2} comes with the Perl script
% \xfile{pdfatfi.pl} that should be installed in such a way
% that it can be called as \texttt{pdfatfi}.
% Example (linux):
% \begin{quote}
%   |chmod +x scripts/oberdiek/pdfatfi.pl|\\
%   |cp scripts/oberdiek/pdfatfi.pl /usr/local/bin/|
% \end{quote}
%
% \subsection{Package installation}
%
% \paragraph{Unpacking.} The \xfile{.dtx} file is a self-extracting
% \docstrip\ archive. The files are extracted by running the
% \xfile{.dtx} through \plainTeX:
% \begin{quote}
%   \verb|tex ifpdf.dtx|
% \end{quote}
%
% \paragraph{TDS.} Now the different files must be moved into
% the different directories in your installation TDS tree
% (also known as \xfile{texmf} tree):
% \begin{quote}
% \def\t{^^A
% \begin{tabular}{@{}>{\ttfamily}l@{ $\rightarrow$ }>{\ttfamily}l@{}}
%   ifpdf.sty & tex/generic/oberdiek/ifpdf.sty\\
%   ifpdf.pdf & doc/latex/oberdiek/ifpdf.pdf\\
%   test/ifpdf-test1.tex & doc/latex/oberdiek/test/ifpdf-test1.tex\\
%   ifpdf.dtx & source/latex/oberdiek/ifpdf.dtx\\
% \end{tabular}^^A
% }^^A
% \sbox0{\t}^^A
% \ifdim\wd0>\linewidth
%   \begingroup
%     \advance\linewidth by\leftmargin
%     \advance\linewidth by\rightmargin
%   \edef\x{\endgroup
%     \def\noexpand\lw{\the\linewidth}^^A
%   }\x
%   \def\lwbox{^^A
%     \leavevmode
%     \hbox to \linewidth{^^A
%       \kern-\leftmargin\relax
%       \hss
%       \usebox0
%       \hss
%       \kern-\rightmargin\relax
%     }^^A
%   }^^A
%   \ifdim\wd0>\lw
%     \sbox0{\small\t}^^A
%     \ifdim\wd0>\linewidth
%       \ifdim\wd0>\lw
%         \sbox0{\footnotesize\t}^^A
%         \ifdim\wd0>\linewidth
%           \ifdim\wd0>\lw
%             \sbox0{\scriptsize\t}^^A
%             \ifdim\wd0>\linewidth
%               \ifdim\wd0>\lw
%                 \sbox0{\tiny\t}^^A
%                 \ifdim\wd0>\linewidth
%                   \lwbox
%                 \else
%                   \usebox0
%                 \fi
%               \else
%                 \lwbox
%               \fi
%             \else
%               \usebox0
%             \fi
%           \else
%             \lwbox
%           \fi
%         \else
%           \usebox0
%         \fi
%       \else
%         \lwbox
%       \fi
%     \else
%       \usebox0
%     \fi
%   \else
%     \lwbox
%   \fi
% \else
%   \usebox0
% \fi
% \end{quote}
% If you have a \xfile{docstrip.cfg} that configures and enables \docstrip's
% TDS installing feature, then some files can already be in the right
% place, see the documentation of \docstrip.
%
% \subsection{Refresh file name databases}
%
% If your \TeX~distribution
% (\teTeX, \mikTeX, \dots) relies on file name databases, you must refresh
% these. For example, \teTeX\ users run \verb|texhash| or
% \verb|mktexlsr|.
%
% \subsection{Some details for the interested}
%
% \paragraph{Attached source.}
%
% The PDF documentation on CTAN also includes the
% \xfile{.dtx} source file. It can be extracted by
% AcrobatReader 6 or higher. Another option is \textsf{pdftk},
% e.g. unpack the file into the current directory:
% \begin{quote}
%   \verb|pdftk ifpdf.pdf unpack_files output .|
% \end{quote}
%
% \paragraph{Unpacking with \LaTeX.}
% The \xfile{.dtx} chooses its action depending on the format:
% \begin{description}
% \item[\plainTeX:] Run \docstrip\ and extract the files.
% \item[\LaTeX:] Generate the documentation.
% \end{description}
% If you insist on using \LaTeX\ for \docstrip\ (really,
% \docstrip\ does not need \LaTeX), then inform the autodetect routine
% about your intention:
% \begin{quote}
%   \verb|latex \let\install=y\input{ifpdf.dtx}|
% \end{quote}
% Do not forget to quote the argument according to the demands
% of your shell.
%
% \paragraph{Generating the documentation.}
% You can use both the \xfile{.dtx} or the \xfile{.drv} to generate
% the documentation. The process can be configured by the
% configuration file \xfile{ltxdoc.cfg}. For instance, put this
% line into this file, if you want to have A4 as paper format:
% \begin{quote}
%   \verb|\PassOptionsToClass{a4paper}{article}|
% \end{quote}
% An example follows how to generate the
% documentation with pdf\LaTeX:
% \begin{quote}
%\begin{verbatim}
%pdflatex ifpdf.dtx
%makeindex -s gind.ist ifpdf.idx
%pdflatex ifpdf.dtx
%makeindex -s gind.ist ifpdf.idx
%pdflatex ifpdf.dtx
%\end{verbatim}
% \end{quote}
%
% \section{Catalogue}
%
% The following XML file can be used as source for the
% \href{http://mirror.ctan.org/help/Catalogue/catalogue.html}{\TeX\ Catalogue}.
% The elements \texttt{caption} and \texttt{description} are imported
% from the original XML file from the Catalogue.
% The name of the XML file in the Catalogue is \xfile{ifpdf.xml}.
%    \begin{macrocode}
%<*catalogue>
<?xml version='1.0' encoding='us-ascii'?>
<!DOCTYPE entry SYSTEM 'catalogue.dtd'>
<entry datestamp='$Date$' modifier='$Author$' id='ifpdf'>
  <name>ifpdf</name>
  <caption>Provides the \ifpdf conditional.</caption>
  <authorref id='auth:oberdiek'/>
  <copyright owner='Heiko Oberdiek' year='2001,2005-2011'/>
  <license type='lppl1.3'/>
  <version number='2.3'/>
  <description>
      This package looks for pdfTeX in pdf mode and implements and
      sets the switch <tt>\ifpdf</tt>. The detection is based on the
      value of <tt>\pdfoutput</tt> (which the package will not
      change).  The package works with plain or LaTeX formats.  To use
      it with LaTeX simply <tt>\usepackage{ifpdf}</tt>. Then use
      <tt>\ifpdf ... \else ... \fi</tt>.
    <p/>
    The package is part of the <xref refid='oberdiek'>oberdiek</xref> bundle.
  </description>
  <documentation details='Package documentation'
      href='ctan:/macros/latex/contrib/oberdiek/ifpdf.pdf'/>
  <ctan file='true' path='/macros/latex/contrib/oberdiek/ifpdf.dtx'/>
  <miktex location='oberdiek'/>
  <texlive location='oberdiek'/>
  <install path='/macros/latex/contrib/oberdiek/oberdiek.tds.zip'/>
</entry>
%</catalogue>
%    \end{macrocode}
%
% \begin{History}
%   \begin{Version}{2001/06/14 v1.0}
%   \item
%     First public version.
%   \end{Version}
%   \begin{Version}{2001/07/14 v1.1}
%   \item
%     Documentation addition: global options
%   \end{Version}
%   \begin{Version}{2001/09/26 v1.2}
%   \item
%     Documentation typo corrected.
%   \item
%     Version number corrected.
%   \item
%     Line number in log entry removed.
%   \end{Version}
%   \begin{Version}{2005/07/22 v1.3}
%   \item
%     Some source code comments from Robin Fairbairns added.
%   \item
%     Bug fix for negative values of \cs{pdfoutput}
%     (Oleg Katsitadze)
%   \item
%     LPPL 1.3
%   \item
%     Installation section with locations added.
%   \end{Version}
%   \begin{Version}{2006/02/20 v1.4}
%   \item
%     DTX framework.
%   \item
%     More robust check in case of undefined \cs{pdfoutput}.
%   \item
%     Extended documentation.
%   \end{Version}
%   \begin{Version}{2007/09/09 v1.5}
%   \item
%     Catcode settings added.
%   \end{Version}
%   \begin{Version}{2007/12/12 v1.6}
%   \item
%     Minor update.
%   \end{Version}
%   \begin{Version}{2008/12/12 v1.7}
%   \item
%     Fix in documentation for \cs{boolean} (found by S. Venkataraman).
%   \item
%     Code is not changed.
%   \end{Version}
%   \begin{Version}{2009/04/10 v2.0}
%   \item
%     Support for \LuaTeX\ 0.40 added.
%   \item
%     Checks, whether \cs{pdfoutput} was changed.
%   \end{Version}
%   \begin{Version}{2010/01/28 v2.1}
%   \item
%     Compatibility to ini\TeX\ added.
%   \end{Version}
%   \begin{Version}{2010/09/13 v2.2}
%   \item
%     Code with \cs{escapechar} rewritten because of
%     LuaTeX bug: \cs{escapechar} does not work.
%   \end{Version}
%   \begin{Version}{2011/01/30 v2.3}
%   \item
%     Already loaded package files are not input in \hologo{plainTeX}.
%   \end{Version}
% \end{History}
%
% \PrintIndex
%
% \Finale
\endinput

%        (quote the arguments according to the demands of your shell)
%
% Documentation:
%    (a) If ifpdf.drv is present:
%           latex ifpdf.drv
%    (b) Without ifpdf.drv:
%           latex ifpdf.dtx; ...
%    The class ltxdoc loads the configuration file ltxdoc.cfg
%    if available. Here you can specify further options, e.g.
%    use A4 as paper format:
%       \PassOptionsToClass{a4paper}{article}
%
%    Programm calls to get the documentation (example):
%       pdflatex ifpdf.dtx
%       makeindex -s gind.ist ifpdf.idx
%       pdflatex ifpdf.dtx
%       makeindex -s gind.ist ifpdf.idx
%       pdflatex ifpdf.dtx
%
% Installation:
%    TDS:tex/generic/oberdiek/ifpdf.sty
%    TDS:doc/latex/oberdiek/ifpdf.pdf
%    TDS:doc/latex/oberdiek/test/ifpdf-test1.tex
%    TDS:source/latex/oberdiek/ifpdf.dtx
%
%<*ignore>
\begingroup
  \catcode123=1 %
  \catcode125=2 %
  \def\x{LaTeX2e}%
\expandafter\endgroup
\ifcase 0\ifx\install y1\fi\expandafter
         \ifx\csname processbatchFile\endcsname\relax\else1\fi
         \ifx\fmtname\x\else 1\fi\relax
\else\csname fi\endcsname
%</ignore>
%<*install>
\input docstrip.tex
\Msg{************************************************************************}
\Msg{* Installation}
\Msg{* Package: ifpdf 2011/01/30 v2.3 Provides the ifpdf switch (HO)}
\Msg{************************************************************************}

\keepsilent
\askforoverwritefalse

\let\MetaPrefix\relax
\preamble

This is a generated file.

Project: ifpdf
Version: 2011/01/30 v2.3

Copyright (C) 2001, 2005-2011 by
   Heiko Oberdiek <heiko.oberdiek at googlemail.com>

This work may be distributed and/or modified under the
conditions of the LaTeX Project Public License, either
version 1.3c of this license or (at your option) any later
version. This version of this license is in
   http://www.latex-project.org/lppl/lppl-1-3c.txt
and the latest version of this license is in
   http://www.latex-project.org/lppl.txt
and version 1.3 or later is part of all distributions of
LaTeX version 2005/12/01 or later.

This work has the LPPL maintenance status "maintained".

This Current Maintainer of this work is Heiko Oberdiek.

The Base Interpreter refers to any `TeX-Format',
because some files are installed in TDS:tex/generic//.

This work consists of the main source file ifpdf.dtx
and the derived files
   ifpdf.sty, ifpdf.pdf, ifpdf.ins, ifpdf.drv, ifpdf-test1.tex.

\endpreamble
\let\MetaPrefix\DoubleperCent

\generate{%
  \file{ifpdf.ins}{\from{ifpdf.dtx}{install}}%
  \file{ifpdf.drv}{\from{ifpdf.dtx}{driver}}%
  \usedir{tex/generic/oberdiek}%
  \file{ifpdf.sty}{\from{ifpdf.dtx}{package}}%
  \usedir{doc/latex/oberdiek/test}%
  \file{ifpdf-test1.tex}{\from{ifpdf.dtx}{test1}}%
  \nopreamble
  \nopostamble
  \usedir{source/latex/oberdiek/catalogue}%
  \file{ifpdf.xml}{\from{ifpdf.dtx}{catalogue}}%
}

\catcode32=13\relax% active space
\let =\space%
\Msg{************************************************************************}
\Msg{*}
\Msg{* To finish the installation you have to move the following}
\Msg{* file into a directory searched by TeX:}
\Msg{*}
\Msg{*     ifpdf.sty}
\Msg{*}
\Msg{* To produce the documentation run the file `ifpdf.drv'}
\Msg{* through LaTeX.}
\Msg{*}
\Msg{* Happy TeXing!}
\Msg{*}
\Msg{************************************************************************}

\endbatchfile
%</install>
%<*ignore>
\fi
%</ignore>
%<*driver>
\NeedsTeXFormat{LaTeX2e}
\ProvidesFile{ifpdf.drv}%
  [2011/01/30 v2.3 Provides the ifpdf switch (HO)]%
\documentclass{ltxdoc}
\usepackage{holtxdoc}[2011/11/22]
\begin{document}
  \DocInput{ifpdf.dtx}%
\end{document}
%</driver>
% \fi
%
% \CheckSum{600}
%
% \CharacterTable
%  {Upper-case    \A\B\C\D\E\F\G\H\I\J\K\L\M\N\O\P\Q\R\S\T\U\V\W\X\Y\Z
%   Lower-case    \a\b\c\d\e\f\g\h\i\j\k\l\m\n\o\p\q\r\s\t\u\v\w\x\y\z
%   Digits        \0\1\2\3\4\5\6\7\8\9
%   Exclamation   \!     Double quote  \"     Hash (number) \#
%   Dollar        \$     Percent       \%     Ampersand     \&
%   Acute accent  \'     Left paren    \(     Right paren   \)
%   Asterisk      \*     Plus          \+     Comma         \,
%   Minus         \-     Point         \.     Solidus       \/
%   Colon         \:     Semicolon     \;     Less than     \<
%   Equals        \=     Greater than  \>     Question mark \?
%   Commercial at \@     Left bracket  \[     Backslash     \\
%   Right bracket \]     Circumflex    \^     Underscore    \_
%   Grave accent  \`     Left brace    \{     Vertical bar  \|
%   Right brace   \}     Tilde         \~}
%
% \GetFileInfo{ifpdf.drv}
%
% \title{The \xpackage{ifpdf} package}
% \date{2011/01/30 v2.3}
% \author{Heiko Oberdiek\\\xemail{heiko.oberdiek at googlemail.com}}
%
% \maketitle
%
% \begin{abstract}
% This package looks for \pdfTeX\ in pdf mode and
% implements and sets the switch \cs{ifpdf}.
% The detection is based on \cs{pdfoutput} and
% the package will not change this value.
% It works with plain or \LaTeX\ formats.
% \end{abstract}
%
% \tableofcontents
%
% \hypersetup{bookmarksopenlevel=2}
% \section{Documentation}
%
% \subsection{Introduction}
%
% It is commonly known that H\`an Th\^e Th\`anh's \pdfTeX\ generates
% PDF output directly and many people uses \pdfTeX\ for this purpose.
% However the DVI output was never thrown away. In contrary, he
% new features for typesetting that works in both PDF and DVI mode.
%
% In the meantime many \TeX\ distributions replace the traditional
% \TeX\ binary with \pdfTeX. Then, for example, called as |latex|
% \pdfTeX\ works in DVI mode with the \LaTeX\ format preloaded,
% called as |pdflatex| \pdfTeX\ starts in PDF mode.
%
% Often packages or users want to know, whether the current document
% is typset by \pdfTeX\ in PDF mode, because the different modes
% have different capabilities (color setting, graphics inclusion,
% \dots). For this purpose \pdfTeX's
% \cs{pdfoutput} can be asked.
%
% As regulary reader of \TeX\ newsgroups and mailing lists I could
% observe many problems with this task. Common errors are:
% \begin{itemize}
% \item
%   \pdfTeX\ has \emph{two} modes. Using \pdfTeX\ does not mean
%   that the user always want to have PDF mode. For example, the
%   PostScript support is better in DVI mode in conjunction with a
%   PostScript aware DVI driver (e.g. dvips). Also the additional
%   typesetting features are mode independent and also available
%   in DVI mode.
% \item
%   \LaTeX's \cs{@ifundefined} inherited the side effect from
%   \cs{csname}. Unknown commands are defined with the meaning
%   of \cs{relax}. If it is checked, whether \cs{pdfoutput} is
%   defined, then this should not be forgotten.
% \item
%   Having \cs{pdfoutput} does not automatically mean PDF mode.
%   Also the value of \cs{pdfoutput} must be asked.
% \item
%   \cs{pdfoutput} must not be destroyed in some way. Later
%   code and packages are fooled then and will perhaps make
%   wrong decisions. For example they may drop support for
%   PDF mode, because they do not know that \pdfTeX\ is
%   running at all.
% \end{itemize}
%
% Robin Fairbairns provides an entry for this topic in his
% excellent FAQ (\url{http://www.tex.ac.uk/faq}):
% \href{http://www.tex.ac.uk/cgi-bin/texfaq2html?label=ifpdf}^^A
% {Am I using PDFTeX?}
%
% \subsection{Usage}
%
% The package \xpackage{ifpdf} can be used with both \plainTeX\
% and \LaTeX:
% \begin{description}
% \item[\plainTeX:] |\input ifpdf.sty|
% \item[\LaTeXe:]   |\usepackage{ifpdf}|
% \end{description}
%
% \DescribeMacro{\ifpdf}
% The package provides the switch \cs{ifpdf}:
% \begin{quote}
%   |\ifpdf|\\
%   \hspace*{1.5em}\dots\ do things, if \pdfTeX\ is running in pdf mode \dots\\
%   |\else|\\
%   \hspace*{1.5em}\dots\ other \TeX\ or \pdfTeX\ in dvi mode \dots\\
%   |\fi|
% \end{quote}
% Users of the package \xpackage{ifthen} can use the switch as boolean:
% \begin{quote}
%   |\boolean{pdf}|
% \end{quote}
% The package can also be used to set global
% documentclass options:
% \begin{quote}
%   |\RequirePackage{ifpdf}|\\
%   |\ifpdf|\\
%   \hspace*{1.5em}|\documentclass[pdftex,|\dots|]{|\dots|}|\\
%   |\else|\\
%   \hspace*{1.5em}|\documentclass[|\dots|]{|\dots|}|\\
%   |\fi|
% \end{quote}
%
% \subsection{Specification}
%
% The package have the following properties:
% \begin{itemize}
% \item It asks the setting of \cs{pdfoutput} for detecting
%   \pdfTeX\ in PDF mode.
% \item It never changes \cs{pdfoutput}.
% \item If \cs{pdfoutput} is undefined or has the meaning \cs{relax},
%   but the engine provides the primitive \cs{pdfoutput},
%   then \cs{pdfoutput} is enabled or restored if possible
%   (only \LuaTeX, version 0.36.0 or higher).
% \item It can be used with many formats including
%   \plainTeX\ and \LaTeX.
% \end{itemize}
% The mode detection implements the following algorithm:
% \begin{quote}
% \def\={${:}{=}$}
% \def\>{\null\qquad\ignorespaces}
% \def\comment#1{^^A
%   \textrm{\itshape\% #1}^^A
% }
% \sffamily
% if undefined(\cs{pdfoutput})\\
% \> \cs{ifpdf} \= \cs{iffalse} \comment{\pdfTeX is not running}\\
% else\\
% \> if \cs{pdfoutput} $\leq$ 0\\
% \>\> \cs{ifpdf} \= \cs{iffalse} \comment{\pdfTeX\ in DVI mode}\\
% \> else\\
% \>\> \cs{ifpdf} \= \cs{iftrue} \comment{\pdfTeX\ in PDF mode}\\
% \> fi\\
% fi
% \end{quote}
% The function \textsf{undefined} checks both cases, undefined command
% and \cs{relax}.
%
% \hypersetup{bookmarksopenlevel=1}
%
% \StopEventually{
% }
%
% \section{Implementation}
%
%    \begin{macrocode}
%<*package>
%    \end{macrocode}
%
% \subsection{Reload check and package identification}
%    Reload check, especially if the package is not used with \LaTeX.
%    \begin{macrocode}
\begingroup\catcode61\catcode48\catcode32=10\relax%
  \catcode13=5 % ^^M
  \endlinechar=13 %
  \catcode35=6 % #
  \catcode39=12 % '
  \catcode44=12 % ,
  \catcode45=12 % -
  \catcode46=12 % .
  \catcode58=12 % :
  \catcode64=11 % @
  \catcode123=1 % {
  \catcode125=2 % }
  \expandafter\let\expandafter\x\csname ver@ifpdf.sty\endcsname
  \ifx\x\relax % plain-TeX, first loading
  \else
    \def\empty{}%
    \ifx\x\empty % LaTeX, first loading,
      % variable is initialized, but \ProvidesPackage not yet seen
    \else
      \expandafter\ifx\csname PackageInfo\endcsname\relax
        \def\x#1#2{%
          \immediate\write-1{Package #1 Info: #2.}%
        }%
      \else
        \def\x#1#2{\PackageInfo{#1}{#2, stopped}}%
      \fi
      \x{ifpdf}{The package is already loaded}%
      \aftergroup\endinput
    \fi
  \fi
\endgroup%
%    \end{macrocode}
%    Package identification:
%    \begin{macrocode}
\begingroup\catcode61\catcode48\catcode32=10\relax%
  \catcode13=5 % ^^M
  \endlinechar=13 %
  \catcode35=6 % #
  \catcode39=12 % '
  \catcode40=12 % (
  \catcode41=12 % )
  \catcode44=12 % ,
  \catcode45=12 % -
  \catcode46=12 % .
  \catcode47=12 % /
  \catcode58=12 % :
  \catcode64=11 % @
  \catcode91=12 % [
  \catcode93=12 % ]
  \catcode123=1 % {
  \catcode125=2 % }
  \expandafter\ifx\csname ProvidesPackage\endcsname\relax
    \def\x#1#2#3[#4]{\endgroup
      \immediate\write-1{Package: #3 #4}%
      \xdef#1{#4}%
    }%
  \else
    \def\x#1#2[#3]{\endgroup
      #2[{#3}]%
      \ifx#1\@undefined
        \xdef#1{#3}%
      \fi
      \ifx#1\relax
        \xdef#1{#3}%
      \fi
    }%
  \fi
\expandafter\x\csname ver@ifpdf.sty\endcsname
\ProvidesPackage{ifpdf}%
  [2011/01/30 v2.3 Provides the ifpdf switch (HO)]%
%    \end{macrocode}
%
% \subsection{Catcodes}
%
%    \begin{macrocode}
\begingroup\catcode61\catcode48\catcode32=10\relax%
  \catcode13=5 % ^^M
  \endlinechar=13 %
  \catcode123=1 % {
  \catcode125=2 % }
  \catcode64=11 % @
  \def\x{\endgroup
    \expandafter\edef\csname ifpdf@AtEnd\endcsname{%
      \endlinechar=\the\endlinechar\relax
      \catcode13=\the\catcode13\relax
      \catcode32=\the\catcode32\relax
      \catcode35=\the\catcode35\relax
      \catcode61=\the\catcode61\relax
      \catcode64=\the\catcode64\relax
      \catcode123=\the\catcode123\relax
      \catcode125=\the\catcode125\relax
    }%
  }%
\x\catcode61\catcode48\catcode32=10\relax%
\catcode13=5 % ^^M
\endlinechar=13 %
\catcode35=6 % #
\catcode64=11 % @
\catcode123=1 % {
\catcode125=2 % }
\def\TMP@EnsureCode#1#2{%
  \edef\ifpdf@AtEnd{%
    \ifpdf@AtEnd
    \catcode#1=\the\catcode#1\relax
  }%
  \catcode#1=#2\relax
}
\TMP@EnsureCode{10}{12}% ^^J
\TMP@EnsureCode{39}{12}% '
\TMP@EnsureCode{40}{12}% (
\TMP@EnsureCode{41}{12}% )
\TMP@EnsureCode{44}{12}% ,
\TMP@EnsureCode{45}{12}% -
\TMP@EnsureCode{46}{12}% .
\TMP@EnsureCode{47}{12}% /
\TMP@EnsureCode{58}{12}% :
\TMP@EnsureCode{60}{12}% <
\TMP@EnsureCode{91}{12}% [
\TMP@EnsureCode{93}{12}% ]
\TMP@EnsureCode{94}{7}% ^
\TMP@EnsureCode{96}{12}% `
\edef\ifpdf@AtEnd{\ifpdf@AtEnd\noexpand\endinput}
%    \end{macrocode}
%
% \subsection{Check for previously defined \cs{ifpdf}}
%
%    \begin{macrocode}
\begingroup
  \expandafter\ifx\csname ifpdf\endcsname\relax
  \else
    \edef\i/{\expandafter\string\csname ifpdf\endcsname}%
    \expandafter\ifx\csname PackageError\endcsname\relax
      \def\x#1#2{%
        \edef\z{#2}%
        \expandafter\errhelp\expandafter{\z}%
        \errmessage{Package ifpdf Error: #1}%
      }%
      \def\y{^^J}%
      \newlinechar=10 %
    \else
      \def\x#1#2{%
        \PackageError{ifpdf}{#1}{#2}%
      }%
      \def\y{\MessageBreak}%
    \fi
    \x{Name clash, \i/ is already defined}{%
      Incompatible versions of \i/ can cause problems,\y
      therefore package loading is aborted.%
    }%
    \endgroup
    \expandafter\ifpdf@AtEnd
  \fi%
\endgroup
%    \end{macrocode}
%
% \subsection{\cs{pdfoutput} and \LuaTeX}
%
%    It might happen, that \LuaTeX\ is running, but \cs{pdfoutput}
%    does not exist. In version 0.40 only \cs{directlua}
%    is available at startup time. The enabling Lua function was
%    already added in version 0.36. Thus we can ignore older versions,
%    here \cs{pdfoutput} is available at startup time.
%
%    \begin{macrocode}
\begingroup
  \def\skip#1\relax\begingroup{}%
  \expandafter\ifx\csname pdfoutput\endcsname\relax
  \else
    \expandafter\skip
  \fi
  \expandafter\ifx\csname directlua\endcsname\relax
    \expandafter\skip
  \fi
\endgroup
\begingroup\expandafter\expandafter\expandafter\endgroup
\expandafter\ifx\csname RequirePackage\endcsname\relax
  \def\TMP@RequirePackage#1[#2]{%
    \begingroup\expandafter\expandafter\expandafter\endgroup
    \expandafter\ifx\csname ver@#1.sty\endcsname\relax
      \input #1.sty\relax
    \fi
  }%
  \TMP@RequirePackage{ifluatex}[2009/04/10]%
\else
  \RequirePackage{ifluatex}[2009/04/10]%
\fi
\ifluatex
  \ifnum\luatexversion<36 %
%    \end{macrocode}
%    Unhappily \LuaTeX's \cs{primitive} (derived from
%    pdfTeX's \cs{pdfprimitive}) cannot be used:
%    \begin{quote}
%    |\protected\gdef\pdfoutput{\primitive\pdfoutput}|
%    \end{quote}
%    Setting a value works, but getting fails,
%    because \TeX\ does no longer see it as number. It
%    is unexpandable and breaks numerical contexts.
%    This was fixed in \hologo{pdfTeX} 1.40.10
%    (bugfix \#4289: ``|\primitive\pdfoutput| cannot be queried'').
%    \begin{macrocode}
  \else
    \begingroup
      \directlua{tex.enableprimitives('ifpdf', {'pdfoutput'})}%
      \global\let\pdfoutput\ifpdfpdfoutput
    \endgroup
  \fi
\fi
\relax\begingroup\endgroup
%    \end{macrocode}
%
% \subsection{\cs{ifpdf}}
%
%    \begin{macro}{\ifpdf}
%    Create and set the switch. \cs{newif} initializes the
%    switch with \cs{iffalse}. \cs{newif} is \cs{outer}
%    in \plainTeX.
%    \begin{macrocode}
\begingroup\expandafter\expandafter\expandafter\endgroup
\expandafter\ifx\csname newif\endcsname\relax
  \edef\pdffalse{%
    \let
    \expandafter\noexpand\csname ifpdf\endcsname
    \expandafter\noexpand\csname iffalse\endcsname
  }%
  \edef\pdftrue{%
    \let
    \expandafter\noexpand\csname ifpdf\endcsname
    \expandafter\noexpand\csname iftrue\endcsname
  }%
  \pdffalse
\else
  \csname newif\expandafter\endcsname\csname ifpdf\endcsname
\fi
%    \end{macrocode}
%    Test \cs{pdfoutput}. Is it  defined and different from
%    \cs{relax}? Someone could have used \LaTeX\ internal
%    \cs{@ifundefined}, or something else involving.
%    Notice, \cs{csname} is executed inside a group for the test
%    to cancel the side effect of \cs{csname}.
%    \begin{macrocode}
\begingroup\expandafter\expandafter\expandafter\endgroup
\expandafter\ifx\csname pdfoutput\endcsname\relax
\else
  \ifnum\pdfoutput<1 %
%    \end{macrocode}
%    \cs{pdfoutput}|=0| or negative, so not generating pdf.
%    \begin{macrocode}
  \else
    \pdftrue
  \fi
\fi
%    \end{macrocode}
%    \end{macro}
%
% \subsection{Test for fool attempts}
%
%    \begin{macrocode}
\begingroup
  \expandafter\ifx\csname pdfoutput\endcsname\relax
  \else
    \escapechar=92 %
    \edef\m{\meaning\pdfoutput}%
    \edef\p{\string\pdfoutput}%
    \ifx\m\p
    \else
      \expandafter\ifx\csname PackageWarningNoLine\endcsname\relax
        \def\PackageWarningNoLine#1#2{%
          \immediate\write16{%
            Package `#1' Warning: #2.%
          }%
        }%
      \fi
      \PackageWarningNoLine{ifpdf}{%
        Someone has redefined \string\pdfoutput%
      }%
    \fi
  \fi
\endgroup
%    \end{macrocode}
%
% \subsection{Protocol entry}
%
%     Log comment:
%    \begin{macrocode}
\begingroup
  \expandafter\ifx\csname PackageInfo\endcsname\relax
    \def\x#1#2{%
      \immediate\write-1{Package #1 Info: #2.}%
    }%
  \else
    \let\x\PackageInfo
    \expandafter\let\csname on@line\endcsname\empty
  \fi
  \x{ifpdf}{pdfTeX in PDF mode is \ifpdf\else not \fi detected}%
\endgroup
%    \end{macrocode}
%    \begin{macrocode}
\ifpdf@AtEnd%
%    \end{macrocode}
%    \begin{macrocode}
%</package>
%    \end{macrocode}
%
% \section{Test}
%
% \subsection{Catcode checks for loading}
%
%    \begin{macrocode}
%<*test1>
%    \end{macrocode}
%    \begin{macrocode}
\catcode`\{=1 %
\catcode`\}=2 %
\catcode`\#=6 %
\catcode`\@=11 %
\expandafter\ifx\csname count@\endcsname\relax
  \countdef\count@=255 %
\fi
\expandafter\ifx\csname @gobble\endcsname\relax
  \long\def\@gobble#1{}%
\fi
\expandafter\ifx\csname @firstofone\endcsname\relax
  \long\def\@firstofone#1{#1}%
\fi
\expandafter\ifx\csname loop\endcsname\relax
  \expandafter\@firstofone
\else
  \expandafter\@gobble
\fi
{%
  \def\loop#1\repeat{%
    \def\body{#1}%
    \iterate
  }%
  \def\iterate{%
    \body
      \let\next\iterate
    \else
      \let\next\relax
    \fi
    \next
  }%
  \let\repeat=\fi
}%
\def\RestoreCatcodes{}
\count@=0 %
\loop
  \edef\RestoreCatcodes{%
    \RestoreCatcodes
    \catcode\the\count@=\the\catcode\count@\relax
  }%
\ifnum\count@<255 %
  \advance\count@ 1 %
\repeat

\def\RangeCatcodeInvalid#1#2{%
  \count@=#1\relax
  \loop
    \catcode\count@=15 %
  \ifnum\count@<#2\relax
    \advance\count@ 1 %
  \repeat
}
\def\RangeCatcodeCheck#1#2#3{%
  \count@=#1\relax
  \loop
    \ifnum#3=\catcode\count@
    \else
      \errmessage{%
        Character \the\count@\space
        with wrong catcode \the\catcode\count@\space
        instead of \number#3%
      }%
    \fi
  \ifnum\count@<#2\relax
    \advance\count@ 1 %
  \repeat
}
\def\space{ }
\expandafter\ifx\csname LoadCommand\endcsname\relax
  \def\LoadCommand{\input ifpdf.sty\relax}%
\fi
\def\Test{%
  \RangeCatcodeInvalid{0}{47}%
  \RangeCatcodeInvalid{58}{64}%
  \RangeCatcodeInvalid{91}{96}%
  \RangeCatcodeInvalid{123}{255}%
  \catcode`\@=12 %
  \catcode`\\=0 %
  \catcode`\%=14 %
  \LoadCommand
  \RangeCatcodeCheck{0}{36}{15}%
  \RangeCatcodeCheck{37}{37}{14}%
  \RangeCatcodeCheck{38}{47}{15}%
  \RangeCatcodeCheck{48}{57}{12}%
  \RangeCatcodeCheck{58}{63}{15}%
  \RangeCatcodeCheck{64}{64}{12}%
  \RangeCatcodeCheck{65}{90}{11}%
  \RangeCatcodeCheck{91}{91}{15}%
  \RangeCatcodeCheck{92}{92}{0}%
  \RangeCatcodeCheck{93}{96}{15}%
  \RangeCatcodeCheck{97}{122}{11}%
  \RangeCatcodeCheck{123}{255}{15}%
  \RestoreCatcodes
}
\Test
\csname @@end\endcsname
\end
%    \end{macrocode}
%    \begin{macrocode}
%</test1>
%    \end{macrocode}
%
% \section{Installation}
%
% \subsection{Download}
%
% \paragraph{Package.} This package is available on
% CTAN\footnote{\url{ftp://ftp.ctan.org/tex-archive/}}:
% \begin{description}
% \item[\CTAN{macros/latex/contrib/oberdiek/ifpdf.dtx}] The source file.
% \item[\CTAN{macros/latex/contrib/oberdiek/ifpdf.pdf}] Documentation.
% \end{description}
%
%
% \paragraph{Bundle.} All the packages of the bundle `oberdiek'
% are also available in a TDS compliant ZIP archive. There
% the packages are already unpacked and the documentation files
% are generated. The files and directories obey the TDS standard.
% \begin{description}
% \item[\CTAN{install/macros/latex/contrib/oberdiek.tds.zip}]
% \end{description}
% \emph{TDS} refers to the standard ``A Directory Structure
% for \TeX\ Files'' (\CTAN{tds/tds.pdf}). Directories
% with \xfile{texmf} in their name are usually organized this way.
%
% \subsection{Bundle installation}
%
% \paragraph{Unpacking.} Unpack the \xfile{oberdiek.tds.zip} in the
% TDS tree (also known as \xfile{texmf} tree) of your choice.
% Example (linux):
% \begin{quote}
%   |unzip oberdiek.tds.zip -d ~/texmf|
% \end{quote}
%
% \paragraph{Script installation.}
% Check the directory \xfile{TDS:scripts/oberdiek/} for
% scripts that need further installation steps.
% Package \xpackage{attachfile2} comes with the Perl script
% \xfile{pdfatfi.pl} that should be installed in such a way
% that it can be called as \texttt{pdfatfi}.
% Example (linux):
% \begin{quote}
%   |chmod +x scripts/oberdiek/pdfatfi.pl|\\
%   |cp scripts/oberdiek/pdfatfi.pl /usr/local/bin/|
% \end{quote}
%
% \subsection{Package installation}
%
% \paragraph{Unpacking.} The \xfile{.dtx} file is a self-extracting
% \docstrip\ archive. The files are extracted by running the
% \xfile{.dtx} through \plainTeX:
% \begin{quote}
%   \verb|tex ifpdf.dtx|
% \end{quote}
%
% \paragraph{TDS.} Now the different files must be moved into
% the different directories in your installation TDS tree
% (also known as \xfile{texmf} tree):
% \begin{quote}
% \def\t{^^A
% \begin{tabular}{@{}>{\ttfamily}l@{ $\rightarrow$ }>{\ttfamily}l@{}}
%   ifpdf.sty & tex/generic/oberdiek/ifpdf.sty\\
%   ifpdf.pdf & doc/latex/oberdiek/ifpdf.pdf\\
%   test/ifpdf-test1.tex & doc/latex/oberdiek/test/ifpdf-test1.tex\\
%   ifpdf.dtx & source/latex/oberdiek/ifpdf.dtx\\
% \end{tabular}^^A
% }^^A
% \sbox0{\t}^^A
% \ifdim\wd0>\linewidth
%   \begingroup
%     \advance\linewidth by\leftmargin
%     \advance\linewidth by\rightmargin
%   \edef\x{\endgroup
%     \def\noexpand\lw{\the\linewidth}^^A
%   }\x
%   \def\lwbox{^^A
%     \leavevmode
%     \hbox to \linewidth{^^A
%       \kern-\leftmargin\relax
%       \hss
%       \usebox0
%       \hss
%       \kern-\rightmargin\relax
%     }^^A
%   }^^A
%   \ifdim\wd0>\lw
%     \sbox0{\small\t}^^A
%     \ifdim\wd0>\linewidth
%       \ifdim\wd0>\lw
%         \sbox0{\footnotesize\t}^^A
%         \ifdim\wd0>\linewidth
%           \ifdim\wd0>\lw
%             \sbox0{\scriptsize\t}^^A
%             \ifdim\wd0>\linewidth
%               \ifdim\wd0>\lw
%                 \sbox0{\tiny\t}^^A
%                 \ifdim\wd0>\linewidth
%                   \lwbox
%                 \else
%                   \usebox0
%                 \fi
%               \else
%                 \lwbox
%               \fi
%             \else
%               \usebox0
%             \fi
%           \else
%             \lwbox
%           \fi
%         \else
%           \usebox0
%         \fi
%       \else
%         \lwbox
%       \fi
%     \else
%       \usebox0
%     \fi
%   \else
%     \lwbox
%   \fi
% \else
%   \usebox0
% \fi
% \end{quote}
% If you have a \xfile{docstrip.cfg} that configures and enables \docstrip's
% TDS installing feature, then some files can already be in the right
% place, see the documentation of \docstrip.
%
% \subsection{Refresh file name databases}
%
% If your \TeX~distribution
% (\teTeX, \mikTeX, \dots) relies on file name databases, you must refresh
% these. For example, \teTeX\ users run \verb|texhash| or
% \verb|mktexlsr|.
%
% \subsection{Some details for the interested}
%
% \paragraph{Attached source.}
%
% The PDF documentation on CTAN also includes the
% \xfile{.dtx} source file. It can be extracted by
% AcrobatReader 6 or higher. Another option is \textsf{pdftk},
% e.g. unpack the file into the current directory:
% \begin{quote}
%   \verb|pdftk ifpdf.pdf unpack_files output .|
% \end{quote}
%
% \paragraph{Unpacking with \LaTeX.}
% The \xfile{.dtx} chooses its action depending on the format:
% \begin{description}
% \item[\plainTeX:] Run \docstrip\ and extract the files.
% \item[\LaTeX:] Generate the documentation.
% \end{description}
% If you insist on using \LaTeX\ for \docstrip\ (really,
% \docstrip\ does not need \LaTeX), then inform the autodetect routine
% about your intention:
% \begin{quote}
%   \verb|latex \let\install=y% \iffalse meta-comment
%
% File: ifpdf.dtx
% Version: 2011/01/30 v2.3
% Info: Provides the ifpdf switch
%
% Copyright (C) 2001, 2005-2011 by
%    Heiko Oberdiek <heiko.oberdiek at googlemail.com>
%
% This work may be distributed and/or modified under the
% conditions of the LaTeX Project Public License, either
% version 1.3c of this license or (at your option) any later
% version. This version of this license is in
%    http://www.latex-project.org/lppl/lppl-1-3c.txt
% and the latest version of this license is in
%    http://www.latex-project.org/lppl.txt
% and version 1.3 or later is part of all distributions of
% LaTeX version 2005/12/01 or later.
%
% This work has the LPPL maintenance status "maintained".
%
% This Current Maintainer of this work is Heiko Oberdiek.
%
% The Base Interpreter refers to any `TeX-Format',
% because some files are installed in TDS:tex/generic//.
%
% This work consists of the main source file ifpdf.dtx
% and the derived files
%    ifpdf.sty, ifpdf.pdf, ifpdf.ins, ifpdf.drv, ifpdf-test1.tex.
%
% Distribution:
%    CTAN:macros/latex/contrib/oberdiek/ifpdf.dtx
%    CTAN:macros/latex/contrib/oberdiek/ifpdf.pdf
%
% Unpacking:
%    (a) If ifpdf.ins is present:
%           tex ifpdf.ins
%    (b) Without ifpdf.ins:
%           tex ifpdf.dtx
%    (c) If you insist on using LaTeX
%           latex \let\install=y\input{ifpdf.dtx}
%        (quote the arguments according to the demands of your shell)
%
% Documentation:
%    (a) If ifpdf.drv is present:
%           latex ifpdf.drv
%    (b) Without ifpdf.drv:
%           latex ifpdf.dtx; ...
%    The class ltxdoc loads the configuration file ltxdoc.cfg
%    if available. Here you can specify further options, e.g.
%    use A4 as paper format:
%       \PassOptionsToClass{a4paper}{article}
%
%    Programm calls to get the documentation (example):
%       pdflatex ifpdf.dtx
%       makeindex -s gind.ist ifpdf.idx
%       pdflatex ifpdf.dtx
%       makeindex -s gind.ist ifpdf.idx
%       pdflatex ifpdf.dtx
%
% Installation:
%    TDS:tex/generic/oberdiek/ifpdf.sty
%    TDS:doc/latex/oberdiek/ifpdf.pdf
%    TDS:doc/latex/oberdiek/test/ifpdf-test1.tex
%    TDS:source/latex/oberdiek/ifpdf.dtx
%
%<*ignore>
\begingroup
  \catcode123=1 %
  \catcode125=2 %
  \def\x{LaTeX2e}%
\expandafter\endgroup
\ifcase 0\ifx\install y1\fi\expandafter
         \ifx\csname processbatchFile\endcsname\relax\else1\fi
         \ifx\fmtname\x\else 1\fi\relax
\else\csname fi\endcsname
%</ignore>
%<*install>
\input docstrip.tex
\Msg{************************************************************************}
\Msg{* Installation}
\Msg{* Package: ifpdf 2011/01/30 v2.3 Provides the ifpdf switch (HO)}
\Msg{************************************************************************}

\keepsilent
\askforoverwritefalse

\let\MetaPrefix\relax
\preamble

This is a generated file.

Project: ifpdf
Version: 2011/01/30 v2.3

Copyright (C) 2001, 2005-2011 by
   Heiko Oberdiek <heiko.oberdiek at googlemail.com>

This work may be distributed and/or modified under the
conditions of the LaTeX Project Public License, either
version 1.3c of this license or (at your option) any later
version. This version of this license is in
   http://www.latex-project.org/lppl/lppl-1-3c.txt
and the latest version of this license is in
   http://www.latex-project.org/lppl.txt
and version 1.3 or later is part of all distributions of
LaTeX version 2005/12/01 or later.

This work has the LPPL maintenance status "maintained".

This Current Maintainer of this work is Heiko Oberdiek.

The Base Interpreter refers to any `TeX-Format',
because some files are installed in TDS:tex/generic//.

This work consists of the main source file ifpdf.dtx
and the derived files
   ifpdf.sty, ifpdf.pdf, ifpdf.ins, ifpdf.drv, ifpdf-test1.tex.

\endpreamble
\let\MetaPrefix\DoubleperCent

\generate{%
  \file{ifpdf.ins}{\from{ifpdf.dtx}{install}}%
  \file{ifpdf.drv}{\from{ifpdf.dtx}{driver}}%
  \usedir{tex/generic/oberdiek}%
  \file{ifpdf.sty}{\from{ifpdf.dtx}{package}}%
  \usedir{doc/latex/oberdiek/test}%
  \file{ifpdf-test1.tex}{\from{ifpdf.dtx}{test1}}%
  \nopreamble
  \nopostamble
  \usedir{source/latex/oberdiek/catalogue}%
  \file{ifpdf.xml}{\from{ifpdf.dtx}{catalogue}}%
}

\catcode32=13\relax% active space
\let =\space%
\Msg{************************************************************************}
\Msg{*}
\Msg{* To finish the installation you have to move the following}
\Msg{* file into a directory searched by TeX:}
\Msg{*}
\Msg{*     ifpdf.sty}
\Msg{*}
\Msg{* To produce the documentation run the file `ifpdf.drv'}
\Msg{* through LaTeX.}
\Msg{*}
\Msg{* Happy TeXing!}
\Msg{*}
\Msg{************************************************************************}

\endbatchfile
%</install>
%<*ignore>
\fi
%</ignore>
%<*driver>
\NeedsTeXFormat{LaTeX2e}
\ProvidesFile{ifpdf.drv}%
  [2011/01/30 v2.3 Provides the ifpdf switch (HO)]%
\documentclass{ltxdoc}
\usepackage{holtxdoc}[2011/11/22]
\begin{document}
  \DocInput{ifpdf.dtx}%
\end{document}
%</driver>
% \fi
%
% \CheckSum{600}
%
% \CharacterTable
%  {Upper-case    \A\B\C\D\E\F\G\H\I\J\K\L\M\N\O\P\Q\R\S\T\U\V\W\X\Y\Z
%   Lower-case    \a\b\c\d\e\f\g\h\i\j\k\l\m\n\o\p\q\r\s\t\u\v\w\x\y\z
%   Digits        \0\1\2\3\4\5\6\7\8\9
%   Exclamation   \!     Double quote  \"     Hash (number) \#
%   Dollar        \$     Percent       \%     Ampersand     \&
%   Acute accent  \'     Left paren    \(     Right paren   \)
%   Asterisk      \*     Plus          \+     Comma         \,
%   Minus         \-     Point         \.     Solidus       \/
%   Colon         \:     Semicolon     \;     Less than     \<
%   Equals        \=     Greater than  \>     Question mark \?
%   Commercial at \@     Left bracket  \[     Backslash     \\
%   Right bracket \]     Circumflex    \^     Underscore    \_
%   Grave accent  \`     Left brace    \{     Vertical bar  \|
%   Right brace   \}     Tilde         \~}
%
% \GetFileInfo{ifpdf.drv}
%
% \title{The \xpackage{ifpdf} package}
% \date{2011/01/30 v2.3}
% \author{Heiko Oberdiek\\\xemail{heiko.oberdiek at googlemail.com}}
%
% \maketitle
%
% \begin{abstract}
% This package looks for \pdfTeX\ in pdf mode and
% implements and sets the switch \cs{ifpdf}.
% The detection is based on \cs{pdfoutput} and
% the package will not change this value.
% It works with plain or \LaTeX\ formats.
% \end{abstract}
%
% \tableofcontents
%
% \hypersetup{bookmarksopenlevel=2}
% \section{Documentation}
%
% \subsection{Introduction}
%
% It is commonly known that H\`an Th\^e Th\`anh's \pdfTeX\ generates
% PDF output directly and many people uses \pdfTeX\ for this purpose.
% However the DVI output was never thrown away. In contrary, he
% new features for typesetting that works in both PDF and DVI mode.
%
% In the meantime many \TeX\ distributions replace the traditional
% \TeX\ binary with \pdfTeX. Then, for example, called as |latex|
% \pdfTeX\ works in DVI mode with the \LaTeX\ format preloaded,
% called as |pdflatex| \pdfTeX\ starts in PDF mode.
%
% Often packages or users want to know, whether the current document
% is typset by \pdfTeX\ in PDF mode, because the different modes
% have different capabilities (color setting, graphics inclusion,
% \dots). For this purpose \pdfTeX's
% \cs{pdfoutput} can be asked.
%
% As regulary reader of \TeX\ newsgroups and mailing lists I could
% observe many problems with this task. Common errors are:
% \begin{itemize}
% \item
%   \pdfTeX\ has \emph{two} modes. Using \pdfTeX\ does not mean
%   that the user always want to have PDF mode. For example, the
%   PostScript support is better in DVI mode in conjunction with a
%   PostScript aware DVI driver (e.g. dvips). Also the additional
%   typesetting features are mode independent and also available
%   in DVI mode.
% \item
%   \LaTeX's \cs{@ifundefined} inherited the side effect from
%   \cs{csname}. Unknown commands are defined with the meaning
%   of \cs{relax}. If it is checked, whether \cs{pdfoutput} is
%   defined, then this should not be forgotten.
% \item
%   Having \cs{pdfoutput} does not automatically mean PDF mode.
%   Also the value of \cs{pdfoutput} must be asked.
% \item
%   \cs{pdfoutput} must not be destroyed in some way. Later
%   code and packages are fooled then and will perhaps make
%   wrong decisions. For example they may drop support for
%   PDF mode, because they do not know that \pdfTeX\ is
%   running at all.
% \end{itemize}
%
% Robin Fairbairns provides an entry for this topic in his
% excellent FAQ (\url{http://www.tex.ac.uk/faq}):
% \href{http://www.tex.ac.uk/cgi-bin/texfaq2html?label=ifpdf}^^A
% {Am I using PDFTeX?}
%
% \subsection{Usage}
%
% The package \xpackage{ifpdf} can be used with both \plainTeX\
% and \LaTeX:
% \begin{description}
% \item[\plainTeX:] |\input ifpdf.sty|
% \item[\LaTeXe:]   |\usepackage{ifpdf}|
% \end{description}
%
% \DescribeMacro{\ifpdf}
% The package provides the switch \cs{ifpdf}:
% \begin{quote}
%   |\ifpdf|\\
%   \hspace*{1.5em}\dots\ do things, if \pdfTeX\ is running in pdf mode \dots\\
%   |\else|\\
%   \hspace*{1.5em}\dots\ other \TeX\ or \pdfTeX\ in dvi mode \dots\\
%   |\fi|
% \end{quote}
% Users of the package \xpackage{ifthen} can use the switch as boolean:
% \begin{quote}
%   |\boolean{pdf}|
% \end{quote}
% The package can also be used to set global
% documentclass options:
% \begin{quote}
%   |\RequirePackage{ifpdf}|\\
%   |\ifpdf|\\
%   \hspace*{1.5em}|\documentclass[pdftex,|\dots|]{|\dots|}|\\
%   |\else|\\
%   \hspace*{1.5em}|\documentclass[|\dots|]{|\dots|}|\\
%   |\fi|
% \end{quote}
%
% \subsection{Specification}
%
% The package have the following properties:
% \begin{itemize}
% \item It asks the setting of \cs{pdfoutput} for detecting
%   \pdfTeX\ in PDF mode.
% \item It never changes \cs{pdfoutput}.
% \item If \cs{pdfoutput} is undefined or has the meaning \cs{relax},
%   but the engine provides the primitive \cs{pdfoutput},
%   then \cs{pdfoutput} is enabled or restored if possible
%   (only \LuaTeX, version 0.36.0 or higher).
% \item It can be used with many formats including
%   \plainTeX\ and \LaTeX.
% \end{itemize}
% The mode detection implements the following algorithm:
% \begin{quote}
% \def\={${:}{=}$}
% \def\>{\null\qquad\ignorespaces}
% \def\comment#1{^^A
%   \textrm{\itshape\% #1}^^A
% }
% \sffamily
% if undefined(\cs{pdfoutput})\\
% \> \cs{ifpdf} \= \cs{iffalse} \comment{\pdfTeX is not running}\\
% else\\
% \> if \cs{pdfoutput} $\leq$ 0\\
% \>\> \cs{ifpdf} \= \cs{iffalse} \comment{\pdfTeX\ in DVI mode}\\
% \> else\\
% \>\> \cs{ifpdf} \= \cs{iftrue} \comment{\pdfTeX\ in PDF mode}\\
% \> fi\\
% fi
% \end{quote}
% The function \textsf{undefined} checks both cases, undefined command
% and \cs{relax}.
%
% \hypersetup{bookmarksopenlevel=1}
%
% \StopEventually{
% }
%
% \section{Implementation}
%
%    \begin{macrocode}
%<*package>
%    \end{macrocode}
%
% \subsection{Reload check and package identification}
%    Reload check, especially if the package is not used with \LaTeX.
%    \begin{macrocode}
\begingroup\catcode61\catcode48\catcode32=10\relax%
  \catcode13=5 % ^^M
  \endlinechar=13 %
  \catcode35=6 % #
  \catcode39=12 % '
  \catcode44=12 % ,
  \catcode45=12 % -
  \catcode46=12 % .
  \catcode58=12 % :
  \catcode64=11 % @
  \catcode123=1 % {
  \catcode125=2 % }
  \expandafter\let\expandafter\x\csname ver@ifpdf.sty\endcsname
  \ifx\x\relax % plain-TeX, first loading
  \else
    \def\empty{}%
    \ifx\x\empty % LaTeX, first loading,
      % variable is initialized, but \ProvidesPackage not yet seen
    \else
      \expandafter\ifx\csname PackageInfo\endcsname\relax
        \def\x#1#2{%
          \immediate\write-1{Package #1 Info: #2.}%
        }%
      \else
        \def\x#1#2{\PackageInfo{#1}{#2, stopped}}%
      \fi
      \x{ifpdf}{The package is already loaded}%
      \aftergroup\endinput
    \fi
  \fi
\endgroup%
%    \end{macrocode}
%    Package identification:
%    \begin{macrocode}
\begingroup\catcode61\catcode48\catcode32=10\relax%
  \catcode13=5 % ^^M
  \endlinechar=13 %
  \catcode35=6 % #
  \catcode39=12 % '
  \catcode40=12 % (
  \catcode41=12 % )
  \catcode44=12 % ,
  \catcode45=12 % -
  \catcode46=12 % .
  \catcode47=12 % /
  \catcode58=12 % :
  \catcode64=11 % @
  \catcode91=12 % [
  \catcode93=12 % ]
  \catcode123=1 % {
  \catcode125=2 % }
  \expandafter\ifx\csname ProvidesPackage\endcsname\relax
    \def\x#1#2#3[#4]{\endgroup
      \immediate\write-1{Package: #3 #4}%
      \xdef#1{#4}%
    }%
  \else
    \def\x#1#2[#3]{\endgroup
      #2[{#3}]%
      \ifx#1\@undefined
        \xdef#1{#3}%
      \fi
      \ifx#1\relax
        \xdef#1{#3}%
      \fi
    }%
  \fi
\expandafter\x\csname ver@ifpdf.sty\endcsname
\ProvidesPackage{ifpdf}%
  [2011/01/30 v2.3 Provides the ifpdf switch (HO)]%
%    \end{macrocode}
%
% \subsection{Catcodes}
%
%    \begin{macrocode}
\begingroup\catcode61\catcode48\catcode32=10\relax%
  \catcode13=5 % ^^M
  \endlinechar=13 %
  \catcode123=1 % {
  \catcode125=2 % }
  \catcode64=11 % @
  \def\x{\endgroup
    \expandafter\edef\csname ifpdf@AtEnd\endcsname{%
      \endlinechar=\the\endlinechar\relax
      \catcode13=\the\catcode13\relax
      \catcode32=\the\catcode32\relax
      \catcode35=\the\catcode35\relax
      \catcode61=\the\catcode61\relax
      \catcode64=\the\catcode64\relax
      \catcode123=\the\catcode123\relax
      \catcode125=\the\catcode125\relax
    }%
  }%
\x\catcode61\catcode48\catcode32=10\relax%
\catcode13=5 % ^^M
\endlinechar=13 %
\catcode35=6 % #
\catcode64=11 % @
\catcode123=1 % {
\catcode125=2 % }
\def\TMP@EnsureCode#1#2{%
  \edef\ifpdf@AtEnd{%
    \ifpdf@AtEnd
    \catcode#1=\the\catcode#1\relax
  }%
  \catcode#1=#2\relax
}
\TMP@EnsureCode{10}{12}% ^^J
\TMP@EnsureCode{39}{12}% '
\TMP@EnsureCode{40}{12}% (
\TMP@EnsureCode{41}{12}% )
\TMP@EnsureCode{44}{12}% ,
\TMP@EnsureCode{45}{12}% -
\TMP@EnsureCode{46}{12}% .
\TMP@EnsureCode{47}{12}% /
\TMP@EnsureCode{58}{12}% :
\TMP@EnsureCode{60}{12}% <
\TMP@EnsureCode{91}{12}% [
\TMP@EnsureCode{93}{12}% ]
\TMP@EnsureCode{94}{7}% ^
\TMP@EnsureCode{96}{12}% `
\edef\ifpdf@AtEnd{\ifpdf@AtEnd\noexpand\endinput}
%    \end{macrocode}
%
% \subsection{Check for previously defined \cs{ifpdf}}
%
%    \begin{macrocode}
\begingroup
  \expandafter\ifx\csname ifpdf\endcsname\relax
  \else
    \edef\i/{\expandafter\string\csname ifpdf\endcsname}%
    \expandafter\ifx\csname PackageError\endcsname\relax
      \def\x#1#2{%
        \edef\z{#2}%
        \expandafter\errhelp\expandafter{\z}%
        \errmessage{Package ifpdf Error: #1}%
      }%
      \def\y{^^J}%
      \newlinechar=10 %
    \else
      \def\x#1#2{%
        \PackageError{ifpdf}{#1}{#2}%
      }%
      \def\y{\MessageBreak}%
    \fi
    \x{Name clash, \i/ is already defined}{%
      Incompatible versions of \i/ can cause problems,\y
      therefore package loading is aborted.%
    }%
    \endgroup
    \expandafter\ifpdf@AtEnd
  \fi%
\endgroup
%    \end{macrocode}
%
% \subsection{\cs{pdfoutput} and \LuaTeX}
%
%    It might happen, that \LuaTeX\ is running, but \cs{pdfoutput}
%    does not exist. In version 0.40 only \cs{directlua}
%    is available at startup time. The enabling Lua function was
%    already added in version 0.36. Thus we can ignore older versions,
%    here \cs{pdfoutput} is available at startup time.
%
%    \begin{macrocode}
\begingroup
  \def\skip#1\relax\begingroup{}%
  \expandafter\ifx\csname pdfoutput\endcsname\relax
  \else
    \expandafter\skip
  \fi
  \expandafter\ifx\csname directlua\endcsname\relax
    \expandafter\skip
  \fi
\endgroup
\begingroup\expandafter\expandafter\expandafter\endgroup
\expandafter\ifx\csname RequirePackage\endcsname\relax
  \def\TMP@RequirePackage#1[#2]{%
    \begingroup\expandafter\expandafter\expandafter\endgroup
    \expandafter\ifx\csname ver@#1.sty\endcsname\relax
      \input #1.sty\relax
    \fi
  }%
  \TMP@RequirePackage{ifluatex}[2009/04/10]%
\else
  \RequirePackage{ifluatex}[2009/04/10]%
\fi
\ifluatex
  \ifnum\luatexversion<36 %
%    \end{macrocode}
%    Unhappily \LuaTeX's \cs{primitive} (derived from
%    pdfTeX's \cs{pdfprimitive}) cannot be used:
%    \begin{quote}
%    |\protected\gdef\pdfoutput{\primitive\pdfoutput}|
%    \end{quote}
%    Setting a value works, but getting fails,
%    because \TeX\ does no longer see it as number. It
%    is unexpandable and breaks numerical contexts.
%    This was fixed in \hologo{pdfTeX} 1.40.10
%    (bugfix \#4289: ``|\primitive\pdfoutput| cannot be queried'').
%    \begin{macrocode}
  \else
    \begingroup
      \directlua{tex.enableprimitives('ifpdf', {'pdfoutput'})}%
      \global\let\pdfoutput\ifpdfpdfoutput
    \endgroup
  \fi
\fi
\relax\begingroup\endgroup
%    \end{macrocode}
%
% \subsection{\cs{ifpdf}}
%
%    \begin{macro}{\ifpdf}
%    Create and set the switch. \cs{newif} initializes the
%    switch with \cs{iffalse}. \cs{newif} is \cs{outer}
%    in \plainTeX.
%    \begin{macrocode}
\begingroup\expandafter\expandafter\expandafter\endgroup
\expandafter\ifx\csname newif\endcsname\relax
  \edef\pdffalse{%
    \let
    \expandafter\noexpand\csname ifpdf\endcsname
    \expandafter\noexpand\csname iffalse\endcsname
  }%
  \edef\pdftrue{%
    \let
    \expandafter\noexpand\csname ifpdf\endcsname
    \expandafter\noexpand\csname iftrue\endcsname
  }%
  \pdffalse
\else
  \csname newif\expandafter\endcsname\csname ifpdf\endcsname
\fi
%    \end{macrocode}
%    Test \cs{pdfoutput}. Is it  defined and different from
%    \cs{relax}? Someone could have used \LaTeX\ internal
%    \cs{@ifundefined}, or something else involving.
%    Notice, \cs{csname} is executed inside a group for the test
%    to cancel the side effect of \cs{csname}.
%    \begin{macrocode}
\begingroup\expandafter\expandafter\expandafter\endgroup
\expandafter\ifx\csname pdfoutput\endcsname\relax
\else
  \ifnum\pdfoutput<1 %
%    \end{macrocode}
%    \cs{pdfoutput}|=0| or negative, so not generating pdf.
%    \begin{macrocode}
  \else
    \pdftrue
  \fi
\fi
%    \end{macrocode}
%    \end{macro}
%
% \subsection{Test for fool attempts}
%
%    \begin{macrocode}
\begingroup
  \expandafter\ifx\csname pdfoutput\endcsname\relax
  \else
    \escapechar=92 %
    \edef\m{\meaning\pdfoutput}%
    \edef\p{\string\pdfoutput}%
    \ifx\m\p
    \else
      \expandafter\ifx\csname PackageWarningNoLine\endcsname\relax
        \def\PackageWarningNoLine#1#2{%
          \immediate\write16{%
            Package `#1' Warning: #2.%
          }%
        }%
      \fi
      \PackageWarningNoLine{ifpdf}{%
        Someone has redefined \string\pdfoutput%
      }%
    \fi
  \fi
\endgroup
%    \end{macrocode}
%
% \subsection{Protocol entry}
%
%     Log comment:
%    \begin{macrocode}
\begingroup
  \expandafter\ifx\csname PackageInfo\endcsname\relax
    \def\x#1#2{%
      \immediate\write-1{Package #1 Info: #2.}%
    }%
  \else
    \let\x\PackageInfo
    \expandafter\let\csname on@line\endcsname\empty
  \fi
  \x{ifpdf}{pdfTeX in PDF mode is \ifpdf\else not \fi detected}%
\endgroup
%    \end{macrocode}
%    \begin{macrocode}
\ifpdf@AtEnd%
%    \end{macrocode}
%    \begin{macrocode}
%</package>
%    \end{macrocode}
%
% \section{Test}
%
% \subsection{Catcode checks for loading}
%
%    \begin{macrocode}
%<*test1>
%    \end{macrocode}
%    \begin{macrocode}
\catcode`\{=1 %
\catcode`\}=2 %
\catcode`\#=6 %
\catcode`\@=11 %
\expandafter\ifx\csname count@\endcsname\relax
  \countdef\count@=255 %
\fi
\expandafter\ifx\csname @gobble\endcsname\relax
  \long\def\@gobble#1{}%
\fi
\expandafter\ifx\csname @firstofone\endcsname\relax
  \long\def\@firstofone#1{#1}%
\fi
\expandafter\ifx\csname loop\endcsname\relax
  \expandafter\@firstofone
\else
  \expandafter\@gobble
\fi
{%
  \def\loop#1\repeat{%
    \def\body{#1}%
    \iterate
  }%
  \def\iterate{%
    \body
      \let\next\iterate
    \else
      \let\next\relax
    \fi
    \next
  }%
  \let\repeat=\fi
}%
\def\RestoreCatcodes{}
\count@=0 %
\loop
  \edef\RestoreCatcodes{%
    \RestoreCatcodes
    \catcode\the\count@=\the\catcode\count@\relax
  }%
\ifnum\count@<255 %
  \advance\count@ 1 %
\repeat

\def\RangeCatcodeInvalid#1#2{%
  \count@=#1\relax
  \loop
    \catcode\count@=15 %
  \ifnum\count@<#2\relax
    \advance\count@ 1 %
  \repeat
}
\def\RangeCatcodeCheck#1#2#3{%
  \count@=#1\relax
  \loop
    \ifnum#3=\catcode\count@
    \else
      \errmessage{%
        Character \the\count@\space
        with wrong catcode \the\catcode\count@\space
        instead of \number#3%
      }%
    \fi
  \ifnum\count@<#2\relax
    \advance\count@ 1 %
  \repeat
}
\def\space{ }
\expandafter\ifx\csname LoadCommand\endcsname\relax
  \def\LoadCommand{\input ifpdf.sty\relax}%
\fi
\def\Test{%
  \RangeCatcodeInvalid{0}{47}%
  \RangeCatcodeInvalid{58}{64}%
  \RangeCatcodeInvalid{91}{96}%
  \RangeCatcodeInvalid{123}{255}%
  \catcode`\@=12 %
  \catcode`\\=0 %
  \catcode`\%=14 %
  \LoadCommand
  \RangeCatcodeCheck{0}{36}{15}%
  \RangeCatcodeCheck{37}{37}{14}%
  \RangeCatcodeCheck{38}{47}{15}%
  \RangeCatcodeCheck{48}{57}{12}%
  \RangeCatcodeCheck{58}{63}{15}%
  \RangeCatcodeCheck{64}{64}{12}%
  \RangeCatcodeCheck{65}{90}{11}%
  \RangeCatcodeCheck{91}{91}{15}%
  \RangeCatcodeCheck{92}{92}{0}%
  \RangeCatcodeCheck{93}{96}{15}%
  \RangeCatcodeCheck{97}{122}{11}%
  \RangeCatcodeCheck{123}{255}{15}%
  \RestoreCatcodes
}
\Test
\csname @@end\endcsname
\end
%    \end{macrocode}
%    \begin{macrocode}
%</test1>
%    \end{macrocode}
%
% \section{Installation}
%
% \subsection{Download}
%
% \paragraph{Package.} This package is available on
% CTAN\footnote{\url{ftp://ftp.ctan.org/tex-archive/}}:
% \begin{description}
% \item[\CTAN{macros/latex/contrib/oberdiek/ifpdf.dtx}] The source file.
% \item[\CTAN{macros/latex/contrib/oberdiek/ifpdf.pdf}] Documentation.
% \end{description}
%
%
% \paragraph{Bundle.} All the packages of the bundle `oberdiek'
% are also available in a TDS compliant ZIP archive. There
% the packages are already unpacked and the documentation files
% are generated. The files and directories obey the TDS standard.
% \begin{description}
% \item[\CTAN{install/macros/latex/contrib/oberdiek.tds.zip}]
% \end{description}
% \emph{TDS} refers to the standard ``A Directory Structure
% for \TeX\ Files'' (\CTAN{tds/tds.pdf}). Directories
% with \xfile{texmf} in their name are usually organized this way.
%
% \subsection{Bundle installation}
%
% \paragraph{Unpacking.} Unpack the \xfile{oberdiek.tds.zip} in the
% TDS tree (also known as \xfile{texmf} tree) of your choice.
% Example (linux):
% \begin{quote}
%   |unzip oberdiek.tds.zip -d ~/texmf|
% \end{quote}
%
% \paragraph{Script installation.}
% Check the directory \xfile{TDS:scripts/oberdiek/} for
% scripts that need further installation steps.
% Package \xpackage{attachfile2} comes with the Perl script
% \xfile{pdfatfi.pl} that should be installed in such a way
% that it can be called as \texttt{pdfatfi}.
% Example (linux):
% \begin{quote}
%   |chmod +x scripts/oberdiek/pdfatfi.pl|\\
%   |cp scripts/oberdiek/pdfatfi.pl /usr/local/bin/|
% \end{quote}
%
% \subsection{Package installation}
%
% \paragraph{Unpacking.} The \xfile{.dtx} file is a self-extracting
% \docstrip\ archive. The files are extracted by running the
% \xfile{.dtx} through \plainTeX:
% \begin{quote}
%   \verb|tex ifpdf.dtx|
% \end{quote}
%
% \paragraph{TDS.} Now the different files must be moved into
% the different directories in your installation TDS tree
% (also known as \xfile{texmf} tree):
% \begin{quote}
% \def\t{^^A
% \begin{tabular}{@{}>{\ttfamily}l@{ $\rightarrow$ }>{\ttfamily}l@{}}
%   ifpdf.sty & tex/generic/oberdiek/ifpdf.sty\\
%   ifpdf.pdf & doc/latex/oberdiek/ifpdf.pdf\\
%   test/ifpdf-test1.tex & doc/latex/oberdiek/test/ifpdf-test1.tex\\
%   ifpdf.dtx & source/latex/oberdiek/ifpdf.dtx\\
% \end{tabular}^^A
% }^^A
% \sbox0{\t}^^A
% \ifdim\wd0>\linewidth
%   \begingroup
%     \advance\linewidth by\leftmargin
%     \advance\linewidth by\rightmargin
%   \edef\x{\endgroup
%     \def\noexpand\lw{\the\linewidth}^^A
%   }\x
%   \def\lwbox{^^A
%     \leavevmode
%     \hbox to \linewidth{^^A
%       \kern-\leftmargin\relax
%       \hss
%       \usebox0
%       \hss
%       \kern-\rightmargin\relax
%     }^^A
%   }^^A
%   \ifdim\wd0>\lw
%     \sbox0{\small\t}^^A
%     \ifdim\wd0>\linewidth
%       \ifdim\wd0>\lw
%         \sbox0{\footnotesize\t}^^A
%         \ifdim\wd0>\linewidth
%           \ifdim\wd0>\lw
%             \sbox0{\scriptsize\t}^^A
%             \ifdim\wd0>\linewidth
%               \ifdim\wd0>\lw
%                 \sbox0{\tiny\t}^^A
%                 \ifdim\wd0>\linewidth
%                   \lwbox
%                 \else
%                   \usebox0
%                 \fi
%               \else
%                 \lwbox
%               \fi
%             \else
%               \usebox0
%             \fi
%           \else
%             \lwbox
%           \fi
%         \else
%           \usebox0
%         \fi
%       \else
%         \lwbox
%       \fi
%     \else
%       \usebox0
%     \fi
%   \else
%     \lwbox
%   \fi
% \else
%   \usebox0
% \fi
% \end{quote}
% If you have a \xfile{docstrip.cfg} that configures and enables \docstrip's
% TDS installing feature, then some files can already be in the right
% place, see the documentation of \docstrip.
%
% \subsection{Refresh file name databases}
%
% If your \TeX~distribution
% (\teTeX, \mikTeX, \dots) relies on file name databases, you must refresh
% these. For example, \teTeX\ users run \verb|texhash| or
% \verb|mktexlsr|.
%
% \subsection{Some details for the interested}
%
% \paragraph{Attached source.}
%
% The PDF documentation on CTAN also includes the
% \xfile{.dtx} source file. It can be extracted by
% AcrobatReader 6 or higher. Another option is \textsf{pdftk},
% e.g. unpack the file into the current directory:
% \begin{quote}
%   \verb|pdftk ifpdf.pdf unpack_files output .|
% \end{quote}
%
% \paragraph{Unpacking with \LaTeX.}
% The \xfile{.dtx} chooses its action depending on the format:
% \begin{description}
% \item[\plainTeX:] Run \docstrip\ and extract the files.
% \item[\LaTeX:] Generate the documentation.
% \end{description}
% If you insist on using \LaTeX\ for \docstrip\ (really,
% \docstrip\ does not need \LaTeX), then inform the autodetect routine
% about your intention:
% \begin{quote}
%   \verb|latex \let\install=y\input{ifpdf.dtx}|
% \end{quote}
% Do not forget to quote the argument according to the demands
% of your shell.
%
% \paragraph{Generating the documentation.}
% You can use both the \xfile{.dtx} or the \xfile{.drv} to generate
% the documentation. The process can be configured by the
% configuration file \xfile{ltxdoc.cfg}. For instance, put this
% line into this file, if you want to have A4 as paper format:
% \begin{quote}
%   \verb|\PassOptionsToClass{a4paper}{article}|
% \end{quote}
% An example follows how to generate the
% documentation with pdf\LaTeX:
% \begin{quote}
%\begin{verbatim}
%pdflatex ifpdf.dtx
%makeindex -s gind.ist ifpdf.idx
%pdflatex ifpdf.dtx
%makeindex -s gind.ist ifpdf.idx
%pdflatex ifpdf.dtx
%\end{verbatim}
% \end{quote}
%
% \section{Catalogue}
%
% The following XML file can be used as source for the
% \href{http://mirror.ctan.org/help/Catalogue/catalogue.html}{\TeX\ Catalogue}.
% The elements \texttt{caption} and \texttt{description} are imported
% from the original XML file from the Catalogue.
% The name of the XML file in the Catalogue is \xfile{ifpdf.xml}.
%    \begin{macrocode}
%<*catalogue>
<?xml version='1.0' encoding='us-ascii'?>
<!DOCTYPE entry SYSTEM 'catalogue.dtd'>
<entry datestamp='$Date$' modifier='$Author$' id='ifpdf'>
  <name>ifpdf</name>
  <caption>Provides the \ifpdf conditional.</caption>
  <authorref id='auth:oberdiek'/>
  <copyright owner='Heiko Oberdiek' year='2001,2005-2011'/>
  <license type='lppl1.3'/>
  <version number='2.3'/>
  <description>
      This package looks for pdfTeX in pdf mode and implements and
      sets the switch <tt>\ifpdf</tt>. The detection is based on the
      value of <tt>\pdfoutput</tt> (which the package will not
      change).  The package works with plain or LaTeX formats.  To use
      it with LaTeX simply <tt>\usepackage{ifpdf}</tt>. Then use
      <tt>\ifpdf ... \else ... \fi</tt>.
    <p/>
    The package is part of the <xref refid='oberdiek'>oberdiek</xref> bundle.
  </description>
  <documentation details='Package documentation'
      href='ctan:/macros/latex/contrib/oberdiek/ifpdf.pdf'/>
  <ctan file='true' path='/macros/latex/contrib/oberdiek/ifpdf.dtx'/>
  <miktex location='oberdiek'/>
  <texlive location='oberdiek'/>
  <install path='/macros/latex/contrib/oberdiek/oberdiek.tds.zip'/>
</entry>
%</catalogue>
%    \end{macrocode}
%
% \begin{History}
%   \begin{Version}{2001/06/14 v1.0}
%   \item
%     First public version.
%   \end{Version}
%   \begin{Version}{2001/07/14 v1.1}
%   \item
%     Documentation addition: global options
%   \end{Version}
%   \begin{Version}{2001/09/26 v1.2}
%   \item
%     Documentation typo corrected.
%   \item
%     Version number corrected.
%   \item
%     Line number in log entry removed.
%   \end{Version}
%   \begin{Version}{2005/07/22 v1.3}
%   \item
%     Some source code comments from Robin Fairbairns added.
%   \item
%     Bug fix for negative values of \cs{pdfoutput}
%     (Oleg Katsitadze)
%   \item
%     LPPL 1.3
%   \item
%     Installation section with locations added.
%   \end{Version}
%   \begin{Version}{2006/02/20 v1.4}
%   \item
%     DTX framework.
%   \item
%     More robust check in case of undefined \cs{pdfoutput}.
%   \item
%     Extended documentation.
%   \end{Version}
%   \begin{Version}{2007/09/09 v1.5}
%   \item
%     Catcode settings added.
%   \end{Version}
%   \begin{Version}{2007/12/12 v1.6}
%   \item
%     Minor update.
%   \end{Version}
%   \begin{Version}{2008/12/12 v1.7}
%   \item
%     Fix in documentation for \cs{boolean} (found by S. Venkataraman).
%   \item
%     Code is not changed.
%   \end{Version}
%   \begin{Version}{2009/04/10 v2.0}
%   \item
%     Support for \LuaTeX\ 0.40 added.
%   \item
%     Checks, whether \cs{pdfoutput} was changed.
%   \end{Version}
%   \begin{Version}{2010/01/28 v2.1}
%   \item
%     Compatibility to ini\TeX\ added.
%   \end{Version}
%   \begin{Version}{2010/09/13 v2.2}
%   \item
%     Code with \cs{escapechar} rewritten because of
%     LuaTeX bug: \cs{escapechar} does not work.
%   \end{Version}
%   \begin{Version}{2011/01/30 v2.3}
%   \item
%     Already loaded package files are not input in \hologo{plainTeX}.
%   \end{Version}
% \end{History}
%
% \PrintIndex
%
% \Finale
\endinput
|
% \end{quote}
% Do not forget to quote the argument according to the demands
% of your shell.
%
% \paragraph{Generating the documentation.}
% You can use both the \xfile{.dtx} or the \xfile{.drv} to generate
% the documentation. The process can be configured by the
% configuration file \xfile{ltxdoc.cfg}. For instance, put this
% line into this file, if you want to have A4 as paper format:
% \begin{quote}
%   \verb|\PassOptionsToClass{a4paper}{article}|
% \end{quote}
% An example follows how to generate the
% documentation with pdf\LaTeX:
% \begin{quote}
%\begin{verbatim}
%pdflatex ifpdf.dtx
%makeindex -s gind.ist ifpdf.idx
%pdflatex ifpdf.dtx
%makeindex -s gind.ist ifpdf.idx
%pdflatex ifpdf.dtx
%\end{verbatim}
% \end{quote}
%
% \section{Catalogue}
%
% The following XML file can be used as source for the
% \href{http://mirror.ctan.org/help/Catalogue/catalogue.html}{\TeX\ Catalogue}.
% The elements \texttt{caption} and \texttt{description} are imported
% from the original XML file from the Catalogue.
% The name of the XML file in the Catalogue is \xfile{ifpdf.xml}.
%    \begin{macrocode}
%<*catalogue>
<?xml version='1.0' encoding='us-ascii'?>
<!DOCTYPE entry SYSTEM 'catalogue.dtd'>
<entry datestamp='$Date$' modifier='$Author$' id='ifpdf'>
  <name>ifpdf</name>
  <caption>Provides the \ifpdf conditional.</caption>
  <authorref id='auth:oberdiek'/>
  <copyright owner='Heiko Oberdiek' year='2001,2005-2011'/>
  <license type='lppl1.3'/>
  <version number='2.3'/>
  <description>
      This package looks for pdfTeX in pdf mode and implements and
      sets the switch <tt>\ifpdf</tt>. The detection is based on the
      value of <tt>\pdfoutput</tt> (which the package will not
      change).  The package works with plain or LaTeX formats.  To use
      it with LaTeX simply <tt>\usepackage{ifpdf}</tt>. Then use
      <tt>\ifpdf ... \else ... \fi</tt>.
    <p/>
    The package is part of the <xref refid='oberdiek'>oberdiek</xref> bundle.
  </description>
  <documentation details='Package documentation'
      href='ctan:/macros/latex/contrib/oberdiek/ifpdf.pdf'/>
  <ctan file='true' path='/macros/latex/contrib/oberdiek/ifpdf.dtx'/>
  <miktex location='oberdiek'/>
  <texlive location='oberdiek'/>
  <install path='/macros/latex/contrib/oberdiek/oberdiek.tds.zip'/>
</entry>
%</catalogue>
%    \end{macrocode}
%
% \begin{History}
%   \begin{Version}{2001/06/14 v1.0}
%   \item
%     First public version.
%   \end{Version}
%   \begin{Version}{2001/07/14 v1.1}
%   \item
%     Documentation addition: global options
%   \end{Version}
%   \begin{Version}{2001/09/26 v1.2}
%   \item
%     Documentation typo corrected.
%   \item
%     Version number corrected.
%   \item
%     Line number in log entry removed.
%   \end{Version}
%   \begin{Version}{2005/07/22 v1.3}
%   \item
%     Some source code comments from Robin Fairbairns added.
%   \item
%     Bug fix for negative values of \cs{pdfoutput}
%     (Oleg Katsitadze)
%   \item
%     LPPL 1.3
%   \item
%     Installation section with locations added.
%   \end{Version}
%   \begin{Version}{2006/02/20 v1.4}
%   \item
%     DTX framework.
%   \item
%     More robust check in case of undefined \cs{pdfoutput}.
%   \item
%     Extended documentation.
%   \end{Version}
%   \begin{Version}{2007/09/09 v1.5}
%   \item
%     Catcode settings added.
%   \end{Version}
%   \begin{Version}{2007/12/12 v1.6}
%   \item
%     Minor update.
%   \end{Version}
%   \begin{Version}{2008/12/12 v1.7}
%   \item
%     Fix in documentation for \cs{boolean} (found by S. Venkataraman).
%   \item
%     Code is not changed.
%   \end{Version}
%   \begin{Version}{2009/04/10 v2.0}
%   \item
%     Support for \LuaTeX\ 0.40 added.
%   \item
%     Checks, whether \cs{pdfoutput} was changed.
%   \end{Version}
%   \begin{Version}{2010/01/28 v2.1}
%   \item
%     Compatibility to ini\TeX\ added.
%   \end{Version}
%   \begin{Version}{2010/09/13 v2.2}
%   \item
%     Code with \cs{escapechar} rewritten because of
%     LuaTeX bug: \cs{escapechar} does not work.
%   \end{Version}
%   \begin{Version}{2011/01/30 v2.3}
%   \item
%     Already loaded package files are not input in \hologo{plainTeX}.
%   \end{Version}
% \end{History}
%
% \PrintIndex
%
% \Finale
\endinput
|
% \end{quote}
% Do not forget to quote the argument according to the demands
% of your shell.
%
% \paragraph{Generating the documentation.}
% You can use both the \xfile{.dtx} or the \xfile{.drv} to generate
% the documentation. The process can be configured by the
% configuration file \xfile{ltxdoc.cfg}. For instance, put this
% line into this file, if you want to have A4 as paper format:
% \begin{quote}
%   \verb|\PassOptionsToClass{a4paper}{article}|
% \end{quote}
% An example follows how to generate the
% documentation with pdf\LaTeX:
% \begin{quote}
%\begin{verbatim}
%pdflatex ifpdf.dtx
%makeindex -s gind.ist ifpdf.idx
%pdflatex ifpdf.dtx
%makeindex -s gind.ist ifpdf.idx
%pdflatex ifpdf.dtx
%\end{verbatim}
% \end{quote}
%
% \section{Catalogue}
%
% The following XML file can be used as source for the
% \href{http://mirror.ctan.org/help/Catalogue/catalogue.html}{\TeX\ Catalogue}.
% The elements \texttt{caption} and \texttt{description} are imported
% from the original XML file from the Catalogue.
% The name of the XML file in the Catalogue is \xfile{ifpdf.xml}.
%    \begin{macrocode}
%<*catalogue>
<?xml version='1.0' encoding='us-ascii'?>
<!DOCTYPE entry SYSTEM 'catalogue.dtd'>
<entry datestamp='$Date$' modifier='$Author$' id='ifpdf'>
  <name>ifpdf</name>
  <caption>Provides the \ifpdf conditional.</caption>
  <authorref id='auth:oberdiek'/>
  <copyright owner='Heiko Oberdiek' year='2001,2005-2011'/>
  <license type='lppl1.3'/>
  <version number='2.3'/>
  <description>
      This package looks for pdfTeX in pdf mode and implements and
      sets the switch <tt>\ifpdf</tt>. The detection is based on the
      value of <tt>\pdfoutput</tt> (which the package will not
      change).  The package works with plain or LaTeX formats.  To use
      it with LaTeX simply <tt>\usepackage{ifpdf}</tt>. Then use
      <tt>\ifpdf ... \else ... \fi</tt>.
    <p/>
    The package is part of the <xref refid='oberdiek'>oberdiek</xref> bundle.
  </description>
  <documentation details='Package documentation'
      href='ctan:/macros/latex/contrib/oberdiek/ifpdf.pdf'/>
  <ctan file='true' path='/macros/latex/contrib/oberdiek/ifpdf.dtx'/>
  <miktex location='oberdiek'/>
  <texlive location='oberdiek'/>
  <install path='/macros/latex/contrib/oberdiek/oberdiek.tds.zip'/>
</entry>
%</catalogue>
%    \end{macrocode}
%
% \begin{History}
%   \begin{Version}{2001/06/14 v1.0}
%   \item
%     First public version.
%   \end{Version}
%   \begin{Version}{2001/07/14 v1.1}
%   \item
%     Documentation addition: global options
%   \end{Version}
%   \begin{Version}{2001/09/26 v1.2}
%   \item
%     Documentation typo corrected.
%   \item
%     Version number corrected.
%   \item
%     Line number in log entry removed.
%   \end{Version}
%   \begin{Version}{2005/07/22 v1.3}
%   \item
%     Some source code comments from Robin Fairbairns added.
%   \item
%     Bug fix for negative values of \cs{pdfoutput}
%     (Oleg Katsitadze)
%   \item
%     LPPL 1.3
%   \item
%     Installation section with locations added.
%   \end{Version}
%   \begin{Version}{2006/02/20 v1.4}
%   \item
%     DTX framework.
%   \item
%     More robust check in case of undefined \cs{pdfoutput}.
%   \item
%     Extended documentation.
%   \end{Version}
%   \begin{Version}{2007/09/09 v1.5}
%   \item
%     Catcode settings added.
%   \end{Version}
%   \begin{Version}{2007/12/12 v1.6}
%   \item
%     Minor update.
%   \end{Version}
%   \begin{Version}{2008/12/12 v1.7}
%   \item
%     Fix in documentation for \cs{boolean} (found by S. Venkataraman).
%   \item
%     Code is not changed.
%   \end{Version}
%   \begin{Version}{2009/04/10 v2.0}
%   \item
%     Support for \LuaTeX\ 0.40 added.
%   \item
%     Checks, whether \cs{pdfoutput} was changed.
%   \end{Version}
%   \begin{Version}{2010/01/28 v2.1}
%   \item
%     Compatibility to ini\TeX\ added.
%   \end{Version}
%   \begin{Version}{2010/09/13 v2.2}
%   \item
%     Code with \cs{escapechar} rewritten because of
%     LuaTeX bug: \cs{escapechar} does not work.
%   \end{Version}
%   \begin{Version}{2011/01/30 v2.3}
%   \item
%     Already loaded package files are not input in \hologo{plainTeX}.
%   \end{Version}
% \end{History}
%
% \PrintIndex
%
% \Finale
\endinput
|
% \end{quote}
% Do not forget to quote the argument according to the demands
% of your shell.
%
% \paragraph{Generating the documentation.}
% You can use both the \xfile{.dtx} or the \xfile{.drv} to generate
% the documentation. The process can be configured by the
% configuration file \xfile{ltxdoc.cfg}. For instance, put this
% line into this file, if you want to have A4 as paper format:
% \begin{quote}
%   \verb|\PassOptionsToClass{a4paper}{article}|
% \end{quote}
% An example follows how to generate the
% documentation with pdf\LaTeX:
% \begin{quote}
%\begin{verbatim}
%pdflatex ifpdf.dtx
%makeindex -s gind.ist ifpdf.idx
%pdflatex ifpdf.dtx
%makeindex -s gind.ist ifpdf.idx
%pdflatex ifpdf.dtx
%\end{verbatim}
% \end{quote}
%
% \section{Catalogue}
%
% The following XML file can be used as source for the
% \href{http://mirror.ctan.org/help/Catalogue/catalogue.html}{\TeX\ Catalogue}.
% The elements \texttt{caption} and \texttt{description} are imported
% from the original XML file from the Catalogue.
% The name of the XML file in the Catalogue is \xfile{ifpdf.xml}.
%    \begin{macrocode}
%<*catalogue>
<?xml version='1.0' encoding='us-ascii'?>
<!DOCTYPE entry SYSTEM 'catalogue.dtd'>
<entry datestamp='$Date$' modifier='$Author$' id='ifpdf'>
  <name>ifpdf</name>
  <caption>Provides the \ifpdf conditional.</caption>
  <authorref id='auth:oberdiek'/>
  <copyright owner='Heiko Oberdiek' year='2001,2005-2011'/>
  <license type='lppl1.3'/>
  <version number='2.3'/>
  <description>
      This package looks for pdfTeX in pdf mode and implements and
      sets the switch <tt>\ifpdf</tt>. The detection is based on the
      value of <tt>\pdfoutput</tt> (which the package will not
      change).  The package works with plain or LaTeX formats.  To use
      it with LaTeX simply <tt>\usepackage{ifpdf}</tt>. Then use
      <tt>\ifpdf ... \else ... \fi</tt>.
    <p/>
    The package is part of the <xref refid='oberdiek'>oberdiek</xref> bundle.
  </description>
  <documentation details='Package documentation'
      href='ctan:/macros/latex/contrib/oberdiek/ifpdf.pdf'/>
  <ctan file='true' path='/macros/latex/contrib/oberdiek/ifpdf.dtx'/>
  <miktex location='oberdiek'/>
  <texlive location='oberdiek'/>
  <install path='/macros/latex/contrib/oberdiek/oberdiek.tds.zip'/>
</entry>
%</catalogue>
%    \end{macrocode}
%
% \begin{History}
%   \begin{Version}{2001/06/14 v1.0}
%   \item
%     First public version.
%   \end{Version}
%   \begin{Version}{2001/07/14 v1.1}
%   \item
%     Documentation addition: global options
%   \end{Version}
%   \begin{Version}{2001/09/26 v1.2}
%   \item
%     Documentation typo corrected.
%   \item
%     Version number corrected.
%   \item
%     Line number in log entry removed.
%   \end{Version}
%   \begin{Version}{2005/07/22 v1.3}
%   \item
%     Some source code comments from Robin Fairbairns added.
%   \item
%     Bug fix for negative values of \cs{pdfoutput}
%     (Oleg Katsitadze)
%   \item
%     LPPL 1.3
%   \item
%     Installation section with locations added.
%   \end{Version}
%   \begin{Version}{2006/02/20 v1.4}
%   \item
%     DTX framework.
%   \item
%     More robust check in case of undefined \cs{pdfoutput}.
%   \item
%     Extended documentation.
%   \end{Version}
%   \begin{Version}{2007/09/09 v1.5}
%   \item
%     Catcode settings added.
%   \end{Version}
%   \begin{Version}{2007/12/12 v1.6}
%   \item
%     Minor update.
%   \end{Version}
%   \begin{Version}{2008/12/12 v1.7}
%   \item
%     Fix in documentation for \cs{boolean} (found by S. Venkataraman).
%   \item
%     Code is not changed.
%   \end{Version}
%   \begin{Version}{2009/04/10 v2.0}
%   \item
%     Support for \LuaTeX\ 0.40 added.
%   \item
%     Checks, whether \cs{pdfoutput} was changed.
%   \end{Version}
%   \begin{Version}{2010/01/28 v2.1}
%   \item
%     Compatibility to ini\TeX\ added.
%   \end{Version}
%   \begin{Version}{2010/09/13 v2.2}
%   \item
%     Code with \cs{escapechar} rewritten because of
%     LuaTeX bug: \cs{escapechar} does not work.
%   \end{Version}
%   \begin{Version}{2011/01/30 v2.3}
%   \item
%     Already loaded package files are not input in \hologo{plainTeX}.
%   \end{Version}
% \end{History}
%
% \PrintIndex
%
% \Finale
\endinput
